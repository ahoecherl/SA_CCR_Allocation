    

    
    \hypertarget{exemplary-sa-ccr-allocation-under-consideration-of-an-initial-margin-threshold}{%
\section{Exemplary SA-CCR allocation under consideration of an initial
margin
threshold}\label{exemplary-sa-ccr-allocation-under-consideration-of-an-initial-margin-threshold}}

Our goal is to perform an Euler allocation for the minimal example of a
one trade portfolio. We use the same 200Bn IRS as in previous examples.

    \begin{tcolorbox}[breakable, size=fbox, boxrule=1pt, pad at break*=1mm,colback=cellbackground, colframe=cellborder]
\prompt{In}{incolor}{In}{\boxspacing}
\begin{Verbatim}[commandchars=\\\{\}]
\PY{n}{irs} \PY{o}{=} \PY{n}{IRS}\PY{p}{(}\PY{n}{notional}\PY{o}{=}\PY{l+m+mi}{200000000000}\PY{p}{,}
          \PY{n}{index}\PY{o}{=}\PY{n}{InterestRateIndex}\PY{o}{.}\PY{n}{USDLIBOR3M}\PY{p}{,}
          \PY{n}{timeToSwapStart}\PY{o}{=}\PY{n}{ql}\PY{o}{.}\PY{n}{Period}\PY{p}{(}\PY{l+m+mi}{2}\PY{p}{,} \PY{n}{ql}\PY{o}{.}\PY{n}{Days}\PY{p}{)}\PY{p}{,}
          \PY{n}{timeToSwapEnd}\PY{o}{=}\PY{n}{ql}\PY{o}{.}\PY{n}{Period}\PY{p}{(}\PY{l+m+mi}{10}\PY{p}{,} \PY{n}{ql}\PY{o}{.}\PY{n}{Years}\PY{p}{)}\PY{p}{)}
\end{Verbatim}
\end{tcolorbox}

    \begin{tcolorbox}[breakable, size=fbox, boxrule=1pt, pad at break*=1mm,colback=cellbackground, colframe=cellborder]
\prompt{In}{incolor}{In}{\boxspacing}
\begin{Verbatim}[commandchars=\\\{\}]
\PY{n}{ca} \PY{o}{=} \PY{n}{CollateralAgreement}\PY{p}{(}\PY{n}{threshold}\PY{o}{=}\PY{l+m+mi}{2000000000}\PY{p}{,}
                         \PY{n}{mta}\PY{o}{=}\PY{l+m+mi}{0}\PY{p}{)}
\PY{n}{ca}\PY{o}{.}\PY{n}{link\PYZus{}sa\PYZus{}ccr\PYZus{}instance}\PY{p}{(}\PY{n}{SA\PYZus{}CCR}\PY{p}{(}\PY{n}{ca}\PY{p}{)}\PY{p}{)}
\PY{n}{ca}\PY{o}{.}\PY{n}{add\PYZus{}trades}\PY{p}{(}\PY{n}{irs}\PY{p}{)}
\end{Verbatim}
\end{tcolorbox}

    The inclusion of the threshold raises the ead since it lowers the
available overcollataralization.

    \begin{tcolorbox}[breakable, size=fbox, boxrule=1pt, pad at break*=1mm,colback=cellbackground, colframe=cellborder]
\prompt{In}{incolor}{In}{\boxspacing}
\begin{Verbatim}[commandchars=\\\{\}]
\PY{n}{ead\PYZus{}with\PYZus{}threshold} \PY{o}{=} \PY{n}{ca}\PY{o}{.}\PY{n}{get\PYZus{}sa\PYZus{}ccr\PYZus{}model}\PY{p}{(}\PY{p}{)}\PY{o}{.}\PY{n}{get\PYZus{}risk\PYZus{}measure}\PY{p}{(}\PY{p}{)}
\PY{n+nb}{print}\PY{p}{(}\PY{n}{ead\PYZus{}with\PYZus{}threshold}\PY{p}{)}
\PY{n}{ca}\PY{o}{.}\PY{n}{threshold} \PY{o}{=} \PY{l+m+mi}{0}
\PY{n}{ead\PYZus{}no\PYZus{}threshold} \PY{o}{=} \PY{n}{ca}\PY{o}{.}\PY{n}{get\PYZus{}sa\PYZus{}ccr\PYZus{}model}\PY{p}{(}\PY{p}{)}\PY{o}{.}\PY{n}{get\PYZus{}risk\PYZus{}measure}\PY{p}{(}\PY{p}{)}
\PY{n+nb}{print}\PY{p}{(}\PY{n}{ead\PYZus{}no\PYZus{}threshold}\PY{p}{)}
\PY{n}{ca}\PY{o}{.}\PY{n}{threshold} \PY{o}{=} \PY{l+m+mi}{2000000000}
\end{Verbatim}
\end{tcolorbox}

    \begin{Verbatim}[commandchars=\\\{\}]
817798882.3982577
582881953.4648591
    \end{Verbatim}

    The EAD with threshold is 817798882.40 while the EAD without threshold
is 582881953.46

    
    When trying to allocate with threshold we realize that the allocation is
not working due to the missing homogeneity of C.

    \begin{tcolorbox}[breakable, size=fbox, boxrule=1pt, pad at break*=1mm,colback=cellbackground, colframe=cellborder]
\prompt{In}{incolor}{In}{\boxspacing}
\begin{Verbatim}[commandchars=\\\{\}]
\PY{n}{eulerAllocator} \PY{o}{=} \PY{n}{EulerAllocator}\PY{p}{(}\PY{n}{ca}\PY{p}{)}
\PY{n}{allocated\PYZus{}value} \PY{o}{=} \PY{n}{eulerAllocator}\PY{o}{.}\PY{n}{allocate\PYZus{}ead}\PY{p}{(}\PY{p}{)}\PY{p}{[}\PY{n}{irs}\PY{p}{]}
\PY{n+nb}{print}\PY{p}{(}\PY{n}{allocated\PYZus{}value}\PY{p}{)}
\end{Verbatim}
\end{tcolorbox}

    \begin{Verbatim}[commandchars=\\\{\}]
526604164.2713547
    \end{Verbatim}

    The value of 526604164.27 that has been allocated to the single trade in
the portfolio is far off from the portfolios EAD of 817798882.40

    
    If we instead allocate without threshold, the allocation works.

    \begin{tcolorbox}[breakable, size=fbox, boxrule=1pt, pad at break*=1mm,colback=cellbackground, colframe=cellborder]
\prompt{In}{incolor}{In}{\boxspacing}
\begin{Verbatim}[commandchars=\\\{\}]
\PY{n}{ca}\PY{o}{.}\PY{n}{threshold} \PY{o}{=} \PY{l+m+mi}{0}
\PY{n}{allocated\PYZus{}value}\PY{o}{=}\PY{n}{eulerAllocator}\PY{o}{.}\PY{n}{allocate\PYZus{}ead}\PY{p}{(}\PY{p}{)}\PY{p}{[}\PY{n}{irs}\PY{p}{]}
\PY{n+nb}{print}\PY{p}{(}\PY{n}{allocated\PYZus{}value}\PY{p}{)}
\end{Verbatim}
\end{tcolorbox}

    \begin{Verbatim}[commandchars=\\\{\}]
582887839.6499157
    \end{Verbatim}

    The value of 582887839.65 that has been allocated to the single trade in
the portfolio equals the portfolios EAD without threshold of
582881953.46

    
    A reasonable approach to allocate an SA-CCR EAD under consideration of a
threshold could be to allocate without threshold and then scale
accordingly:

\begin{align}
\label{eq:C threshold scaling}
    X_{t\text{, TH}} = X_{t\text{, no TH}} \frac{EAD_{\text{TH}}}{EAD_{\text{no TH}}}
\end{align}

    \begin{tcolorbox}[breakable, size=fbox, boxrule=1pt, pad at break*=1mm,colback=cellbackground, colframe=cellborder]
\prompt{In}{incolor}{In}{\boxspacing}
\begin{Verbatim}[commandchars=\\\{\}]
\PY{n+nb}{print}\PY{p}{(}\PY{n}{allocated\PYZus{}value} \PY{o}{*} \PY{p}{(}\PY{n}{ead\PYZus{}with\PYZus{}threshold}\PY{o}{/}\PY{n}{ead\PYZus{}no\PYZus{}threshold}\PY{p}{)}\PY{p}{)}
\end{Verbatim}
\end{tcolorbox}

    \begin{Verbatim}[commandchars=\\\{\}]
817807140.8724346
    \end{Verbatim}

    However, this approach does coincide with a loss of precision. If the
\(C_{calc}\) is below the threshold then \(C\) is 0 and exhibits
homogeneity, even when taking the threshold into account. We can
construct an example that shows, that the approach in
\ref{eq:C threshold scaling} does not yield the correct allocation which
can be calculated when taking the threshold into account since the IM is
below the IM threshold.

We set up an IRS and an Equity option that have a similar EAD on their
own. Here we calculate with a threshold of 50Mn which is a common value
as it is the maximum amount permitted by the regulator.
\todo{Cite source for this}

    \begin{tcolorbox}[breakable, size=fbox, boxrule=1pt, pad at break*=1mm,colback=cellbackground, colframe=cellborder]
\prompt{In}{incolor}{In}{\boxspacing}
\begin{Verbatim}[commandchars=\\\{\}]
\PY{n}{irs} \PY{o}{=} \PY{n}{IRS}\PY{p}{(}\PY{n}{notional}\PY{o}{=}\PY{l+m+mi}{100000000}\PY{p}{,}
          \PY{n}{index} \PY{o}{=} \PY{n}{InterestRateIndex}\PY{o}{.}\PY{n}{USDLIBOR3M}\PY{p}{,}
          \PY{n}{timeToSwapStart}\PY{o}{=}\PY{n}{ql}\PY{o}{.}\PY{n}{Period}\PY{p}{(}\PY{l+m+mi}{2}\PY{p}{,} \PY{n}{ql}\PY{o}{.}\PY{n}{Days}\PY{p}{)}\PY{p}{,}
          \PY{n}{timeToSwapEnd}\PY{o}{=}\PY{n}{ql}\PY{o}{.}\PY{n}{Period}\PY{p}{(}\PY{l+m+mi}{10}\PY{p}{,} \PY{n}{ql}\PY{o}{.}\PY{n}{Years}\PY{p}{)}\PY{p}{)}

\PY{n}{ca} \PY{o}{=} \PY{n}{CollateralAgreement}\PY{p}{(}\PY{n}{threshold}\PY{o}{=}\PY{l+m+mi}{50000000}\PY{p}{)}
\PY{n}{ca}\PY{o}{.}\PY{n}{link\PYZus{}sa\PYZus{}ccr\PYZus{}instance}\PY{p}{(}\PY{n}{SA\PYZus{}CCR}\PY{p}{(}\PY{n}{ca}\PY{p}{)}\PY{p}{)}
\PY{n}{ca}\PY{o}{.}\PY{n}{add\PYZus{}trades}\PY{p}{(}\PY{n}{irs}\PY{p}{)}
\end{Verbatim}
\end{tcolorbox}

    \begin{tcolorbox}[breakable, size=fbox, boxrule=1pt, pad at break*=1mm,colback=cellbackground, colframe=cellborder]
\prompt{In}{incolor}{In}{\boxspacing}
\begin{Verbatim}[commandchars=\\\{\}]
\PY{n+nb}{print}\PY{p}{(}\PY{n}{ca}\PY{o}{.}\PY{n}{get\PYZus{}im\PYZus{}model}\PY{p}{(}\PY{p}{)}\PY{o}{.}\PY{n}{get\PYZus{}risk\PYZus{}measure}\PY{p}{(}\PY{p}{)}\PY{p}{)}
\PY{n+nb}{print}\PY{p}{(}\PY{n}{ca}\PY{o}{.}\PY{n}{get\PYZus{}sa\PYZus{}ccr\PYZus{}model}\PY{p}{(}\PY{p}{)}\PY{o}{.}\PY{n}{get\PYZus{}risk\PYZus{}measure}\PY{p}{(}\PY{p}{)}\PY{p}{)}
\end{Verbatim}
\end{tcolorbox}

    \begin{Verbatim}[commandchars=\\\{\}]
4519078.535740282
1651404.7245623078
    \end{Verbatim}

    Through optimization it can be identified that for a one year long call
option on Adidas an underlying stock count of 403690 stocks results in
the same EAD as for the IRS above.

    
    \begin{tcolorbox}[breakable, size=fbox, boxrule=1pt, pad at break*=1mm,colback=cellbackground, colframe=cellborder]
\prompt{In}{incolor}{In}{\boxspacing}
\begin{Verbatim}[commandchars=\\\{\}]
\PY{n}{eqOpt} \PY{o}{=} \PY{n}{EquityOption}\PY{p}{(}\PY{n}{notional} \PY{o}{=} \PY{l+m+mi}{403106}\PY{p}{,}
                     \PY{n}{maturity} \PY{o}{=} \PY{n}{ql}\PY{o}{.}\PY{n}{Period}\PY{p}{(}\PY{l+m+mi}{1}\PY{p}{,} \PY{n}{ql}\PY{o}{.}\PY{n}{Years}\PY{p}{)}\PY{p}{,}
                     \PY{n}{underlying} \PY{o}{=} \PY{n}{Stock}\PY{o}{.}\PY{n}{ADS}\PY{p}{,}
                     \PY{n}{tradeType} \PY{o}{=} \PY{n}{TradeType}\PY{o}{.}\PY{n}{CALL}\PY{p}{,}
                     \PY{n}{tradeDirection} \PY{o}{=} \PY{n}{TradeDirection}\PY{o}{.}\PY{n}{LONG}\PY{p}{)}
\end{Verbatim}
\end{tcolorbox}

    \begin{tcolorbox}[breakable, size=fbox, boxrule=1pt, pad at break*=1mm,colback=cellbackground, colframe=cellborder]
\prompt{In}{incolor}{In}{\boxspacing}
\begin{Verbatim}[commandchars=\\\{\}]
\PY{n}{eqOpt} \PY{o}{=} \PY{n}{EquityOption}\PY{p}{(}\PY{n}{notional} \PY{o}{=} \PY{n}{eqOptNot}\PY{o}{.}\PY{n}{x}\PY{p}{[}\PY{l+m+mi}{0}\PY{p}{]}\PY{p}{)}
\PY{n}{ca2} \PY{o}{=} \PY{n}{CollateralAgreement}\PY{p}{(}\PY{n}{threshold} \PY{o}{=} \PY{l+m+mi}{50000000}\PY{p}{)}
\PY{n}{ca2}\PY{o}{.}\PY{n}{link\PYZus{}sa\PYZus{}ccr\PYZus{}instance}\PY{p}{(}\PY{n}{SA\PYZus{}CCR}\PY{p}{(}\PY{n}{ca2}\PY{p}{)}\PY{p}{)}
\PY{n}{ca2}\PY{o}{.}\PY{n}{add\PYZus{}trades}\PY{p}{(}\PY{n}{eqOpt}\PY{p}{)}
\end{Verbatim}
\end{tcolorbox}

    The initial margin of this single trade is also far below the 50Mn
threshold but differs significantly from the IM of the IRS.

    \begin{tcolorbox}[breakable, size=fbox, boxrule=1pt, pad at break*=1mm,colback=cellbackground, colframe=cellborder]
\prompt{In}{incolor}{In}{\boxspacing}
\begin{Verbatim}[commandchars=\\\{\}]
\PY{n+nb}{print}\PY{p}{(}\PY{l+s+s1}{\PYZsq{}}\PY{l+s+s1}{EAD: }\PY{l+s+s1}{\PYZsq{}} \PY{o}{+} \PY{n+nb}{str}\PY{p}{(}\PY{n}{ca2}\PY{o}{.}\PY{n}{get\PYZus{}sa\PYZus{}ccr\PYZus{}model}\PY{p}{(}\PY{p}{)}\PY{o}{.}\PY{n}{get\PYZus{}risk\PYZus{}measure}\PY{p}{(}\PY{p}{)}\PY{p}{)}\PY{p}{)}
\PY{n+nb}{print}\PY{p}{(}\PY{l+s+s1}{\PYZsq{}}\PY{l+s+s1}{IM:  }\PY{l+s+s1}{\PYZsq{}} \PY{o}{+} \PY{n+nb}{str}\PY{p}{(}\PY{n}{ca2}\PY{o}{.}\PY{n}{get\PYZus{}im\PYZus{}model}\PY{p}{(}\PY{p}{)}\PY{o}{.}\PY{n}{get\PYZus{}risk\PYZus{}measure}\PY{p}{(}\PY{p}{)}\PY{p}{)}\PY{p}{)}
\end{Verbatim}
\end{tcolorbox}

    \begin{Verbatim}[commandchars=\\\{\}]
EAD: 1651404.7245622769
IM:  2806744.4625389306
    \end{Verbatim}

    Given the market data, difference in model, risk horizon etc. the SA CCR
EAD model calculates the same risk for the two trades when calculated
individually, while the ISDA SIMM IM model evaluates the IRS to be 61\%
riskier.

    
    When putting both trades in a common portfolio we observe, that the EAD
and the IM of this joint portfolio is the sum of the two separate
portfolios. This is not surprising since both, the SA CCR and ISDA SIMM
model do not recognize any hedge effect between different asset classes.

    \begin{tcolorbox}[breakable, size=fbox, boxrule=1pt, pad at break*=1mm,colback=cellbackground, colframe=cellborder]
\prompt{In}{incolor}{In}{\boxspacing}
\begin{Verbatim}[commandchars=\\\{\}]
\PY{n}{ca3} \PY{o}{=} \PY{n}{CollateralAgreement}\PY{p}{(}\PY{n}{threshold}\PY{o}{=}\PY{l+m+mi}{50000000}\PY{p}{)}
\PY{n}{ca3}\PY{o}{.}\PY{n}{link\PYZus{}sa\PYZus{}ccr\PYZus{}instance}\PY{p}{(}\PY{n}{SA\PYZus{}CCR}\PY{p}{(}\PY{n}{ca3}\PY{p}{)}\PY{p}{)}
\PY{n}{ca3}\PY{o}{.}\PY{n}{add\PYZus{}trades}\PY{p}{(}\PY{p}{[}\PY{n}{irs}\PY{p}{,} \PY{n}{eqOpt}\PY{p}{]}\PY{p}{)}
\end{Verbatim}
\end{tcolorbox}

    The EAD of the combined portfolio is 3302809 USD. The calculated IM of
the combined portfolio is 7325822 USD.

    
    \begin{tcolorbox}[breakable, size=fbox, boxrule=1pt, pad at break*=1mm,colback=cellbackground, colframe=cellborder]
\prompt{In}{incolor}{In}{\boxspacing}
\begin{Verbatim}[commandchars=\\\{\}]
\PY{n}{eulerAllocator} \PY{o}{=} \PY{n}{EulerAllocator}\PY{p}{(}\PY{n}{ca3}\PY{p}{)}
\PY{n}{ca3}\PY{o}{.}\PY{n}{threshold} \PY{o}{=} \PY{l+m+mi}{50000000}
\PY{n}{allocation} \PY{o}{=} \PY{n}{eulerAllocator}\PY{o}{.}\PY{n}{allocate\PYZus{}ead}\PY{p}{(}\PY{p}{)}
\end{Verbatim}
\end{tcolorbox}

    Euler allocation allocates 1651404 USD of the EAD to the IRS.Euler
allocation allocates 1651404 USD of the EAD to the equity option.The sum
of these two allocations is 3302809 USD, Which is close to the EAD
calculated for the portfolio of 3302809 USD.

    
    Due to the high threshold, no IM is exchanged. Only VM is exchanged
which is not overcollateralization and therefore only reduces the \(RC\)
in formula \todo{reference RC+PFE formula} to 0 but does not impact the
\(PFE\).

    The available collateral is 1452396 USD. Due to the hight threshold this
consists only of the VM which is 1452396 USD.

    
    The 50/50 allocation is certainly the correct result since there are no
hedge effects between the two trades and they both have the same stand
alone EAD. However, when applying formula \ref{eq:C threshold scaling}
we yield a different result.

    \begin{tcolorbox}[breakable, size=fbox, boxrule=1pt, pad at break*=1mm,colback=cellbackground, colframe=cellborder]
\prompt{In}{incolor}{In}{\boxspacing}
\begin{Verbatim}[commandchars=\\\{\}]
\PY{n}{ead\PYZus{}with\PYZus{}threshold} \PY{o}{=} \PY{n}{ca3}\PY{o}{.}\PY{n}{get\PYZus{}sa\PYZus{}ccr\PYZus{}model}\PY{p}{(}\PY{p}{)}\PY{o}{.}\PY{n}{get\PYZus{}risk\PYZus{}measure}\PY{p}{(}\PY{p}{)}
\PY{n}{ca3}\PY{o}{.}\PY{n}{threshold}\PY{o}{=}\PY{l+m+mi}{0}
\PY{n}{ead\PYZus{}no\PYZus{}threshold} \PY{o}{=} \PY{n}{ca3}\PY{o}{.}\PY{n}{get\PYZus{}sa\PYZus{}ccr\PYZus{}model}\PY{p}{(}\PY{p}{)}\PY{o}{.}\PY{n}{get\PYZus{}risk\PYZus{}measure}\PY{p}{(}\PY{p}{)}
\PY{n}{allocation} \PY{o}{=} \PY{n}{eulerAllocator}\PY{o}{.}\PY{n}{allocate\PYZus{}ead}\PY{p}{(}\PY{p}{)}
\PY{n}{factor} \PY{o}{=} \PY{n}{ead\PYZus{}with\PYZus{}threshold}\PY{o}{/}\PY{n}{ead\PYZus{}no\PYZus{}threshold}
\PY{n+nb}{print}\PY{p}{(}\PY{n}{allocation}\PY{p}{[}\PY{n}{irs}\PY{p}{]}\PY{o}{*}\PY{n}{factor}\PY{p}{)}
\PY{n}{ca3}\PY{o}{.}\PY{n}{threshold} \PY{o}{=} \PY{l+m+mi}{50000000}
\end{Verbatim}
\end{tcolorbox}

    \begin{Verbatim}[commandchars=\\\{\}]
1154590.5057480328
    \end{Verbatim}

    Euler allocation allocates 1154590 USD of the EAD to the IRS. Euler
allocation allocates 2148228 USD of the EAD to the equity option. The
sum of these two allocations is 3302818 USD, which is close to the EAD
calculated for the portfolio of 3302809 USD.

    
    Therefore, the approximation of \ref{eq:C threshold scaling} should only
be used if the \(IM_{calc} > TH_{IM}\).

When we exceed the threshold we can see that the Euler allocation with
threshold returns the same results as the Euler allocation without
thresholds. For this we decrease the IM threshold from 10Mn to 1Mn.

    \begin{tcolorbox}[breakable, size=fbox, boxrule=1pt, pad at break*=1mm,colback=cellbackground, colframe=cellborder]
\prompt{In}{incolor}{In}{\boxspacing}
\begin{Verbatim}[commandchars=\\\{\}]
\PY{n}{ca3}\PY{o}{.}\PY{n}{threshold} \PY{o}{=} \PY{l+m+mi}{1000000}
\PY{n}{allocation\PYZus{}10Mn\PYZus{}threshold} \PY{o}{=} \PY{n}{eulerAllocator}\PY{o}{.}\PY{n}{allocate\PYZus{}ead}\PY{p}{(}\PY{p}{)}

\PY{n}{ca3}\PY{o}{.}\PY{n}{threshold} \PY{o}{=} \PY{l+m+mi}{100000}
\PY{n}{allocation\PYZus{}1Mn\PYZus{}threshold} \PY{o}{=} \PY{n}{eulerAllocator}\PY{o}{.}\PY{n}{allocate\PYZus{}ead}\PY{p}{(}\PY{p}{)}

\PY{n}{ca3}\PY{o}{.}\PY{n}{threshold} \PY{o}{=} \PY{l+m+mi}{0}
\PY{n}{allocation\PYZus{}no\PYZus{}threshold} \PY{o}{=} \PY{n}{eulerAllocator}\PY{o}{.}\PY{n}{allocate\PYZus{}ead}\PY{p}{(}\PY{p}{)}
\end{Verbatim}
\end{tcolorbox}


