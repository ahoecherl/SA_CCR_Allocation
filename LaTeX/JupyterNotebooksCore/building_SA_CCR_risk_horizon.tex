    

    
    \hypertarget{risk-horizon}{%
\subsubsection{Risk Horizon}\label{risk-horizon}}

For unmargined transaction the margining factor is

\[MF^{\text{unmargined}}_i = \sqrt{\frac{\min\left(M_i;1\text{ year}\right)}{1\text{ year}}}\]

This factor can be used to scale down a risk weight calibrated for a 1
year horizon to a shorter period.

With margining the margin period of risk (MPOR) is:

\begin{itemize}
\tightlist
\item
  10 business days for small, uncleared OTC portfolios
\item
  5 business days for cleared derivatives
\item
  20 business days for netting sets with more than 5000 transactions
  that are not with a central counterparty
\item
  and doubling this period for portfolios with outstanding disputes
\end{itemize}

The margining factor is then

\[ MF^{\text{margined}}_i = \frac{3}{2}\sqrt{\frac{MPOR_i}{1\text{ year}}} \]

At this point we need to introduce a collateral agreement object. For
simplicities sake we will not differentiate between collateral and
netting sets in this thesis. All trades that are covered by the same
collateral agreement are also admissible for netting with each other.
(Also refer to the introduction of close out netting above). To take
into account the different parameters determining the risk horizon a
couple of parameters are required to create a collateral agreement. As
an example, below we are setting up a collateral agreement for uncleared
derivatives without exchange of variation margin or initial margin.

    \begin{tcolorbox}[breakable, size=fbox, boxrule=1pt, pad at break*=1mm,colback=cellbackground, colframe=cellborder]
\prompt{In}{incolor}{In}{\boxspacing}
\begin{Verbatim}[commandchars=\\\{\}]
\PY{n}{ca} \PY{o}{=} \PY{n}{CollateralAgreement}\PY{p}{(}
        \PY{n}{margining}\PY{o}{=}\PY{n}{Margining}\PY{o}{.}\PY{n}{UNMARGINED}\PY{p}{,}
        \PY{n}{clearing}\PY{o}{=}\PY{n}{Clearing}\PY{o}{.}\PY{n}{UNCLEARED}\PY{p}{,}
        \PY{n}{initialMargining}\PY{o}{=}\PY{n}{InitialMargining}\PY{o}{.}\PY{n}{NO\PYZus{}IM}\PY{p}{,}
        \PY{n}{tradecount}\PY{o}{=}\PY{n}{Tradecount}\PY{o}{.}\PY{n}{UNDER\PYZus{}FIVE\PYZus{}THOUSAND}\PY{p}{,}
        \PY{n}{dispute}\PY{o}{=}\PY{n}{Dispute}\PY{o}{.}\PY{n}{NO\PYZus{}OUTSTANDING\PYZus{}DISPUTES}\PY{p}{,}
        \PY{n}{threshold}\PY{o}{=}\PY{l+m+mf}{0.0}\PY{p}{,}      \PY{c+c1}{\PYZsh{}Threshold to trigger a margin call}
        \PY{n}{mta}\PY{o}{=}\PY{l+m+mf}{0.0}\PY{p}{,}            \PY{c+c1}{\PYZsh{}received initial margin}
        \PY{p}{)}
\end{Verbatim}
\end{tcolorbox}

    With this collateral set object we can define a function for calculation
the margining factor:

    For trades of differing maturity let's compare the margining factor for
the three most common scenarios:

\begin{enumerate}
\def\labelenumi{\arabic{enumi}.}
\tightlist
\item
  No margining
\item
  Bilateral margining
\item
  Centrally cleared
\end{enumerate}
 
            
\prompt{Out}{outcolor}{Out}{}
    
    \centering{\begin{tabular}{lrrrrr}
\toprule
{} &  Three days &  Two weeks &  Six months &  One year &  Ten years \\
\midrule
No margining        &      0.2000 &     0.2000 &      0.7071 &    1.0000 &     1.0000 \\
Bilateral margining &      0.3000 &     0.3000 &      0.3000 &    0.3000 &     0.3000 \\
Centrally cleared   &      0.2121 &     0.2121 &      0.2121 &    0.2121 &     0.2121 \\
\bottomrule
\end{tabular}
}

    


