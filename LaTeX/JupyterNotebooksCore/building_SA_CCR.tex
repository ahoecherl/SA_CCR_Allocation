    

    
    For analysis we create an SA-CCR object that implements SA-CCR as
specified in \cite{SACCR}

    When using \emph{SA-CCR} the exposure at default (EAD) has to be
calculated as:

\begin{align*}
EAD &= \alpha * (RC + PFE)\\
\\
\text{where} \qquad \alpha&=1.4 \\
RC&: \text{Replacement Cost} \\
PFE&: \text{Potential Future Exposure}
\end{align*}

    \begin{tcolorbox}[breakable, size=fbox, boxrule=1pt, pad at break*=1mm,colback=cellbackground, colframe=cellborder]
\prompt{In}{incolor}{In}{\boxspacing}
\begin{Verbatim}[commandchars=\\\{\}]
\PY{n}{SA\PYZus{}CCR}\PY{o}{.}\PY{n}{calculate\PYZus{}sa\PYZus{}ccr\PYZus{}ead}\PY{p}{(}\PY{n}{rc} \PY{o}{=} \PY{l+m+mi}{10}\PY{p}{,} \PY{n}{pfe} \PY{o}{=} \PY{l+m+mi}{20}\PY{p}{)}
\end{Verbatim}
\end{tcolorbox}

            \begin{tcolorbox}[breakable, size=fbox, boxrule=.5pt, pad at break*=1mm, opacityfill=0]
\prompt{Out}{outcolor}{Out}{\boxspacing}
\begin{Verbatim}[commandchars=\\\{\}]
42.0
\end{Verbatim}
\end{tcolorbox}
        
    \hypertarget{relation-of-rc-and-pfe}{%
\paragraph{Relation of RC and PFE}\label{relation-of-rc-and-pfe}}

The purpose of the RC is to assess the immediate loss suffered by the
default of a counterparty. It is based on the current MtM of the
derivative less the accessible collateral. If a bank has posted
collateral to non-segregated accounts of a counterparty this collateral
is also assumed to be lost in case of a default which increases the
replacement cost.

The potential future exposure (PFE) on the other hand assesses how the
RC might develop in the future. The future being defined as during the
next year. If the RC today is 0 but is likely to be larger than 0 in the
near future the estimated EAD should take this expected increase in RC
into account.

See also Paragraph 130 and 131 of \cite{SACCR}

    Paragraph 130 - case without margining:

\begin{quote}
For unmargined transactions, the \emph{RC} intends to capture the loss
that would occur if a counterparty were to default and were closed out
of its transactions immediately. The \emph{PFE} add-on represents a
potential conservative increase in exposure over a one-year time horizon
from the present date (i.e.~the calculation date).
\end{quote}

Paragraph 131 - case with margining:

\begin{quote}
For margined trades, the \emph{RC} intends to capture the loss that
would occur if a counterparty were to default at the present or at a
future time, assuming that the closeout and replacement of transactions
occur instantaneously. However, there may be a period (the margin period
of risk) between the last exchange of collateral before default and
replacement of the trades in the market. The \emph{PFE} add-on
represents the potential change in value of the trades during this time
period.
\end{quote}

    \hypertarget{definition-of-potential-future-exposure-pfe}{%
\paragraph{Definition of Potential Future Exposure
(PFE)}\label{definition-of-potential-future-exposure-pfe}}

\begin{align*}
PFE &= \text{multiplier} * AddOn^{\text{aggregate}} \\
\\
\text{where} \qquad AddOn^{\text{aggregate}} &: \text{aggregate add-on component} \\
\text{multiplier} &: f(V,C,AddOn^{\text{aggregate}})
\end{align*}

\(AddOn\) is calculated differently for each asset \(a\) class. Since no
netting is allowed between asset classes the aggregate is calculated as:

\[AddOn^{\text{aggregate}} = \sum_{a}AddOn^{a}\]

Collateralization is taken into account of the PFE calculation through
the multiplier that uses the collateral held as an input. As
overcollateralization e.g.~through IM increases, the multiplier
decreases. However, the multiplier is floored at 5\%.

\begin{align*}
\text{multiplier} &= \min \left\{ 1; Floor + (1-Floor) \exp\left(\frac{V-C}{2(1-Floor)AddOn^{\text{aggregate}}}\right) \right\} \\
\text{where} \qquad Floor &= 5\%
\end{align*}

It is important to note that the multiplier can only be below 1 if
\(C>V\), i.e.~if the portfolio is overcollateralized. If the portfolio
is overcollateralized, the \emph{AddOn} comes into play. The idea behind
the \emph{AddOn} is related to the idea of value at risk. The higher the
\emph{AddOn} the faster the SA-CCR model expects the positions to lose
in value. Therefore, the higher the value, the higher the multiplier.

In the example below the current NPV of the portfolio is 30 and the
received collateral (\emph{IM} + \emph{VM}) is 37. The portfolio is
overcollateralized as it should be when initial margin is used. On the
other hand, as collateralization decreases e.g.~V=C since only VM is
exchanged or C=0 in the case of an uncollateralized portfolio the
multiplier increases.


