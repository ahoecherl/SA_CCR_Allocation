    

    
    \hypertarget{addon-calculation}{%
\section{AddOn calculation}\label{addon-calculation}}

Most of the SA-CCR logic is hidden inside the AddOn calculation. At
first it is important to define the following four data parameters:

\hypertarget{m_i}{%
\subparagraph{\texorpdfstring{\(M_i\)}{M\_i}}\label{m_i}}

\begin{quote}
Maturity of the derivative contract. If the underlying of a derivative
is another derivative - e.g.~in the case of a swaption the maturity date
of the underlying needs to be chosen.
\end{quote}

\hypertarget{s_i}{%
\subparagraph{\texorpdfstring{\(S_i\)}{S\_i}}\label{s_i}}

\begin{quote}
For interest rate and credit derivatives the start date of the time
periodreferenced by an interst rate or credit contract. If the
derivtives underlying is another interest rate or credit intsrument (eg
swaption or bond option) \(S_i\) is the start date of the underlying
instead.
\end{quote}

\hypertarget{e_i}{%
\subparagraph{\texorpdfstring{\(E_i\)}{E\_i}}\label{e_i}}

\begin{quote}
Defined as \(S_i\) but referencing the end date instead of the start
date.
\end{quote}

\hypertarget{t_i}{%
\subparagraph{\texorpdfstring{\(T_i\)}{T\_i}}\label{t_i}}

\begin{quote}
For options across all asset classes this is the latest contractual
exercise date.
\end{quote}

    \hypertarget{trade-level-adjusted-notional}{%
\subsection{Trade level adjusted
notional}\label{trade-level-adjusted-notional}}

Each trade \(i\) has a trade level adjusted notional \(d_i^a\) assigned
to it. This is calculated differently for the different asset classes.

\hypertarget{interest-rate-and-credit-derivatives}{%
\paragraph{Interest rate and credit
derivatives}\label{interest-rate-and-credit-derivatives}}

The notional of the trade is usually a well defined value in domestic
currency for interest rate and credit derivatives. It is multiplied by a
supervisory duration factor. The basic idea is, that the value of the
derivative can change more the longer the remaining

\begin{align*}
d_i &= \text{Notional}_i * SD_i \\
\\
\text{where} \qquad SD_i &=\frac{\exp\left(-0.05 * S_i\right)-\exp\left(-0.05 * E_i\right)}{0.05}
\end{align*}

\hypertarget{fx-derivatives}{%
\paragraph{FX derivatives}\label{fx-derivatives}}

While the wording in the BCBS paper is a bit more specific we will just
assume that every FX traded derivative has a USD leg and set the
notional equal the to USD notional.

\hypertarget{equity-and-commodity-derivatives}{%
\paragraph{Equity and commodity
derivatives}\label{equity-and-commodity-derivatives}}

The notional is defined as the price of the underlying. Therefore, it
fluctuates over time.

\hypertarget{notional-of-exotic-derivatives}{%
\paragraph{Notional of exotic
derivatives}\label{notional-of-exotic-derivatives}}

For more exotic derivatives which do have adjustable notionals,
resetting notionals etc. detailed handling of the notional is defined in
paragraph 158.

    Within this thesis we investigate only equity and interest rate
derivatives. For these we can make a few exemplary calculations of the
trade level adjusted notional.

For equity trades determining the trade level adjusted notional is
trivial as it always is the spot price of the underlying. As an example
consider the two trades defined below:

    \begin{tcolorbox}[breakable, size=fbox, boxrule=1pt, pad at break*=1mm,colback=cellbackground, colframe=cellborder]
\prompt{In}{incolor}{In}{\boxspacing}
\begin{Verbatim}[commandchars=\\\{\}]
\PY{c+c1}{\PYZsh{}When the strike is not set explicitly an at the money option is created with K = S(t0)}
\PY{n}{eqOption1} \PY{o}{=} \PY{n}{EquityOption}\PY{p}{(}\PY{n}{maturity} \PY{o}{=} \PY{n}{ql}\PY{o}{.}\PY{n}{Period}\PY{p}{(}\PY{l+m+mi}{1}\PY{p}{,} \PY{n}{ql}\PY{o}{.}\PY{n}{Years}\PY{p}{)}\PY{p}{,}
                         \PY{n}{tradeType}\PY{o}{=} \PY{n}{TradeType}\PY{o}{.}\PY{n}{CALL}\PY{p}{,}
                         \PY{n}{tradeDirection}\PY{o}{=} \PY{n}{TradeDirection}\PY{o}{.}\PY{n}{LONG}\PY{p}{,}
                         \PY{n}{underlying}\PY{o}{=} \PY{n}{Stock}\PY{o}{.}\PY{n}{ADS}\PY{p}{)}

\PY{n}{eqOption2} \PY{o}{=} \PY{n}{EquityOption}\PY{p}{(}\PY{n}{maturity} \PY{o}{=} \PY{n}{ql}\PY{o}{.}\PY{n}{Period}\PY{p}{(}\PY{l+m+mi}{1}\PY{p}{,}   \PY{n}{ql}\PY{o}{.}\PY{n}{Years}\PY{p}{)}\PY{p}{,}
                         \PY{n}{tradeType}\PY{o}{=} \PY{n}{TradeType}\PY{o}{.}\PY{n}{PUT}\PY{p}{,}
                         \PY{n}{tradeDirection}\PY{o}{=} \PY{n}{TradeDirection}\PY{o}{.}\PY{n}{SHORT}\PY{p}{,}
                         \PY{n}{underlying}\PY{o}{=} \PY{n}{Stock}\PY{o}{.}\PY{n}{ADS}\PY{p}{,}
                         \PY{n}{strike} \PY{o}{=} \PY{l+m+mi}{60}\PY{p}{)}
\end{Verbatim}
\end{tcolorbox}

    Let the spot price of Adidas stock be 42. Then, the adjusted notional of
\texttt{eqOption1}, an at the money call on Adidas, is 42 and the
adjusted notional of \texttt{eqOption2}, a short in the money put on
Adidas, is also 42.

    
    For interest rate derivatives such as interest rate swaps or swaptions
on the other hand, the notional is adjusted by the supervisory duration
factor. As the supervisory duration depends on \emph{S} and \emph{E} it
is important to understand how these are determined for the different
interest rate derivatives.

\begin{longtable}[]{@{}lll@{}}
\toprule
\begin{minipage}[b]{0.24\columnwidth}\raggedright
Trade Type\strut
\end{minipage} & \begin{minipage}[b]{0.32\columnwidth}\raggedright
\emph{S}\strut
\end{minipage} & \begin{minipage}[b]{0.36\columnwidth}\raggedright
\emph{E}\strut
\end{minipage}\tabularnewline
\midrule
\endhead
\begin{minipage}[t]{0.24\columnwidth}\raggedright
\textbf{Interest Rate Swap}\strut
\end{minipage} & \begin{minipage}[t]{0.32\columnwidth}\raggedright
Current date\strut
\end{minipage} & \begin{minipage}[t]{0.36\columnwidth}\raggedright
Maturity date\strut
\end{minipage}\tabularnewline
\begin{minipage}[t]{0.24\columnwidth}\raggedright
\textbf{Forward starting IRS}\strut
\end{minipage} & \begin{minipage}[t]{0.32\columnwidth}\raggedright
Start date of the underlying swap\strut
\end{minipage} & \begin{minipage}[t]{0.36\columnwidth}\raggedright
Maturity date of the underlying swap\strut
\end{minipage}\tabularnewline
\begin{minipage}[t]{0.24\columnwidth}\raggedright
\textbf{Swaption}\strut
\end{minipage} & \begin{minipage}[t]{0.32\columnwidth}\raggedright
Start date of the underlying swap\strut
\end{minipage} & \begin{minipage}[t]{0.36\columnwidth}\raggedright
Maturity data of the underlying swap\strut
\end{minipage}\tabularnewline
\bottomrule
\end{longtable}

    \hypertarget{supervisory-delta-adjustments-delta_i}{%
\subsection{\texorpdfstring{Supervisory delta adjustments:
\(\delta_i\)}{Supervisory delta adjustments: \textbackslash{}delta\_i}}\label{supervisory-delta-adjustments-delta_i}}

For linear derivatives \(\delta\) is 1 for long derivatives and -1 for
short derivatives.

For options \(\delta\) is defined as under Black-Scholes:

\begin{align*}
\delta_{\text{long Call}} &= +\Phi\left(\frac{\ln\left(P_i / K_i \right) + 0.5 * \sigma_i^2 * T_i}{\sigma_i * \sqrt{T_i}}\right) \\
\\
\text{where} \qquad \Phi &: \text{standard normal cdf} \\
\sigma_i &: \text{supervisory volatility as defined in Table 2 in paragraph 183}
\end{align*}

This delta is multiplied by -1 in case of a long Put option or a short
Call option. This formula is used for both, equity options and
swaptions.

No detail is given at this point on the delta calculation of CDO
tranches as these are not in the scope of this thesis.

    In the case of an european equity option the parametrization is quite
straight forward.

\(\sigma_i\): 1.2 is the supervisory volatility for a single stock
option

\(K_i\): The strike of the option

\(P_i\): The spot price of the underlying stock

\(T_i\): The maturity of the option

A swaption on the other hand is parametrized as follows for calculation
of its supervisory delta:

\(\sigma_i\): 0.5 is the supervisory volatility for any interst rate
option.

\(K_i\): The strike of the option is the fixed rate of the underlying
swap

\(P_i\): Is the current par rate of the underlying (forward starting)
swap

\(T_i\): The maturity of the option. Please note the difference to
\(E_i\) used for calculation of the adjusted notional, which is the
maturity of the underlying swap.

SA-CCR uses the same Black-Scholes based formula for Swaps as it uses
for Equities. It differentiates options in two dimensions. Whether they
are \emph{bought} or \emph{sold} and whether they are \emph{Call} or
\emph{Put} options (Compare paragraph 159).

SA-CCR defines an option as a call option if it rises in value as the
underlying rises in value. A fixed payer swap rises in value as the
underlying interest rate rises in value. Therefore, an option to buy a
fixed payer swap at a predetermined strike also rises in value as the
underlying interest rate rises in value. Therefore, a swaption on a
payer swap is considered a \emph{Call} under SA-CCR, while a swaption on
a receiver swap is considered a \emph{Put}.

    For the at the money option \texttt{eqOption1} that was set up above we
yield a supervisory delta adjustment of 0.7257.

    
    For an examplary short european swaption that has a par swap as
underlying (i.e.~the NPV of the swap is 0) that is set up as follows:

    \begin{tcolorbox}[breakable, size=fbox, boxrule=1pt, pad at break*=1mm,colback=cellbackground, colframe=cellborder]
\prompt{In}{incolor}{In}{\boxspacing}
\begin{Verbatim}[commandchars=\\\{\}]
\PY{n}{swap} \PY{o}{=} \PY{n}{IRS}\PY{p}{(}\PY{n}{notional}\PY{o}{=}\PY{l+m+mi}{100}\PY{p}{,}
           \PY{n}{timeToSwapStart}\PY{o}{=}\PY{n}{ql}\PY{o}{.}\PY{n}{Period}\PY{p}{(}\PY{l+m+mi}{1}\PY{p}{,} \PY{n}{ql}\PY{o}{.}\PY{n}{Years}\PY{p}{)}\PY{p}{,}
           \PY{n}{timeToSwapEnd}\PY{o}{=}\PY{n}{ql}\PY{o}{.}\PY{n}{Period}\PY{p}{(}\PY{l+m+mi}{3}\PY{p}{,} \PY{n}{ql}\PY{o}{.}\PY{n}{Years}\PY{p}{)}\PY{p}{,}
           \PY{n}{swapDirection}\PY{o}{=}\PY{n}{SwapDirection}\PY{o}{.}\PY{n}{PAYER}\PY{p}{,}
           \PY{n}{index}\PY{o}{=}\PY{n}{InterestRateIndex}\PY{o}{.}\PY{n}{EURIBOR6M}
          \PY{p}{)}

\PY{n}{swaption} \PY{o}{=} \PY{n}{Swaption}\PY{p}{(}\PY{n}{underlyingSwap}\PY{o}{=}\PY{n}{swap}\PY{p}{,}
                    \PY{n}{optionMaturity}\PY{o}{=}\PY{n}{ql}\PY{o}{.}\PY{n}{Period}\PY{p}{(}\PY{l+m+mi}{1}\PY{p}{,} \PY{n}{ql}\PY{o}{.}\PY{n}{Years}\PY{p}{)}\PY{p}{,}
                    \PY{n}{tradeDirection}\PY{o}{=}\PY{n}{TradeDirection}\PY{o}{.}\PY{n}{SHORT}\PY{p}{)}

\PY{n}{SA\PYZus{}CCR}\PY{o}{.}\PY{n}{calculate\PYZus{}sa\PYZus{}ccr\PYZus{}delta}\PY{p}{(}\PY{n}{swaption}\PY{p}{)}
\end{Verbatim}
\end{tcolorbox}

            \begin{tcolorbox}[breakable, size=fbox, boxrule=.5pt, pad at break*=1mm, opacityfill=0]
\prompt{Out}{outcolor}{Out}{\boxspacing}
\begin{Verbatim}[commandchars=\\\{\}]
-0.5987063256828626
\end{Verbatim}
\end{tcolorbox}
        
    we yield a regulatory delta of -0.5987.

    

