\documentclass[11pt]{article}

    \usepackage[breakable]{tcolorbox}
    \usepackage{parskip} % Stop auto-indenting (to mimic markdown behaviour)
    
    \usepackage{iftex}
    \ifPDFTeX
    	\usepackage[T1]{fontenc}
    	\usepackage{mathpazo}
    \else
    	\usepackage{fontspec}
    \fi

    % Basic figure setup, for now with no caption control since it's done
    % automatically by Pandoc (which extracts ![](path) syntax from Markdown).
    \usepackage{graphicx}
    % Maintain compatibility with old templates. Remove in nbconvert 6.0
    \let\Oldincludegraphics\includegraphics
    % Ensure that by default, figures have no caption (until we provide a
    % proper Figure object with a Caption API and a way to capture that
    % in the conversion process - todo).
    \usepackage{caption}
    \DeclareCaptionFormat{nocaption}{}
    \captionsetup{format=nocaption,aboveskip=0pt,belowskip=0pt}

    \usepackage[Export]{adjustbox} % Used to constrain images to a maximum size
    \adjustboxset{max size={0.9\linewidth}{0.9\paperheight}}
    \usepackage{float}
    \floatplacement{figure}{H} % forces figures to be placed at the correct location
    \usepackage{xcolor} % Allow colors to be defined
    \usepackage{enumerate} % Needed for markdown enumerations to work
    \usepackage{geometry} % Used to adjust the document margins
    \usepackage{amsmath} % Equations
    \usepackage{amssymb} % Equations
    \usepackage{textcomp} % defines textquotesingle
    % Hack from http://tex.stackexchange.com/a/47451/13684:
    \AtBeginDocument{%
        \def\PYZsq{\textquotesingle}% Upright quotes in Pygmentized code
    }
    \usepackage{upquote} % Upright quotes for verbatim code
    \usepackage{eurosym} % defines \euro
    \usepackage[mathletters]{ucs} % Extended unicode (utf-8) support
    \usepackage{fancyvrb} % verbatim replacement that allows latex
    \usepackage{grffile} % extends the file name processing of package graphics 
                         % to support a larger range
    \makeatletter % fix for grffile with XeLaTeX
    \def\Gread@@xetex#1{%
      \IfFileExists{"\Gin@base".bb}%
      {\Gread@eps{\Gin@base.bb}}%
      {\Gread@@xetex@aux#1}%
    }
    \makeatother

    % The hyperref package gives us a pdf with properly built
    % internal navigation ('pdf bookmarks' for the table of contents,
    % internal cross-reference links, web links for URLs, etc.)
    \usepackage{hyperref}
    % The default LaTeX title has an obnoxious amount of whitespace. By default,
    % titling removes some of it. It also provides customization options.
    \usepackage{titling}
    \usepackage{longtable} % longtable support required by pandoc >1.10
    \usepackage{booktabs}  % table support for pandoc > 1.12.2
    \usepackage[inline]{enumitem} % IRkernel/repr support (it uses the enumerate* environment)
    \usepackage[normalem]{ulem} % ulem is needed to support strikethroughs (\sout)
                                % normalem makes italics be italics, not underlines
    \usepackage{mathrsfs}
    

    
    % Colors for the hyperref package
    \definecolor{urlcolor}{rgb}{0,.145,.698}
    \definecolor{linkcolor}{rgb}{.71,0.21,0.01}
    \definecolor{citecolor}{rgb}{.12,.54,.11}

    % ANSI colors
    \definecolor{ansi-black}{HTML}{3E424D}
    \definecolor{ansi-black-intense}{HTML}{282C36}
    \definecolor{ansi-red}{HTML}{E75C58}
    \definecolor{ansi-red-intense}{HTML}{B22B31}
    \definecolor{ansi-green}{HTML}{00A250}
    \definecolor{ansi-green-intense}{HTML}{007427}
    \definecolor{ansi-yellow}{HTML}{DDB62B}
    \definecolor{ansi-yellow-intense}{HTML}{B27D12}
    \definecolor{ansi-blue}{HTML}{208FFB}
    \definecolor{ansi-blue-intense}{HTML}{0065CA}
    \definecolor{ansi-magenta}{HTML}{D160C4}
    \definecolor{ansi-magenta-intense}{HTML}{A03196}
    \definecolor{ansi-cyan}{HTML}{60C6C8}
    \definecolor{ansi-cyan-intense}{HTML}{258F8F}
    \definecolor{ansi-white}{HTML}{C5C1B4}
    \definecolor{ansi-white-intense}{HTML}{A1A6B2}
    \definecolor{ansi-default-inverse-fg}{HTML}{FFFFFF}
    \definecolor{ansi-default-inverse-bg}{HTML}{000000}

    % commands and environments needed by pandoc snippets
    % extracted from the output of `pandoc -s`
    \providecommand{\tightlist}{%
      \setlength{\itemsep}{0pt}\setlength{\parskip}{0pt}}
    \DefineVerbatimEnvironment{Highlighting}{Verbatim}{commandchars=\\\{\}}
    % Add ',fontsize=\small' for more characters per line
    \newenvironment{Shaded}{}{}
    \newcommand{\KeywordTok}[1]{\textcolor[rgb]{0.00,0.44,0.13}{\textbf{{#1}}}}
    \newcommand{\DataTypeTok}[1]{\textcolor[rgb]{0.56,0.13,0.00}{{#1}}}
    \newcommand{\DecValTok}[1]{\textcolor[rgb]{0.25,0.63,0.44}{{#1}}}
    \newcommand{\BaseNTok}[1]{\textcolor[rgb]{0.25,0.63,0.44}{{#1}}}
    \newcommand{\FloatTok}[1]{\textcolor[rgb]{0.25,0.63,0.44}{{#1}}}
    \newcommand{\CharTok}[1]{\textcolor[rgb]{0.25,0.44,0.63}{{#1}}}
    \newcommand{\StringTok}[1]{\textcolor[rgb]{0.25,0.44,0.63}{{#1}}}
    \newcommand{\CommentTok}[1]{\textcolor[rgb]{0.38,0.63,0.69}{\textit{{#1}}}}
    \newcommand{\OtherTok}[1]{\textcolor[rgb]{0.00,0.44,0.13}{{#1}}}
    \newcommand{\AlertTok}[1]{\textcolor[rgb]{1.00,0.00,0.00}{\textbf{{#1}}}}
    \newcommand{\FunctionTok}[1]{\textcolor[rgb]{0.02,0.16,0.49}{{#1}}}
    \newcommand{\RegionMarkerTok}[1]{{#1}}
    \newcommand{\ErrorTok}[1]{\textcolor[rgb]{1.00,0.00,0.00}{\textbf{{#1}}}}
    \newcommand{\NormalTok}[1]{{#1}}
    
    % Additional commands for more recent versions of Pandoc
    \newcommand{\ConstantTok}[1]{\textcolor[rgb]{0.53,0.00,0.00}{{#1}}}
    \newcommand{\SpecialCharTok}[1]{\textcolor[rgb]{0.25,0.44,0.63}{{#1}}}
    \newcommand{\VerbatimStringTok}[1]{\textcolor[rgb]{0.25,0.44,0.63}{{#1}}}
    \newcommand{\SpecialStringTok}[1]{\textcolor[rgb]{0.73,0.40,0.53}{{#1}}}
    \newcommand{\ImportTok}[1]{{#1}}
    \newcommand{\DocumentationTok}[1]{\textcolor[rgb]{0.73,0.13,0.13}{\textit{{#1}}}}
    \newcommand{\AnnotationTok}[1]{\textcolor[rgb]{0.38,0.63,0.69}{\textbf{\textit{{#1}}}}}
    \newcommand{\CommentVarTok}[1]{\textcolor[rgb]{0.38,0.63,0.69}{\textbf{\textit{{#1}}}}}
    \newcommand{\VariableTok}[1]{\textcolor[rgb]{0.10,0.09,0.49}{{#1}}}
    \newcommand{\ControlFlowTok}[1]{\textcolor[rgb]{0.00,0.44,0.13}{\textbf{{#1}}}}
    \newcommand{\OperatorTok}[1]{\textcolor[rgb]{0.40,0.40,0.40}{{#1}}}
    \newcommand{\BuiltInTok}[1]{{#1}}
    \newcommand{\ExtensionTok}[1]{{#1}}
    \newcommand{\PreprocessorTok}[1]{\textcolor[rgb]{0.74,0.48,0.00}{{#1}}}
    \newcommand{\AttributeTok}[1]{\textcolor[rgb]{0.49,0.56,0.16}{{#1}}}
    \newcommand{\InformationTok}[1]{\textcolor[rgb]{0.38,0.63,0.69}{\textbf{\textit{{#1}}}}}
    \newcommand{\WarningTok}[1]{\textcolor[rgb]{0.38,0.63,0.69}{\textbf{\textit{{#1}}}}}
    
    
    % Define a nice break command that doesn't care if a line doesn't already
    % exist.
    \def\br{\hspace*{\fill} \\* }
    % Math Jax compatibility definitions
    \def\gt{>}
    \def\lt{<}
    \let\Oldtex\TeX
    \let\Oldlatex\LaTeX
    \renewcommand{\TeX}{\textrm{\Oldtex}}
    \renewcommand{\LaTeX}{\textrm{\Oldlatex}}
    % Document parameters
    % Document title
    \title{SA\_CCR\_examplary\_calculation}
    
    
    
    
    
% Pygments definitions
\makeatletter
\def\PY@reset{\let\PY@it=\relax \let\PY@bf=\relax%
    \let\PY@ul=\relax \let\PY@tc=\relax%
    \let\PY@bc=\relax \let\PY@ff=\relax}
\def\PY@tok#1{\csname PY@tok@#1\endcsname}
\def\PY@toks#1+{\ifx\relax#1\empty\else%
    \PY@tok{#1}\expandafter\PY@toks\fi}
\def\PY@do#1{\PY@bc{\PY@tc{\PY@ul{%
    \PY@it{\PY@bf{\PY@ff{#1}}}}}}}
\def\PY#1#2{\PY@reset\PY@toks#1+\relax+\PY@do{#2}}

\expandafter\def\csname PY@tok@w\endcsname{\def\PY@tc##1{\textcolor[rgb]{0.73,0.73,0.73}{##1}}}
\expandafter\def\csname PY@tok@c\endcsname{\let\PY@it=\textit\def\PY@tc##1{\textcolor[rgb]{0.25,0.50,0.50}{##1}}}
\expandafter\def\csname PY@tok@cp\endcsname{\def\PY@tc##1{\textcolor[rgb]{0.74,0.48,0.00}{##1}}}
\expandafter\def\csname PY@tok@k\endcsname{\let\PY@bf=\textbf\def\PY@tc##1{\textcolor[rgb]{0.00,0.50,0.00}{##1}}}
\expandafter\def\csname PY@tok@kp\endcsname{\def\PY@tc##1{\textcolor[rgb]{0.00,0.50,0.00}{##1}}}
\expandafter\def\csname PY@tok@kt\endcsname{\def\PY@tc##1{\textcolor[rgb]{0.69,0.00,0.25}{##1}}}
\expandafter\def\csname PY@tok@o\endcsname{\def\PY@tc##1{\textcolor[rgb]{0.40,0.40,0.40}{##1}}}
\expandafter\def\csname PY@tok@ow\endcsname{\let\PY@bf=\textbf\def\PY@tc##1{\textcolor[rgb]{0.67,0.13,1.00}{##1}}}
\expandafter\def\csname PY@tok@nb\endcsname{\def\PY@tc##1{\textcolor[rgb]{0.00,0.50,0.00}{##1}}}
\expandafter\def\csname PY@tok@nf\endcsname{\def\PY@tc##1{\textcolor[rgb]{0.00,0.00,1.00}{##1}}}
\expandafter\def\csname PY@tok@nc\endcsname{\let\PY@bf=\textbf\def\PY@tc##1{\textcolor[rgb]{0.00,0.00,1.00}{##1}}}
\expandafter\def\csname PY@tok@nn\endcsname{\let\PY@bf=\textbf\def\PY@tc##1{\textcolor[rgb]{0.00,0.00,1.00}{##1}}}
\expandafter\def\csname PY@tok@ne\endcsname{\let\PY@bf=\textbf\def\PY@tc##1{\textcolor[rgb]{0.82,0.25,0.23}{##1}}}
\expandafter\def\csname PY@tok@nv\endcsname{\def\PY@tc##1{\textcolor[rgb]{0.10,0.09,0.49}{##1}}}
\expandafter\def\csname PY@tok@no\endcsname{\def\PY@tc##1{\textcolor[rgb]{0.53,0.00,0.00}{##1}}}
\expandafter\def\csname PY@tok@nl\endcsname{\def\PY@tc##1{\textcolor[rgb]{0.63,0.63,0.00}{##1}}}
\expandafter\def\csname PY@tok@ni\endcsname{\let\PY@bf=\textbf\def\PY@tc##1{\textcolor[rgb]{0.60,0.60,0.60}{##1}}}
\expandafter\def\csname PY@tok@na\endcsname{\def\PY@tc##1{\textcolor[rgb]{0.49,0.56,0.16}{##1}}}
\expandafter\def\csname PY@tok@nt\endcsname{\let\PY@bf=\textbf\def\PY@tc##1{\textcolor[rgb]{0.00,0.50,0.00}{##1}}}
\expandafter\def\csname PY@tok@nd\endcsname{\def\PY@tc##1{\textcolor[rgb]{0.67,0.13,1.00}{##1}}}
\expandafter\def\csname PY@tok@s\endcsname{\def\PY@tc##1{\textcolor[rgb]{0.73,0.13,0.13}{##1}}}
\expandafter\def\csname PY@tok@sd\endcsname{\let\PY@it=\textit\def\PY@tc##1{\textcolor[rgb]{0.73,0.13,0.13}{##1}}}
\expandafter\def\csname PY@tok@si\endcsname{\let\PY@bf=\textbf\def\PY@tc##1{\textcolor[rgb]{0.73,0.40,0.53}{##1}}}
\expandafter\def\csname PY@tok@se\endcsname{\let\PY@bf=\textbf\def\PY@tc##1{\textcolor[rgb]{0.73,0.40,0.13}{##1}}}
\expandafter\def\csname PY@tok@sr\endcsname{\def\PY@tc##1{\textcolor[rgb]{0.73,0.40,0.53}{##1}}}
\expandafter\def\csname PY@tok@ss\endcsname{\def\PY@tc##1{\textcolor[rgb]{0.10,0.09,0.49}{##1}}}
\expandafter\def\csname PY@tok@sx\endcsname{\def\PY@tc##1{\textcolor[rgb]{0.00,0.50,0.00}{##1}}}
\expandafter\def\csname PY@tok@m\endcsname{\def\PY@tc##1{\textcolor[rgb]{0.40,0.40,0.40}{##1}}}
\expandafter\def\csname PY@tok@gh\endcsname{\let\PY@bf=\textbf\def\PY@tc##1{\textcolor[rgb]{0.00,0.00,0.50}{##1}}}
\expandafter\def\csname PY@tok@gu\endcsname{\let\PY@bf=\textbf\def\PY@tc##1{\textcolor[rgb]{0.50,0.00,0.50}{##1}}}
\expandafter\def\csname PY@tok@gd\endcsname{\def\PY@tc##1{\textcolor[rgb]{0.63,0.00,0.00}{##1}}}
\expandafter\def\csname PY@tok@gi\endcsname{\def\PY@tc##1{\textcolor[rgb]{0.00,0.63,0.00}{##1}}}
\expandafter\def\csname PY@tok@gr\endcsname{\def\PY@tc##1{\textcolor[rgb]{1.00,0.00,0.00}{##1}}}
\expandafter\def\csname PY@tok@ge\endcsname{\let\PY@it=\textit}
\expandafter\def\csname PY@tok@gs\endcsname{\let\PY@bf=\textbf}
\expandafter\def\csname PY@tok@gp\endcsname{\let\PY@bf=\textbf\def\PY@tc##1{\textcolor[rgb]{0.00,0.00,0.50}{##1}}}
\expandafter\def\csname PY@tok@go\endcsname{\def\PY@tc##1{\textcolor[rgb]{0.53,0.53,0.53}{##1}}}
\expandafter\def\csname PY@tok@gt\endcsname{\def\PY@tc##1{\textcolor[rgb]{0.00,0.27,0.87}{##1}}}
\expandafter\def\csname PY@tok@err\endcsname{\def\PY@bc##1{\setlength{\fboxsep}{0pt}\fcolorbox[rgb]{1.00,0.00,0.00}{1,1,1}{\strut ##1}}}
\expandafter\def\csname PY@tok@kc\endcsname{\let\PY@bf=\textbf\def\PY@tc##1{\textcolor[rgb]{0.00,0.50,0.00}{##1}}}
\expandafter\def\csname PY@tok@kd\endcsname{\let\PY@bf=\textbf\def\PY@tc##1{\textcolor[rgb]{0.00,0.50,0.00}{##1}}}
\expandafter\def\csname PY@tok@kn\endcsname{\let\PY@bf=\textbf\def\PY@tc##1{\textcolor[rgb]{0.00,0.50,0.00}{##1}}}
\expandafter\def\csname PY@tok@kr\endcsname{\let\PY@bf=\textbf\def\PY@tc##1{\textcolor[rgb]{0.00,0.50,0.00}{##1}}}
\expandafter\def\csname PY@tok@bp\endcsname{\def\PY@tc##1{\textcolor[rgb]{0.00,0.50,0.00}{##1}}}
\expandafter\def\csname PY@tok@fm\endcsname{\def\PY@tc##1{\textcolor[rgb]{0.00,0.00,1.00}{##1}}}
\expandafter\def\csname PY@tok@vc\endcsname{\def\PY@tc##1{\textcolor[rgb]{0.10,0.09,0.49}{##1}}}
\expandafter\def\csname PY@tok@vg\endcsname{\def\PY@tc##1{\textcolor[rgb]{0.10,0.09,0.49}{##1}}}
\expandafter\def\csname PY@tok@vi\endcsname{\def\PY@tc##1{\textcolor[rgb]{0.10,0.09,0.49}{##1}}}
\expandafter\def\csname PY@tok@vm\endcsname{\def\PY@tc##1{\textcolor[rgb]{0.10,0.09,0.49}{##1}}}
\expandafter\def\csname PY@tok@sa\endcsname{\def\PY@tc##1{\textcolor[rgb]{0.73,0.13,0.13}{##1}}}
\expandafter\def\csname PY@tok@sb\endcsname{\def\PY@tc##1{\textcolor[rgb]{0.73,0.13,0.13}{##1}}}
\expandafter\def\csname PY@tok@sc\endcsname{\def\PY@tc##1{\textcolor[rgb]{0.73,0.13,0.13}{##1}}}
\expandafter\def\csname PY@tok@dl\endcsname{\def\PY@tc##1{\textcolor[rgb]{0.73,0.13,0.13}{##1}}}
\expandafter\def\csname PY@tok@s2\endcsname{\def\PY@tc##1{\textcolor[rgb]{0.73,0.13,0.13}{##1}}}
\expandafter\def\csname PY@tok@sh\endcsname{\def\PY@tc##1{\textcolor[rgb]{0.73,0.13,0.13}{##1}}}
\expandafter\def\csname PY@tok@s1\endcsname{\def\PY@tc##1{\textcolor[rgb]{0.73,0.13,0.13}{##1}}}
\expandafter\def\csname PY@tok@mb\endcsname{\def\PY@tc##1{\textcolor[rgb]{0.40,0.40,0.40}{##1}}}
\expandafter\def\csname PY@tok@mf\endcsname{\def\PY@tc##1{\textcolor[rgb]{0.40,0.40,0.40}{##1}}}
\expandafter\def\csname PY@tok@mh\endcsname{\def\PY@tc##1{\textcolor[rgb]{0.40,0.40,0.40}{##1}}}
\expandafter\def\csname PY@tok@mi\endcsname{\def\PY@tc##1{\textcolor[rgb]{0.40,0.40,0.40}{##1}}}
\expandafter\def\csname PY@tok@il\endcsname{\def\PY@tc##1{\textcolor[rgb]{0.40,0.40,0.40}{##1}}}
\expandafter\def\csname PY@tok@mo\endcsname{\def\PY@tc##1{\textcolor[rgb]{0.40,0.40,0.40}{##1}}}
\expandafter\def\csname PY@tok@ch\endcsname{\let\PY@it=\textit\def\PY@tc##1{\textcolor[rgb]{0.25,0.50,0.50}{##1}}}
\expandafter\def\csname PY@tok@cm\endcsname{\let\PY@it=\textit\def\PY@tc##1{\textcolor[rgb]{0.25,0.50,0.50}{##1}}}
\expandafter\def\csname PY@tok@cpf\endcsname{\let\PY@it=\textit\def\PY@tc##1{\textcolor[rgb]{0.25,0.50,0.50}{##1}}}
\expandafter\def\csname PY@tok@c1\endcsname{\let\PY@it=\textit\def\PY@tc##1{\textcolor[rgb]{0.25,0.50,0.50}{##1}}}
\expandafter\def\csname PY@tok@cs\endcsname{\let\PY@it=\textit\def\PY@tc##1{\textcolor[rgb]{0.25,0.50,0.50}{##1}}}

\def\PYZbs{\char`\\}
\def\PYZus{\char`\_}
\def\PYZob{\char`\{}
\def\PYZcb{\char`\}}
\def\PYZca{\char`\^}
\def\PYZam{\char`\&}
\def\PYZlt{\char`\<}
\def\PYZgt{\char`\>}
\def\PYZsh{\char`\#}
\def\PYZpc{\char`\%}
\def\PYZdl{\char`\$}
\def\PYZhy{\char`\-}
\def\PYZsq{\char`\'}
\def\PYZdq{\char`\"}
\def\PYZti{\char`\~}
% for compatibility with earlier versions
\def\PYZat{@}
\def\PYZlb{[}
\def\PYZrb{]}
\makeatother


    % For linebreaks inside Verbatim environment from package fancyvrb. 
    \makeatletter
        \newbox\Wrappedcontinuationbox 
        \newbox\Wrappedvisiblespacebox 
        \newcommand*\Wrappedvisiblespace {\textcolor{red}{\textvisiblespace}} 
        \newcommand*\Wrappedcontinuationsymbol {\textcolor{red}{\llap{\tiny$\m@th\hookrightarrow$}}} 
        \newcommand*\Wrappedcontinuationindent {3ex } 
        \newcommand*\Wrappedafterbreak {\kern\Wrappedcontinuationindent\copy\Wrappedcontinuationbox} 
        % Take advantage of the already applied Pygments mark-up to insert 
        % potential linebreaks for TeX processing. 
        %        {, <, #, %, $, ' and ": go to next line. 
        %        _, }, ^, &, >, - and ~: stay at end of broken line. 
        % Use of \textquotesingle for straight quote. 
        \newcommand*\Wrappedbreaksatspecials {% 
            \def\PYGZus{\discretionary{\char`\_}{\Wrappedafterbreak}{\char`\_}}% 
            \def\PYGZob{\discretionary{}{\Wrappedafterbreak\char`\{}{\char`\{}}% 
            \def\PYGZcb{\discretionary{\char`\}}{\Wrappedafterbreak}{\char`\}}}% 
            \def\PYGZca{\discretionary{\char`\^}{\Wrappedafterbreak}{\char`\^}}% 
            \def\PYGZam{\discretionary{\char`\&}{\Wrappedafterbreak}{\char`\&}}% 
            \def\PYGZlt{\discretionary{}{\Wrappedafterbreak\char`\<}{\char`\<}}% 
            \def\PYGZgt{\discretionary{\char`\>}{\Wrappedafterbreak}{\char`\>}}% 
            \def\PYGZsh{\discretionary{}{\Wrappedafterbreak\char`\#}{\char`\#}}% 
            \def\PYGZpc{\discretionary{}{\Wrappedafterbreak\char`\%}{\char`\%}}% 
            \def\PYGZdl{\discretionary{}{\Wrappedafterbreak\char`\$}{\char`\$}}% 
            \def\PYGZhy{\discretionary{\char`\-}{\Wrappedafterbreak}{\char`\-}}% 
            \def\PYGZsq{\discretionary{}{\Wrappedafterbreak\textquotesingle}{\textquotesingle}}% 
            \def\PYGZdq{\discretionary{}{\Wrappedafterbreak\char`\"}{\char`\"}}% 
            \def\PYGZti{\discretionary{\char`\~}{\Wrappedafterbreak}{\char`\~}}% 
        } 
        % Some characters . , ; ? ! / are not pygmentized. 
        % This macro makes them "active" and they will insert potential linebreaks 
        \newcommand*\Wrappedbreaksatpunct {% 
            \lccode`\~`\.\lowercase{\def~}{\discretionary{\hbox{\char`\.}}{\Wrappedafterbreak}{\hbox{\char`\.}}}% 
            \lccode`\~`\,\lowercase{\def~}{\discretionary{\hbox{\char`\,}}{\Wrappedafterbreak}{\hbox{\char`\,}}}% 
            \lccode`\~`\;\lowercase{\def~}{\discretionary{\hbox{\char`\;}}{\Wrappedafterbreak}{\hbox{\char`\;}}}% 
            \lccode`\~`\:\lowercase{\def~}{\discretionary{\hbox{\char`\:}}{\Wrappedafterbreak}{\hbox{\char`\:}}}% 
            \lccode`\~`\?\lowercase{\def~}{\discretionary{\hbox{\char`\?}}{\Wrappedafterbreak}{\hbox{\char`\?}}}% 
            \lccode`\~`\!\lowercase{\def~}{\discretionary{\hbox{\char`\!}}{\Wrappedafterbreak}{\hbox{\char`\!}}}% 
            \lccode`\~`\/\lowercase{\def~}{\discretionary{\hbox{\char`\/}}{\Wrappedafterbreak}{\hbox{\char`\/}}}% 
            \catcode`\.\active
            \catcode`\,\active 
            \catcode`\;\active
            \catcode`\:\active
            \catcode`\?\active
            \catcode`\!\active
            \catcode`\/\active 
            \lccode`\~`\~ 	
        }
    \makeatother

    \let\OriginalVerbatim=\Verbatim
    \makeatletter
    \renewcommand{\Verbatim}[1][1]{%
        %\parskip\z@skip
        \sbox\Wrappedcontinuationbox {\Wrappedcontinuationsymbol}%
        \sbox\Wrappedvisiblespacebox {\FV@SetupFont\Wrappedvisiblespace}%
        \def\FancyVerbFormatLine ##1{\hsize\linewidth
            \vtop{\raggedright\hyphenpenalty\z@\exhyphenpenalty\z@
                \doublehyphendemerits\z@\finalhyphendemerits\z@
                \strut ##1\strut}%
        }%
        % If the linebreak is at a space, the latter will be displayed as visible
        % space at end of first line, and a continuation symbol starts next line.
        % Stretch/shrink are however usually zero for typewriter font.
        \def\FV@Space {%
            \nobreak\hskip\z@ plus\fontdimen3\font minus\fontdimen4\font
            \discretionary{\copy\Wrappedvisiblespacebox}{\Wrappedafterbreak}
            {\kern\fontdimen2\font}%
        }%
        
        % Allow breaks at special characters using \PYG... macros.
        \Wrappedbreaksatspecials
        % Breaks at punctuation characters . , ; ? ! and / need catcode=\active 	
        \OriginalVerbatim[#1,codes*=\Wrappedbreaksatpunct]%
    }
    \makeatother

    % Exact colors from NB
    \definecolor{incolor}{HTML}{303F9F}
    \definecolor{outcolor}{HTML}{D84315}
    \definecolor{cellborder}{HTML}{CFCFCF}
    \definecolor{cellbackground}{HTML}{F7F7F7}
    
    % prompt
    \makeatletter
    \newcommand{\boxspacing}{\kern\kvtcb@left@rule\kern\kvtcb@boxsep}
    \makeatother
    \newcommand{\prompt}[4]{
        \ttfamily\llap{{\color{#2}[#3]:\hspace{3pt}#4}}\vspace{-\baselineskip}
    }
    

    
    % Prevent overflowing lines due to hard-to-break entities
    \sloppy 
    % Setup hyperref package
    \hypersetup{
      breaklinks=true,  % so long urls are correctly broken across lines
      colorlinks=true,
      urlcolor=urlcolor,
      linkcolor=linkcolor,
      citecolor=citecolor,
      }
    % Slightly bigger margins than the latex defaults
    
    \geometry{verbose,tmargin=1in,bmargin=1in,lmargin=1in,rmargin=1in}
    
    

\begin{document}
    
    \maketitle
    
    

    
    \hypertarget{sa-ccr}{%
\section{SA-CCR}\label{sa-ccr}}

    When using \emph{SA-CCR} the exposure at default (EAD) has to be
calculated as:

\begin{align*}
EAD &= \alpha * (RC + PFE)\\
\\
\text{where} \qquad \alpha&=1.4 \\
RC&: \text{Replacement Cost} \\
PFE&: \text{Potential Future Exposure}
\end{align*}

Let's set this relationship up in Python

    \begin{tcolorbox}[breakable, size=fbox, boxrule=1pt, pad at break*=1mm,colback=cellbackground, colframe=cellborder]
\prompt{In}{incolor}{1}{\boxspacing}
\begin{Verbatim}[commandchars=\\\{\}]
\PY{c+c1}{\PYZsh{}hidden}

\PY{c+c1}{\PYZsh{} Imports}
\PY{k+kn}{from} \PY{n+nn}{datetime} \PY{k+kn}{import} \PY{n}{datetime}\PY{p}{,} \PY{n}{timedelta}
\PY{k+kn}{from} \PY{n+nn}{enum} \PY{k+kn}{import} \PY{n}{Enum}
\PY{k+kn}{from} \PY{n+nn}{math} \PY{k+kn}{import} \PY{n}{exp}\PY{p}{,} \PY{n}{log}\PY{p}{,} \PY{n}{sqrt}
\PY{k+kn}{from} \PY{n+nn}{typing} \PY{k+kn}{import} \PY{n}{List}
\PY{k+kn}{import} \PY{n+nn}{numpy}
\PY{k+kn}{from} \PY{n+nn}{pandas} \PY{k+kn}{import} \PY{n}{read\PYZus{}csv}\PY{p}{,} \PY{n}{DataFrame}\PY{p}{,} \PY{n}{Series}
\PY{k+kn}{from} \PY{n+nn}{scipy}\PY{n+nn}{.}\PY{n+nn}{stats} \PY{k+kn}{import} \PY{n}{norm}
\PY{k+kn}{from} \PY{n+nn}{utilities}\PY{n+nn}{.}\PY{n+nn}{Enums} \PY{k+kn}{import} \PY{n}{TradeType}\PY{p}{,} \PY{n}{TradeDirection}\PY{p}{,} \PY{n}{SwapDirection}\PY{p}{,} \PY{n}{AssetClass}\PY{p}{,} \PY{n}{MaturityBucket}\PY{p}{,} \PY{n}{SubClass}\PY{p}{,} \PY{n}{CreditSubClass}\PY{p}{,} \PYZbs{}
    \PY{n}{EquitySubClass}\PY{p}{,} \PY{n}{Currency}\PY{p}{,} \PY{n}{CurrencyPair}\PY{p}{,} \PY{n}{Stock}
\PY{k+kn}{from} \PY{n+nn}{collateralAgreement} \PY{k+kn}{import} \PY{n}{CollateralAgreement}\PY{p}{,} \PY{n}{Margining}\PY{p}{,} \PY{n}{Clearing}\PY{p}{,} \PY{n}{Tradecount}\PY{p}{,} \PY{n}{Dispute}
\PY{k+kn}{from} \PY{n+nn}{instruments}\PY{n+nn}{.}\PY{n+nn}{Trade} \PY{k+kn}{import} \PY{n}{Trade}
\PY{k+kn}{from} \PY{n+nn}{instruments}\PY{n+nn}{.}\PY{n+nn}{equity\PYZus{}instruments}\PY{n+nn}{.}\PY{n+nn}{equityForward} \PY{k+kn}{import} \PY{n}{EquityForward}
\PY{k+kn}{from} \PY{n+nn}{instruments}\PY{n+nn}{.}\PY{n+nn}{fx\PYZus{}instruments}\PY{n+nn}{.}\PY{n+nn}{fxForward} \PY{k+kn}{import} \PY{n}{FxForward}
\PY{k+kn}{from} \PY{n+nn}{instruments}\PY{n+nn}{.}\PY{n+nn}{fx\PYZus{}instruments}\PY{n+nn}{.}\PY{n+nn}{fxOption} \PY{k+kn}{import} \PY{n}{FxOption}

\PY{k+kn}{from} \PY{n+nn}{marketdata}\PY{n+nn}{.}\PY{n+nn}{init\PYZus{}marketdata} \PY{k+kn}{import} \PY{n}{today}
\PY{n}{today}
\end{Verbatim}
\end{tcolorbox}

    \begin{Verbatim}[commandchars=\\\{\}]

        ---------------------------------------------------------------------------

        ModuleNotFoundError                       Traceback (most recent call last)

        <ipython-input-1-eba5627208cd> in <module>
          9 from pandas import read\_csv, DataFrame, Series
         10 from scipy.stats import norm
    ---> 11 from utilities.Enums import TradeType, TradeDirection, SwapDirection, AssetClass, MaturityBucket, SubClass, CreditSubClass, \textbackslash{}
         12     EquitySubClass, Currency, CurrencyPair, Stock
         13 from collateralAgreement import CollateralAgreement, Margining, Clearing, Tradecount, Dispute
    

        ModuleNotFoundError: No module named 'utilities'

    \end{Verbatim}

    \begin{tcolorbox}[breakable, size=fbox, boxrule=1pt, pad at break*=1mm,colback=cellbackground, colframe=cellborder]
\prompt{In}{incolor}{ }{\boxspacing}
\begin{Verbatim}[commandchars=\\\{\}]
\PY{k}{def} \PY{n+nf}{calculate\PYZus{}sa\PYZus{}ccr\PYZus{}ead}\PY{p}{(}\PY{n}{rc}\PY{p}{:}\PY{n+nb}{float} \PY{p}{,} \PY{n}{pfe}\PY{p}{:} \PY{n+nb}{float}\PY{p}{)} \PY{o}{\PYZhy{}}\PY{o}{\PYZgt{}} \PY{n+nb}{float}\PY{p}{:}
    \PY{l+s+sd}{\PYZdq{}\PYZdq{}\PYZdq{}}
\PY{l+s+sd}{    Calculate EAD of SA\PYZhy{}CCR as defined in paragraph 186.}
\PY{l+s+sd}{               }
\PY{l+s+sd}{    :param rc: Replacement Cost}
\PY{l+s+sd}{    :param pfe: Potential Future Exposure}
\PY{l+s+sd}{    :return: Exposure at default according to SA\PYZhy{}CCR}
\PY{l+s+sd}{    \PYZdq{}\PYZdq{}\PYZdq{}}

    \PY{n}{alpha} \PY{o}{=} \PY{l+m+mf}{1.4} \PY{c+c1}{\PYZsh{}Carried over from alpha used for IMM}
    \PY{k}{return} \PY{n}{alpha}\PY{o}{*}\PY{p}{(}\PY{n}{rc}\PY{o}{+}\PY{n}{pfe}\PY{p}{)}

\PY{n}{calculate\PYZus{}sa\PYZus{}ccr\PYZus{}ead}\PY{p}{(}\PY{l+m+mf}{10.0}\PY{p}{,}\PY{l+m+mf}{15.0}\PY{p}{)}
\end{Verbatim}
\end{tcolorbox}

    SA-CCR differentiates between the following asset classes:

\begin{itemize}
\tightlist
\item
  Interest rate derivatives
\item
  Foreign exchange derivatives
\item
  Credit derivatives
\item
  Equity derivatives
\item
  Commodity derivatives
\end{itemize}

Each of these asset classes is differentiated again into different
hedging sets - Netting of expsoure is only possible within a common
hedging set. The definition of a hedging set depends on the asset class.
For FX derivatives derivatives based on the same currency pair form a
hedging set. For Equity derivatives on the other hand all trades are in
a common hedging set. For all asset classes, derivatives that have a
basis as an underlying and volatility derivatives form two seperate
hedging sets.

This should be moving

    \hypertarget{relation-of-rc-and-pfe}{%
\subsubsection{Relation of RC and PFE}\label{relation-of-rc-and-pfe}}

The purpose of the RC is to assess the imidiate loss suffered by the
default of a counterparty. It is based on the current MtM of the
derivative less the accessible collateral. If a bank has posted
collateral to non-segregated accounts of a counterparty this collateral
is also assumed to be lost in case of a default which increases the
replacement cost.

The potential future exposure (PFE) on the other hand assesses how the
RC might develop in the future. The future being defined as during the
next year. If the RC today is 0 but is likely to be larger than 0 in the
near future the estimated EAD should take this expected increase in RC
into account.

See also Paragraph 130 and 131 of BCBS(2014):

Paragraph 130 - case without margining:

\begin{quote}
For unmargined transactions, the \emph{RC} intends to caputre the loss
that would occur if a counterparty were to default and were closed out
of its transactions immediately. The \emph{PFE} add-on represents a
potential conversative increase in expousre over a one-year time horizon
from the present date (i.e.~the calculation date).
\end{quote}

Paragraph 131 - case with margining:

\begin{quote}
For margined trades, the \emph{RC} intends to capture the loss that
would occur if a counterparty were to default at the present or at a
future time, assuming that the closeout and replacement of transactions
occur instantaneously. However, there may be a period (the margin period
of risk) between the last exchange of collateral before default and
replacement of the trades in the market. The \emph{PFE} add-on
represents the potential change in value of the trades during this time
period.
\end{quote}

    \hypertarget{definition-of-potential-future-exposure-pfe}{%
\subsection{Definition of Potential Future Exposure
(PFE)}\label{definition-of-potential-future-exposure-pfe}}

\begin{align*}
PFE &= \text{multiplier} * AddOn^{\text{aggregate}} \\
\\
\text{where} \qquad AddOn^{\text{aggregate}} &: \text{aggregate add-on component} \\
\text{multiplier} &: f(V,C,AddOn^{\text{aggregate}})
\end{align*}

\(AddOn\) is calculated differently for each asset \(a\) class. Since no
netting is allowed between asset classes the aggregate is calculated as:

\[AddOn^{\text{aggregate}} = \sum_{a}AddOn^{a}\]

Collateralization is taken into account of the PFE calculation through
the multiplier that uses the collateral held as an input. As
overcollateralization e.g.~through IM increases, the multiplier
decreases. However, the multiplier is floored at 5\%.

\begin{align*}
\text{multiplier} &= \min \left\{ 1; Floor + (1-Floor) \exp\left(\frac{V-C}{2(1-Floor)AddOn^{\text{aggregate}}}\right) \right\} \\
\text{where} \qquad Floor &= 5\%
\end{align*}

    \begin{tcolorbox}[breakable, size=fbox, boxrule=1pt, pad at break*=1mm,colback=cellbackground, colframe=cellborder]
\prompt{In}{incolor}{ }{\boxspacing}
\begin{Verbatim}[commandchars=\\\{\}]
\PY{k}{def} \PY{n+nf}{multiplier}\PY{p}{(}\PY{n}{v}\PY{p}{:}\PY{n+nb}{float}\PY{p}{,}\PY{n}{c}\PY{p}{:}\PY{n+nb}{float}\PY{p}{,} \PY{n}{addOn\PYZus{}aggregate}\PY{p}{:}\PY{n+nb}{float}\PY{p}{,} \PY{n}{floor}\PY{p}{:}\PY{n+nb}{float} \PY{o}{=} \PY{l+m+mf}{0.05}\PY{p}{)} \PY{o}{\PYZhy{}}\PY{o}{\PYZgt{}} \PY{n+nb}{float}\PY{p}{:}
    \PY{l+s+sd}{\PYZdq{}\PYZdq{}\PYZdq{}}
\PY{l+s+sd}{    Multiplier calculation for SA\PYZhy{}CCR}
\PY{l+s+sd}{    }
\PY{l+s+sd}{    :param v: Current value of the derivative transactions in the netting set}
\PY{l+s+sd}{    :param c: Haircut value of the net collateral held}
\PY{l+s+sd}{    :param addOn\PYZus{}aggregate: Aggregated (summed up) AddOn over all asset classes}
\PY{l+s+sd}{    :param floor: Regulatory floor for the multiplier. Set to 5\PYZpc{} in paragraph 149}
\PY{l+s+sd}{    :return: Multiplier for PFE calculation according to SA\PYZhy{}CCR}
\PY{l+s+sd}{    \PYZdq{}\PYZdq{}\PYZdq{}}
    
    \PY{k}{return} \PY{n+nb}{min}\PY{p}{(}\PY{l+m+mi}{1}\PY{p}{,} \PY{n}{floor}\PY{o}{+}\PY{p}{(}\PY{l+m+mi}{1}\PY{o}{\PYZhy{}}\PY{n}{floor}\PY{p}{)}\PY{o}{*}\PY{n}{exp}\PY{p}{(}\PY{p}{(}\PY{n}{v}\PY{o}{\PYZhy{}}\PY{n}{c}\PY{p}{)}\PY{o}{/}\PY{p}{(}\PY{l+m+mi}{2}\PY{o}{*}\PY{p}{(}\PY{l+m+mi}{1}\PY{o}{\PYZhy{}}\PY{n}{floor}\PY{p}{)}\PY{o}{*}\PY{n}{addOn\PYZus{}aggregate}\PY{p}{)}\PY{p}{)}\PY{p}{)}
\end{Verbatim}
\end{tcolorbox}

    \hypertarget{addon-calculation}{%
\section{AddOn calculation}\label{addon-calculation}}

Most of the SA-CCR logic is hidden inside the AddOn calculation. At
first it is important to define the following four data parameters:

\hypertarget{m_i}{%
\paragraph{\texorpdfstring{\(M_i\)}{M\_i}}\label{m_i}}

\begin{quote}
Maturity of the derivative contract. If the underlying of a derivative
is another derivative - e.g.~in the case of a swaption the maturity date
of the underlying needs to be chosen.
\end{quote}

\hypertarget{s_i}{%
\paragraph{\texorpdfstring{\(S_i\)}{S\_i}}\label{s_i}}

\begin{quote}
For interest rate and credit derivatives the start date of the time
periodreferenced by an interst rate or credit contract. If the
derivtives underlying is another interest rate or credit intsrument (eg
swaption or bond option) \(S_i\) is the start date of the underlying
instead.
\end{quote}

\hypertarget{e_i}{%
\paragraph{\texorpdfstring{\(E_i\)}{E\_i}}\label{e_i}}

\begin{quote}
Defined as \(S_i\) but referencing the end date instead of the start
date.
\end{quote}

\hypertarget{t_i}{%
\paragraph{\texorpdfstring{\(T_i\)}{T\_i}}\label{t_i}}

\begin{quote}
For options across all asset classes this is the latest contractual
exercise date.
\end{quote}

    Let's set up some trades at this point:

\begin{itemize}
\tightlist
\item
  An Equity Option that is defined by a strike, maturity and current
  value of the underlying. Additional characteristics e.g.~if it is an
  american or european option are irrelevant for SA-CCR calculations
\item
  A running intere rate swap
\item
  A forward starting interest rate swap
\item
  A swaption with the previous swap as underlying
\end{itemize}

    \begin{tcolorbox}[breakable, size=fbox, boxrule=1pt, pad at break*=1mm,colback=cellbackground, colframe=cellborder]
\prompt{In}{incolor}{ }{\boxspacing}
\begin{Verbatim}[commandchars=\\\{\}]
\PY{k+kn}{from} \PY{n+nn}{instruments}\PY{n+nn}{.}\PY{n+nn}{equity\PYZus{}instruments}\PY{n+nn}{.}\PY{n+nn}{equityOption} \PY{k+kn}{import} \PY{n}{EquityOption}
\PY{k+kn}{from} \PY{n+nn}{instruments}\PY{n+nn}{.}\PY{n+nn}{interestRateInstrument}\PY{n+nn}{.}\PY{n+nn}{interestRateSwap} \PY{k+kn}{import} \PY{n}{InterestRateSwap}
\PY{k+kn}{from} \PY{n+nn}{instruments}\PY{n+nn}{.}\PY{n+nn}{interestRateInstrument}\PY{n+nn}{.}\PY{n+nn}{swaption} \PY{k+kn}{import} \PY{n}{Swaption}

\PY{n}{eq\PYZus{}option} \PY{o}{=} \PY{n}{EquityOption}\PY{p}{(}\PY{n}{notional}\PY{o}{=}\PY{l+m+mi}{20}\PY{p}{,} \PY{n}{K}\PY{o}{=}\PY{l+m+mi}{35}\PY{p}{,} \PY{n}{mat\PYZus{}in\PYZus{}days}\PY{o}{=}\PY{l+m+mi}{720}\PY{p}{)}
\PY{n+nb}{print}\PY{p}{(}\PY{n+nb}{str}\PY{p}{(}\PY{n}{eq\PYZus{}option}\PY{p}{)}\PY{p}{)}

\PY{n}{swap} \PY{o}{=} \PY{n}{InterestRateSwap}\PY{p}{(}\PY{n}{notional} \PY{o}{=} \PY{l+m+mi}{2000000}\PY{p}{,} \PY{n}{currency} \PY{o}{=} \PY{n}{Currency}\PY{o}{.}\PY{n}{EUR}\PY{p}{,} \PY{n}{timeToSwapStart\PYZus{}in\PYZus{}days}\PY{o}{=}\PY{l+m+mi}{0}\PY{p}{,} \PY{n}{timeToSwapEnd\PYZus{}in\PYZus{}days}\PY{o}{=}\PY{l+m+mi}{4}\PY{o}{*}\PY{l+m+mi}{360}\PY{p}{,} \PY{n}{swapDirection}\PY{o}{=}\PY{n}{SwapDirection}\PY{o}{.}\PY{n}{RECEIVER}\PY{p}{)}

\PY{n}{fw\PYZus{}swap} \PY{o}{=} \PY{n}{InterestRateSwap}\PY{p}{(}\PY{n}{notional} \PY{o}{=} \PY{l+m+mi}{1000000}\PY{p}{,}\PY{n}{currency} \PY{o}{=} \PY{n}{Currency}\PY{o}{.}\PY{n}{USD}\PY{p}{,} \PY{n}{timeToSwapStart\PYZus{}in\PYZus{}days}\PY{o}{=}\PY{l+m+mi}{1}\PY{o}{*}\PY{l+m+mi}{360}\PY{p}{,} \PY{n}{timeToSwapEnd\PYZus{}in\PYZus{}days}\PY{o}{=}\PY{l+m+mi}{3}\PY{o}{*}\PY{l+m+mi}{360}\PY{p}{,} \PY{n}{swapDirection} \PY{o}{=} \PY{n}{SwapDirection}\PY{o}{.}\PY{n}{PAYER}\PY{p}{)}
\PY{n+nb}{print}\PY{p}{(}\PY{n}{fw\PYZus{}swap}\PY{p}{)}

\PY{n}{swaption} \PY{o}{=} \PY{n}{Swaption}\PY{p}{(}\PY{n}{underlyingSwap}\PY{o}{=}\PY{n}{fw\PYZus{}swap}\PY{p}{,}
                    \PY{n}{optionMaturity\PYZus{}in\PYZus{}days}\PY{o}{=}\PY{l+m+mi}{1}\PY{o}{*}\PY{l+m+mi}{360}\PY{p}{,}
                    \PY{n}{tradeDirection} \PY{o}{=} \PY{n}{TradeDirection}\PY{o}{.}\PY{n}{SHORT}\PY{p}{,}
                    \PY{n}{strikeFixedRate} \PY{o}{=} \PY{l+m+mf}{0.014}\PY{p}{)}
\PY{n+nb}{print}\PY{p}{(}\PY{n}{swaption}\PY{p}{)}
\end{Verbatim}
\end{tcolorbox}

    \hypertarget{trade-level-adjusted-notional}{%
\subsection{Trade level adjusted
notional}\label{trade-level-adjusted-notional}}

Each trade \(i\) has a trade level adjusted notional \(d_i^a\) assigned
to it. This is calculated differently for the different asset classes.

\hypertarget{interest-rate-and-credit-derivatives}{%
\paragraph{Interest rate and credit
derivatives}\label{interest-rate-and-credit-derivatives}}

The notional of the trade is usually a well defined value in domestic
currency for interest rate and credit derivatives. It is multiplied by a
supervisory duration factor. The basic idea is, that the value of the
derivative can change more the longer the remaining

\begin{align*}
d_i &= \text{Notional}_i * SD_i \\
\\
\text{where} \qquad SD_i &=\frac{\exp\left(-0.05 * S_i\right)-\exp\left(-0.05 * E_i\right)}{0.05}
\end{align*}

\hypertarget{fx-derivatives}{%
\paragraph{FX derivatives}\label{fx-derivatives}}

While the wording in the BCBS paper is a bit more specific we will just
assume that every FX traded derivative has a USD leg and set the
notional equal the to USD notional.

\hypertarget{equity-and-commodity-derivatives}{%
\paragraph{Equity and commodity
derivatives}\label{equity-and-commodity-derivatives}}

The notional is defined as the price of the underlying. Therefore, it
fluctuates over time.

\hypertarget{notional-of-exotic-derivatives}{%
\paragraph{Notional of exotic
derivatives}\label{notional-of-exotic-derivatives}}

For more exotic derivatives which do have adjustable notionals,
resetting notionals etc. detailed handling of the notional is defined in
paragraph 158.

    \begin{tcolorbox}[breakable, size=fbox, boxrule=1pt, pad at break*=1mm,colback=cellbackground, colframe=cellborder]
\prompt{In}{incolor}{ }{\boxspacing}
\begin{Verbatim}[commandchars=\\\{\}]
\PY{k}{def} \PY{n+nf}{trade\PYZus{}level\PYZus{}adjusted\PYZus{}notional}\PY{p}{(}\PY{n}{trade}\PY{p}{:} \PY{n}{Trade}\PY{p}{)}\PY{p}{:}
    \PY{l+s+sd}{\PYZdq{}\PYZdq{}\PYZdq{}}
\PY{l+s+sd}{    Calculates the trade level adjusted notional as defined in paragraph 157}
\PY{l+s+sd}{    }
\PY{l+s+sd}{    :param trade: Trade object for which the adjusted notional should be calculated}
\PY{l+s+sd}{    :param underlying\PYZus{}price: }
\PY{l+s+sd}{        required for FX trades (exchange rate to domestic currency)}
\PY{l+s+sd}{        required for EQ trades (current price of one share)}
\PY{l+s+sd}{        required for CO trades (current price of fone unit of commodity)}
\PY{l+s+sd}{        not required for IR trades and CR trades}
\PY{l+s+sd}{    }
\PY{l+s+sd}{    :return: adjusted notional as defined in paragraph 157}
\PY{l+s+sd}{    \PYZdq{}\PYZdq{}\PYZdq{}}
    
    \PY{k}{if} \PY{n}{trade}\PY{o}{.}\PY{n}{assetClass} \PY{o+ow}{in} \PY{p}{(}\PY{n}{AssetClass}\PY{o}{.}\PY{n}{IR}\PY{p}{,} \PY{n}{AssetClass}\PY{o}{.}\PY{n}{CR}\PY{p}{)}\PY{p}{:}
        \PY{n}{sd} \PY{o}{=} \PY{p}{(}\PY{n}{exp}\PY{p}{(}\PY{o}{\PYZhy{}}\PY{l+m+mf}{0.05}\PY{o}{*}\PY{n}{trade}\PY{o}{.}\PY{n}{s}\PY{p}{)}\PY{o}{\PYZhy{}}\PY{n}{exp}\PY{p}{(}\PY{o}{\PYZhy{}}\PY{l+m+mf}{0.05}\PY{o}{*}\PY{n}{trade}\PY{o}{.}\PY{n}{e}\PY{p}{)}\PY{p}{)}\PY{o}{/}\PY{l+m+mf}{0.05}
        \PY{k}{return} \PY{n}{trade}\PY{o}{.}\PY{n}{notional}\PY{o}{*}\PY{n}{sd}
    
    \PY{c+c1}{\PYZsh{} This assumes that trade.Notional is defined as quantity of underlying shares for equity and foreign currency notional for FX}
    \PY{k}{if} \PY{n}{trade}\PY{o}{.}\PY{n}{assetClass} \PY{o}{==} \PY{n}{AssetClass}\PY{o}{.}\PY{n}{EQ}\PY{p}{:}
        \PY{k}{return} \PY{n}{trade}\PY{o}{.}\PY{n}{notional}\PY{o}{*}\PY{n}{trade}\PY{o}{.}\PY{n}{S}
    
    \PY{k}{if} \PY{n}{trade}\PY{o}{.}\PY{n}{assetClass} \PY{o}{==} \PY{n}{AssetClass}\PY{o}{.}\PY{n}{FX}\PY{p}{:}
        \PY{k}{return} \PY{n}{trade}\PY{o}{.}\PY{n}{notional}
\end{Verbatim}
\end{tcolorbox}

    \hypertarget{supervisory-delta-adjustments-delta_i}{%
\subsection{\texorpdfstring{Supervisory delta adjustments:
\(\delta_i\)}{Supervisory delta adjustments: \textbackslash{}delta\_i}}\label{supervisory-delta-adjustments-delta_i}}

For linear derivatives \(\delta\) is 1 for long derivatives and -1 for
short derivatives.

For options \(\delta\) is defined as under Black-Scholes:

\begin{align*}
\delta_{\text{long Call}} &= +\Phi\left(\frac{\ln\left(P_i / K_i \right) + 0.5 * \sigma_i^2 * T_i}{\sigma_i * \sqrt{T_i}}\right) \\
\\
\text{where} \qquad \Phi &: \text{standard normal cdf} \\
\sigma_i &: \text{supervisory volatility as defined in Table 2 in paragraph 183}
\end{align*}

This delta is multiplied by -1 in case of a long Put option or a short
Call option.

No detail is given at this point on the delta calculation of CDO
tranches as these are not in the scope of this thesis.

    \begin{tcolorbox}[breakable, size=fbox, boxrule=1pt, pad at break*=1mm,colback=cellbackground, colframe=cellborder]
\prompt{In}{incolor}{ }{\boxspacing}
\begin{Verbatim}[commandchars=\\\{\}]
\PY{k}{def} \PY{n+nf}{get\PYZus{}supervisory\PYZus{}volatility}\PY{p}{(}\PY{n}{trade}\PY{p}{:} \PY{n}{Trade}\PY{p}{)} \PY{o}{\PYZhy{}}\PY{o}{\PYZgt{}} \PY{n+nb}{float}\PY{p}{:}
    \PY{n}{supervisory\PYZus{}parameter} \PY{o}{=} \PY{n}{read\PYZus{}csv}\PY{p}{(}\PY{l+s+s1}{\PYZsq{}}\PY{l+s+s1}{supervisory\PYZus{}parameters.csv}\PY{l+s+s1}{\PYZsq{}}\PY{p}{)}
    \PY{n}{sigma} \PY{o}{=} \PY{k+kc}{None}
    \PY{k}{if} \PY{n}{trade}\PY{o}{.}\PY{n}{subClass} \PY{o+ow}{is} \PY{k+kc}{None}\PY{p}{:}
        \PY{n}{sigma} \PY{o}{=} \PY{n}{supervisory\PYZus{}parameter}\PY{p}{[}\PY{p}{(}\PY{n}{supervisory\PYZus{}parameter}\PY{o}{.}\PY{n}{AssetClass} \PY{o}{==} \PY{n}{trade}\PY{o}{.}\PY{n}{assetClass}\PY{o}{.}\PY{n}{value}\PY{p}{)}\PY{p}{]}\PY{p}{[}\PY{l+s+s1}{\PYZsq{}}\PY{l+s+s1}{SupervisoryOptionVolatility}\PY{l+s+s1}{\PYZsq{}}\PY{p}{]}\PY{o}{.}\PY{n}{iloc}\PY{p}{[}\PY{l+m+mi}{0}\PY{p}{]}
    \PY{k}{elif} \PY{n}{trade}\PY{o}{.}\PY{n}{subClass} \PY{o+ow}{is} \PY{o+ow}{not} \PY{k+kc}{None}\PY{p}{:}
        \PY{n}{sigma} \PY{o}{=} \PY{n}{supervisory\PYZus{}parameter}\PY{p}{[}\PY{p}{(}\PY{n}{supervisory\PYZus{}parameter}\PY{o}{.}\PY{n}{AssetClass} \PY{o}{==} \PY{n}{trade}\PY{o}{.}\PY{n}{assetClass}\PY{o}{.}\PY{n}{value}\PY{p}{)} \PY{o}{\PYZam{}} \PY{p}{(}\PY{n}{supervisory\PYZus{}parameter}\PY{o}{.}\PY{n}{SubClass} \PY{o}{==} \PY{n}{trade}\PY{o}{.}\PY{n}{subClass}\PY{o}{.}\PY{n}{value}\PY{p}{)}\PY{p}{]}\PY{p}{[}\PY{l+s+s1}{\PYZsq{}}\PY{l+s+s1}{SupervisoryOptionVolatility}\PY{l+s+s1}{\PYZsq{}}\PY{p}{]}\PY{o}{.}\PY{n}{iloc}\PY{p}{[}\PY{l+m+mi}{0}\PY{p}{]}
    \PY{k}{return} \PY{n}{sigma}

\PY{k}{def} \PY{n+nf}{calculate\PYZus{}sa\PYZus{}ccr\PYZus{}delta}\PY{p}{(}\PY{n}{trade}\PY{p}{)} \PY{o}{\PYZhy{}}\PY{o}{\PYZgt{}} \PY{n+nb}{float}\PY{p}{:}
    \PY{k}{if} \PY{p}{(}\PY{n}{trade}\PY{o}{.}\PY{n}{tradeType} \PY{o}{==} \PY{n}{TradeType}\PY{o}{.}\PY{n}{LINEAR}\PY{p}{)} \PY{o+ow}{and} \PY{p}{(}\PY{n}{trade}\PY{o}{.}\PY{n}{tradeDirection} \PY{o}{==} \PY{n}{TradeDirection}\PY{o}{.}\PY{n}{LONG}\PY{p}{)}\PY{p}{:}
        \PY{k}{return} \PY{l+m+mi}{1}
    \PY{k}{elif} \PY{p}{(}\PY{n}{trade}\PY{o}{.}\PY{n}{tradeType} \PY{o}{==} \PY{n}{TradeType}\PY{o}{.}\PY{n}{LINEAR}\PY{p}{)} \PY{o+ow}{and} \PY{p}{(}\PY{n}{trade}\PY{o}{.}\PY{n}{tradeDirection} \PY{o}{==} \PY{n}{TradeDirection}\PY{o}{.}\PY{n}{SHORT}\PY{p}{)}\PY{p}{:}
        \PY{k}{return} \PY{o}{\PYZhy{}}\PY{l+m+mi}{1}
    \PY{k}{elif} \PY{n}{trade}\PY{o}{.}\PY{n}{tradeType} \PY{o+ow}{in} \PY{p}{(}\PY{n}{TradeType}\PY{o}{.}\PY{n}{CALL}\PY{p}{,} \PY{n}{TradeType}\PY{o}{.}\PY{n}{PUT}\PY{p}{)}\PY{p}{:}
        \PY{n}{d1\PYZus{}mult} \PY{o}{=} \PY{o}{+}\PY{l+m+mi}{1} \PY{k}{if} \PY{n}{trade}\PY{o}{.}\PY{n}{tradeType} \PY{o}{==} \PY{n}{TradeType}\PY{o}{.}\PY{n}{CALL} \PY{k}{else} \PY{o}{\PYZhy{}}\PY{l+m+mi}{1}
        
        \PY{k}{if} \PY{n}{trade}\PY{o}{.}\PY{n}{tradeType} \PY{o}{==} \PY{n}{TradeType}\PY{o}{.}\PY{n}{CALL} \PY{o+ow}{and} \PY{n}{trade}\PY{o}{.}\PY{n}{tradeDirection} \PY{o}{==} \PY{n}{TradeDirection}\PY{o}{.}\PY{n}{LONG}\PY{p}{:} \PY{n}{n\PYZus{}mult} \PY{o}{=} \PY{l+m+mi}{1}
        \PY{k}{elif} \PY{n}{trade}\PY{o}{.}\PY{n}{tradeType} \PY{o}{==} \PY{n}{TradeType}\PY{o}{.}\PY{n}{CALL} \PY{o+ow}{and} \PY{n}{trade}\PY{o}{.}\PY{n}{tradeDirection} \PY{o}{==} \PY{n}{TradeDirection}\PY{o}{.}\PY{n}{SHORT}\PY{p}{:} \PY{n}{n\PYZus{}mult} \PY{o}{=} \PY{o}{\PYZhy{}}\PY{l+m+mi}{1}
        \PY{k}{elif} \PY{n}{trade}\PY{o}{.}\PY{n}{tradeType} \PY{o}{==} \PY{n}{TradeType}\PY{o}{.}\PY{n}{PUT} \PY{o+ow}{and} \PY{n}{trade}\PY{o}{.}\PY{n}{tradeDirection} \PY{o}{==} \PY{n}{TradeDirection}\PY{o}{.}\PY{n}{LONG}\PY{p}{:} \PY{n}{n\PYZus{}mult} \PY{o}{=} \PY{o}{\PYZhy{}}\PY{l+m+mi}{1}
        \PY{k}{else}\PY{p}{:} \PY{n}{n\PYZus{}mult} \PY{o}{=} \PY{o}{+}\PY{l+m+mi}{1}
        
        \PY{n}{sigma} \PY{o}{=} \PY{n}{get\PYZus{}supervisory\PYZus{}volatility}\PY{p}{(}\PY{n}{trade}\PY{p}{)}
        
        \PY{n}{d1} \PY{o}{=} \PY{p}{(}\PY{n}{log}\PY{p}{(}\PY{n}{trade}\PY{o}{.}\PY{n}{S}\PY{o}{/}\PY{n}{trade}\PY{o}{.}\PY{n}{K}\PY{p}{)} \PY{o}{+} \PY{l+m+mf}{0.5} \PY{o}{*} \PY{n}{sigma}\PY{o}{*}\PY{o}{*}\PY{l+m+mi}{2} \PY{o}{*} \PY{n}{trade}\PY{o}{.}\PY{n}{t}\PY{p}{)}\PY{o}{/}\PY{p}{(}\PY{n}{sigma}\PY{o}{*}\PY{n}{sqrt}\PY{p}{(}\PY{n}{trade}\PY{o}{.}\PY{n}{t}\PY{p}{)}\PY{p}{)}
        
        \PY{n}{delta} \PY{o}{=} \PY{n}{n\PYZus{}mult}\PY{o}{*}\PY{n}{norm}\PY{o}{.}\PY{n}{cdf}\PY{p}{(}\PY{n}{d1\PYZus{}mult}\PY{o}{*}\PY{n}{d1}\PY{p}{)}
        \PY{k}{return} \PY{n}{delta}
\end{Verbatim}
\end{tcolorbox}

    Let's try if our delta function works for the different trade types we
have set up.

\begin{enumerate}
\def\labelenumi{\arabic{enumi}.}
\tightlist
\item
  An equity option
\item
  An interest rate swap
\item
  A forward starting interest rate swap
\item
  A swaption
\end{enumerate}

    \begin{tcolorbox}[breakable, size=fbox, boxrule=1pt, pad at break*=1mm,colback=cellbackground, colframe=cellborder]
\prompt{In}{incolor}{ }{\boxspacing}
\begin{Verbatim}[commandchars=\\\{\}]
\PY{n+nb}{print}\PY{p}{(}\PY{n}{eq\PYZus{}option}\PY{p}{)}
\PY{n}{calculate\PYZus{}sa\PYZus{}ccr\PYZus{}delta}\PY{p}{(}\PY{n}{eq\PYZus{}option}\PY{p}{)}
\end{Verbatim}
\end{tcolorbox}

    \begin{tcolorbox}[breakable, size=fbox, boxrule=1pt, pad at break*=1mm,colback=cellbackground, colframe=cellborder]
\prompt{In}{incolor}{ }{\boxspacing}
\begin{Verbatim}[commandchars=\\\{\}]
\PY{n+nb}{print}\PY{p}{(}\PY{n}{swap}\PY{p}{)}
\PY{n}{calculate\PYZus{}sa\PYZus{}ccr\PYZus{}delta}\PY{p}{(}\PY{n}{swap}\PY{p}{)}
\end{Verbatim}
\end{tcolorbox}

    \begin{tcolorbox}[breakable, size=fbox, boxrule=1pt, pad at break*=1mm,colback=cellbackground, colframe=cellborder]
\prompt{In}{incolor}{ }{\boxspacing}
\begin{Verbatim}[commandchars=\\\{\}]
\PY{n+nb}{print}\PY{p}{(}\PY{n}{fw\PYZus{}swap}\PY{p}{)}
\PY{n}{calculate\PYZus{}sa\PYZus{}ccr\PYZus{}delta}\PY{p}{(}\PY{n}{fw\PYZus{}swap}\PY{p}{)}
\end{Verbatim}
\end{tcolorbox}

    \begin{tcolorbox}[breakable, size=fbox, boxrule=1pt, pad at break*=1mm,colback=cellbackground, colframe=cellborder]
\prompt{In}{incolor}{ }{\boxspacing}
\begin{Verbatim}[commandchars=\\\{\}]
\PY{n+nb}{print}\PY{p}{(}\PY{n}{swaption}\PY{p}{)}
\PY{n}{calculate\PYZus{}sa\PYZus{}ccr\PYZus{}delta}\PY{p}{(}\PY{n}{swaption}\PY{p}{)}
\end{Verbatim}
\end{tcolorbox}

    The formula used for the regulatory delta is equivalent to the
Black-Scholes formula for delta under the assumption of a risk free rate
of 0. As a sanity check we can check if the Delta of a long call is
about 0.5 if the call is just about to mature.

    \begin{tcolorbox}[breakable, size=fbox, boxrule=1pt, pad at break*=1mm,colback=cellbackground, colframe=cellborder]
\prompt{In}{incolor}{ }{\boxspacing}
\begin{Verbatim}[commandchars=\\\{\}]
\PY{n}{eq\PYZus{}option2} \PY{o}{=} \PY{n}{EquityOption}\PY{p}{(}\PY{n}{notional} \PY{o}{=} \PY{l+m+mi}{20}\PY{p}{,} \PY{n}{K} \PY{o}{=} \PY{l+m+mi}{42}\PY{p}{,} \PY{n}{mat\PYZus{}in\PYZus{}days} \PY{o}{=} \PY{l+m+mi}{1}\PY{p}{,} \PY{n}{tradeType}\PY{o}{=}\PY{n}{TradeType}\PY{o}{.}\PY{n}{CALL}\PY{p}{,} \PY{n}{tradeDirection}\PY{o}{=}\PY{n}{TradeDirection}\PY{o}{.}\PY{n}{LONG}\PY{p}{)}
\PY{n+nb}{print}\PY{p}{(}\PY{n}{eq\PYZus{}option2}\PY{p}{)}
\PY{n}{calculate\PYZus{}sa\PYZus{}ccr\PYZus{}delta}\PY{p}{(}\PY{n}{eq\PYZus{}option2}\PY{p}{)}
\end{Verbatim}
\end{tcolorbox}

    Trying to validate methodology by replicating Example 1 of Annex 4a.
Setting up trades:

    \begin{tcolorbox}[breakable, size=fbox, boxrule=1pt, pad at break*=1mm,colback=cellbackground, colframe=cellborder]
\prompt{In}{incolor}{ }{\boxspacing}
\begin{Verbatim}[commandchars=\\\{\}]
\PY{n}{trade1} \PY{o}{=} \PY{n}{InterestRateSwap}\PY{p}{(}\PY{n}{notional} \PY{o}{=} \PY{l+m+mi}{10000000}\PY{p}{,} \PY{n}{currency}\PY{o}{=}\PY{n}{Currency}\PY{o}{.}\PY{n}{USD}\PY{p}{,} \PY{n}{timeToSwapStart\PYZus{}in\PYZus{}days}\PY{o}{=}\PY{l+m+mi}{0}\PY{p}{,} \PY{n}{timeToSwapEnd\PYZus{}in\PYZus{}days}\PY{o}{=}\PY{l+m+mi}{10}\PY{o}{*}\PY{l+m+mi}{360}\PY{p}{,} \PY{n}{swapDirection}\PY{o}{=}\PY{n}{SwapDirection}\PY{o}{.}\PY{n}{PAYER}\PY{p}{)}
\PY{n}{trade2} \PY{o}{=} \PY{n}{InterestRateSwap}\PY{p}{(}\PY{n}{notional} \PY{o}{=} \PY{l+m+mi}{10000000}\PY{p}{,} \PY{n}{currency}\PY{o}{=}\PY{n}{Currency}\PY{o}{.}\PY{n}{USD}\PY{p}{,} \PY{n}{timeToSwapStart\PYZus{}in\PYZus{}days}\PY{o}{=}\PY{l+m+mi}{0}\PY{p}{,} \PY{n}{timeToSwapEnd\PYZus{}in\PYZus{}days}\PY{o}{=}\PY{l+m+mi}{4}\PY{o}{*}\PY{l+m+mi}{360}\PY{p}{,} \PY{n}{swapDirection}\PY{o}{=}\PY{n}{SwapDirection}\PY{o}{.}\PY{n}{RECEIVER}\PY{p}{)}
\PY{n}{trade3\PYZus{}ul} \PY{o}{=} \PY{n}{InterestRateSwap}\PY{p}{(}\PY{n}{notional} \PY{o}{=} \PY{l+m+mi}{5000000}\PY{p}{,} \PY{n}{currency}\PY{o}{=}\PY{n}{Currency}\PY{o}{.}\PY{n}{EUR}\PY{p}{,} \PY{n}{timeToSwapStart\PYZus{}in\PYZus{}days}\PY{o}{=}\PY{l+m+mi}{1}\PY{o}{*}\PY{l+m+mi}{360}\PY{p}{,} \PY{n}{timeToSwapEnd\PYZus{}in\PYZus{}days}\PY{o}{=}\PY{l+m+mi}{11}\PY{o}{*}\PY{l+m+mi}{360}\PY{p}{,} \PY{n}{swapDirection}\PY{o}{=}\PY{n}{SwapDirection}\PY{o}{.}\PY{n}{RECEIVER}\PY{p}{)}
\PY{n}{trade3} \PY{o}{=} \PY{n}{Swaption}\PY{p}{(}\PY{n}{underlyingSwap}\PY{o}{=}\PY{n}{trade3\PYZus{}ul}\PY{p}{,} \PY{n}{optionMaturity\PYZus{}in\PYZus{}days}\PY{o}{=}\PY{l+m+mi}{1}\PY{o}{*}\PY{l+m+mi}{360}\PY{p}{,} \PY{n}{tradeDirection}\PY{o}{=}\PY{n}{TradeDirection}\PY{o}{.}\PY{n}{LONG}\PY{p}{,} \PY{n}{strikeFixedRate}\PY{o}{=}\PY{l+m+mf}{0.05}\PY{p}{)}
\end{Verbatim}
\end{tcolorbox}

    Calculating trade level adjusted notional and supervisory delta. Should
be the following:

\begin{longtable}[]{@{}lll@{}}
\toprule
Trade & Adjusted notional (thousands) & Supervisory delta\tabularnewline
\midrule
\endhead
Trade 1 & 78694 & 1\tabularnewline
Trade 2 & 36254 & -1\tabularnewline
Trade 3 & 37428 & -0.27\tabularnewline
\bottomrule
\end{longtable}

    \begin{tcolorbox}[breakable, size=fbox, boxrule=1pt, pad at break*=1mm,colback=cellbackground, colframe=cellborder]
\prompt{In}{incolor}{ }{\boxspacing}
\begin{Verbatim}[commandchars=\\\{\}]
\PY{n}{ind} \PY{o}{=} \PY{p}{[}\PY{l+s+s1}{\PYZsq{}}\PY{l+s+s1}{Trade 1}\PY{l+s+s1}{\PYZsq{}}\PY{p}{,} \PY{l+s+s1}{\PYZsq{}}\PY{l+s+s1}{Trade 2}\PY{l+s+s1}{\PYZsq{}}\PY{p}{,} \PY{l+s+s1}{\PYZsq{}}\PY{l+s+s1}{Trade 3}\PY{l+s+s1}{\PYZsq{}}\PY{p}{]}
\PY{n}{data} \PY{o}{=} \PY{p}{\PYZob{}}\PY{l+s+s1}{\PYZsq{}}\PY{l+s+s1}{Adjusted notional (thousands)}\PY{l+s+s1}{\PYZsq{}}\PY{p}{:} \PY{p}{[}\PY{n+nb}{round}\PY{p}{(}\PY{n}{trade\PYZus{}level\PYZus{}adjusted\PYZus{}notional}\PY{p}{(}\PY{n}{trade1}\PY{p}{)}\PY{o}{/}\PY{l+m+mi}{1000}\PY{p}{)}\PY{p}{,}
                                          \PY{n+nb}{round}\PY{p}{(}\PY{n}{trade\PYZus{}level\PYZus{}adjusted\PYZus{}notional}\PY{p}{(}\PY{n}{trade2}\PY{p}{)}\PY{o}{/}\PY{l+m+mi}{1000}\PY{p}{)}\PY{p}{,}
                                          \PY{n+nb}{round}\PY{p}{(}\PY{n}{trade\PYZus{}level\PYZus{}adjusted\PYZus{}notional}\PY{p}{(}\PY{n}{trade3}\PY{p}{)}\PY{o}{/}\PY{l+m+mi}{1000}\PY{p}{)}\PY{p}{]}\PY{p}{,}
        \PY{l+s+s1}{\PYZsq{}}\PY{l+s+s1}{Supervisory delta}\PY{l+s+s1}{\PYZsq{}}\PY{p}{:} \PY{p}{[}\PY{n+nb}{round}\PY{p}{(}\PY{n}{calculate\PYZus{}sa\PYZus{}ccr\PYZus{}delta}\PY{p}{(}\PY{n}{trade1}\PY{p}{)}\PY{p}{)}\PY{p}{,}
                              \PY{n+nb}{round}\PY{p}{(}\PY{n}{calculate\PYZus{}sa\PYZus{}ccr\PYZus{}delta}\PY{p}{(}\PY{n}{trade2}\PY{p}{)}\PY{p}{)}\PY{p}{,}
                              \PY{n+nb}{round}\PY{p}{(}\PY{n}{calculate\PYZus{}sa\PYZus{}ccr\PYZus{}delta}\PY{p}{(}\PY{n}{trade3}\PY{p}{)}\PY{p}{,}\PY{l+m+mi}{2}\PY{p}{)}\PY{p}{]}\PY{p}{\PYZcb{}}
\PY{n}{df} \PY{o}{=} \PY{n}{DataFrame}\PY{p}{(}\PY{n}{data} \PY{o}{=} \PY{n}{data}\PY{p}{,} \PY{n}{index} \PY{o}{=} \PY{n}{ind}\PY{p}{)}
\PY{n}{df}
\end{Verbatim}
\end{tcolorbox}

    The implemented formulas correctly replicate the results given in the
first Example of Annex 4a.

    SA-CCR uses the same Black-Scholes based formula for Swaps as it uses
for Equities. It differentiates options in two dimensions. Whether they
are \emph{bought} or \emph{sold} and whether they are \emph{Call} or
\emph{Put} options (Compare paragraph 159).

SA-CCR defines an option as a call option if it rises in value as the
underlying rises in value. A fixed payer swap rises in value as the
underlying interest rate rises in value. Therefore, an option to buy a
fixed payer swap at a predetermined strike also rises in value as the
underlying interest rate rises in value. Therefore, a swaption on a
payer swap is considered a \emph{Call} under SA-CCR, while a swaption on
a receiver swap is considered a \emph{Put}.

    \hypertarget{risk-horizon}{%
\subsection{Risk horizon}\label{risk-horizon}}

For unmargined transaction the margining factor is

\[MF^{\text{unmargined}}_i = \sqrt{\frac{\min\left(M_i;1\text{ year}\right)}{1\text{ year}}}\]

This factor can be used to scale down a risk weight calibrated for a 1
year horizon to a shorter period.

With margining the margin period of risk (MPOR) is:

\begin{itemize}
\tightlist
\item
  10 business days for small, uncleared OTC portfolios
\item
  5 business days for cleared derivatives
\item
  20 business days for netting sets with more than 5000 transactions
  that are not with a central counterparty
\item
  and doubling this period for portfolios with outstanding disputes
\end{itemize}

The margining factor is then

\[ MF^{\text{margined}}_i = \frac{3}{2}\sqrt{\frac{MPOR_i}{1\text{ year}}} \]

    At this point we need to introduce a collateral agreement object. For
simplicities sake we will not differentiate between collateral and
netting sets in this thesis. All trades that are covered by the same
collateral agreement are also admissible for netting with each other.
(Also refer to the introduction of close out netting above)

The default properties of a collateral agreement are displayed below:

    \begin{tcolorbox}[breakable, size=fbox, boxrule=1pt, pad at break*=1mm,colback=cellbackground, colframe=cellborder]
\prompt{In}{incolor}{1}{\boxspacing}
\begin{Verbatim}[commandchars=\\\{\}]
\PY{n}{ca} \PY{o}{=} \PY{n}{CollateralAgreement}\PY{p}{(}\PY{p}{)}
\PY{n+nb}{print}\PY{p}{(}\PY{n}{ca}\PY{p}{)}
\end{Verbatim}
\end{tcolorbox}

    \begin{Verbatim}[commandchars=\\\{\}]

        ---------------------------------------------------------------------------

        NameError                                 Traceback (most recent call last)

        <ipython-input-1-ec8bb2860cd7> in <module>
    ----> 1 ca = CollateralAgreement()
          2 print(ca)
          3 
    

        NameError: name 'CollateralAgreement' is not defined

    \end{Verbatim}

    And they may be changed at the initialization of the collateral
agreement or afterwards. The latter case is displayed in the next cell:

    \begin{tcolorbox}[breakable, size=fbox, boxrule=1pt, pad at break*=1mm,colback=cellbackground, colframe=cellborder]
\prompt{In}{incolor}{ }{\boxspacing}
\begin{Verbatim}[commandchars=\\\{\}]
\PY{n}{ca}\PY{o}{.}\PY{n}{margining}\PY{o}{=} \PY{n}{Margining}\PY{o}{.}\PY{n}{MARGINED}
\PY{n}{ca}\PY{o}{.}\PY{n}{clearing}\PY{o}{=} \PY{n}{Clearing}\PY{o}{.}\PY{n}{CLEARED}
\PY{n}{ca}\PY{o}{.}\PY{n}{vm} \PY{o}{=} \PY{o}{\PYZhy{}}\PY{l+m+mi}{3000}
\PY{n}{ca}\PY{o}{.}\PY{n}{posted\PYZus{}im} \PY{o}{=} \PY{l+m+mi}{1000}
\PY{n}{ca}\PY{o}{.}\PY{n}{received\PYZus{}im} \PY{o}{=} \PY{l+m+mi}{1000}

\PY{n+nb}{print}\PY{p}{(}\PY{n}{ca}\PY{p}{)}
\end{Verbatim}
\end{tcolorbox}

    With this collateral set object we can define a function for calculation
the margining factor:

    \begin{tcolorbox}[breakable, size=fbox, boxrule=1pt, pad at break*=1mm,colback=cellbackground, colframe=cellborder]
\prompt{In}{incolor}{ }{\boxspacing}
\begin{Verbatim}[commandchars=\\\{\}]
\PY{k}{def} \PY{n+nf}{margining\PYZus{}factor}\PY{p}{(}\PY{n}{trade} \PY{p}{:} \PY{n}{Trade}\PY{p}{,} \PY{n}{ca} \PY{p}{:} \PY{n}{CollateralAgreement}\PY{p}{)} \PY{o}{\PYZhy{}}\PY{o}{\PYZgt{}} \PY{n+nb}{float}\PY{p}{:}
    \PY{k}{if} \PY{n}{ca}\PY{o}{.}\PY{n}{margining} \PY{o}{==} \PY{n}{Margining}\PY{o}{.}\PY{n}{UNMARGINED}\PY{p}{:}
        \PY{n}{floored\PYZus{}maturity} \PY{o}{=} \PY{n+nb}{max}\PY{p}{(}\PY{l+m+mi}{10}\PY{o}{/}\PY{l+m+mi}{250}\PY{p}{,} \PY{n}{trade}\PY{o}{.}\PY{n}{m}\PY{p}{)} \PY{c+c1}{\PYZsh{}as Everything is measured in years 10/250 is equal to 10 business days.}
        \PY{n}{mf\PYZus{}unmargined} \PY{o}{=} \PY{n}{sqrt}\PY{p}{(}\PY{n+nb}{min}\PY{p}{(}\PY{n}{floored\PYZus{}maturity}\PY{p}{,} \PY{l+m+mi}{1}\PY{p}{)}\PY{o}{/}\PY{l+m+mi}{1}\PY{p}{)}
        \PY{k}{return} \PY{n}{mf\PYZus{}unmargined}
    \PY{k}{if} \PY{n}{ca}\PY{o}{.}\PY{n}{margining} \PY{o}{==} \PY{n}{Margining}\PY{o}{.}\PY{n}{MARGINED}\PY{p}{:}
        \PY{c+c1}{\PYZsh{} Compare paragraph 164 to see how MPOR needs to be set for margined trades.}
        \PY{k}{if} \PY{n}{ca}\PY{o}{.}\PY{n}{clearing}\PY{o}{==}\PY{n}{Clearing}\PY{o}{.}\PY{n}{CLEARED}\PY{p}{:}
            \PY{n}{MPOR} \PY{o}{=} \PY{l+m+mi}{5}
        \PY{k}{elif} \PY{n}{ca}\PY{o}{.}\PY{n}{tradecount}\PY{o}{==}\PY{n}{Tradecount}\PY{o}{.}\PY{n}{OVER\PYZus{}FIVE\PYZus{}THOUSAND}\PY{p}{:}
            \PY{n}{MPOR} \PY{o}{=} \PY{l+m+mi}{20}
        \PY{k}{else}\PY{p}{:} \PY{n}{MPOR} \PY{o}{=} \PY{l+m+mi}{10}
        \PY{k}{if} \PY{n}{ca}\PY{o}{.}\PY{n}{dispute} \PY{o}{==} \PY{n}{Dispute}\PY{o}{.}\PY{n}{OUTSTANDING\PYZus{}DISPUTES}\PY{p}{:}
            \PY{n}{MPOR} \PY{o}{=} \PY{n}{MPOR}\PY{o}{*}\PY{l+m+mi}{2}
        
        \PY{n}{mf\PYZus{}margined} \PY{o}{=} \PY{l+m+mi}{3}\PY{o}{/}\PY{l+m+mi}{2} \PY{o}{*} \PY{n}{sqrt}\PY{p}{(}\PY{n}{MPOR}\PY{o}{/}\PY{l+m+mi}{250}\PY{p}{)} \PY{c+c1}{\PYZsh{}Since MPOR above is defined in Business days and not year fractions division by 250 is necessary.}
        \PY{k}{return} \PY{n}{mf\PYZus{}margined}
\end{Verbatim}
\end{tcolorbox}

    For trades of differing maturity let's compare the margining factor for
the three most common scenarios:

\begin{enumerate}
\def\labelenumi{\arabic{enumi}.}
\tightlist
\item
  No margining
\item
  Bilateral uncleared
\item
  Centrally cleared
\end{enumerate}

    \begin{tcolorbox}[breakable, size=fbox, boxrule=1pt, pad at break*=1mm,colback=cellbackground, colframe=cellborder]
\prompt{In}{incolor}{ }{\boxspacing}
\begin{Verbatim}[commandchars=\\\{\}]
\PY{n}{one\PYZus{}day} \PY{o}{=} \PY{n}{InterestRateSwap}\PY{p}{(}\PY{n}{notional} \PY{o}{=} \PY{l+m+mi}{1000000}\PY{p}{,} \PY{n}{currency} \PY{o}{=} \PY{n}{Currency}\PY{o}{.}\PY{n}{USD}\PY{p}{,} \PY{n}{timeToSwapStart\PYZus{}in\PYZus{}days}\PY{o}{=}\PY{l+m+mi}{0}\PY{p}{,} \PY{n}{timeToSwapEnd\PYZus{}in\PYZus{}days}\PY{o}{=}\PY{l+m+mi}{1}\PY{p}{,} \PY{n}{swapDirection}\PY{o}{=}\PY{n}{SwapDirection}\PY{o}{.}\PY{n}{PAYER}\PY{p}{)}
\PY{n}{two\PYZus{}weeks} \PY{o}{=} \PY{n}{InterestRateSwap}\PY{p}{(}\PY{n}{notional} \PY{o}{=} \PY{l+m+mi}{1000000}\PY{p}{,} \PY{n}{currency} \PY{o}{=} \PY{n}{Currency}\PY{o}{.}\PY{n}{USD}\PY{p}{,} \PY{n}{timeToSwapStart\PYZus{}in\PYZus{}days}\PY{o}{=}\PY{l+m+mi}{0}\PY{p}{,} \PY{n}{timeToSwapEnd\PYZus{}in\PYZus{}days}\PY{o}{=}\PY{l+m+mi}{10}\PY{p}{,} \PY{n}{swapDirection}\PY{o}{=}\PY{n}{SwapDirection}\PY{o}{.}\PY{n}{PAYER}\PY{p}{)}
\PY{n}{six\PYZus{}months} \PY{o}{=} \PY{n}{InterestRateSwap}\PY{p}{(}\PY{n}{notional} \PY{o}{=} \PY{l+m+mi}{1000000}\PY{p}{,} \PY{n}{currency} \PY{o}{=} \PY{n}{Currency}\PY{o}{.}\PY{n}{USD}\PY{p}{,} \PY{n}{timeToSwapStart\PYZus{}in\PYZus{}days}\PY{o}{=}\PY{l+m+mi}{0}\PY{p}{,} \PY{n}{timeToSwapEnd\PYZus{}in\PYZus{}days}\PY{o}{=}\PY{l+m+mi}{125}\PY{p}{,} \PY{n}{swapDirection}\PY{o}{=}\PY{n}{SwapDirection}\PY{o}{.}\PY{n}{PAYER}\PY{p}{)}
\PY{n}{one\PYZus{}year} \PY{o}{=} \PY{n}{InterestRateSwap}\PY{p}{(}\PY{n}{notional} \PY{o}{=} \PY{l+m+mi}{1000000}\PY{p}{,} \PY{n}{currency} \PY{o}{=} \PY{n}{Currency}\PY{o}{.}\PY{n}{USD}\PY{p}{,} \PY{n}{timeToSwapStart\PYZus{}in\PYZus{}days}\PY{o}{=}\PY{l+m+mi}{0}\PY{p}{,} \PY{n}{timeToSwapEnd\PYZus{}in\PYZus{}days}\PY{o}{=}\PY{l+m+mi}{250}\PY{p}{,} \PY{n}{swapDirection}\PY{o}{=}\PY{n}{SwapDirection}\PY{o}{.}\PY{n}{PAYER}\PY{p}{)}
\PY{n}{ten\PYZus{}years} \PY{o}{=} \PY{n}{InterestRateSwap}\PY{p}{(}\PY{n}{notional} \PY{o}{=} \PY{l+m+mi}{1000000}\PY{p}{,} \PY{n}{currency} \PY{o}{=} \PY{n}{Currency}\PY{o}{.}\PY{n}{USD}\PY{p}{,} \PY{n}{timeToSwapStart\PYZus{}in\PYZus{}days}\PY{o}{=}\PY{l+m+mi}{0}\PY{p}{,} \PY{n}{timeToSwapEnd\PYZus{}in\PYZus{}days}\PY{o}{=}\PY{l+m+mi}{2500}\PY{p}{,} \PY{n}{swapDirection}\PY{o}{=}\PY{n}{SwapDirection}\PY{o}{.}\PY{n}{PAYER}\PY{p}{)}

\PY{n}{no\PYZus{}margining} \PY{o}{=} \PY{n}{CollateralAgreement}\PY{p}{(}\PY{p}{)}
\PY{n}{bilateral\PYZus{}margining} \PY{o}{=} \PY{n}{CollateralAgreement}\PY{p}{(}\PY{n}{margining} \PY{o}{=} \PY{n}{Margining}\PY{o}{.}\PY{n}{MARGINED}\PY{p}{)}
\PY{n}{central\PYZus{}clearing} \PY{o}{=} \PY{n}{CollateralAgreement}\PY{p}{(}\PY{n}{margining} \PY{o}{=} \PY{n}{Margining}\PY{o}{.}\PY{n}{MARGINED}\PY{p}{,} \PY{n}{clearing} \PY{o}{=} \PY{n}{Clearing}\PY{o}{.}\PY{n}{CLEARED}\PY{p}{)}

\PY{n}{trades} \PY{o}{=} \PY{p}{[}\PY{n}{one\PYZus{}day}\PY{p}{,} \PY{n}{two\PYZus{}weeks}\PY{p}{,} \PY{n}{six\PYZus{}months}\PY{p}{,} \PY{n}{one\PYZus{}year}\PY{p}{,} \PY{n}{ten\PYZus{}years}\PY{p}{]}
\PY{n}{cas} \PY{o}{=} \PY{p}{[}\PY{n}{no\PYZus{}margining}\PY{p}{,} \PY{n}{bilateral\PYZus{}margining}\PY{p}{,} \PY{n}{central\PYZus{}clearing}\PY{p}{]}
\PY{n}{ar} \PY{o}{=} \PY{n}{numpy}\PY{o}{.}\PY{n}{empty}\PY{p}{(}\PY{p}{[}\PY{l+m+mi}{3}\PY{p}{,}\PY{l+m+mi}{5}\PY{p}{]}\PY{p}{)}
\PY{n}{i} \PY{o}{=} \PY{l+m+mi}{0}
\PY{k}{for} \PY{n}{t} \PY{o+ow}{in} \PY{n}{trades}\PY{p}{:}
    \PY{n}{j} \PY{o}{=} \PY{l+m+mi}{0}
    \PY{k}{for} \PY{n}{ca} \PY{o+ow}{in} \PY{n}{cas}\PY{p}{:}
        \PY{n}{ar}\PY{p}{[}\PY{n}{j}\PY{p}{,}\PY{n}{i}\PY{p}{]} \PY{o}{=} \PY{n}{margining\PYZus{}factor}\PY{p}{(}\PY{n}{t}\PY{p}{,}\PY{n}{ca}\PY{p}{)}
        \PY{n}{j} \PY{o}{+}\PY{o}{=} \PY{l+m+mi}{1}
    \PY{n}{i} \PY{o}{+}\PY{o}{=} \PY{l+m+mi}{1}

\PY{n}{DataFrame}\PY{p}{(}\PY{n}{index}\PY{o}{=}\PY{p}{[}\PY{l+s+s1}{\PYZsq{}}\PY{l+s+s1}{No margining}\PY{l+s+s1}{\PYZsq{}}\PY{p}{,} \PY{l+s+s1}{\PYZsq{}}\PY{l+s+s1}{Bilateral uncleared}\PY{l+s+s1}{\PYZsq{}}\PY{p}{,} \PY{l+s+s1}{\PYZsq{}}\PY{l+s+s1}{Centrally cleared}\PY{l+s+s1}{\PYZsq{}}\PY{p}{]}\PY{p}{,}
          \PY{n}{columns}\PY{o}{=}\PY{p}{[}\PY{l+s+s1}{\PYZsq{}}\PY{l+s+s1}{One day}\PY{l+s+s1}{\PYZsq{}}\PY{p}{,} \PY{l+s+s1}{\PYZsq{}}\PY{l+s+s1}{Two weeks}\PY{l+s+s1}{\PYZsq{}}\PY{p}{,} \PY{l+s+s1}{\PYZsq{}}\PY{l+s+s1}{Six months}\PY{l+s+s1}{\PYZsq{}}\PY{p}{,} \PY{l+s+s1}{\PYZsq{}}\PY{l+s+s1}{One year}\PY{l+s+s1}{\PYZsq{}}\PY{p}{,} \PY{l+s+s1}{\PYZsq{}}\PY{l+s+s1}{Ten years}\PY{l+s+s1}{\PYZsq{}}\PY{p}{]}\PY{p}{,}
          \PY{n}{data} \PY{o}{=} \PY{n}{ar}\PY{p}{)}
\end{Verbatim}
\end{tcolorbox}

    \hypertarget{addon-for-interest-rate-derivatives}{%
\subsection{AddOn for interest rate
derivatives}\label{addon-for-interest-rate-derivatives}}

\hypertarget{step-1---calculation-of-effective-notional-d_jkir}{%
\paragraph{\texorpdfstring{Step 1 - calculation of effective notional
\(D_{jk}^{IR}\)}{Step 1 - calculation of effective notional D\_\{jk\}\^{}\{IR\}}}\label{step-1---calculation-of-effective-notional-d_jkir}}

\begin{align*}
D_{jk}^{IR} &= \sum_{i\in\left\{Ccy_j, MB_k\right\}}{\delta_i*d_i^{IR}*MF_i}
\end{align*}

Here, the notation \(i\in\left\{Ccy_j, MB_k\right\}\) refers to trades
whose underlying is the interest rate of a common currency \(j\) and
which mature in a common maturity bucket \(k\)

    \begin{tcolorbox}[breakable, size=fbox, boxrule=1pt, pad at break*=1mm,colback=cellbackground, colframe=cellborder]
\prompt{In}{incolor}{ }{\boxspacing}
\begin{Verbatim}[commandchars=\\\{\}]
\PY{k}{def} \PY{n+nf}{get\PYZus{}supervisory\PYZus{}factor}\PY{p}{(}\PY{n}{assetClass}\PY{p}{:} \PY{n}{AssetClass}\PY{p}{,} \PY{n}{subClass}\PY{p}{:} \PY{n}{SubClass} \PY{o}{=} \PY{k+kc}{None}\PY{p}{)} \PY{o}{\PYZhy{}}\PY{o}{\PYZgt{}} \PY{n+nb}{float}\PY{p}{:}
    \PY{n}{supervisory\PYZus{}parameter} \PY{o}{=} \PY{n}{read\PYZus{}csv}\PY{p}{(}\PY{l+s+s1}{\PYZsq{}}\PY{l+s+s1}{supervisory\PYZus{}parameters.csv}\PY{l+s+s1}{\PYZsq{}}\PY{p}{)}
    \PY{n}{sf} \PY{o}{=} \PY{k+kc}{None}
    \PY{k}{if} \PY{n}{subClass} \PY{o+ow}{is} \PY{k+kc}{None}\PY{p}{:}
        \PY{n}{sf} \PY{o}{=} \PY{n}{supervisory\PYZus{}parameter}\PY{p}{[}\PY{p}{(}\PY{n}{supervisory\PYZus{}parameter}\PY{o}{.}\PY{n}{AssetClass} \PY{o}{==} \PY{n}{assetClass}\PY{o}{.}\PY{n}{value}\PY{p}{)}\PY{p}{]}\PY{p}{[}\PY{l+s+s1}{\PYZsq{}}\PY{l+s+s1}{Supervisory factor}\PY{l+s+s1}{\PYZsq{}}\PY{p}{]}\PY{o}{.}\PY{n}{iloc}\PY{p}{[}\PY{l+m+mi}{0}\PY{p}{]}
    \PY{k}{elif} \PY{n}{subClass} \PY{o+ow}{is} \PY{o+ow}{not} \PY{k+kc}{None}\PY{p}{:}
        \PY{n}{sf} \PY{o}{=} \PY{n}{supervisory\PYZus{}parameter}\PY{p}{[}\PY{p}{(}\PY{n}{supervisory\PYZus{}parameter}\PY{o}{.}\PY{n}{AssetClass} \PY{o}{==} \PY{n}{assetClass}\PY{o}{.}\PY{n}{value}\PY{p}{)} \PY{o}{\PYZam{}} \PY{p}{(}\PY{n}{supervisory\PYZus{}parameter}\PY{o}{.}\PY{n}{SubClass} \PY{o}{==} \PY{n}{subClass}\PY{o}{.}\PY{n}{value}\PY{p}{)}\PY{p}{]}\PY{p}{[}\PY{l+s+s1}{\PYZsq{}}\PY{l+s+s1}{Supervisory factor}\PY{l+s+s1}{\PYZsq{}}\PY{p}{]}\PY{o}{.}\PY{n}{iloc}\PY{p}{[}\PY{l+m+mi}{0}\PY{p}{]}
    \PY{k}{return} \PY{n}{sf}

\PY{k}{def} \PY{n+nf}{interest\PYZus{}rate\PYZus{}addOn}\PY{p}{(}\PY{n}{trades}\PY{p}{:} \PY{n}{List}\PY{p}{[}\PY{n}{Trade}\PY{p}{]}\PY{p}{,} \PY{n}{ca}\PY{p}{:} \PY{n}{CollateralAgreement}\PY{p}{)} \PY{o}{\PYZhy{}}\PY{o}{\PYZgt{}} \PY{n+nb}{float}\PY{p}{:}
    \PY{n}{bucketed\PYZus{}trades} \PY{o}{=} \PY{p}{\PYZob{}}\PY{p}{\PYZcb{}}
    \PY{n}{en\PYZus{}cur\PYZus{}mat} \PY{o}{=} \PY{p}{\PYZob{}}\PY{p}{\PYZcb{}}
    \PY{n}{add\PYZus{}on\PYZus{}cur} \PY{o}{=} \PY{n}{Series}\PY{p}{(}\PY{p}{)}
    \PY{n}{currencies} \PY{o}{=} \PY{n+nb}{set}\PY{p}{(}\PY{p}{)}
    
    \PY{k}{for} \PY{n}{t} \PY{o+ow}{in} \PY{n}{trades}\PY{p}{:}
        \PY{n}{key} \PY{o}{=} \PY{p}{(}\PY{n}{t}\PY{o}{.}\PY{n}{currency}\PY{p}{,} \PY{n}{t}\PY{o}{.}\PY{n}{get\PYZus{}maturity\PYZus{}bucket}\PY{p}{(}\PY{p}{)}\PY{p}{)}
        \PY{n}{currencies}\PY{o}{.}\PY{n}{add}\PY{p}{(}\PY{n}{t}\PY{o}{.}\PY{n}{currency}\PY{p}{)}
        \PY{k}{if} \PY{n}{key} \PY{o+ow}{in} \PY{n}{bucketed\PYZus{}trades}\PY{p}{:}
            \PY{n}{bucketed\PYZus{}trades}\PY{p}{[}\PY{n}{key}\PY{p}{]}\PY{o}{.}\PY{n}{append}\PY{p}{(}\PY{n}{t}\PY{p}{)}
        \PY{k}{else}\PY{p}{:}
            \PY{n}{bucketed\PYZus{}trades}\PY{p}{[}\PY{n}{key}\PY{p}{]} \PY{o}{=} \PY{p}{[}\PY{n}{t}\PY{p}{]}
    
    \PY{k}{for} \PY{n}{key}\PY{p}{,} \PY{n}{trades} \PY{o+ow}{in} \PY{n}{bucketed\PYZus{}trades}\PY{o}{.}\PY{n}{items}\PY{p}{(}\PY{p}{)}\PY{p}{:}
        \PY{n}{effective\PYZus{}notional} \PY{o}{=} \PY{l+m+mi}{0}
        \PY{k}{for} \PY{n}{t} \PY{o+ow}{in} \PY{n}{trades}\PY{p}{:}
            \PY{n}{effective\PYZus{}notional} \PY{o}{+}\PY{o}{=} \PY{n}{calculate\PYZus{}sa\PYZus{}ccr\PYZus{}delta}\PY{p}{(}\PY{n}{t}\PY{p}{)} \PY{o}{*} \PY{n}{trade\PYZus{}level\PYZus{}adjusted\PYZus{}notional}\PY{p}{(}\PY{n}{t}\PY{p}{)} \PY{o}{*} \PY{n}{margining\PYZus{}factor}\PY{p}{(}\PY{n}{t}\PY{p}{,} \PY{n}{ca}\PY{p}{)}
        \PY{n}{en\PYZus{}cur\PYZus{}mat}\PY{p}{[}\PY{n}{key}\PY{p}{]} \PY{o}{=} \PY{n}{effective\PYZus{}notional}
        
    \PY{k}{for} \PY{n}{cur} \PY{o+ow}{in} \PY{n}{currencies}\PY{p}{:}
        \PY{n}{d\PYZus{}1} \PY{o}{=} \PY{n}{en\PYZus{}cur\PYZus{}mat}\PY{o}{.}\PY{n}{get}\PY{p}{(}\PY{p}{(}\PY{n}{cur}\PY{p}{,} \PY{n}{MaturityBucket}\PY{o}{.}\PY{n}{ONE}\PY{p}{)}\PY{p}{,}\PY{l+m+mi}{0}\PY{p}{)}
        \PY{n}{d\PYZus{}2} \PY{o}{=} \PY{n}{en\PYZus{}cur\PYZus{}mat}\PY{o}{.}\PY{n}{get}\PY{p}{(}\PY{p}{(}\PY{n}{cur}\PY{p}{,} \PY{n}{MaturityBucket}\PY{o}{.}\PY{n}{TWO}\PY{p}{)}\PY{p}{,}\PY{l+m+mi}{0}\PY{p}{)}
        \PY{n}{d\PYZus{}3} \PY{o}{=} \PY{n}{en\PYZus{}cur\PYZus{}mat}\PY{o}{.}\PY{n}{get}\PY{p}{(}\PY{p}{(}\PY{n}{cur}\PY{p}{,} \PY{n}{MaturityBucket}\PY{o}{.}\PY{n}{THREE}\PY{p}{)}\PY{p}{,}\PY{l+m+mi}{0}\PY{p}{)}
        
        \PY{n}{en\PYZus{}cur} \PY{o}{=} \PY{n}{sqrt}\PY{p}{(}\PY{n}{d\PYZus{}1}\PY{o}{*}\PY{o}{*}\PY{l+m+mi}{2} \PY{o}{+} \PY{n}{d\PYZus{}2}\PY{o}{*}\PY{o}{*}\PY{l+m+mi}{2} \PY{o}{+} \PY{n}{d\PYZus{}3}\PY{o}{*}\PY{o}{*}\PY{l+m+mi}{2} \PY{o}{+} \PY{l+m+mf}{1.4}\PY{o}{*}\PY{n}{d\PYZus{}1} \PY{o}{*}\PY{n}{d\PYZus{}2} \PY{o}{+} \PY{l+m+mf}{1.4}\PY{o}{*}\PY{n}{d\PYZus{}2}\PY{o}{*}\PY{n}{d\PYZus{}3} \PY{o}{+} \PY{l+m+mf}{0.6}\PY{o}{*}\PY{n}{d\PYZus{}1}\PY{o}{*}\PY{n}{d\PYZus{}3}\PY{p}{)}
        
        \PY{n}{add\PYZus{}on\PYZus{}cur}\PY{p}{[}\PY{n}{cur}\PY{p}{]} \PY{o}{=} \PY{n}{get\PYZus{}supervisory\PYZus{}factor}\PY{p}{(}\PY{n}{assetClass} \PY{o}{=} \PY{n}{AssetClass}\PY{o}{.}\PY{n}{IR}\PY{p}{)} \PY{o}{*} \PY{n}{en\PYZus{}cur}

    \PY{k}{return} \PY{n}{add\PYZus{}on\PYZus{}cur}\PY{o}{.}\PY{n}{sum}\PY{p}{(}\PY{p}{)}
\end{Verbatim}
\end{tcolorbox}

    For a quick sanity check we can again check against the example 1 in
Appendix 4a. The \(AddOn_{IR}\) should be 347 thousand USD.

    \begin{tcolorbox}[breakable, size=fbox, boxrule=1pt, pad at break*=1mm,colback=cellbackground, colframe=cellborder]
\prompt{In}{incolor}{ }{\boxspacing}
\begin{Verbatim}[commandchars=\\\{\}]
\PY{n+nb}{round}\PY{p}{(}\PY{n}{interest\PYZus{}rate\PYZus{}addOn}\PY{p}{(}\PY{p}{[}\PY{n}{trade1}\PY{p}{,} \PY{n}{trade2}\PY{p}{,} \PY{n}{trade3}\PY{p}{]}\PY{p}{,} \PY{n}{CollateralAgreement}\PY{p}{(}\PY{p}{)}\PY{p}{)}\PY{o}{/}\PY{l+m+mi}{1000}\PY{p}{)}
\end{Verbatim}
\end{tcolorbox}

    Next, we are setting up a function to calculate the \(AddOn\) for
derivatives from the FX asset class. The approach is a little simpler
than that for the IR asset class as no differentiaten between time
buckets is necessary. All derivatives on a common currency pair can be
set off against each other.

\begin{align*}
AddOn^{FX} &= \sum_j{AddOn_{HS_j}^{FX}} \\
AddOn^{FX}_{HS_j} &= SF_j^{FX} * | \text{EffectiveNotional}_j^{FX} | \\
\text{EffectiveNotional}_j^{FX} &= \sum_{i \in HS_j}{\delta_i * d_i^{FX} * MF_i^{type}}
\end{align*}

With the hedging sets \(i \in HS\) referencing all derivatives on a
common currency pair.

    \begin{tcolorbox}[breakable, size=fbox, boxrule=1pt, pad at break*=1mm,colback=cellbackground, colframe=cellborder]
\prompt{In}{incolor}{ }{\boxspacing}
\begin{Verbatim}[commandchars=\\\{\}]
\PY{k}{def} \PY{n+nf}{fx\PYZus{}addOn}\PY{p}{(}\PY{n}{trades}\PY{p}{:} \PY{n}{List}\PY{p}{[}\PY{n}{Trade}\PY{p}{]}\PY{p}{,} \PY{n}{ca}\PY{p}{:} \PY{n}{CollateralAgreement}\PY{p}{)} \PY{o}{\PYZhy{}}\PY{o}{\PYZgt{}} \PY{n+nb}{float}\PY{p}{:}
    \PY{n}{bucketed\PYZus{}trades} \PY{o}{=} \PY{p}{\PYZob{}}\PY{p}{\PYZcb{}}
    \PY{n}{en\PYZus{}cur} \PY{o}{=} \PY{p}{\PYZob{}}\PY{p}{\PYZcb{}}
    \PY{n}{add\PYZus{}on\PYZus{}cur} \PY{o}{=} \PY{n}{Series}\PY{p}{(}\PY{p}{)}
    \PY{n}{currencyPairs} \PY{o}{=} \PY{n+nb}{set}\PY{p}{(}\PY{p}{)}
    
    \PY{k}{for} \PY{n}{t} \PY{o+ow}{in} \PY{n}{trades}\PY{p}{:}
        \PY{n}{key} \PY{o}{=} \PY{p}{(}\PY{n}{t}\PY{o}{.}\PY{n}{currencyPair}\PY{p}{)}
        \PY{n}{currencyPairs}\PY{o}{.}\PY{n}{add}\PY{p}{(}\PY{n}{t}\PY{o}{.}\PY{n}{currencyPair}\PY{p}{)}
        \PY{k}{if} \PY{n}{key} \PY{o+ow}{in} \PY{n}{bucketed\PYZus{}trades}\PY{p}{:}
            \PY{n}{bucketed\PYZus{}trades}\PY{p}{[}\PY{n}{key}\PY{p}{]}\PY{o}{.}\PY{n}{append}\PY{p}{(}\PY{n}{t}\PY{p}{)}
        \PY{k}{else}\PY{p}{:}
            \PY{n}{bucketed\PYZus{}trades}\PY{p}{[}\PY{n}{key}\PY{p}{]} \PY{o}{=} \PY{p}{[}\PY{n}{t}\PY{p}{]}
            
    \PY{k}{for} \PY{n}{key}\PY{p}{,} \PY{n}{trades} \PY{o+ow}{in} \PY{n}{bucketed\PYZus{}trades}\PY{o}{.}\PY{n}{items}\PY{p}{(}\PY{p}{)}\PY{p}{:}
        \PY{n}{effective\PYZus{}notional} \PY{o}{=} \PY{l+m+mi}{0}
        \PY{k}{for} \PY{n}{t} \PY{o+ow}{in} \PY{n}{trades}\PY{p}{:}
            \PY{n}{effective\PYZus{}notional} \PY{o}{+}\PY{o}{=} \PY{n}{calculate\PYZus{}sa\PYZus{}ccr\PYZus{}delta}\PY{p}{(}\PY{n}{t}\PY{p}{)} \PY{o}{*} \PY{n}{trade\PYZus{}level\PYZus{}adjusted\PYZus{}notional}\PY{p}{(}\PY{n}{t}\PY{p}{)} \PY{o}{*} \PY{n}{margining\PYZus{}factor}\PY{p}{(}\PY{n}{t}\PY{p}{,} \PY{n}{ca}\PY{p}{)}
        \PY{n}{en\PYZus{}cur}\PY{p}{[}\PY{n}{key}\PY{p}{]} \PY{o}{=} \PY{n}{effective\PYZus{}notional}
        
    \PY{k}{for} \PY{n}{cur} \PY{o+ow}{in} \PY{n}{currencyPairs}\PY{p}{:}
        
        \PY{n}{add\PYZus{}on\PYZus{}cur}\PY{p}{[}\PY{n}{cur}\PY{p}{]} \PY{o}{=} \PY{n}{get\PYZus{}supervisory\PYZus{}factor}\PY{p}{(}\PY{n}{assetClass} \PY{o}{=} \PY{n}{AssetClass}\PY{o}{.}\PY{n}{FX}\PY{p}{)} \PY{o}{*} \PY{n+nb}{abs}\PY{p}{(}\PY{n}{en\PYZus{}cur}\PY{p}{[}\PY{n}{cur}\PY{p}{]}\PY{p}{)}
        
    \PY{k}{return} \PY{n}{add\PYZus{}on\PYZus{}cur}\PY{o}{.}\PY{n}{sum}\PY{p}{(}\PY{p}{)}

\PY{n}{fx\PYZus{}trade1} \PY{o}{=} \PY{n}{FxForward}\PY{p}{(}\PY{n}{notional} \PY{o}{=} \PY{l+m+mi}{1000000}\PY{p}{,} \PY{n}{currencyPair} \PY{o}{=} \PY{n}{CurrencyPair}\PY{o}{.}\PY{n}{GBPUSD}\PY{p}{,} \PY{n}{tradeDirection} \PY{o}{=} \PY{n}{TradeDirection}\PY{o}{.}\PY{n}{LONG}\PY{p}{,} \PY{n}{m}\PY{o}{=}\PY{l+m+mi}{3}\PY{p}{)}
\PY{n}{fx\PYZus{}trade2} \PY{o}{=} \PY{n}{FxOption}\PY{p}{(}\PY{n}{notional}\PY{o}{=} \PY{l+m+mi}{2000000}\PY{p}{,} \PY{n}{currencyPair} \PY{o}{=} \PY{n}{CurrencyPair}\PY{o}{.}\PY{n}{GBPUSD}\PY{p}{,} \PY{n}{tradeDirection} \PY{o}{=} \PY{n}{TradeDirection}\PY{o}{.}\PY{n}{LONG}\PY{p}{,} \PY{n}{tradeType} \PY{o}{=} \PY{n}{TradeType}\PY{o}{.}\PY{n}{PUT}\PY{p}{,} \PY{n}{m} \PY{o}{=} \PY{l+m+mi}{2}\PY{p}{,} \PY{n}{strike} \PY{o}{=} \PY{l+m+mf}{1.1}\PY{p}{,} \PY{n}{currentForwardFxRate}\PY{o}{=}\PY{l+m+mf}{1.07}\PY{p}{)}

\PY{n}{fx\PYZus{}addOn}\PY{p}{(}\PY{p}{[}\PY{n}{fx\PYZus{}trade1}\PY{p}{,} \PY{n}{fx\PYZus{}trade2}\PY{p}{]}\PY{p}{,} \PY{n}{CollateralAgreement}\PY{p}{(}\PY{p}{)}\PY{p}{)}
\end{Verbatim}
\end{tcolorbox}

    Since credit and commodity derivatives will not be regarded in this
thesis the final \(AddOn\) component that we will set up is for the
equity asset class. The formula is somwhat simplified since we're only
considering single stock options and no index options. With this
assumption we can always use the same correlation factor.

    \begin{tcolorbox}[breakable, size=fbox, boxrule=1pt, pad at break*=1mm,colback=cellbackground, colframe=cellborder]
\prompt{In}{incolor}{ }{\boxspacing}
\begin{Verbatim}[commandchars=\\\{\}]
\PY{k}{def} \PY{n+nf}{equity\PYZus{}addOn}\PY{p}{(}\PY{n}{trades}\PY{p}{:} \PY{n}{List}\PY{p}{[}\PY{n}{Trade}\PY{p}{]}\PY{p}{,} \PY{n}{ca}\PY{p}{:} \PY{n}{CollateralAgreement}\PY{p}{)}\PY{p}{:}
    \PY{n}{bucketed\PYZus{}trades} \PY{o}{=} \PY{p}{\PYZob{}}\PY{p}{\PYZcb{}}
    \PY{n}{en\PYZus{}eq} \PY{o}{=} \PY{p}{\PYZob{}}\PY{p}{\PYZcb{}}
    \PY{n}{add\PYZus{}on\PYZus{}eq} \PY{o}{=} \PY{n}{Series}\PY{p}{(}\PY{p}{)}
    \PY{n}{equities} \PY{o}{=} \PY{n+nb}{set}\PY{p}{(}\PY{p}{)}
    
    \PY{k}{for} \PY{n}{t} \PY{o+ow}{in} \PY{n}{trades}\PY{p}{:}
        \PY{n}{key} \PY{o}{=} \PY{p}{(}\PY{n}{t}\PY{o}{.}\PY{n}{underlying}\PY{p}{)}
        \PY{n}{equities}\PY{o}{.}\PY{n}{add}\PY{p}{(}\PY{n}{t}\PY{o}{.}\PY{n}{underlying}\PY{p}{)}
        \PY{k}{if} \PY{n}{key} \PY{o+ow}{in} \PY{n}{bucketed\PYZus{}trades}\PY{p}{:}
            \PY{n}{bucketed\PYZus{}trades}\PY{p}{[}\PY{n}{key}\PY{p}{]}\PY{o}{.}\PY{n}{append}\PY{p}{(}\PY{n}{t}\PY{p}{)}
        \PY{k}{else}\PY{p}{:}
            \PY{n}{bucketed\PYZus{}trades}\PY{p}{[}\PY{n}{key}\PY{p}{]} \PY{o}{=} \PY{p}{[}\PY{n}{t}\PY{p}{]}
            
    \PY{k}{for} \PY{n}{key}\PY{p}{,} \PY{n}{trades} \PY{o+ow}{in} \PY{n}{bucketed\PYZus{}trades}\PY{o}{.}\PY{n}{items}\PY{p}{(}\PY{p}{)}\PY{p}{:}
        \PY{n}{effective\PYZus{}notional} \PY{o}{=} \PY{l+m+mi}{0}
        \PY{k}{for} \PY{n}{t} \PY{o+ow}{in} \PY{n}{trades}\PY{p}{:}
            \PY{n}{effective\PYZus{}notional} \PY{o}{+}\PY{o}{=} \PY{n}{calculate\PYZus{}sa\PYZus{}ccr\PYZus{}delta}\PY{p}{(}\PY{n}{t}\PY{p}{)} \PY{o}{*} \PY{n}{trade\PYZus{}level\PYZus{}adjusted\PYZus{}notional}\PY{p}{(}\PY{n}{t}\PY{p}{)} \PY{o}{*} \PY{n}{margining\PYZus{}factor}\PY{p}{(}\PY{n}{t}\PY{p}{,} \PY{n}{ca}\PY{p}{)}
        \PY{n}{en\PYZus{}eq}\PY{p}{[}\PY{n}{key}\PY{p}{]} \PY{o}{=} \PY{n}{effective\PYZus{}notional}
        
    \PY{k}{for} \PY{n}{eq} \PY{o+ow}{in} \PY{n}{equities}\PY{p}{:}
        
        \PY{n}{add\PYZus{}on\PYZus{}eq}\PY{p}{[}\PY{n}{eq}\PY{p}{]} \PY{o}{=} \PY{n}{get\PYZus{}supervisory\PYZus{}factor}\PY{p}{(}\PY{n}{assetClass} \PY{o}{=} \PY{n}{AssetClass}\PY{o}{.}\PY{n}{EQ}\PY{p}{,} \PY{n}{subClass} \PY{o}{=} \PY{n}{EquitySubClass}\PY{o}{.}\PY{n}{SINGLE\PYZus{}NAME}\PY{p}{)} \PY{o}{*} \PY{n}{en\PYZus{}eq}\PY{p}{[}\PY{n}{eq}\PY{p}{]}
    
    \PY{n}{add\PYZus{}on\PYZus{}aggr} \PY{o}{=} \PY{n}{sqrt}\PY{p}{(} \PY{p}{(}\PY{l+m+mf}{0.5}\PY{o}{*}\PY{n}{add\PYZus{}on\PYZus{}eq}\PY{o}{.}\PY{n}{sum}\PY{p}{(}\PY{p}{)}\PY{p}{)}\PY{o}{*}\PY{o}{*}\PY{l+m+mi}{2} \PY{o}{+} \PY{p}{(}\PY{l+m+mf}{0.75}\PY{o}{*}\PY{n}{add\PYZus{}on\PYZus{}eq}\PY{o}{*}\PY{o}{*}\PY{l+m+mi}{2}\PY{p}{)}\PY{o}{.}\PY{n}{sum}\PY{p}{(}\PY{p}{)}\PY{p}{)}
    \PY{k}{return} \PY{n}{add\PYZus{}on\PYZus{}aggr}

\PY{c+c1}{\PYZsh{} eq\PYZus{}trade\PYZus{}1 = EquityOption(notional = 20, S = 20 , K =18, m=3)}
\PY{c+c1}{\PYZsh{} eq\PYZus{}trade\PYZus{}2 = EquityOption(notional = 30, S = 18, K = 19, m=3, underlying = Stock.DBK, tradeType=TradeType.PUT)}

\PY{c+c1}{\PYZsh{} equity\PYZus{}addOn([eq\PYZus{}trade\PYZus{}1, eq\PYZus{}trade\PYZus{}2], CollateralAgreement())}
\end{Verbatim}
\end{tcolorbox}

    Equity add on calculation is validation with one example set in
\texttt{test/equity\_addOn\_validation.xlsx}. The result in the excel is
444.13 USD. Which as can be seen below is in line with the output of the
function.

    \begin{tcolorbox}[breakable, size=fbox, boxrule=1pt, pad at break*=1mm,colback=cellbackground, colframe=cellborder]
\prompt{In}{incolor}{ }{\boxspacing}
\begin{Verbatim}[commandchars=\\\{\}]
\PY{n}{eq\PYZus{}trade1} \PY{o}{=} \PY{n}{EquityOption}\PY{p}{(}\PY{n}{notional} \PY{o}{=} \PY{l+m+mi}{20}\PY{p}{,} \PY{n}{K}\PY{o}{=}\PY{l+m+mi}{91}\PY{p}{,} \PY{n}{mat\PYZus{}in\PYZus{}days}\PY{o}{=}\PY{l+m+mi}{2}\PY{o}{*}\PY{l+m+mi}{360}\PY{p}{,} \PY{n}{tradeDirection}\PY{o}{=} \PY{n}{TradeDirection}\PY{o}{.}\PY{n}{SHORT}\PY{p}{,} \PY{n}{underlying}\PY{o}{=} \PY{n}{Stock}\PY{o}{.}\PY{n}{ADS}\PY{p}{)}
\PY{n}{eq\PYZus{}trade2} \PY{o}{=} \PY{n}{EquityOption}\PY{p}{(}\PY{n}{notional} \PY{o}{=} \PY{l+m+mi}{30}\PY{p}{,} \PY{n}{K}\PY{o}{=}\PY{l+m+mi}{35}\PY{p}{,} \PY{n}{mat\PYZus{}in\PYZus{}days}\PY{o}{=}\PY{l+m+mi}{2}\PY{o}{*}\PY{l+m+mi}{360}\PY{p}{,} \PY{n}{tradeType} \PY{o}{=} \PY{n}{TradeType}\PY{o}{.}\PY{n}{PUT}\PY{p}{,} \PY{n}{tradeDirection} \PY{o}{=} \PY{n}{TradeDirection}\PY{o}{.}\PY{n}{LONG}\PY{p}{,} \PY{n}{underlying} \PY{o}{=} \PY{n}{Stock}\PY{o}{.}\PY{n}{DBK}\PY{p}{)}
\PY{n}{eq\PYZus{}trade3} \PY{o}{=} \PY{n}{EquityForward}\PY{p}{(}\PY{n}{notional} \PY{o}{=} \PY{l+m+mi}{30}\PY{p}{,} \PY{n}{m}\PY{o}{=}\PY{l+m+mf}{1.5}\PY{p}{,} \PY{n}{S}\PY{o}{=}\PY{l+m+mi}{28}\PY{p}{,} \PY{n}{tradeDirection} \PY{o}{=} \PY{n}{TradeDirection}\PY{o}{.}\PY{n}{LONG}\PY{p}{,} \PY{n}{underlying} \PY{o}{=} \PY{n}{Stock}\PY{o}{.}\PY{n}{DBK}\PY{p}{)}

\PY{n}{equity\PYZus{}addOn}\PY{p}{(}\PY{p}{[}\PY{n}{eq\PYZus{}trade1}\PY{p}{,} \PY{n}{eq\PYZus{}trade2}\PY{p}{,} \PY{n}{eq\PYZus{}trade3}\PY{p}{]}\PY{p}{,} \PY{n}{CollateralAgreement}\PY{p}{(}\PY{p}{)}\PY{p}{)}
\end{Verbatim}
\end{tcolorbox}

    Finally we are able to define a function for the PFE that consumes
trades and a collateral agreement.

    \begin{tcolorbox}[breakable, size=fbox, boxrule=1pt, pad at break*=1mm,colback=cellbackground, colframe=cellborder]
\prompt{In}{incolor}{ }{\boxspacing}
\begin{Verbatim}[commandchars=\\\{\}]
\PY{k}{def} \PY{n+nf}{calculate\PYZus{}pfe}\PY{p}{(}\PY{n}{trades}\PY{p}{:} \PY{n}{List}\PY{p}{[}\PY{n}{Trade}\PY{p}{]}\PY{p}{,} \PY{n}{ca} \PY{p}{:} \PY{n}{CollateralAgreement}\PY{p}{)}\PY{p}{:}
    \PY{n}{ac\PYZus{}bucketing} \PY{o}{=} \PY{p}{\PYZob{}}\PY{p}{\PYZcb{}}
    \PY{n}{addOns} \PY{o}{=} \PY{n}{Series}\PY{p}{(}\PY{p}{)}
    \PY{n}{V} \PY{o}{=} \PY{l+m+mi}{0}
    \PY{k}{for} \PY{n}{t} \PY{o+ow}{in} \PY{n}{trades}\PY{p}{:}
        \PY{k}{if} \PY{n}{t}\PY{o}{.}\PY{n}{assetClass} \PY{o+ow}{in} \PY{n}{ac\PYZus{}bucketing}\PY{p}{:}
            \PY{n}{ac\PYZus{}bucketing}\PY{p}{[}\PY{n}{t}\PY{o}{.}\PY{n}{assetClass}\PY{p}{]}\PY{o}{.}\PY{n}{append}\PY{p}{(}\PY{n}{t}\PY{p}{)}
        \PY{k}{else}\PY{p}{:}
            \PY{n}{ac\PYZus{}bucketing}\PY{p}{[}\PY{n}{t}\PY{o}{.}\PY{n}{assetClass}\PY{p}{]} \PY{o}{=} \PY{p}{[}\PY{n}{t}\PY{p}{]}
    \PY{k}{for} \PY{n}{ac}\PY{p}{,} \PY{n}{ac\PYZus{}trades} \PY{o+ow}{in} \PY{n}{ac\PYZus{}bucketing}\PY{o}{.}\PY{n}{items}\PY{p}{(}\PY{p}{)}\PY{p}{:}
        \PY{k}{if} \PY{n}{ac} \PY{o}{==} \PY{n}{AssetClass}\PY{o}{.}\PY{n}{EQ}\PY{p}{:}
            \PY{n}{addOns}\PY{p}{[}\PY{n}{ac}\PY{p}{]} \PY{o}{=} \PY{n}{equity\PYZus{}addOn}\PY{p}{(}\PY{n}{ac\PYZus{}trades}\PY{p}{,} \PY{n}{ca}\PY{p}{)}
        \PY{k}{elif} \PY{n}{ac} \PY{o}{==} \PY{n}{AssetClass}\PY{o}{.}\PY{n}{IR}\PY{p}{:}
            \PY{n}{addOns}\PY{p}{[}\PY{n}{ac}\PY{p}{]} \PY{o}{=} \PY{n}{interest\PYZus{}rate\PYZus{}addOn}\PY{p}{(}\PY{n}{ac\PYZus{}trades}\PY{p}{,} \PY{n}{ca}\PY{p}{)}
        \PY{k}{elif} \PY{n}{ac} \PY{o}{==} \PY{n}{AssetClass}\PY{o}{.}\PY{n}{FX}\PY{p}{:}
            \PY{n}{addOns}\PY{p}{[}\PY{n}{ac}\PY{p}{]} \PY{o}{=} \PY{n}{fx\PYZus{}addOn}\PY{p}{(}\PY{n}{ac\PYZus{}trades}\PY{p}{,} \PY{n}{ca}\PY{p}{)}

    \PY{k}{for} \PY{n}{t} \PY{o+ow}{in} \PY{n}{trades}\PY{p}{:}
        \PY{n}{V} \PY{o}{+}\PY{o}{=} \PY{n}{t}\PY{o}{.}\PY{n}{get\PYZus{}price}\PY{p}{(}\PY{p}{)}
    
    \PY{n}{C} \PY{o}{=} \PY{n}{ca}\PY{o}{.}\PY{n}{get\PYZus{}C}\PY{p}{(}\PY{p}{)}
    \PY{n}{aggregate\PYZus{}addOn} \PY{o}{=} \PY{n}{addOns}\PY{o}{.}\PY{n}{sum}\PY{p}{(}\PY{p}{)}
    \PY{n}{multiplier\PYZus{}var} \PY{o}{=} \PY{n}{multiplier}\PY{p}{(}\PY{n}{V}\PY{p}{,} \PY{n}{C}\PY{p}{,} \PY{n}{aggregate\PYZus{}addOn}\PY{p}{)}
    
    \PY{n}{PFE} \PY{o}{=} \PY{n}{multiplier\PYZus{}var}\PY{o}{*}\PY{n}{aggregate\PYZus{}addOn}
    \PY{k}{return} \PY{p}{\PYZob{}}\PY{l+s+s1}{\PYZsq{}}\PY{l+s+s1}{PFE}\PY{l+s+s1}{\PYZsq{}}\PY{p}{:} \PY{n}{PFE}\PY{p}{,}
            \PY{l+s+s1}{\PYZsq{}}\PY{l+s+s1}{multiplier}\PY{l+s+s1}{\PYZsq{}}\PY{p}{:} \PY{n}{multiplier\PYZus{}var}\PY{p}{,}
            \PY{l+s+s1}{\PYZsq{}}\PY{l+s+s1}{AddOn\PYZus{}agg}\PY{l+s+s1}{\PYZsq{}}\PY{p}{:} \PY{n}{aggregate\PYZus{}addOn}\PY{p}{\PYZcb{}}

\PY{n}{calculate\PYZus{}pfe}\PY{p}{(}\PY{p}{[}\PY{n}{eq\PYZus{}trade1}\PY{p}{]}\PY{p}{,} \PY{n}{CollateralAgreement}\PY{p}{(}\PY{p}{)}\PY{p}{)}
\PY{n}{calculate\PYZus{}pfe}\PY{p}{(}\PY{p}{[}\PY{n}{trade1}\PY{p}{,} \PY{n}{trade2}\PY{p}{,} \PY{n}{trade3}\PY{p}{]}\PY{p}{,} \PY{n}{CollateralAgreement}\PY{p}{(}\PY{p}{)}\PY{p}{)}
\end{Verbatim}
\end{tcolorbox}

    \hypertarget{definition-of-replacement-cost-rc}{%
\subsection{Definition of Replacement Cost
(RC)}\label{definition-of-replacement-cost-rc}}

The replacement cost for unmargined trades is defined as

\begin{align*}
RC &= \max\{V-C;0\} \\
\\
\text{where} \qquad V&: \text{Value of derivative transactions} \\
C&: \text{Net collateral held according to NICA methodology}
\end{align*}

\hypertarget{rc-under-margining}{%
\subsubsection{RC under margining}\label{rc-under-margining}}

To consider margining RC is adjusted to be

\begin{align*}
RC &= \max\{V-C; TH+MTA-NICA;0\} \\
\\
\text{where} \qquad TH&: \text{Threshold} \\
MTA&: \text{Minimum Transfer Amount} \\
NICA&: \text{Net Independent Collateral Amount}
\end{align*}

Here, the term \(TH+MTA-NICA\) represents the maximum exposure that does
not trigger a new variation margin call.

While threshold and the minimum transfer amount are well established
parameters of a collateral agreement (compare e.g.~Gregory - The XVA
Challenge) the net independent collateral amount is a term coined for
SA-CCR.

\hypertarget{net-independendent-margin-amount---nica}{%
\subsubsection{Net independendent margin amount -
NICA}\label{net-independendent-margin-amount---nica}}

Independent margin is margin that exceeds the current Variation Margin
of the portfolio - this could be a voluntary or accidental
overcollateralization or a mandated initial margin amount. If margin is
posted to segregated accounts NICA can only become positive.

Margin on a segregated account can surely be recovered if the
counterparty to which the margin has been pledged defaults. Therefore
this margin can not be lost through a default of the counterparty and
can only be used by the counterparty to which it has been plegded in
case of a default of the pledgor. Generally speaking VM is transferred
to non-segregated accounts while IM is transferred to segregated
accounts. It is important to note that over-collateralization or initial
margin, sometimes also referred to as \emph{independent amount} that are
deposited on an unsegregated account of the counterparty have to be
deducted from the NICA.

For bilateral initial margin calculation the regulator requires, that
initial margin is payed to segregated accounts. In the case of a
bilateral initial margin agreement the NICA would therefore be equal the
the IM received from the counterparty i.e.
\(NICA = IM_{received from Counterparty}\).

On the other hand, there can also be voluntary, one-sided initial margin
agreements in place between counterparties that differ drastically in
size and default probability. Large investment banks can for example
require hedge funds to post initial margin to an unsegregated account of
the investment bank to reduce the counterparty credit risk from the
investement banks point of view. In this case the NICA would be
calculated as
\(NICA_{\text{Investment Bank}} = IM_{\text{posted by Hedge Fund}}\)
while the NICA of the hedge fund would be calculated as
\$NICA\_\{\text{Hedge Fund}\} = - IM\_\{\text{posted by Hedge Fund}\}
\$.

If the hedge fund would instead post the IM on a segregated account the
NICA of the investment bank remains unchanged while the NICA of the
hedge fund would increase to 0.

    Finally, let's set RC up in Python:

    \begin{tcolorbox}[breakable, size=fbox, boxrule=1pt, pad at break*=1mm,colback=cellbackground, colframe=cellborder]
\prompt{In}{incolor}{ }{\boxspacing}
\begin{Verbatim}[commandchars=\\\{\}]
\PY{k}{def} \PY{n+nf}{calculate\PYZus{}rc}\PY{p}{(}\PY{n}{trades}\PY{p}{:} \PY{n}{List}\PY{p}{[}\PY{n}{Trade}\PY{p}{]}\PY{p}{,} \PY{n}{ca}\PY{p}{:} \PY{n}{CollateralAgreement}\PY{p}{)} \PY{o}{\PYZhy{}}\PY{o}{\PYZgt{}} \PY{n+nb}{float}\PY{p}{:}
    \PY{l+s+sd}{\PYZdq{}\PYZdq{}\PYZdq{}}
\PY{l+s+sd}{    }
\PY{l+s+sd}{    :param v: Current value of the derivative transactions in the netting set}
\PY{l+s+sd}{    :param c: Haircut value of the net collateral held}
\PY{l+s+sd}{    :param th: Threshold set in the collateral agreement}
\PY{l+s+sd}{    :param mta: Minimum transfer amount set in the collateral agreement}
\PY{l+s+sd}{    :param nica: Current net independent collateral amount (compare paragraph 143)}
\PY{l+s+sd}{    :return: Replacement Cost as defined in paragraph 144}
\PY{l+s+sd}{    \PYZdq{}\PYZdq{}\PYZdq{}}
    \PY{n}{v} \PY{o}{=} \PY{l+m+mi}{0}
    \PY{k}{for} \PY{n}{t} \PY{o+ow}{in} \PY{n}{trades}\PY{p}{:}
        \PY{n}{v} \PY{o}{+}\PY{o}{=} \PY{n}{t}\PY{o}{.}\PY{n}{get\PYZus{}price}\PY{p}{(}\PY{p}{)}
    \PY{n}{c} \PY{o}{=} \PY{n}{ca}\PY{o}{.}\PY{n}{get\PYZus{}C}\PY{p}{(}\PY{p}{)}
    \PY{n}{th} \PY{o}{=} \PY{n}{ca}\PY{o}{.}\PY{n}{threshold}
    \PY{n}{mta} \PY{o}{=} \PY{n}{ca}\PY{o}{.}\PY{n}{mta}
    \PY{n}{nica} \PY{o}{=} \PY{n}{ca}\PY{o}{.}\PY{n}{get\PYZus{}nica}\PY{p}{(}\PY{p}{)}
    \PY{n}{result} \PY{o}{=} \PY{n+nb}{max}\PY{p}{(}\PY{n}{v}\PY{o}{\PYZhy{}}\PY{n}{c}\PY{p}{,} \PY{n}{th}\PY{o}{+}\PY{n}{mta}\PY{o}{\PYZhy{}}\PY{n}{nica}\PY{p}{,} \PY{l+m+mi}{0}\PY{p}{)}
    \PY{k}{return} \PY{n}{result}

\PY{n}{calculate\PYZus{}rc}\PY{p}{(}\PY{p}{[}\PY{n}{trade1}\PY{p}{,} \PY{n}{trade2}\PY{p}{,} \PY{n}{trade3}\PY{p}{]}\PY{p}{,} \PY{n}{CollateralAgreement}\PY{p}{(}\PY{p}{)}\PY{p}{)}
\end{Verbatim}
\end{tcolorbox}

    With all pieces in place we are now able to calculate the EAD according
to SA-CCR given a list of trades and a collateral agreement that they
are netted and margined under.

    \begin{tcolorbox}[breakable, size=fbox, boxrule=1pt, pad at break*=1mm,colback=cellbackground, colframe=cellborder]
\prompt{In}{incolor}{ }{\boxspacing}
\begin{Verbatim}[commandchars=\\\{\}]
\PY{n}{trades} \PY{o}{=} \PY{p}{[}\PY{n}{trade1}\PY{p}{,} \PY{n}{trade2}\PY{p}{,} \PY{n}{trade3}\PY{p}{]}
\PY{n}{ca} \PY{o}{=} \PY{n}{CollateralAgreement}\PY{p}{(}\PY{p}{)}
\PY{n}{calculate\PYZus{}sa\PYZus{}ccr\PYZus{}ead}\PY{p}{(}\PY{n}{rc} \PY{o}{=} \PY{n}{calculate\PYZus{}rc}\PY{p}{(}\PY{n}{trades}\PY{p}{,} \PY{n}{ca}\PY{p}{)}\PY{p}{,} \PY{n}{pfe} \PY{o}{=} \PY{n}{calculate\PYZus{}pfe}\PY{p}{(}\PY{n}{trades}\PY{p}{,} \PY{n}{ca}\PY{p}{)}\PY{p}{[}\PY{l+s+s1}{\PYZsq{}}\PY{l+s+s1}{PFE}\PY{l+s+s1}{\PYZsq{}}\PY{p}{]}\PY{p}{)}
\end{Verbatim}
\end{tcolorbox}

    \begin{tcolorbox}[breakable, size=fbox, boxrule=1pt, pad at break*=1mm,colback=cellbackground, colframe=cellborder]
\prompt{In}{incolor}{1}{\boxspacing}
\begin{Verbatim}[commandchars=\\\{\}]
\PY{k+kn}{import} \PY{n+nn}{jupyterUtils}

\PY{n}{jupyterUtils}\PY{o}{.}\PY{n}{export}\PY{p}{(}\PY{l+s+s2}{\PYZdq{}}\PY{l+s+s2}{SA\PYZus{}CCR\PYZus{}examplary\PYZus{}calculation.ipynb}\PY{l+s+s2}{\PYZdq{}}\PY{p}{)}
\end{Verbatim}
\end{tcolorbox}

    \begin{Verbatim}[commandchars=\\\{\}]

        ---------------------------------------------------------------------------

        FileNotFoundError                         Traceback (most recent call last)

        <ipython-input-1-461b9d2a39b6> in <module>
          1 import jupyterUtils
          2 
    ----> 3 jupyterUtils.export("SA\_CCR\_examplary\_calculation.ipynb")
    

        C:\textbackslash{}Oxford\textbackslash{}Master\_Thesis\textbackslash{}Allocation\_Thesis\textbackslash{}SA\_CCR\_Allocation\textbackslash{}analysis\textbackslash{}jupyterUtils.py in export(filename)
         75 
         76     lines\_to\_write = lines\_to\_write[1:-5]
    ---> 77     with open(toBeCreatedFile, 'w+', encoding="utf8") as out\_file:
         78         out\_file.writelines(lines\_to\_write)
         79 
    

        FileNotFoundError: [Errno 2] No such file or directory: 'C:\textbackslash{}\textbackslash{}Oxford\textbackslash{}\textbackslash{}Master Thesis\textbackslash{}\textbackslash{}Allocation Thesis\textbackslash{}\textbackslash{}SA\_CCR\_Allocation\textbackslash{}\textbackslash{}LaTeX\textbackslash{}\textbackslash{}JupyterNotebooksCore\textbackslash{}\textbackslash{}SA\_CCR\_examplary\_calculation.tex'

    \end{Verbatim}


    % Add a bibliography block to the postdoc
    
    
    
\end{document}
