\documentclass[11pt]{article}

    \usepackage[breakable]{tcolorbox}
    \usepackage{parskip} % Stop auto-indenting (to mimic markdown behaviour)
    
    \usepackage{iftex}
    \ifPDFTeX
    	\usepackage[T1]{fontenc}
    	\usepackage{mathpazo}
    \else
    	\usepackage{fontspec}
    \fi

    % Basic figure setup, for now with no caption control since it's done
    % automatically by Pandoc (which extracts ![](path) syntax from Markdown).
    \usepackage{graphicx}
    % Maintain compatibility with old templates. Remove in nbconvert 6.0
    \let\Oldincludegraphics\includegraphics
    % Ensure that by default, figures have no caption (until we provide a
    % proper Figure object with a Caption API and a way to capture that
    % in the conversion process - todo).
    \usepackage{caption}
    \DeclareCaptionFormat{nocaption}{}
    \captionsetup{format=nocaption,aboveskip=0pt,belowskip=0pt}

    \usepackage[Export]{adjustbox} % Used to constrain images to a maximum size
    \adjustboxset{max size={0.9\linewidth}{0.9\paperheight}}
    \usepackage{float}
    \floatplacement{figure}{H} % forces figures to be placed at the correct location
    \usepackage{xcolor} % Allow colors to be defined
    \usepackage{enumerate} % Needed for markdown enumerations to work
    \usepackage{geometry} % Used to adjust the document margins
    \usepackage{amsmath} % Equations
    \usepackage{amssymb} % Equations
    \usepackage{textcomp} % defines textquotesingle
    % Hack from http://tex.stackexchange.com/a/47451/13684:
    \AtBeginDocument{%
        \def\PYZsq{\textquotesingle}% Upright quotes in Pygmentized code
    }
    \usepackage{upquote} % Upright quotes for verbatim code
    \usepackage{eurosym} % defines \euro
    \usepackage[mathletters]{ucs} % Extended unicode (utf-8) support
    \usepackage{fancyvrb} % verbatim replacement that allows latex
    \usepackage{grffile} % extends the file name processing of package graphics 
                         % to support a larger range
    \makeatletter % fix for grffile with XeLaTeX
    \def\Gread@@xetex#1{%
      \IfFileExists{"\Gin@base".bb}%
      {\Gread@eps{\Gin@base.bb}}%
      {\Gread@@xetex@aux#1}%
    }
    \makeatother

    % The hyperref package gives us a pdf with properly built
    % internal navigation ('pdf bookmarks' for the table of contents,
    % internal cross-reference links, web links for URLs, etc.)
    \usepackage{hyperref}
    % The default LaTeX title has an obnoxious amount of whitespace. By default,
    % titling removes some of it. It also provides customization options.
    \usepackage{titling}
    \usepackage{longtable} % longtable support required by pandoc >1.10
    \usepackage{booktabs}  % table support for pandoc > 1.12.2
    \usepackage[inline]{enumitem} % IRkernel/repr support (it uses the enumerate* environment)
    \usepackage[normalem]{ulem} % ulem is needed to support strikethroughs (\sout)
                                % normalem makes italics be italics, not underlines
    \usepackage{mathrsfs}
    

    
    % Colors for the hyperref package
    \definecolor{urlcolor}{rgb}{0,.145,.698}
    \definecolor{linkcolor}{rgb}{.71,0.21,0.01}
    \definecolor{citecolor}{rgb}{.12,.54,.11}

    % ANSI colors
    \definecolor{ansi-black}{HTML}{3E424D}
    \definecolor{ansi-black-intense}{HTML}{282C36}
    \definecolor{ansi-red}{HTML}{E75C58}
    \definecolor{ansi-red-intense}{HTML}{B22B31}
    \definecolor{ansi-green}{HTML}{00A250}
    \definecolor{ansi-green-intense}{HTML}{007427}
    \definecolor{ansi-yellow}{HTML}{DDB62B}
    \definecolor{ansi-yellow-intense}{HTML}{B27D12}
    \definecolor{ansi-blue}{HTML}{208FFB}
    \definecolor{ansi-blue-intense}{HTML}{0065CA}
    \definecolor{ansi-magenta}{HTML}{D160C4}
    \definecolor{ansi-magenta-intense}{HTML}{A03196}
    \definecolor{ansi-cyan}{HTML}{60C6C8}
    \definecolor{ansi-cyan-intense}{HTML}{258F8F}
    \definecolor{ansi-white}{HTML}{C5C1B4}
    \definecolor{ansi-white-intense}{HTML}{A1A6B2}
    \definecolor{ansi-default-inverse-fg}{HTML}{FFFFFF}
    \definecolor{ansi-default-inverse-bg}{HTML}{000000}

    % commands and environments needed by pandoc snippets
    % extracted from the output of `pandoc -s`
    \providecommand{\tightlist}{%
      \setlength{\itemsep}{0pt}\setlength{\parskip}{0pt}}
    \DefineVerbatimEnvironment{Highlighting}{Verbatim}{commandchars=\\\{\}}
    % Add ',fontsize=\small' for more characters per line
    \newenvironment{Shaded}{}{}
    \newcommand{\KeywordTok}[1]{\textcolor[rgb]{0.00,0.44,0.13}{\textbf{{#1}}}}
    \newcommand{\DataTypeTok}[1]{\textcolor[rgb]{0.56,0.13,0.00}{{#1}}}
    \newcommand{\DecValTok}[1]{\textcolor[rgb]{0.25,0.63,0.44}{{#1}}}
    \newcommand{\BaseNTok}[1]{\textcolor[rgb]{0.25,0.63,0.44}{{#1}}}
    \newcommand{\FloatTok}[1]{\textcolor[rgb]{0.25,0.63,0.44}{{#1}}}
    \newcommand{\CharTok}[1]{\textcolor[rgb]{0.25,0.44,0.63}{{#1}}}
    \newcommand{\StringTok}[1]{\textcolor[rgb]{0.25,0.44,0.63}{{#1}}}
    \newcommand{\CommentTok}[1]{\textcolor[rgb]{0.38,0.63,0.69}{\textit{{#1}}}}
    \newcommand{\OtherTok}[1]{\textcolor[rgb]{0.00,0.44,0.13}{{#1}}}
    \newcommand{\AlertTok}[1]{\textcolor[rgb]{1.00,0.00,0.00}{\textbf{{#1}}}}
    \newcommand{\FunctionTok}[1]{\textcolor[rgb]{0.02,0.16,0.49}{{#1}}}
    \newcommand{\RegionMarkerTok}[1]{{#1}}
    \newcommand{\ErrorTok}[1]{\textcolor[rgb]{1.00,0.00,0.00}{\textbf{{#1}}}}
    \newcommand{\NormalTok}[1]{{#1}}
    
    % Additional commands for more recent versions of Pandoc
    \newcommand{\ConstantTok}[1]{\textcolor[rgb]{0.53,0.00,0.00}{{#1}}}
    \newcommand{\SpecialCharTok}[1]{\textcolor[rgb]{0.25,0.44,0.63}{{#1}}}
    \newcommand{\VerbatimStringTok}[1]{\textcolor[rgb]{0.25,0.44,0.63}{{#1}}}
    \newcommand{\SpecialStringTok}[1]{\textcolor[rgb]{0.73,0.40,0.53}{{#1}}}
    \newcommand{\ImportTok}[1]{{#1}}
    \newcommand{\DocumentationTok}[1]{\textcolor[rgb]{0.73,0.13,0.13}{\textit{{#1}}}}
    \newcommand{\AnnotationTok}[1]{\textcolor[rgb]{0.38,0.63,0.69}{\textbf{\textit{{#1}}}}}
    \newcommand{\CommentVarTok}[1]{\textcolor[rgb]{0.38,0.63,0.69}{\textbf{\textit{{#1}}}}}
    \newcommand{\VariableTok}[1]{\textcolor[rgb]{0.10,0.09,0.49}{{#1}}}
    \newcommand{\ControlFlowTok}[1]{\textcolor[rgb]{0.00,0.44,0.13}{\textbf{{#1}}}}
    \newcommand{\OperatorTok}[1]{\textcolor[rgb]{0.40,0.40,0.40}{{#1}}}
    \newcommand{\BuiltInTok}[1]{{#1}}
    \newcommand{\ExtensionTok}[1]{{#1}}
    \newcommand{\PreprocessorTok}[1]{\textcolor[rgb]{0.74,0.48,0.00}{{#1}}}
    \newcommand{\AttributeTok}[1]{\textcolor[rgb]{0.49,0.56,0.16}{{#1}}}
    \newcommand{\InformationTok}[1]{\textcolor[rgb]{0.38,0.63,0.69}{\textbf{\textit{{#1}}}}}
    \newcommand{\WarningTok}[1]{\textcolor[rgb]{0.38,0.63,0.69}{\textbf{\textit{{#1}}}}}
    
    
    % Define a nice break command that doesn't care if a line doesn't already
    % exist.
    \def\br{\hspace*{\fill} \\* }
    % Math Jax compatibility definitions
    \def\gt{>}
    \def\lt{<}
    \let\Oldtex\TeX
    \let\Oldlatex\LaTeX
    \renewcommand{\TeX}{\textrm{\Oldtex}}
    \renewcommand{\LaTeX}{\textrm{\Oldlatex}}
    % Document parameters
    % Document title
    \title{SA-CCR Euler allocation of an exemplary equity portfolio}
    
    
    
    
    
% Pygments definitions
\makeatletter
\def\PY@reset{\let\PY@it=\relax \let\PY@bf=\relax%
    \let\PY@ul=\relax \let\PY@tc=\relax%
    \let\PY@bc=\relax \let\PY@ff=\relax}
\def\PY@tok#1{\csname PY@tok@#1\endcsname}
\def\PY@toks#1+{\ifx\relax#1\empty\else%
    \PY@tok{#1}\expandafter\PY@toks\fi}
\def\PY@do#1{\PY@bc{\PY@tc{\PY@ul{%
    \PY@it{\PY@bf{\PY@ff{#1}}}}}}}
\def\PY#1#2{\PY@reset\PY@toks#1+\relax+\PY@do{#2}}

\expandafter\def\csname PY@tok@w\endcsname{\def\PY@tc##1{\textcolor[rgb]{0.73,0.73,0.73}{##1}}}
\expandafter\def\csname PY@tok@c\endcsname{\let\PY@it=\textit\def\PY@tc##1{\textcolor[rgb]{0.25,0.50,0.50}{##1}}}
\expandafter\def\csname PY@tok@cp\endcsname{\def\PY@tc##1{\textcolor[rgb]{0.74,0.48,0.00}{##1}}}
\expandafter\def\csname PY@tok@k\endcsname{\let\PY@bf=\textbf\def\PY@tc##1{\textcolor[rgb]{0.00,0.50,0.00}{##1}}}
\expandafter\def\csname PY@tok@kp\endcsname{\def\PY@tc##1{\textcolor[rgb]{0.00,0.50,0.00}{##1}}}
\expandafter\def\csname PY@tok@kt\endcsname{\def\PY@tc##1{\textcolor[rgb]{0.69,0.00,0.25}{##1}}}
\expandafter\def\csname PY@tok@o\endcsname{\def\PY@tc##1{\textcolor[rgb]{0.40,0.40,0.40}{##1}}}
\expandafter\def\csname PY@tok@ow\endcsname{\let\PY@bf=\textbf\def\PY@tc##1{\textcolor[rgb]{0.67,0.13,1.00}{##1}}}
\expandafter\def\csname PY@tok@nb\endcsname{\def\PY@tc##1{\textcolor[rgb]{0.00,0.50,0.00}{##1}}}
\expandafter\def\csname PY@tok@nf\endcsname{\def\PY@tc##1{\textcolor[rgb]{0.00,0.00,1.00}{##1}}}
\expandafter\def\csname PY@tok@nc\endcsname{\let\PY@bf=\textbf\def\PY@tc##1{\textcolor[rgb]{0.00,0.00,1.00}{##1}}}
\expandafter\def\csname PY@tok@nn\endcsname{\let\PY@bf=\textbf\def\PY@tc##1{\textcolor[rgb]{0.00,0.00,1.00}{##1}}}
\expandafter\def\csname PY@tok@ne\endcsname{\let\PY@bf=\textbf\def\PY@tc##1{\textcolor[rgb]{0.82,0.25,0.23}{##1}}}
\expandafter\def\csname PY@tok@nv\endcsname{\def\PY@tc##1{\textcolor[rgb]{0.10,0.09,0.49}{##1}}}
\expandafter\def\csname PY@tok@no\endcsname{\def\PY@tc##1{\textcolor[rgb]{0.53,0.00,0.00}{##1}}}
\expandafter\def\csname PY@tok@nl\endcsname{\def\PY@tc##1{\textcolor[rgb]{0.63,0.63,0.00}{##1}}}
\expandafter\def\csname PY@tok@ni\endcsname{\let\PY@bf=\textbf\def\PY@tc##1{\textcolor[rgb]{0.60,0.60,0.60}{##1}}}
\expandafter\def\csname PY@tok@na\endcsname{\def\PY@tc##1{\textcolor[rgb]{0.49,0.56,0.16}{##1}}}
\expandafter\def\csname PY@tok@nt\endcsname{\let\PY@bf=\textbf\def\PY@tc##1{\textcolor[rgb]{0.00,0.50,0.00}{##1}}}
\expandafter\def\csname PY@tok@nd\endcsname{\def\PY@tc##1{\textcolor[rgb]{0.67,0.13,1.00}{##1}}}
\expandafter\def\csname PY@tok@s\endcsname{\def\PY@tc##1{\textcolor[rgb]{0.73,0.13,0.13}{##1}}}
\expandafter\def\csname PY@tok@sd\endcsname{\let\PY@it=\textit\def\PY@tc##1{\textcolor[rgb]{0.73,0.13,0.13}{##1}}}
\expandafter\def\csname PY@tok@si\endcsname{\let\PY@bf=\textbf\def\PY@tc##1{\textcolor[rgb]{0.73,0.40,0.53}{##1}}}
\expandafter\def\csname PY@tok@se\endcsname{\let\PY@bf=\textbf\def\PY@tc##1{\textcolor[rgb]{0.73,0.40,0.13}{##1}}}
\expandafter\def\csname PY@tok@sr\endcsname{\def\PY@tc##1{\textcolor[rgb]{0.73,0.40,0.53}{##1}}}
\expandafter\def\csname PY@tok@ss\endcsname{\def\PY@tc##1{\textcolor[rgb]{0.10,0.09,0.49}{##1}}}
\expandafter\def\csname PY@tok@sx\endcsname{\def\PY@tc##1{\textcolor[rgb]{0.00,0.50,0.00}{##1}}}
\expandafter\def\csname PY@tok@m\endcsname{\def\PY@tc##1{\textcolor[rgb]{0.40,0.40,0.40}{##1}}}
\expandafter\def\csname PY@tok@gh\endcsname{\let\PY@bf=\textbf\def\PY@tc##1{\textcolor[rgb]{0.00,0.00,0.50}{##1}}}
\expandafter\def\csname PY@tok@gu\endcsname{\let\PY@bf=\textbf\def\PY@tc##1{\textcolor[rgb]{0.50,0.00,0.50}{##1}}}
\expandafter\def\csname PY@tok@gd\endcsname{\def\PY@tc##1{\textcolor[rgb]{0.63,0.00,0.00}{##1}}}
\expandafter\def\csname PY@tok@gi\endcsname{\def\PY@tc##1{\textcolor[rgb]{0.00,0.63,0.00}{##1}}}
\expandafter\def\csname PY@tok@gr\endcsname{\def\PY@tc##1{\textcolor[rgb]{1.00,0.00,0.00}{##1}}}
\expandafter\def\csname PY@tok@ge\endcsname{\let\PY@it=\textit}
\expandafter\def\csname PY@tok@gs\endcsname{\let\PY@bf=\textbf}
\expandafter\def\csname PY@tok@gp\endcsname{\let\PY@bf=\textbf\def\PY@tc##1{\textcolor[rgb]{0.00,0.00,0.50}{##1}}}
\expandafter\def\csname PY@tok@go\endcsname{\def\PY@tc##1{\textcolor[rgb]{0.53,0.53,0.53}{##1}}}
\expandafter\def\csname PY@tok@gt\endcsname{\def\PY@tc##1{\textcolor[rgb]{0.00,0.27,0.87}{##1}}}
\expandafter\def\csname PY@tok@err\endcsname{\def\PY@bc##1{\setlength{\fboxsep}{0pt}\fcolorbox[rgb]{1.00,0.00,0.00}{1,1,1}{\strut ##1}}}
\expandafter\def\csname PY@tok@kc\endcsname{\let\PY@bf=\textbf\def\PY@tc##1{\textcolor[rgb]{0.00,0.50,0.00}{##1}}}
\expandafter\def\csname PY@tok@kd\endcsname{\let\PY@bf=\textbf\def\PY@tc##1{\textcolor[rgb]{0.00,0.50,0.00}{##1}}}
\expandafter\def\csname PY@tok@kn\endcsname{\let\PY@bf=\textbf\def\PY@tc##1{\textcolor[rgb]{0.00,0.50,0.00}{##1}}}
\expandafter\def\csname PY@tok@kr\endcsname{\let\PY@bf=\textbf\def\PY@tc##1{\textcolor[rgb]{0.00,0.50,0.00}{##1}}}
\expandafter\def\csname PY@tok@bp\endcsname{\def\PY@tc##1{\textcolor[rgb]{0.00,0.50,0.00}{##1}}}
\expandafter\def\csname PY@tok@fm\endcsname{\def\PY@tc##1{\textcolor[rgb]{0.00,0.00,1.00}{##1}}}
\expandafter\def\csname PY@tok@vc\endcsname{\def\PY@tc##1{\textcolor[rgb]{0.10,0.09,0.49}{##1}}}
\expandafter\def\csname PY@tok@vg\endcsname{\def\PY@tc##1{\textcolor[rgb]{0.10,0.09,0.49}{##1}}}
\expandafter\def\csname PY@tok@vi\endcsname{\def\PY@tc##1{\textcolor[rgb]{0.10,0.09,0.49}{##1}}}
\expandafter\def\csname PY@tok@vm\endcsname{\def\PY@tc##1{\textcolor[rgb]{0.10,0.09,0.49}{##1}}}
\expandafter\def\csname PY@tok@sa\endcsname{\def\PY@tc##1{\textcolor[rgb]{0.73,0.13,0.13}{##1}}}
\expandafter\def\csname PY@tok@sb\endcsname{\def\PY@tc##1{\textcolor[rgb]{0.73,0.13,0.13}{##1}}}
\expandafter\def\csname PY@tok@sc\endcsname{\def\PY@tc##1{\textcolor[rgb]{0.73,0.13,0.13}{##1}}}
\expandafter\def\csname PY@tok@dl\endcsname{\def\PY@tc##1{\textcolor[rgb]{0.73,0.13,0.13}{##1}}}
\expandafter\def\csname PY@tok@s2\endcsname{\def\PY@tc##1{\textcolor[rgb]{0.73,0.13,0.13}{##1}}}
\expandafter\def\csname PY@tok@sh\endcsname{\def\PY@tc##1{\textcolor[rgb]{0.73,0.13,0.13}{##1}}}
\expandafter\def\csname PY@tok@s1\endcsname{\def\PY@tc##1{\textcolor[rgb]{0.73,0.13,0.13}{##1}}}
\expandafter\def\csname PY@tok@mb\endcsname{\def\PY@tc##1{\textcolor[rgb]{0.40,0.40,0.40}{##1}}}
\expandafter\def\csname PY@tok@mf\endcsname{\def\PY@tc##1{\textcolor[rgb]{0.40,0.40,0.40}{##1}}}
\expandafter\def\csname PY@tok@mh\endcsname{\def\PY@tc##1{\textcolor[rgb]{0.40,0.40,0.40}{##1}}}
\expandafter\def\csname PY@tok@mi\endcsname{\def\PY@tc##1{\textcolor[rgb]{0.40,0.40,0.40}{##1}}}
\expandafter\def\csname PY@tok@il\endcsname{\def\PY@tc##1{\textcolor[rgb]{0.40,0.40,0.40}{##1}}}
\expandafter\def\csname PY@tok@mo\endcsname{\def\PY@tc##1{\textcolor[rgb]{0.40,0.40,0.40}{##1}}}
\expandafter\def\csname PY@tok@ch\endcsname{\let\PY@it=\textit\def\PY@tc##1{\textcolor[rgb]{0.25,0.50,0.50}{##1}}}
\expandafter\def\csname PY@tok@cm\endcsname{\let\PY@it=\textit\def\PY@tc##1{\textcolor[rgb]{0.25,0.50,0.50}{##1}}}
\expandafter\def\csname PY@tok@cpf\endcsname{\let\PY@it=\textit\def\PY@tc##1{\textcolor[rgb]{0.25,0.50,0.50}{##1}}}
\expandafter\def\csname PY@tok@c1\endcsname{\let\PY@it=\textit\def\PY@tc##1{\textcolor[rgb]{0.25,0.50,0.50}{##1}}}
\expandafter\def\csname PY@tok@cs\endcsname{\let\PY@it=\textit\def\PY@tc##1{\textcolor[rgb]{0.25,0.50,0.50}{##1}}}

\def\PYZbs{\char`\\}
\def\PYZus{\char`\_}
\def\PYZob{\char`\{}
\def\PYZcb{\char`\}}
\def\PYZca{\char`\^}
\def\PYZam{\char`\&}
\def\PYZlt{\char`\<}
\def\PYZgt{\char`\>}
\def\PYZsh{\char`\#}
\def\PYZpc{\char`\%}
\def\PYZdl{\char`\$}
\def\PYZhy{\char`\-}
\def\PYZsq{\char`\'}
\def\PYZdq{\char`\"}
\def\PYZti{\char`\~}
% for compatibility with earlier versions
\def\PYZat{@}
\def\PYZlb{[}
\def\PYZrb{]}
\makeatother


    % For linebreaks inside Verbatim environment from package fancyvrb. 
    \makeatletter
        \newbox\Wrappedcontinuationbox 
        \newbox\Wrappedvisiblespacebox 
        \newcommand*\Wrappedvisiblespace {\textcolor{red}{\textvisiblespace}} 
        \newcommand*\Wrappedcontinuationsymbol {\textcolor{red}{\llap{\tiny$\m@th\hookrightarrow$}}} 
        \newcommand*\Wrappedcontinuationindent {3ex } 
        \newcommand*\Wrappedafterbreak {\kern\Wrappedcontinuationindent\copy\Wrappedcontinuationbox} 
        % Take advantage of the already applied Pygments mark-up to insert 
        % potential linebreaks for TeX processing. 
        %        {, <, #, %, $, ' and ": go to next line. 
        %        _, }, ^, &, >, - and ~: stay at end of broken line. 
        % Use of \textquotesingle for straight quote. 
        \newcommand*\Wrappedbreaksatspecials {% 
            \def\PYGZus{\discretionary{\char`\_}{\Wrappedafterbreak}{\char`\_}}% 
            \def\PYGZob{\discretionary{}{\Wrappedafterbreak\char`\{}{\char`\{}}% 
            \def\PYGZcb{\discretionary{\char`\}}{\Wrappedafterbreak}{\char`\}}}% 
            \def\PYGZca{\discretionary{\char`\^}{\Wrappedafterbreak}{\char`\^}}% 
            \def\PYGZam{\discretionary{\char`\&}{\Wrappedafterbreak}{\char`\&}}% 
            \def\PYGZlt{\discretionary{}{\Wrappedafterbreak\char`\<}{\char`\<}}% 
            \def\PYGZgt{\discretionary{\char`\>}{\Wrappedafterbreak}{\char`\>}}% 
            \def\PYGZsh{\discretionary{}{\Wrappedafterbreak\char`\#}{\char`\#}}% 
            \def\PYGZpc{\discretionary{}{\Wrappedafterbreak\char`\%}{\char`\%}}% 
            \def\PYGZdl{\discretionary{}{\Wrappedafterbreak\char`\$}{\char`\$}}% 
            \def\PYGZhy{\discretionary{\char`\-}{\Wrappedafterbreak}{\char`\-}}% 
            \def\PYGZsq{\discretionary{}{\Wrappedafterbreak\textquotesingle}{\textquotesingle}}% 
            \def\PYGZdq{\discretionary{}{\Wrappedafterbreak\char`\"}{\char`\"}}% 
            \def\PYGZti{\discretionary{\char`\~}{\Wrappedafterbreak}{\char`\~}}% 
        } 
        % Some characters . , ; ? ! / are not pygmentized. 
        % This macro makes them "active" and they will insert potential linebreaks 
        \newcommand*\Wrappedbreaksatpunct {% 
            \lccode`\~`\.\lowercase{\def~}{\discretionary{\hbox{\char`\.}}{\Wrappedafterbreak}{\hbox{\char`\.}}}% 
            \lccode`\~`\,\lowercase{\def~}{\discretionary{\hbox{\char`\,}}{\Wrappedafterbreak}{\hbox{\char`\,}}}% 
            \lccode`\~`\;\lowercase{\def~}{\discretionary{\hbox{\char`\;}}{\Wrappedafterbreak}{\hbox{\char`\;}}}% 
            \lccode`\~`\:\lowercase{\def~}{\discretionary{\hbox{\char`\:}}{\Wrappedafterbreak}{\hbox{\char`\:}}}% 
            \lccode`\~`\?\lowercase{\def~}{\discretionary{\hbox{\char`\?}}{\Wrappedafterbreak}{\hbox{\char`\?}}}% 
            \lccode`\~`\!\lowercase{\def~}{\discretionary{\hbox{\char`\!}}{\Wrappedafterbreak}{\hbox{\char`\!}}}% 
            \lccode`\~`\/\lowercase{\def~}{\discretionary{\hbox{\char`\/}}{\Wrappedafterbreak}{\hbox{\char`\/}}}% 
            \catcode`\.\active
            \catcode`\,\active 
            \catcode`\;\active
            \catcode`\:\active
            \catcode`\?\active
            \catcode`\!\active
            \catcode`\/\active 
            \lccode`\~`\~ 	
        }
    \makeatother

    \let\OriginalVerbatim=\Verbatim
    \makeatletter
    \renewcommand{\Verbatim}[1][1]{%
        %\parskip\z@skip
        \sbox\Wrappedcontinuationbox {\Wrappedcontinuationsymbol}%
        \sbox\Wrappedvisiblespacebox {\FV@SetupFont\Wrappedvisiblespace}%
        \def\FancyVerbFormatLine ##1{\hsize\linewidth
            \vtop{\raggedright\hyphenpenalty\z@\exhyphenpenalty\z@
                \doublehyphendemerits\z@\finalhyphendemerits\z@
                \strut ##1\strut}%
        }%
        % If the linebreak is at a space, the latter will be displayed as visible
        % space at end of first line, and a continuation symbol starts next line.
        % Stretch/shrink are however usually zero for typewriter font.
        \def\FV@Space {%
            \nobreak\hskip\z@ plus\fontdimen3\font minus\fontdimen4\font
            \discretionary{\copy\Wrappedvisiblespacebox}{\Wrappedafterbreak}
            {\kern\fontdimen2\font}%
        }%
        
        % Allow breaks at special characters using \PYG... macros.
        \Wrappedbreaksatspecials
        % Breaks at punctuation characters . , ; ? ! and / need catcode=\active 	
        \OriginalVerbatim[#1,codes*=\Wrappedbreaksatpunct]%
    }
    \makeatother

    % Exact colors from NB
    \definecolor{incolor}{HTML}{303F9F}
    \definecolor{outcolor}{HTML}{D84315}
    \definecolor{cellborder}{HTML}{CFCFCF}
    \definecolor{cellbackground}{HTML}{F7F7F7}
    
    % prompt
    \makeatletter
    \newcommand{\boxspacing}{\kern\kvtcb@left@rule\kern\kvtcb@boxsep}
    \makeatother
    \newcommand{\prompt}[4]{
        \ttfamily\llap{{\color{#2}[#3]:\hspace{3pt}#4}}\vspace{-\baselineskip}
    }
    

    
    % Prevent overflowing lines due to hard-to-break entities
    \sloppy 
    % Setup hyperref package
    \hypersetup{
      breaklinks=true,  % so long urls are correctly broken across lines
      colorlinks=true,
      urlcolor=urlcolor,
      linkcolor=linkcolor,
      citecolor=citecolor,
      }
    % Slightly bigger margins than the latex defaults
    
    \geometry{verbose,tmargin=1in,bmargin=1in,lmargin=1in,rmargin=1in}
    
    

\begin{document}
    
    \maketitle
    
    

    
    \hypertarget{sa-ccr-euler-allocation-of-an-exemplary-equity-portfolio}{%
\section{SA-CCR Euler allocation of an exemplary equity
portfolio}\label{sa-ccr-euler-allocation-of-an-exemplary-equity-portfolio}}

We set up a collateral agreement with no thresholds or MTA and add two
equity options to it. A big put option on an imagainary stock ADS and a
smaller call option on the same stock. Both options are struck at the
current market price \(S(0)\).

We consider three cases of margining

\begin{enumerate}
\def\labelenumi{\arabic{enumi}.}
\tightlist
\item
  No margining
\item
  VM only
\item
  VM and bilateral IM
\end{enumerate}

    \begin{tcolorbox}[breakable, size=fbox, boxrule=1pt, pad at break*=1mm,colback=cellbackground, colframe=cellborder]
\prompt{In}{incolor}{2}{\boxspacing}
\begin{Verbatim}[commandchars=\\\{\}]
\PY{n}{ca} \PY{o}{=} \PY{n}{CollateralAgreement}\PY{p}{(}\PY{n}{threshold}\PY{o}{=}\PY{l+m+mi}{0}\PY{p}{,}
                         \PY{n}{threshold\PYZus{}vm}\PY{o}{=}\PY{l+m+mi}{0}\PY{p}{,}
                         \PY{n}{mta}\PY{o}{=}\PY{l+m+mi}{0}\PY{p}{,}
                         \PY{n}{margining}\PY{o}{=}\PY{n}{Margining}\PY{o}{.}\PY{n}{UNMARGINED}\PY{p}{,}
                         \PY{n}{initialMargining}\PY{o}{=}\PY{n}{InitialMargining}\PY{o}{.}\PY{n}{NO\PYZus{}IM}\PY{p}{)}
\PY{n}{ca}\PY{o}{.}\PY{n}{link\PYZus{}sa\PYZus{}ccr\PYZus{}instance}\PY{p}{(}\PY{n}{SA\PYZus{}CCR}\PY{p}{(}\PY{n}{ca}\PY{p}{)}\PY{p}{)}

\PY{n}{ca\PYZus{}vm} \PY{o}{=} \PY{n}{CollateralAgreement}\PY{p}{(}\PY{n}{threshold}\PY{o}{=}\PY{l+m+mi}{0}\PY{p}{,}
                            \PY{n}{threshold\PYZus{}vm}\PY{o}{=}\PY{l+m+mi}{0}\PY{p}{,}
                            \PY{n}{mta}\PY{o}{=}\PY{l+m+mi}{0}\PY{p}{,}
                            \PY{n}{margining}\PY{o}{=}\PY{n}{Margining}\PY{o}{.}\PY{n}{MARGINED}\PY{p}{,}
                            \PY{n}{initialMargining}\PY{o}{=}\PY{n}{InitialMargining}\PY{o}{.}\PY{n}{NO\PYZus{}IM}\PY{p}{)}
\PY{n}{ca\PYZus{}vm}\PY{o}{.}\PY{n}{link\PYZus{}sa\PYZus{}ccr\PYZus{}instance}\PY{p}{(}\PY{n}{SA\PYZus{}CCR}\PY{p}{(}\PY{n}{ca\PYZus{}vm}\PY{p}{)}\PY{p}{)}

\PY{n}{ca\PYZus{}im} \PY{o}{=} \PY{n}{CollateralAgreement}\PY{p}{(}\PY{n}{threshold}\PY{o}{=}\PY{l+m+mi}{0}\PY{p}{,}
                            \PY{n}{threshold\PYZus{}vm}\PY{o}{=}\PY{l+m+mi}{0}\PY{p}{,}
                            \PY{n}{mta}\PY{o}{=}\PY{l+m+mi}{0}\PY{p}{,}
                            \PY{n}{margining}\PY{o}{=}\PY{n}{Margining}\PY{o}{.}\PY{n}{MARGINED}\PY{p}{,}
                            \PY{n}{initialMargining}\PY{o}{=}\PY{n}{InitialMargining}\PY{o}{.}\PY{n}{SIMM}\PY{p}{)}
\PY{n}{ca\PYZus{}im}\PY{o}{.}\PY{n}{link\PYZus{}sa\PYZus{}ccr\PYZus{}instance}\PY{p}{(}\PY{n}{SA\PYZus{}CCR}\PY{p}{(}\PY{n}{ca\PYZus{}im}\PY{p}{)}\PY{p}{)}
\end{Verbatim}
\end{tcolorbox}

    \begin{tcolorbox}[breakable, size=fbox, boxrule=1pt, pad at break*=1mm,colback=cellbackground, colframe=cellborder]
\prompt{In}{incolor}{3}{\boxspacing}
\begin{Verbatim}[commandchars=\\\{\}]
\PY{n}{eq\PYZus{}opt\PYZus{}ads\PYZus{}call} \PY{o}{=} \PY{n}{EquityOption}\PY{p}{(}\PY{n}{underlying}\PY{o}{=}\PY{n}{Stock}\PY{o}{.}\PY{n}{ADS}\PY{p}{,}
                               \PY{n}{maturity}\PY{o}{=}\PY{n}{ql}\PY{o}{.}\PY{n}{Period}\PY{p}{(}\PY{l+m+mi}{1}\PY{p}{,} \PY{n}{ql}\PY{o}{.}\PY{n}{Years}\PY{p}{)}\PY{p}{,}
                               \PY{n}{notional}\PY{o}{=}\PY{l+m+mi}{2000000}\PY{p}{,}
                               \PY{n}{tradeType}\PY{o}{=}\PY{n}{TradeType}\PY{o}{.}\PY{n}{CALL}\PY{p}{,}
                               \PY{n}{tradeDirection}\PY{o}{=}\PY{n}{TradeDirection}\PY{o}{.}\PY{n}{LONG}\PY{p}{)}

\PY{n}{eq\PYZus{}opt\PYZus{}ads\PYZus{}put} \PY{o}{=} \PY{n}{EquityOption}\PY{p}{(}\PY{n}{underlying}\PY{o}{=}\PY{n}{Stock}\PY{o}{.}\PY{n}{ADS}\PY{p}{,}
                              \PY{n}{maturity}\PY{o}{=}\PY{n}{ql}\PY{o}{.}\PY{n}{Period}\PY{p}{(}\PY{l+m+mi}{1}\PY{p}{,} \PY{n}{ql}\PY{o}{.}\PY{n}{Years}\PY{p}{)}\PY{p}{,}
                              \PY{n}{notional}\PY{o}{=}\PY{l+m+mi}{3000000}\PY{p}{,}
                              \PY{n}{tradeType}\PY{o}{=}\PY{n}{TradeType}\PY{o}{.}\PY{n}{PUT}\PY{p}{,}
                              \PY{n}{tradeDirection}\PY{o}{=}\PY{n}{TradeDirection}\PY{o}{.}\PY{n}{LONG}\PY{p}{)}
\end{Verbatim}
\end{tcolorbox}

    The current value of these trades is 6,601,467 EUR for the call option
and 10,378,881 EUR for the put option.

    
    When putting these trades in the three portfolios we observe a high
difference between the unmargined and VM margined EAD. This difference
is primarily driven through the RC and therefore through the high
positive PV.

    \begin{tcolorbox}[breakable, size=fbox, boxrule=1pt, pad at break*=1mm,colback=cellbackground, colframe=cellborder]
\prompt{In}{incolor}{5}{\boxspacing}
\begin{Verbatim}[commandchars=\\\{\}]
\PY{n}{ca}\PY{o}{.}\PY{n}{add\PYZus{}trades}\PY{p}{(}\PY{p}{[}\PY{n}{eq\PYZus{}opt\PYZus{}ads\PYZus{}call}\PY{p}{,} \PY{n}{eq\PYZus{}opt\PYZus{}ads\PYZus{}put}\PY{p}{]}\PY{p}{)}
\PY{n}{ca\PYZus{}vm}\PY{o}{.}\PY{n}{add\PYZus{}trades}\PY{p}{(}\PY{p}{[}\PY{n}{eq\PYZus{}opt\PYZus{}ads\PYZus{}call}\PY{p}{,} \PY{n}{eq\PYZus{}opt\PYZus{}ads\PYZus{}put}\PY{p}{]}\PY{p}{)}
\PY{n}{ca\PYZus{}im}\PY{o}{.}\PY{n}{add\PYZus{}trades}\PY{p}{(}\PY{p}{[}\PY{n}{eq\PYZus{}opt\PYZus{}ads\PYZus{}call}\PY{p}{,}\PY{n}{eq\PYZus{}opt\PYZus{}ads\PYZus{}put}\PY{p}{]}\PY{p}{)}
\end{Verbatim}
\end{tcolorbox}

    \begin{tcolorbox}[breakable, size=fbox, boxrule=1pt, pad at break*=1mm,colback=cellbackground, colframe=cellborder]
\prompt{In}{incolor}{6}{\boxspacing}
\begin{Verbatim}[commandchars=\\\{\}]
\PY{n+nb}{print}\PY{p}{(}\PY{l+s+s1}{\PYZsq{}}\PY{l+s+s1}{EAD with no margining: }\PY{l+s+si}{\PYZob{}:,.2f\PYZcb{}}\PY{l+s+s1}{ USD}\PY{l+s+s1}{\PYZsq{}}\PY{o}{.}\PY{n}{format}\PY{p}{(}\PY{n}{ca}\PY{o}{.}\PY{n}{get\PYZus{}sa\PYZus{}ccr\PYZus{}model}\PY{p}{(}\PY{p}{)}\PY{o}{.}\PY{n}{get\PYZus{}ead}\PY{p}{(}\PY{p}{)}\PY{p}{)}\PY{p}{)}
\PY{n+nb}{print}\PY{p}{(}\PY{l+s+s1}{\PYZsq{}}\PY{l+s+s1}{EAD with VM margining: }\PY{l+s+si}{\PYZob{}:,.2f\PYZcb{}}\PY{l+s+s1}{ USD}\PY{l+s+s1}{\PYZsq{}}\PY{o}{.}\PY{n}{format}\PY{p}{(}\PY{n}{ca\PYZus{}vm}\PY{o}{.}\PY{n}{get\PYZus{}sa\PYZus{}ccr\PYZus{}model}\PY{p}{(}\PY{p}{)}\PY{o}{.}\PY{n}{get\PYZus{}ead}\PY{p}{(}\PY{p}{)}\PY{p}{)}\PY{p}{)}
\PY{n+nb}{print}\PY{p}{(}\PY{l+s+s1}{\PYZsq{}}\PY{l+s+s1}{RC with no margining: }\PY{l+s+si}{\PYZob{}:,.2f\PYZcb{}}\PY{l+s+s1}{ USD}\PY{l+s+s1}{\PYZsq{}}\PY{o}{.}\PY{n}{format}\PY{p}{(}\PY{n}{ca}\PY{o}{.}\PY{n}{get\PYZus{}sa\PYZus{}ccr\PYZus{}model}\PY{p}{(}\PY{p}{)}\PY{o}{.}\PY{n}{get\PYZus{}rc}\PY{p}{(}\PY{p}{)}\PY{p}{)}\PY{p}{)}
\PY{n+nb}{print}\PY{p}{(}\PY{l+s+s1}{\PYZsq{}}\PY{l+s+s1}{RC with VM margining: }\PY{l+s+si}{\PYZob{}:,.2f\PYZcb{}}\PY{l+s+s1}{ USD}\PY{l+s+s1}{\PYZsq{}}\PY{o}{.}\PY{n}{format}\PY{p}{(}\PY{n}{ca\PYZus{}vm}\PY{o}{.}\PY{n}{get\PYZus{}sa\PYZus{}ccr\PYZus{}model}\PY{p}{(}\PY{p}{)}\PY{o}{.}\PY{n}{get\PYZus{}rc}\PY{p}{(}\PY{p}{)}\PY{p}{)}\PY{p}{)}
\PY{n+nb}{print}\PY{p}{(}\PY{l+s+s1}{\PYZsq{}}\PY{l+s+s1}{Portfolio PV: }\PY{l+s+si}{\PYZob{}:,.2f\PYZcb{}}\PY{l+s+s1}{ USD}\PY{l+s+s1}{\PYZsq{}}\PY{o}{.}\PY{n}{format}\PY{p}{(}\PY{n}{ca}\PY{o}{.}\PY{n}{get\PYZus{}V}\PY{p}{(}\PY{p}{)}\PY{p}{)}\PY{p}{)}
\end{Verbatim}
\end{tcolorbox}

    \begin{Verbatim}[commandchars=\\\{\}]
EAD with no margining: 37,643,536.02 USD
EAD with VM margining: 3,519,457.62 USD
RC with no margining: 18,508,579.01 USD
RC with VM margining: 0.00 USD
Portfolio PV: 18,508,579.01 USD
    \end{Verbatim}

    Next, we perform an Euler allocation:

    \begin{tcolorbox}[breakable, size=fbox, boxrule=1pt, pad at break*=1mm,colback=cellbackground, colframe=cellborder]
\prompt{In}{incolor}{7}{\boxspacing}
\begin{Verbatim}[commandchars=\\\{\}]
\PY{n}{eulerAllocator1} \PY{o}{=} \PY{n}{EulerAllocator}\PY{p}{(}\PY{n}{ca}\PY{p}{)}
\PY{n}{eulerAllocator2} \PY{o}{=} \PY{n}{EulerAllocator}\PY{p}{(}\PY{n}{ca\PYZus{}vm}\PY{p}{)}
\PY{n}{eulerAllocator3} \PY{o}{=} \PY{n}{EulerAllocator}\PY{p}{(}\PY{n}{ca\PYZus{}im}\PY{p}{)}
\PY{n}{allocation\PYZus{}no\PYZus{}margin} \PY{o}{=} \PY{n}{eulerAllocator1}\PY{o}{.}\PY{n}{allocate\PYZus{}ead}\PY{p}{(}\PY{p}{)}
\PY{n}{allocation\PYZus{}with\PYZus{}vm} \PY{o}{=} \PY{n}{eulerAllocator2}\PY{o}{.}\PY{n}{allocate\PYZus{}ead}\PY{p}{(}\PY{p}{)}
\PY{n}{allocation\PYZus{}with\PYZus{}im\PYZus{}and\PYZus{}im} \PY{o}{=} \PY{n}{eulerAllocator3}\PY{o}{.}\PY{n}{allocate\PYZus{}ead}\PY{p}{(}\PY{p}{)}
\PY{n}{allocation\PYZus{}im} \PY{o}{=} \PY{n}{eulerAllocator3}\PY{o}{.}\PY{n}{allocate\PYZus{}im}\PY{p}{(}\PY{p}{)}
\end{Verbatim}
\end{tcolorbox}

    we can calculate how far the sum of the allocated values deviates from
the risk measure:

    \begin{Verbatim}[commandchars=\\\{\}]
Diff EAD no margin:    0.00 EUR
Diff EAD only VM:      0.00 EUR
Diff EAD VM + IM:   1068.64 EUR
Diff calculated IM:    6.74 EUR
    \end{Verbatim}

    In relation to the EAD of 345,874 EUR the deviation of the allocated EAD
under VM and IM of 1,068.64 EUR is not large but can be improved
nevertheless.

    
    By, default the implemented Euler allocation class uses a forward
difference approach. If we switch over to a central difference approach
the deviation shrinks significantly.

    \begin{tcolorbox}[breakable, size=fbox, boxrule=1pt, pad at break*=1mm,colback=cellbackground, colframe=cellborder]
\prompt{In}{incolor}{11}{\boxspacing}
\begin{Verbatim}[commandchars=\\\{\}]
\PY{n}{eulerAllocator3}\PY{o}{.}\PY{n}{fdApproach2} \PY{o}{=} \PY{n}{FdApproach2}\PY{o}{.}\PY{n}{Central}
\PY{n}{allocation\PYZus{}im} \PY{o}{=} \PY{n}{eulerAllocator3}\PY{o}{.}\PY{n}{allocate\PYZus{}im}\PY{p}{(}\PY{p}{)}
\PY{n}{allocation\PYZus{}with\PYZus{}im\PYZus{}and\PYZus{}im} \PY{o}{=} \PY{n}{eulerAllocator3}\PY{o}{.}\PY{n}{allocate\PYZus{}ead}\PY{p}{(}\PY{p}{)}
\end{Verbatim}
\end{tcolorbox}

    \begin{Verbatim}[commandchars=\\\{\}]
Diff EAD VM + IM:   0.01 EUR
Diff calculated IM: 0.00 EUR
    \end{Verbatim}

    Displaying the allocation results

            \begin{tcolorbox}[breakable, size=fbox, boxrule=.5pt, pad at break*=1mm, opacityfill=0]
\prompt{Out}{outcolor}{14}{\boxspacing}
\begin{Verbatim}[commandchars=\\\{\}]
          2Mn ADS Call 3Mn ADS Put Portfolio Risk Measure
SIMM           -33.75\%     133.75\%         14,231,564 USD
No margin       99.21\%       0.79\%         37,643,536 USD
VM only        232.47\%    -132.47\%          3,519,458 USD
VM+IM          622.10\%    -522.10\%            345,874 USD
\end{Verbatim}
\end{tcolorbox}
        
    For SIMM the Put has the higher risk and the Call is considered a hedge
trade while for SA-CCR with only VM, the Call has the higher risk and
the Put is considered a hedge trade.

The reason for this are the different holding periods between the two
models. If we lower the maturity of the trades to 10 days instead, we
can see that for SA-CCR with only VM the call is considered the hedge
trade.

    \begin{tcolorbox}[breakable, size=fbox, boxrule=1pt, pad at break*=1mm,colback=cellbackground, colframe=cellborder]
\prompt{In}{incolor}{15}{\boxspacing}
\begin{Verbatim}[commandchars=\\\{\}]
\PY{n}{eq\PYZus{}opt\PYZus{}ads\PYZus{}call\PYZus{}10d} \PY{o}{=} \PY{n}{EquityOption}\PY{p}{(}\PY{n}{underlying}\PY{o}{=}\PY{n}{Stock}\PY{o}{.}\PY{n}{ADS}\PY{p}{,}
                               \PY{n}{maturity}\PY{o}{=}\PY{n}{ql}\PY{o}{.}\PY{n}{Period}\PY{p}{(}\PY{l+m+mi}{10}\PY{p}{,} \PY{n}{ql}\PY{o}{.}\PY{n}{Days}\PY{p}{)}\PY{p}{,}
                               \PY{n}{notional}\PY{o}{=}\PY{l+m+mi}{2000000}\PY{p}{,}
                               \PY{n}{tradeType}\PY{o}{=}\PY{n}{TradeType}\PY{o}{.}\PY{n}{CALL}\PY{p}{,}
                               \PY{n}{tradeDirection}\PY{o}{=}\PY{n}{TradeDirection}\PY{o}{.}\PY{n}{LONG}\PY{p}{)}

\PY{n}{eq\PYZus{}opt\PYZus{}ads\PYZus{}put\PYZus{}10d} \PY{o}{=} \PY{n}{EquityOption}\PY{p}{(}\PY{n}{underlying}\PY{o}{=}\PY{n}{Stock}\PY{o}{.}\PY{n}{ADS}\PY{p}{,}
                              \PY{n}{maturity}\PY{o}{=}\PY{n}{ql}\PY{o}{.}\PY{n}{Period}\PY{p}{(}\PY{l+m+mi}{10}\PY{p}{,} \PY{n}{ql}\PY{o}{.}\PY{n}{Days}\PY{p}{)}\PY{p}{,}
                              \PY{n}{notional}\PY{o}{=}\PY{l+m+mi}{3000000}\PY{p}{,}
                              \PY{n}{tradeType}\PY{o}{=}\PY{n}{TradeType}\PY{o}{.}\PY{n}{PUT}\PY{p}{,}
                              \PY{n}{tradeDirection}\PY{o}{=}\PY{n}{TradeDirection}\PY{o}{.}\PY{n}{LONG}\PY{p}{)}
\end{Verbatim}
\end{tcolorbox}

    \begin{tcolorbox}[breakable, size=fbox, boxrule=1pt, pad at break*=1mm,colback=cellbackground, colframe=cellborder]
\prompt{In}{incolor}{16}{\boxspacing}
\begin{Verbatim}[commandchars=\\\{\}]
\PY{n}{ca\PYZus{}vm}\PY{o}{.}\PY{n}{remove\PYZus{}all\PYZus{}trades}\PY{p}{(}\PY{p}{)}
\PY{n}{ca\PYZus{}vm}\PY{o}{.}\PY{n}{add\PYZus{}trades}\PY{p}{(}\PY{p}{[}\PY{n}{eq\PYZus{}opt\PYZus{}ads\PYZus{}call\PYZus{}10d}\PY{p}{,}\PY{n}{eq\PYZus{}opt\PYZus{}ads\PYZus{}put\PYZus{}10d}\PY{p}{]}\PY{p}{)}
\end{Verbatim}
\end{tcolorbox}

    \begin{tcolorbox}[breakable, size=fbox, boxrule=1pt, pad at break*=1mm,colback=cellbackground, colframe=cellborder]
\prompt{In}{incolor}{17}{\boxspacing}
\begin{Verbatim}[commandchars=\\\{\}]
\PY{n}{allocation\PYZus{}with\PYZus{}vm} \PY{o}{=} \PY{n}{eulerAllocator2}\PY{o}{.}\PY{n}{allocate\PYZus{}ead}\PY{p}{(}\PY{p}{)}
\end{Verbatim}
\end{tcolorbox}

            \begin{tcolorbox}[breakable, size=fbox, boxrule=.5pt, pad at break*=1mm, opacityfill=0]
\prompt{Out}{outcolor}{19}{\boxspacing}
\begin{Verbatim}[commandchars=\\\{\}]
        2Mn ADS Call 1W3D 3Mn ADS Put 1W3D Portfolio Risk Measure
VM only          -358.06\%          458.06\%          1,701,707 USD
\end{Verbatim}
\end{tcolorbox}
        
    Going back to 1Y maturity equity options we can see that the allocations
of the preexisting trades can change significantly, when we add another
equity option to the portfolio. We choose a position of 10Mn call
options on an imaginary DBK stock.

    \begin{tcolorbox}[breakable, size=fbox, boxrule=1pt, pad at break*=1mm,colback=cellbackground, colframe=cellborder]
\prompt{In}{incolor}{20}{\boxspacing}
\begin{Verbatim}[commandchars=\\\{\}]
\PY{n}{eq\PYZus{}opt\PYZus{}dbk\PYZus{}call} \PY{o}{=} \PY{n}{EquityOption}\PY{p}{(}\PY{n}{underlying}\PY{o}{=}\PY{n}{Stock}\PY{o}{.}\PY{n}{DBK}\PY{p}{,}
                               \PY{n}{maturity}\PY{o}{=}\PY{n}{ql}\PY{o}{.}\PY{n}{Period}\PY{p}{(}\PY{l+m+mi}{1}\PY{p}{,} \PY{n}{ql}\PY{o}{.}\PY{n}{Years}\PY{p}{)}\PY{p}{,}
                               \PY{n}{notional} \PY{o}{=} \PY{l+m+mi}{10000000}\PY{p}{,}
                               \PY{n}{tradeType}\PY{o}{=}\PY{n}{TradeType}\PY{o}{.}\PY{n}{CALL}\PY{p}{,}
                               \PY{n}{tradeDirection}\PY{o}{=}\PY{n}{TradeDirection}\PY{o}{.}\PY{n}{LONG}\PY{p}{)}

\PY{n}{ca}\PY{o}{.}\PY{n}{remove\PYZus{}all\PYZus{}trades}\PY{p}{(}\PY{p}{)}
\PY{n}{ca}\PY{o}{.}\PY{n}{add\PYZus{}trades}\PY{p}{(}\PY{p}{[}\PY{n}{eq\PYZus{}opt\PYZus{}ads\PYZus{}call}\PY{p}{,}\PY{n}{eq\PYZus{}opt\PYZus{}ads\PYZus{}put}\PY{p}{,}\PY{n}{eq\PYZus{}opt\PYZus{}dbk\PYZus{}call}\PY{p}{]}\PY{p}{)}
\PY{n}{ca\PYZus{}vm}\PY{o}{.}\PY{n}{remove\PYZus{}all\PYZus{}trades}\PY{p}{(}\PY{p}{)}
\PY{n}{ca\PYZus{}vm}\PY{o}{.}\PY{n}{add\PYZus{}trades}\PY{p}{(}\PY{p}{[}\PY{n}{eq\PYZus{}opt\PYZus{}ads\PYZus{}call}\PY{p}{,}\PY{n}{eq\PYZus{}opt\PYZus{}ads\PYZus{}put}\PY{p}{,}\PY{n}{eq\PYZus{}opt\PYZus{}dbk\PYZus{}call}\PY{p}{]}\PY{p}{)}
\PY{n}{ca\PYZus{}im}\PY{o}{.}\PY{n}{remove\PYZus{}all\PYZus{}trades}\PY{p}{(}\PY{p}{)}
\PY{n}{ca\PYZus{}im}\PY{o}{.}\PY{n}{add\PYZus{}trades}\PY{p}{(}\PY{p}{[}\PY{n}{eq\PYZus{}opt\PYZus{}ads\PYZus{}call}\PY{p}{,}\PY{n}{eq\PYZus{}opt\PYZus{}ads\PYZus{}put}\PY{p}{,}\PY{n}{eq\PYZus{}opt\PYZus{}dbk\PYZus{}call}\PY{p}{]}\PY{p}{)}
\end{Verbatim}
\end{tcolorbox}

    \begin{tcolorbox}[breakable, size=fbox, boxrule=1pt, pad at break*=1mm,colback=cellbackground, colframe=cellborder]
\prompt{In}{incolor}{21}{\boxspacing}
\begin{Verbatim}[commandchars=\\\{\}]
\PY{n}{allocation\PYZus{}no\PYZus{}margin} \PY{o}{=} \PY{n}{eulerAllocator1}\PY{o}{.}\PY{n}{allocate\PYZus{}ead}\PY{p}{(}\PY{p}{)}
\PY{n}{allocation\PYZus{}with\PYZus{}vm} \PY{o}{=} \PY{n}{eulerAllocator2}\PY{o}{.}\PY{n}{allocate\PYZus{}ead}\PY{p}{(}\PY{p}{)}
\PY{n}{allocation\PYZus{}with\PYZus{}im\PYZus{}and\PYZus{}im} \PY{o}{=} \PY{n}{eulerAllocator3}\PY{o}{.}\PY{n}{allocate\PYZus{}ead}\PY{p}{(}\PY{p}{)}
\PY{n}{allocation\PYZus{}im} \PY{o}{=} \PY{n}{eulerAllocator3}\PY{o}{.}\PY{n}{allocate\PYZus{}im}\PY{p}{(}\PY{p}{)}
\end{Verbatim}
\end{tcolorbox}

            \begin{tcolorbox}[breakable, size=fbox, boxrule=.5pt, pad at break*=1mm, opacityfill=0]
\prompt{Out}{outcolor}{22}{\boxspacing}
\begin{Verbatim}[commandchars=\\\{\}]
          10Mn DBK Call 2Mn ADS Call 3Mn ADS Put PF Risk Measure
SIMM             63.10\%       15.23\%      21.67\%  27,551,513 USD
No margin        57.45\%       33.16\%       9.39\%  76,295,560 USD
VM only          80.79\%       44.65\%     -25.44\%  10,230,051 USD
VM+IM           106.19\%       86.85\%     -93.04\%   1,847,365 USD
\end{Verbatim}
\end{tcolorbox}
        
    Further analysis of the results shown above may be found in section
\ref{sec:Exemplary Euler allocation of SA-CCR under consideration of margining}


    % Add a bibliography block to the postdoc
    
    
    
\end{document}
