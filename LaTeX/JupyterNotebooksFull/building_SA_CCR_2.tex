\documentclass[11pt]{article}

    \usepackage[breakable]{tcolorbox}
    \usepackage{parskip} % Stop auto-indenting (to mimic markdown behaviour)
    
    \usepackage{iftex}
    \ifPDFTeX
    	\usepackage[T1]{fontenc}
    	\usepackage{mathpazo}
    \else
    	\usepackage{fontspec}
    \fi

    % Basic figure setup, for now with no caption control since it's done
    % automatically by Pandoc (which extracts ![](path) syntax from Markdown).
    \usepackage{graphicx}
    % Maintain compatibility with old templates. Remove in nbconvert 6.0
    \let\Oldincludegraphics\includegraphics
    % Ensure that by default, figures have no caption (until we provide a
    % proper Figure object with a Caption API and a way to capture that
    % in the conversion process - todo).
    \usepackage{caption}
    \DeclareCaptionFormat{nocaption}{}
    \captionsetup{format=nocaption,aboveskip=0pt,belowskip=0pt}

    \usepackage[Export]{adjustbox} % Used to constrain images to a maximum size
    \adjustboxset{max size={0.9\linewidth}{0.9\paperheight}}
    \usepackage{float}
    \floatplacement{figure}{H} % forces figures to be placed at the correct location
    \usepackage{xcolor} % Allow colors to be defined
    \usepackage{enumerate} % Needed for markdown enumerations to work
    \usepackage{geometry} % Used to adjust the document margins
    \usepackage{amsmath} % Equations
    \usepackage{amssymb} % Equations
    \usepackage{textcomp} % defines textquotesingle
    % Hack from http://tex.stackexchange.com/a/47451/13684:
    \AtBeginDocument{%
        \def\PYZsq{\textquotesingle}% Upright quotes in Pygmentized code
    }
    \usepackage{upquote} % Upright quotes for verbatim code
    \usepackage{eurosym} % defines \euro
    \usepackage[mathletters]{ucs} % Extended unicode (utf-8) support
    \usepackage{fancyvrb} % verbatim replacement that allows latex
    \usepackage{grffile} % extends the file name processing of package graphics 
                         % to support a larger range
    \makeatletter % fix for grffile with XeLaTeX
    \def\Gread@@xetex#1{%
      \IfFileExists{"\Gin@base".bb}%
      {\Gread@eps{\Gin@base.bb}}%
      {\Gread@@xetex@aux#1}%
    }
    \makeatother

    % The hyperref package gives us a pdf with properly built
    % internal navigation ('pdf bookmarks' for the table of contents,
    % internal cross-reference links, web links for URLs, etc.)
    \usepackage{hyperref}
    % The default LaTeX title has an obnoxious amount of whitespace. By default,
    % titling removes some of it. It also provides customization options.
    \usepackage{titling}
    \usepackage{longtable} % longtable support required by pandoc >1.10
    \usepackage{booktabs}  % table support for pandoc > 1.12.2
    \usepackage[inline]{enumitem} % IRkernel/repr support (it uses the enumerate* environment)
    \usepackage[normalem]{ulem} % ulem is needed to support strikethroughs (\sout)
                                % normalem makes italics be italics, not underlines
    \usepackage{mathrsfs}
    

    
    % Colors for the hyperref package
    \definecolor{urlcolor}{rgb}{0,.145,.698}
    \definecolor{linkcolor}{rgb}{.71,0.21,0.01}
    \definecolor{citecolor}{rgb}{.12,.54,.11}

    % ANSI colors
    \definecolor{ansi-black}{HTML}{3E424D}
    \definecolor{ansi-black-intense}{HTML}{282C36}
    \definecolor{ansi-red}{HTML}{E75C58}
    \definecolor{ansi-red-intense}{HTML}{B22B31}
    \definecolor{ansi-green}{HTML}{00A250}
    \definecolor{ansi-green-intense}{HTML}{007427}
    \definecolor{ansi-yellow}{HTML}{DDB62B}
    \definecolor{ansi-yellow-intense}{HTML}{B27D12}
    \definecolor{ansi-blue}{HTML}{208FFB}
    \definecolor{ansi-blue-intense}{HTML}{0065CA}
    \definecolor{ansi-magenta}{HTML}{D160C4}
    \definecolor{ansi-magenta-intense}{HTML}{A03196}
    \definecolor{ansi-cyan}{HTML}{60C6C8}
    \definecolor{ansi-cyan-intense}{HTML}{258F8F}
    \definecolor{ansi-white}{HTML}{C5C1B4}
    \definecolor{ansi-white-intense}{HTML}{A1A6B2}
    \definecolor{ansi-default-inverse-fg}{HTML}{FFFFFF}
    \definecolor{ansi-default-inverse-bg}{HTML}{000000}

    % commands and environments needed by pandoc snippets
    % extracted from the output of `pandoc -s`
    \providecommand{\tightlist}{%
      \setlength{\itemsep}{0pt}\setlength{\parskip}{0pt}}
    \DefineVerbatimEnvironment{Highlighting}{Verbatim}{commandchars=\\\{\}}
    % Add ',fontsize=\small' for more characters per line
    \newenvironment{Shaded}{}{}
    \newcommand{\KeywordTok}[1]{\textcolor[rgb]{0.00,0.44,0.13}{\textbf{{#1}}}}
    \newcommand{\DataTypeTok}[1]{\textcolor[rgb]{0.56,0.13,0.00}{{#1}}}
    \newcommand{\DecValTok}[1]{\textcolor[rgb]{0.25,0.63,0.44}{{#1}}}
    \newcommand{\BaseNTok}[1]{\textcolor[rgb]{0.25,0.63,0.44}{{#1}}}
    \newcommand{\FloatTok}[1]{\textcolor[rgb]{0.25,0.63,0.44}{{#1}}}
    \newcommand{\CharTok}[1]{\textcolor[rgb]{0.25,0.44,0.63}{{#1}}}
    \newcommand{\StringTok}[1]{\textcolor[rgb]{0.25,0.44,0.63}{{#1}}}
    \newcommand{\CommentTok}[1]{\textcolor[rgb]{0.38,0.63,0.69}{\textit{{#1}}}}
    \newcommand{\OtherTok}[1]{\textcolor[rgb]{0.00,0.44,0.13}{{#1}}}
    \newcommand{\AlertTok}[1]{\textcolor[rgb]{1.00,0.00,0.00}{\textbf{{#1}}}}
    \newcommand{\FunctionTok}[1]{\textcolor[rgb]{0.02,0.16,0.49}{{#1}}}
    \newcommand{\RegionMarkerTok}[1]{{#1}}
    \newcommand{\ErrorTok}[1]{\textcolor[rgb]{1.00,0.00,0.00}{\textbf{{#1}}}}
    \newcommand{\NormalTok}[1]{{#1}}
    
    % Additional commands for more recent versions of Pandoc
    \newcommand{\ConstantTok}[1]{\textcolor[rgb]{0.53,0.00,0.00}{{#1}}}
    \newcommand{\SpecialCharTok}[1]{\textcolor[rgb]{0.25,0.44,0.63}{{#1}}}
    \newcommand{\VerbatimStringTok}[1]{\textcolor[rgb]{0.25,0.44,0.63}{{#1}}}
    \newcommand{\SpecialStringTok}[1]{\textcolor[rgb]{0.73,0.40,0.53}{{#1}}}
    \newcommand{\ImportTok}[1]{{#1}}
    \newcommand{\DocumentationTok}[1]{\textcolor[rgb]{0.73,0.13,0.13}{\textit{{#1}}}}
    \newcommand{\AnnotationTok}[1]{\textcolor[rgb]{0.38,0.63,0.69}{\textbf{\textit{{#1}}}}}
    \newcommand{\CommentVarTok}[1]{\textcolor[rgb]{0.38,0.63,0.69}{\textbf{\textit{{#1}}}}}
    \newcommand{\VariableTok}[1]{\textcolor[rgb]{0.10,0.09,0.49}{{#1}}}
    \newcommand{\ControlFlowTok}[1]{\textcolor[rgb]{0.00,0.44,0.13}{\textbf{{#1}}}}
    \newcommand{\OperatorTok}[1]{\textcolor[rgb]{0.40,0.40,0.40}{{#1}}}
    \newcommand{\BuiltInTok}[1]{{#1}}
    \newcommand{\ExtensionTok}[1]{{#1}}
    \newcommand{\PreprocessorTok}[1]{\textcolor[rgb]{0.74,0.48,0.00}{{#1}}}
    \newcommand{\AttributeTok}[1]{\textcolor[rgb]{0.49,0.56,0.16}{{#1}}}
    \newcommand{\InformationTok}[1]{\textcolor[rgb]{0.38,0.63,0.69}{\textbf{\textit{{#1}}}}}
    \newcommand{\WarningTok}[1]{\textcolor[rgb]{0.38,0.63,0.69}{\textbf{\textit{{#1}}}}}
    
    
    % Define a nice break command that doesn't care if a line doesn't already
    % exist.
    \def\br{\hspace*{\fill} \\* }
    % Math Jax compatibility definitions
    \def\gt{>}
    \def\lt{<}
    \let\Oldtex\TeX
    \let\Oldlatex\LaTeX
    \renewcommand{\TeX}{\textrm{\Oldtex}}
    \renewcommand{\LaTeX}{\textrm{\Oldlatex}}
    % Document parameters
    % Document title
    \title{building\_SA\_CCR\_2}
    
    
    
    
    
% Pygments definitions
\makeatletter
\def\PY@reset{\let\PY@it=\relax \let\PY@bf=\relax%
    \let\PY@ul=\relax \let\PY@tc=\relax%
    \let\PY@bc=\relax \let\PY@ff=\relax}
\def\PY@tok#1{\csname PY@tok@#1\endcsname}
\def\PY@toks#1+{\ifx\relax#1\empty\else%
    \PY@tok{#1}\expandafter\PY@toks\fi}
\def\PY@do#1{\PY@bc{\PY@tc{\PY@ul{%
    \PY@it{\PY@bf{\PY@ff{#1}}}}}}}
\def\PY#1#2{\PY@reset\PY@toks#1+\relax+\PY@do{#2}}

\expandafter\def\csname PY@tok@w\endcsname{\def\PY@tc##1{\textcolor[rgb]{0.73,0.73,0.73}{##1}}}
\expandafter\def\csname PY@tok@c\endcsname{\let\PY@it=\textit\def\PY@tc##1{\textcolor[rgb]{0.25,0.50,0.50}{##1}}}
\expandafter\def\csname PY@tok@cp\endcsname{\def\PY@tc##1{\textcolor[rgb]{0.74,0.48,0.00}{##1}}}
\expandafter\def\csname PY@tok@k\endcsname{\let\PY@bf=\textbf\def\PY@tc##1{\textcolor[rgb]{0.00,0.50,0.00}{##1}}}
\expandafter\def\csname PY@tok@kp\endcsname{\def\PY@tc##1{\textcolor[rgb]{0.00,0.50,0.00}{##1}}}
\expandafter\def\csname PY@tok@kt\endcsname{\def\PY@tc##1{\textcolor[rgb]{0.69,0.00,0.25}{##1}}}
\expandafter\def\csname PY@tok@o\endcsname{\def\PY@tc##1{\textcolor[rgb]{0.40,0.40,0.40}{##1}}}
\expandafter\def\csname PY@tok@ow\endcsname{\let\PY@bf=\textbf\def\PY@tc##1{\textcolor[rgb]{0.67,0.13,1.00}{##1}}}
\expandafter\def\csname PY@tok@nb\endcsname{\def\PY@tc##1{\textcolor[rgb]{0.00,0.50,0.00}{##1}}}
\expandafter\def\csname PY@tok@nf\endcsname{\def\PY@tc##1{\textcolor[rgb]{0.00,0.00,1.00}{##1}}}
\expandafter\def\csname PY@tok@nc\endcsname{\let\PY@bf=\textbf\def\PY@tc##1{\textcolor[rgb]{0.00,0.00,1.00}{##1}}}
\expandafter\def\csname PY@tok@nn\endcsname{\let\PY@bf=\textbf\def\PY@tc##1{\textcolor[rgb]{0.00,0.00,1.00}{##1}}}
\expandafter\def\csname PY@tok@ne\endcsname{\let\PY@bf=\textbf\def\PY@tc##1{\textcolor[rgb]{0.82,0.25,0.23}{##1}}}
\expandafter\def\csname PY@tok@nv\endcsname{\def\PY@tc##1{\textcolor[rgb]{0.10,0.09,0.49}{##1}}}
\expandafter\def\csname PY@tok@no\endcsname{\def\PY@tc##1{\textcolor[rgb]{0.53,0.00,0.00}{##1}}}
\expandafter\def\csname PY@tok@nl\endcsname{\def\PY@tc##1{\textcolor[rgb]{0.63,0.63,0.00}{##1}}}
\expandafter\def\csname PY@tok@ni\endcsname{\let\PY@bf=\textbf\def\PY@tc##1{\textcolor[rgb]{0.60,0.60,0.60}{##1}}}
\expandafter\def\csname PY@tok@na\endcsname{\def\PY@tc##1{\textcolor[rgb]{0.49,0.56,0.16}{##1}}}
\expandafter\def\csname PY@tok@nt\endcsname{\let\PY@bf=\textbf\def\PY@tc##1{\textcolor[rgb]{0.00,0.50,0.00}{##1}}}
\expandafter\def\csname PY@tok@nd\endcsname{\def\PY@tc##1{\textcolor[rgb]{0.67,0.13,1.00}{##1}}}
\expandafter\def\csname PY@tok@s\endcsname{\def\PY@tc##1{\textcolor[rgb]{0.73,0.13,0.13}{##1}}}
\expandafter\def\csname PY@tok@sd\endcsname{\let\PY@it=\textit\def\PY@tc##1{\textcolor[rgb]{0.73,0.13,0.13}{##1}}}
\expandafter\def\csname PY@tok@si\endcsname{\let\PY@bf=\textbf\def\PY@tc##1{\textcolor[rgb]{0.73,0.40,0.53}{##1}}}
\expandafter\def\csname PY@tok@se\endcsname{\let\PY@bf=\textbf\def\PY@tc##1{\textcolor[rgb]{0.73,0.40,0.13}{##1}}}
\expandafter\def\csname PY@tok@sr\endcsname{\def\PY@tc##1{\textcolor[rgb]{0.73,0.40,0.53}{##1}}}
\expandafter\def\csname PY@tok@ss\endcsname{\def\PY@tc##1{\textcolor[rgb]{0.10,0.09,0.49}{##1}}}
\expandafter\def\csname PY@tok@sx\endcsname{\def\PY@tc##1{\textcolor[rgb]{0.00,0.50,0.00}{##1}}}
\expandafter\def\csname PY@tok@m\endcsname{\def\PY@tc##1{\textcolor[rgb]{0.40,0.40,0.40}{##1}}}
\expandafter\def\csname PY@tok@gh\endcsname{\let\PY@bf=\textbf\def\PY@tc##1{\textcolor[rgb]{0.00,0.00,0.50}{##1}}}
\expandafter\def\csname PY@tok@gu\endcsname{\let\PY@bf=\textbf\def\PY@tc##1{\textcolor[rgb]{0.50,0.00,0.50}{##1}}}
\expandafter\def\csname PY@tok@gd\endcsname{\def\PY@tc##1{\textcolor[rgb]{0.63,0.00,0.00}{##1}}}
\expandafter\def\csname PY@tok@gi\endcsname{\def\PY@tc##1{\textcolor[rgb]{0.00,0.63,0.00}{##1}}}
\expandafter\def\csname PY@tok@gr\endcsname{\def\PY@tc##1{\textcolor[rgb]{1.00,0.00,0.00}{##1}}}
\expandafter\def\csname PY@tok@ge\endcsname{\let\PY@it=\textit}
\expandafter\def\csname PY@tok@gs\endcsname{\let\PY@bf=\textbf}
\expandafter\def\csname PY@tok@gp\endcsname{\let\PY@bf=\textbf\def\PY@tc##1{\textcolor[rgb]{0.00,0.00,0.50}{##1}}}
\expandafter\def\csname PY@tok@go\endcsname{\def\PY@tc##1{\textcolor[rgb]{0.53,0.53,0.53}{##1}}}
\expandafter\def\csname PY@tok@gt\endcsname{\def\PY@tc##1{\textcolor[rgb]{0.00,0.27,0.87}{##1}}}
\expandafter\def\csname PY@tok@err\endcsname{\def\PY@bc##1{\setlength{\fboxsep}{0pt}\fcolorbox[rgb]{1.00,0.00,0.00}{1,1,1}{\strut ##1}}}
\expandafter\def\csname PY@tok@kc\endcsname{\let\PY@bf=\textbf\def\PY@tc##1{\textcolor[rgb]{0.00,0.50,0.00}{##1}}}
\expandafter\def\csname PY@tok@kd\endcsname{\let\PY@bf=\textbf\def\PY@tc##1{\textcolor[rgb]{0.00,0.50,0.00}{##1}}}
\expandafter\def\csname PY@tok@kn\endcsname{\let\PY@bf=\textbf\def\PY@tc##1{\textcolor[rgb]{0.00,0.50,0.00}{##1}}}
\expandafter\def\csname PY@tok@kr\endcsname{\let\PY@bf=\textbf\def\PY@tc##1{\textcolor[rgb]{0.00,0.50,0.00}{##1}}}
\expandafter\def\csname PY@tok@bp\endcsname{\def\PY@tc##1{\textcolor[rgb]{0.00,0.50,0.00}{##1}}}
\expandafter\def\csname PY@tok@fm\endcsname{\def\PY@tc##1{\textcolor[rgb]{0.00,0.00,1.00}{##1}}}
\expandafter\def\csname PY@tok@vc\endcsname{\def\PY@tc##1{\textcolor[rgb]{0.10,0.09,0.49}{##1}}}
\expandafter\def\csname PY@tok@vg\endcsname{\def\PY@tc##1{\textcolor[rgb]{0.10,0.09,0.49}{##1}}}
\expandafter\def\csname PY@tok@vi\endcsname{\def\PY@tc##1{\textcolor[rgb]{0.10,0.09,0.49}{##1}}}
\expandafter\def\csname PY@tok@vm\endcsname{\def\PY@tc##1{\textcolor[rgb]{0.10,0.09,0.49}{##1}}}
\expandafter\def\csname PY@tok@sa\endcsname{\def\PY@tc##1{\textcolor[rgb]{0.73,0.13,0.13}{##1}}}
\expandafter\def\csname PY@tok@sb\endcsname{\def\PY@tc##1{\textcolor[rgb]{0.73,0.13,0.13}{##1}}}
\expandafter\def\csname PY@tok@sc\endcsname{\def\PY@tc##1{\textcolor[rgb]{0.73,0.13,0.13}{##1}}}
\expandafter\def\csname PY@tok@dl\endcsname{\def\PY@tc##1{\textcolor[rgb]{0.73,0.13,0.13}{##1}}}
\expandafter\def\csname PY@tok@s2\endcsname{\def\PY@tc##1{\textcolor[rgb]{0.73,0.13,0.13}{##1}}}
\expandafter\def\csname PY@tok@sh\endcsname{\def\PY@tc##1{\textcolor[rgb]{0.73,0.13,0.13}{##1}}}
\expandafter\def\csname PY@tok@s1\endcsname{\def\PY@tc##1{\textcolor[rgb]{0.73,0.13,0.13}{##1}}}
\expandafter\def\csname PY@tok@mb\endcsname{\def\PY@tc##1{\textcolor[rgb]{0.40,0.40,0.40}{##1}}}
\expandafter\def\csname PY@tok@mf\endcsname{\def\PY@tc##1{\textcolor[rgb]{0.40,0.40,0.40}{##1}}}
\expandafter\def\csname PY@tok@mh\endcsname{\def\PY@tc##1{\textcolor[rgb]{0.40,0.40,0.40}{##1}}}
\expandafter\def\csname PY@tok@mi\endcsname{\def\PY@tc##1{\textcolor[rgb]{0.40,0.40,0.40}{##1}}}
\expandafter\def\csname PY@tok@il\endcsname{\def\PY@tc##1{\textcolor[rgb]{0.40,0.40,0.40}{##1}}}
\expandafter\def\csname PY@tok@mo\endcsname{\def\PY@tc##1{\textcolor[rgb]{0.40,0.40,0.40}{##1}}}
\expandafter\def\csname PY@tok@ch\endcsname{\let\PY@it=\textit\def\PY@tc##1{\textcolor[rgb]{0.25,0.50,0.50}{##1}}}
\expandafter\def\csname PY@tok@cm\endcsname{\let\PY@it=\textit\def\PY@tc##1{\textcolor[rgb]{0.25,0.50,0.50}{##1}}}
\expandafter\def\csname PY@tok@cpf\endcsname{\let\PY@it=\textit\def\PY@tc##1{\textcolor[rgb]{0.25,0.50,0.50}{##1}}}
\expandafter\def\csname PY@tok@c1\endcsname{\let\PY@it=\textit\def\PY@tc##1{\textcolor[rgb]{0.25,0.50,0.50}{##1}}}
\expandafter\def\csname PY@tok@cs\endcsname{\let\PY@it=\textit\def\PY@tc##1{\textcolor[rgb]{0.25,0.50,0.50}{##1}}}

\def\PYZbs{\char`\\}
\def\PYZus{\char`\_}
\def\PYZob{\char`\{}
\def\PYZcb{\char`\}}
\def\PYZca{\char`\^}
\def\PYZam{\char`\&}
\def\PYZlt{\char`\<}
\def\PYZgt{\char`\>}
\def\PYZsh{\char`\#}
\def\PYZpc{\char`\%}
\def\PYZdl{\char`\$}
\def\PYZhy{\char`\-}
\def\PYZsq{\char`\'}
\def\PYZdq{\char`\"}
\def\PYZti{\char`\~}
% for compatibility with earlier versions
\def\PYZat{@}
\def\PYZlb{[}
\def\PYZrb{]}
\makeatother


    % For linebreaks inside Verbatim environment from package fancyvrb. 
    \makeatletter
        \newbox\Wrappedcontinuationbox 
        \newbox\Wrappedvisiblespacebox 
        \newcommand*\Wrappedvisiblespace {\textcolor{red}{\textvisiblespace}} 
        \newcommand*\Wrappedcontinuationsymbol {\textcolor{red}{\llap{\tiny$\m@th\hookrightarrow$}}} 
        \newcommand*\Wrappedcontinuationindent {3ex } 
        \newcommand*\Wrappedafterbreak {\kern\Wrappedcontinuationindent\copy\Wrappedcontinuationbox} 
        % Take advantage of the already applied Pygments mark-up to insert 
        % potential linebreaks for TeX processing. 
        %        {, <, #, %, $, ' and ": go to next line. 
        %        _, }, ^, &, >, - and ~: stay at end of broken line. 
        % Use of \textquotesingle for straight quote. 
        \newcommand*\Wrappedbreaksatspecials {% 
            \def\PYGZus{\discretionary{\char`\_}{\Wrappedafterbreak}{\char`\_}}% 
            \def\PYGZob{\discretionary{}{\Wrappedafterbreak\char`\{}{\char`\{}}% 
            \def\PYGZcb{\discretionary{\char`\}}{\Wrappedafterbreak}{\char`\}}}% 
            \def\PYGZca{\discretionary{\char`\^}{\Wrappedafterbreak}{\char`\^}}% 
            \def\PYGZam{\discretionary{\char`\&}{\Wrappedafterbreak}{\char`\&}}% 
            \def\PYGZlt{\discretionary{}{\Wrappedafterbreak\char`\<}{\char`\<}}% 
            \def\PYGZgt{\discretionary{\char`\>}{\Wrappedafterbreak}{\char`\>}}% 
            \def\PYGZsh{\discretionary{}{\Wrappedafterbreak\char`\#}{\char`\#}}% 
            \def\PYGZpc{\discretionary{}{\Wrappedafterbreak\char`\%}{\char`\%}}% 
            \def\PYGZdl{\discretionary{}{\Wrappedafterbreak\char`\$}{\char`\$}}% 
            \def\PYGZhy{\discretionary{\char`\-}{\Wrappedafterbreak}{\char`\-}}% 
            \def\PYGZsq{\discretionary{}{\Wrappedafterbreak\textquotesingle}{\textquotesingle}}% 
            \def\PYGZdq{\discretionary{}{\Wrappedafterbreak\char`\"}{\char`\"}}% 
            \def\PYGZti{\discretionary{\char`\~}{\Wrappedafterbreak}{\char`\~}}% 
        } 
        % Some characters . , ; ? ! / are not pygmentized. 
        % This macro makes them "active" and they will insert potential linebreaks 
        \newcommand*\Wrappedbreaksatpunct {% 
            \lccode`\~`\.\lowercase{\def~}{\discretionary{\hbox{\char`\.}}{\Wrappedafterbreak}{\hbox{\char`\.}}}% 
            \lccode`\~`\,\lowercase{\def~}{\discretionary{\hbox{\char`\,}}{\Wrappedafterbreak}{\hbox{\char`\,}}}% 
            \lccode`\~`\;\lowercase{\def~}{\discretionary{\hbox{\char`\;}}{\Wrappedafterbreak}{\hbox{\char`\;}}}% 
            \lccode`\~`\:\lowercase{\def~}{\discretionary{\hbox{\char`\:}}{\Wrappedafterbreak}{\hbox{\char`\:}}}% 
            \lccode`\~`\?\lowercase{\def~}{\discretionary{\hbox{\char`\?}}{\Wrappedafterbreak}{\hbox{\char`\?}}}% 
            \lccode`\~`\!\lowercase{\def~}{\discretionary{\hbox{\char`\!}}{\Wrappedafterbreak}{\hbox{\char`\!}}}% 
            \lccode`\~`\/\lowercase{\def~}{\discretionary{\hbox{\char`\/}}{\Wrappedafterbreak}{\hbox{\char`\/}}}% 
            \catcode`\.\active
            \catcode`\,\active 
            \catcode`\;\active
            \catcode`\:\active
            \catcode`\?\active
            \catcode`\!\active
            \catcode`\/\active 
            \lccode`\~`\~ 	
        }
    \makeatother

    \let\OriginalVerbatim=\Verbatim
    \makeatletter
    \renewcommand{\Verbatim}[1][1]{%
        %\parskip\z@skip
        \sbox\Wrappedcontinuationbox {\Wrappedcontinuationsymbol}%
        \sbox\Wrappedvisiblespacebox {\FV@SetupFont\Wrappedvisiblespace}%
        \def\FancyVerbFormatLine ##1{\hsize\linewidth
            \vtop{\raggedright\hyphenpenalty\z@\exhyphenpenalty\z@
                \doublehyphendemerits\z@\finalhyphendemerits\z@
                \strut ##1\strut}%
        }%
        % If the linebreak is at a space, the latter will be displayed as visible
        % space at end of first line, and a continuation symbol starts next line.
        % Stretch/shrink are however usually zero for typewriter font.
        \def\FV@Space {%
            \nobreak\hskip\z@ plus\fontdimen3\font minus\fontdimen4\font
            \discretionary{\copy\Wrappedvisiblespacebox}{\Wrappedafterbreak}
            {\kern\fontdimen2\font}%
        }%
        
        % Allow breaks at special characters using \PYG... macros.
        \Wrappedbreaksatspecials
        % Breaks at punctuation characters . , ; ? ! and / need catcode=\active 	
        \OriginalVerbatim[#1,codes*=\Wrappedbreaksatpunct]%
    }
    \makeatother

    % Exact colors from NB
    \definecolor{incolor}{HTML}{303F9F}
    \definecolor{outcolor}{HTML}{D84315}
    \definecolor{cellborder}{HTML}{CFCFCF}
    \definecolor{cellbackground}{HTML}{F7F7F7}
    
    % prompt
    \makeatletter
    \newcommand{\boxspacing}{\kern\kvtcb@left@rule\kern\kvtcb@boxsep}
    \makeatother
    \newcommand{\prompt}[4]{
        \ttfamily\llap{{\color{#2}[#3]:\hspace{3pt}#4}}\vspace{-\baselineskip}
    }
    

    
    % Prevent overflowing lines due to hard-to-break entities
    \sloppy 
    % Setup hyperref package
    \hypersetup{
      breaklinks=true,  % so long urls are correctly broken across lines
      colorlinks=true,
      urlcolor=urlcolor,
      linkcolor=linkcolor,
      citecolor=citecolor,
      }
    % Slightly bigger margins than the latex defaults
    
    \geometry{verbose,tmargin=1in,bmargin=1in,lmargin=1in,rmargin=1in}
    
    

\begin{document}
    
    \maketitle
    
    

    
    \hypertarget{addon-calculation}{%
\section{AddOn calculation}\label{addon-calculation}}

Most of the SA-CCR logic is hidden inside the AddOn calculation. At
first it is important to define the following four data parameters:

\hypertarget{m_i}{%
\subparagraph{\texorpdfstring{\(M_i\)}{M\_i}}\label{m_i}}

\begin{quote}
Maturity of the derivative contract. If the underlying of a derivative
is another derivative - e.g.~in the case of a swaption the maturity date
of the underlying needs to be chosen.
\end{quote}

\hypertarget{s_i}{%
\subparagraph{\texorpdfstring{\(S_i\)}{S\_i}}\label{s_i}}

\begin{quote}
For interest rate and credit derivatives the start date of the time
periodreferenced by an interst rate or credit contract. If the
derivtives underlying is another interest rate or credit intsrument (eg
swaption or bond option) \(S_i\) is the start date of the underlying
instead.
\end{quote}

\hypertarget{e_i}{%
\subparagraph{\texorpdfstring{\(E_i\)}{E\_i}}\label{e_i}}

\begin{quote}
Defined as \(S_i\) but referencing the end date instead of the start
date.
\end{quote}

\hypertarget{t_i}{%
\subparagraph{\texorpdfstring{\(T_i\)}{T\_i}}\label{t_i}}

\begin{quote}
For options across all asset classes this is the latest contractual
exercise date.
\end{quote}

    \hypertarget{trade-level-adjusted-notional}{%
\subsection{Trade level adjusted
notional}\label{trade-level-adjusted-notional}}

Each trade \(i\) has a trade level adjusted notional \(d_i^a\) assigned
to it. This is calculated differently for the different asset classes.

\hypertarget{interest-rate-and-credit-derivatives}{%
\paragraph{Interest rate and credit
derivatives}\label{interest-rate-and-credit-derivatives}}

The notional of the trade is usually a well defined value in domestic
currency for interest rate and credit derivatives. It is multiplied by a
supervisory duration factor. The basic idea is, that the value of the
derivative can change more the longer the remaining

\begin{align*}
d_i &= \text{Notional}_i * SD_i \\
\\
\text{where} \qquad SD_i &=\frac{\exp\left(-0.05 * S_i\right)-\exp\left(-0.05 * E_i\right)}{0.05}
\end{align*}

\hypertarget{fx-derivatives}{%
\paragraph{FX derivatives}\label{fx-derivatives}}

While the wording in the BCBS paper is a bit more specific we will just
assume that every FX traded derivative has a USD leg and set the
notional equal the to USD notional.

\hypertarget{equity-and-commodity-derivatives}{%
\paragraph{Equity and commodity
derivatives}\label{equity-and-commodity-derivatives}}

The notional is defined as the price of the underlying. Therefore, it
fluctuates over time.

\hypertarget{notional-of-exotic-derivatives}{%
\paragraph{Notional of exotic
derivatives}\label{notional-of-exotic-derivatives}}

For more exotic derivatives which do have adjustable notionals,
resetting notionals etc. detailed handling of the notional is defined in
paragraph 158.

    Within this thesis we investigate only equity and interest rate
derivatives. For these we can make a few exemplary calculations of the
trade level adjusted notional.

For equity trades determining the trade level adjusted notional is
trivial as it always is the spot price of the underlying. As an example
consider the two trades defined below:

    \begin{tcolorbox}[breakable, size=fbox, boxrule=1pt, pad at break*=1mm,colback=cellbackground, colframe=cellborder]
\prompt{In}{incolor}{2}{\boxspacing}
\begin{Verbatim}[commandchars=\\\{\}]
\PY{c+c1}{\PYZsh{}When the strike is not set explicitly an at the money option is created with K = S(t0)}
\PY{n}{eqOption1} \PY{o}{=} \PY{n}{EquityOption}\PY{p}{(}\PY{n}{maturity} \PY{o}{=} \PY{n}{ql}\PY{o}{.}\PY{n}{Period}\PY{p}{(}\PY{l+m+mi}{1}\PY{p}{,} \PY{n}{ql}\PY{o}{.}\PY{n}{Years}\PY{p}{)}\PY{p}{,}
                         \PY{n}{tradeType}\PY{o}{=} \PY{n}{TradeType}\PY{o}{.}\PY{n}{CALL}\PY{p}{,}
                         \PY{n}{tradeDirection}\PY{o}{=} \PY{n}{TradeDirection}\PY{o}{.}\PY{n}{LONG}\PY{p}{,}
                         \PY{n}{underlying}\PY{o}{=} \PY{n}{Stock}\PY{o}{.}\PY{n}{ADS}\PY{p}{)}

\PY{n}{eqOption2} \PY{o}{=} \PY{n}{EquityOption}\PY{p}{(}\PY{n}{maturity} \PY{o}{=} \PY{n}{ql}\PY{o}{.}\PY{n}{Period}\PY{p}{(}\PY{l+m+mi}{1}\PY{p}{,}   \PY{n}{ql}\PY{o}{.}\PY{n}{Years}\PY{p}{)}\PY{p}{,}
                         \PY{n}{tradeType}\PY{o}{=} \PY{n}{TradeType}\PY{o}{.}\PY{n}{PUT}\PY{p}{,}
                         \PY{n}{tradeDirection}\PY{o}{=} \PY{n}{TradeDirection}\PY{o}{.}\PY{n}{SHORT}\PY{p}{,}
                         \PY{n}{underlying}\PY{o}{=} \PY{n}{Stock}\PY{o}{.}\PY{n}{ADS}\PY{p}{,}
                         \PY{n}{strike} \PY{o}{=} \PY{l+m+mi}{60}\PY{p}{)}
\end{Verbatim}
\end{tcolorbox}

    Let the spot price of Adidas stock be 42. Then, the adjusted notional of
\texttt{eqOption1}, an at the money call on Adidas, is 42 and the
adjusted notional of \texttt{eqOption2}, a short in the money put on
Adidas, is also 42.

    
    For interest rate derivatives such as interest rate swaps or swaptions
on the other hand, the notional is adjusted by the supervisory duration
factor. As the supervisory duration depends on \emph{S} and \emph{E} it
is important to understand how these are determined for the different
interest rate derivatives.

\begin{longtable}[]{@{}lll@{}}
\toprule
\begin{minipage}[b]{0.24\columnwidth}\raggedright
Trade Type\strut
\end{minipage} & \begin{minipage}[b]{0.32\columnwidth}\raggedright
\emph{S}\strut
\end{minipage} & \begin{minipage}[b]{0.36\columnwidth}\raggedright
\emph{E}\strut
\end{minipage}\tabularnewline
\midrule
\endhead
\begin{minipage}[t]{0.24\columnwidth}\raggedright
\textbf{Interest Rate Swap}\strut
\end{minipage} & \begin{minipage}[t]{0.32\columnwidth}\raggedright
Current date\strut
\end{minipage} & \begin{minipage}[t]{0.36\columnwidth}\raggedright
Maturity date\strut
\end{minipage}\tabularnewline
\begin{minipage}[t]{0.24\columnwidth}\raggedright
\textbf{Forward starting IRS}\strut
\end{minipage} & \begin{minipage}[t]{0.32\columnwidth}\raggedright
Start date of the underlying swap\strut
\end{minipage} & \begin{minipage}[t]{0.36\columnwidth}\raggedright
Maturity date of the underlying swap\strut
\end{minipage}\tabularnewline
\begin{minipage}[t]{0.24\columnwidth}\raggedright
\textbf{Swaption}\strut
\end{minipage} & \begin{minipage}[t]{0.32\columnwidth}\raggedright
Start date of the underlying swap\strut
\end{minipage} & \begin{minipage}[t]{0.36\columnwidth}\raggedright
Maturity data of the underlying swap\strut
\end{minipage}\tabularnewline
\bottomrule
\end{longtable}

    \hypertarget{supervisory-delta-adjustments-delta_i}{%
\subsection{\texorpdfstring{Supervisory delta adjustments:
\(\delta_i\)}{Supervisory delta adjustments: \textbackslash{}delta\_i}}\label{supervisory-delta-adjustments-delta_i}}

For linear derivatives \(\delta\) is 1 for long derivatives and -1 for
short derivatives.

For options \(\delta\) is defined as under Black-Scholes:

\begin{align*}
\delta_{\text{long Call}} &= +\Phi\left(\frac{\ln\left(P_i / K_i \right) + 0.5 * \sigma_i^2 * T_i}{\sigma_i * \sqrt{T_i}}\right) \\
\\
\text{where} \qquad \Phi &: \text{standard normal cdf} \\
\sigma_i &: \text{supervisory volatility as defined in Table 2 in paragraph 183}
\end{align*}

This delta is multiplied by -1 in case of a long Put option or a short
Call option. This formula is used for both, equity options and
swaptions.

No detail is given at this point on the delta calculation of CDO
tranches as these are not in the scope of this thesis.

    In the case of an european equity option the parametrization is quite
straight forward.

\(\sigma_i\): 1.2 is the supervisory volatility for a single stock
option

\(K_i\): The strike of the option

\(P_i\): The spot price of the underlying stock

\(T_i\): The maturity of the option

A swaption on the other hand is parametrized as follows for calculation
of its supervisory delta:

\(\sigma_i\): 0.5 is the supervisory volatility for any interst rate
option.

\(K_i\): The strike of the option is the fixed rate of the underlying
swap

\(P_i\): Is the current par rate of the underlying (forward starting)
swap

\(T_i\): The maturity of the option. Please note the difference to
\(E_i\) used for calculation of the adjusted notional, which is the
maturity of the underlying swap.

SA-CCR uses the same Black-Scholes based formula for Swaps as it uses
for Equities. It differentiates options in two dimensions. Whether they
are \emph{bought} or \emph{sold} and whether they are \emph{Call} or
\emph{Put} options (Compare paragraph 159).

SA-CCR defines an option as a call option if it rises in value as the
underlying rises in value. A fixed payer swap rises in value as the
underlying interest rate rises in value. Therefore, an option to buy a
fixed payer swap at a predetermined strike also rises in value as the
underlying interest rate rises in value. Therefore, a swaption on a
payer swap is considered a \emph{Call} under SA-CCR, while a swaption on
a receiver swap is considered a \emph{Put}.

    For the at the money option \texttt{eqOption1} that was set up above we
yield a supervisory delta adjustment of 0.7257.

    
    For an examplary short european swaption that has a par swap as
underlying (i.e.~the NPV of the swap is 0) that is set up as follows:

    \begin{tcolorbox}[breakable, size=fbox, boxrule=1pt, pad at break*=1mm,colback=cellbackground, colframe=cellborder]
\prompt{In}{incolor}{9}{\boxspacing}
\begin{Verbatim}[commandchars=\\\{\}]
\PY{n}{swap} \PY{o}{=} \PY{n}{IRS}\PY{p}{(}\PY{n}{notional}\PY{o}{=}\PY{l+m+mi}{100}\PY{p}{,}
           \PY{n}{timeToSwapStart}\PY{o}{=}\PY{n}{ql}\PY{o}{.}\PY{n}{Period}\PY{p}{(}\PY{l+m+mi}{1}\PY{p}{,} \PY{n}{ql}\PY{o}{.}\PY{n}{Years}\PY{p}{)}\PY{p}{,}
           \PY{n}{timeToSwapEnd}\PY{o}{=}\PY{n}{ql}\PY{o}{.}\PY{n}{Period}\PY{p}{(}\PY{l+m+mi}{3}\PY{p}{,} \PY{n}{ql}\PY{o}{.}\PY{n}{Years}\PY{p}{)}\PY{p}{,}
           \PY{n}{swapDirection}\PY{o}{=}\PY{n}{SwapDirection}\PY{o}{.}\PY{n}{PAYER}\PY{p}{,}
           \PY{n}{index}\PY{o}{=}\PY{n}{InterestRateIndex}\PY{o}{.}\PY{n}{EURIBOR6M}
          \PY{p}{)}

\PY{n}{swaption} \PY{o}{=} \PY{n}{Swaption}\PY{p}{(}\PY{n}{underlyingSwap}\PY{o}{=}\PY{n}{swap}\PY{p}{,}
                    \PY{n}{optionMaturity}\PY{o}{=}\PY{n}{ql}\PY{o}{.}\PY{n}{Period}\PY{p}{(}\PY{l+m+mi}{1}\PY{p}{,} \PY{n}{ql}\PY{o}{.}\PY{n}{Years}\PY{p}{)}\PY{p}{,}
                    \PY{n}{tradeDirection}\PY{o}{=}\PY{n}{TradeDirection}\PY{o}{.}\PY{n}{SHORT}\PY{p}{)}

\PY{n}{SA\PYZus{}CCR}\PY{o}{.}\PY{n}{calculate\PYZus{}sa\PYZus{}ccr\PYZus{}delta}\PY{p}{(}\PY{n}{swaption}\PY{p}{)}
\end{Verbatim}
\end{tcolorbox}

            \begin{tcolorbox}[breakable, size=fbox, boxrule=.5pt, pad at break*=1mm, opacityfill=0]
\prompt{Out}{outcolor}{9}{\boxspacing}
\begin{Verbatim}[commandchars=\\\{\}]
-0.5987063256828626
\end{Verbatim}
\end{tcolorbox}
        
    we yield a regulatory delta of -0.5987.

    
    \hypertarget{risk-horizon}{%
\subsubsection{Risk Horizon}\label{risk-horizon}}

For unmargined transaction the margining factor is

\[MF^{\text{unmargined}}_i = \sqrt{\frac{\min\left(M_i;1\text{ year}\right)}{1\text{ year}}}\]

This factor can be used to scale down a risk weight calibrated for a 1
year horizon to a shorter period.

With margining the margin period of risk (MPOR) is:

\begin{itemize}
\tightlist
\item
  10 business days for small, uncleared OTC portfolios
\item
  5 business days for cleared derivatives
\item
  20 business days for netting sets with more than 5000 transactions
  that are not with a central counterparty
\item
  and doubling this period for portfolios with outstanding disputes
\end{itemize}

The margining factor is then

\[ MF^{\text{margined}}_i = \frac{3}{2}\sqrt{\frac{MPOR_i}{1\text{ year}}} \]

At this point we need to introduce a collateral agreement object. For
simplicities sake we will not differentiate between collateral and
netting sets in this thesis. All trades that are covered by the same
collateral agreement are also admissible for netting with each other.
(Also refer to the introduction of close out netting above). To take
into account the different parameters determining the risk horizon a
couple of parameters are required to create a collateral agreement. As
an example, below we are setting up a collateral agreement for uncleared
derivatives without exchange of variation margin or initial margin.

    \begin{tcolorbox}[breakable, size=fbox, boxrule=1pt, pad at break*=1mm,colback=cellbackground, colframe=cellborder]
\prompt{In}{incolor}{17}{\boxspacing}
\begin{Verbatim}[commandchars=\\\{\}]
\PY{n}{ca} \PY{o}{=} \PY{n}{CollateralAgreement}\PY{p}{(}
        \PY{n}{margining}\PY{o}{=}\PY{n}{Margining}\PY{o}{.}\PY{n}{UNMARGINED}\PY{p}{,}
        \PY{n}{clearing}\PY{o}{=}\PY{n}{Clearing}\PY{o}{.}\PY{n}{UNCLEARED}\PY{p}{,}
        \PY{n}{tradecount}\PY{o}{=}\PY{n}{Tradecount}\PY{o}{.}\PY{n}{UNDER\PYZus{}FIVE\PYZus{}THOUSAND}\PY{p}{,}
        \PY{n}{dispute}\PY{o}{=}\PY{n}{Dispute}\PY{o}{.}\PY{n}{NO\PYZus{}OUTSTANDING\PYZus{}DISPUTES}\PY{p}{,}
        \PY{n}{threshold}\PY{o}{=}\PY{l+m+mf}{0.0}\PY{p}{,}      \PY{c+c1}{\PYZsh{}Threshold to trigger a margin call}
        \PY{n}{mta}\PY{o}{=}\PY{l+m+mf}{0.0}\PY{p}{,}            \PY{c+c1}{\PYZsh{}Minimum transfer amount for a margin call}
        \PY{n}{vm}\PY{o}{=}\PY{l+m+mf}{0.0}\PY{p}{,}             \PY{c+c1}{\PYZsh{}Variation margin balance}
        \PY{n}{posted\PYZus{}im}\PY{o}{=}\PY{l+m+mf}{0.0}\PY{p}{,}      \PY{c+c1}{\PYZsh{}posted initial margin}
        \PY{n}{received\PYZus{}im}\PY{o}{=}\PY{l+m+mf}{0.0}     \PY{c+c1}{\PYZsh{}received initial margin}
        \PY{p}{)}
\end{Verbatim}
\end{tcolorbox}

    The SA\_CCR paper provides a small exemplary portfolio in Annex 4a
Example 1. It consists of the following trades:

\begin{longtable}[]{@{}llllllll@{}}
\toprule
\begin{minipage}[b]{0.04\columnwidth}\raggedright
Trade \#\strut
\end{minipage} & \begin{minipage}[b]{0.10\columnwidth}\raggedright
Nature\strut
\end{minipage} & \begin{minipage}[b]{0.09\columnwidth}\raggedright
Residual Maturity\strut
\end{minipage} & \begin{minipage}[b]{0.07\columnwidth}\raggedright
Base Ccy\strut
\end{minipage} & \begin{minipage}[b]{0.11\columnwidth}\raggedright
Notional (tsd)\strut
\end{minipage} & \begin{minipage}[b]{0.11\columnwidth}\raggedright
Pay Leg\strut
\end{minipage} & \begin{minipage}[b]{0.13\columnwidth}\raggedright
Receive Leg\strut
\end{minipage} & \begin{minipage}[b]{0.13\columnwidth}\raggedright
Market value (tsd)\strut
\end{minipage}\tabularnewline
\midrule
\endhead
\begin{minipage}[t]{0.04\columnwidth}\raggedright
1\strut
\end{minipage} & \begin{minipage}[t]{0.10\columnwidth}\raggedright
Interest rate swap\strut
\end{minipage} & \begin{minipage}[t]{0.09\columnwidth}\raggedright
10 years\strut
\end{minipage} & \begin{minipage}[t]{0.07\columnwidth}\raggedright
USD\strut
\end{minipage} & \begin{minipage}[t]{0.11\columnwidth}\raggedright
10000\strut
\end{minipage} & \begin{minipage}[t]{0.11\columnwidth}\raggedright
Fixed\strut
\end{minipage} & \begin{minipage}[t]{0.13\columnwidth}\raggedright
Floating\strut
\end{minipage} & \begin{minipage}[t]{0.13\columnwidth}\raggedright
30\strut
\end{minipage}\tabularnewline
\begin{minipage}[t]{0.04\columnwidth}\raggedright
2\strut
\end{minipage} & \begin{minipage}[t]{0.10\columnwidth}\raggedright
Interest rate swap\strut
\end{minipage} & \begin{minipage}[t]{0.09\columnwidth}\raggedright
4 years\strut
\end{minipage} & \begin{minipage}[t]{0.07\columnwidth}\raggedright
USD\strut
\end{minipage} & \begin{minipage}[t]{0.11\columnwidth}\raggedright
10000\strut
\end{minipage} & \begin{minipage}[t]{0.11\columnwidth}\raggedright
Floating\strut
\end{minipage} & \begin{minipage}[t]{0.13\columnwidth}\raggedright
Fixed\strut
\end{minipage} & \begin{minipage}[t]{0.13\columnwidth}\raggedright
-20\strut
\end{minipage}\tabularnewline
\begin{minipage}[t]{0.04\columnwidth}\raggedright
3\strut
\end{minipage} & \begin{minipage}[t]{0.10\columnwidth}\raggedright
European Swaption\strut
\end{minipage} & \begin{minipage}[t]{0.09\columnwidth}\raggedright
1 into 10 years\strut
\end{minipage} & \begin{minipage}[t]{0.07\columnwidth}\raggedright
EUR\strut
\end{minipage} & \begin{minipage}[t]{0.11\columnwidth}\raggedright
5000\strut
\end{minipage} & \begin{minipage}[t]{0.11\columnwidth}\raggedright
Floating\strut
\end{minipage} & \begin{minipage}[t]{0.13\columnwidth}\raggedright
Fixed\strut
\end{minipage} & \begin{minipage}[t]{0.13\columnwidth}\raggedright
50\strut
\end{minipage}\tabularnewline
\bottomrule
\end{longtable}

To set up this examplary portfolio we need to find fixed rates for the
swaps and underlying swaps to match the desired market values.

    Through optimization and using the market data of the 10th of May 2019
the fixed rates to match the market values in Example 1 were identified.

    For trade 1 the matching fixed rate is 2.3754\%, for trade 1 it is
2.2108\% and for the underlying swap of trade 3 it is 0.1610\%

    

    % Add a bibliography block to the postdoc
    
    
    
\end{document}
