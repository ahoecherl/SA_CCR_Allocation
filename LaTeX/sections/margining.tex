\documentclass[../Thesis_AHoecherl.tex]{subfiles}

\begin{document}
\section{Margining}\label{Margining}

In the context of derivatives, margining refers to the process of posting and receiving collateral in the form of cash or securities to reduce the counterparty credit risk associated with the derivatives.

Counterparty credit risk refers to the risk of a default of the counterparty of a derivative. Derivatives are contracts between two institutions which give raise to future cash flows dependent of the performance of its underlying. These future cash flows may be at risk if the counterparty defaults during the lifetime of the derivative. 

Over the past decades several measures were established in the derivatives markets to mitigate counterparty credit risk.

The most impactful measure is close-out netting. Close-out netting is a contractual agreement of two counterparties how their bilateral derivative contracts have to be settled if one of the counterparties defaults. With close-out netting, in case one of the two counterparties defaults all derivatives which are yet to mature are immediately closed out based on the current market value. The market values of the individual derivatives are summed up and the netted amount needs to be paid by whichever party is trailing across the portfolio. In case of a default this close-out netting procedure takes priority over all other claims of creditors against the defaulted counterparty. Close-out netting has two major advantages. First, the non-defaulting counterparty only suffers a counterparty credit loss, if it is ahead across the entire portfolio of currently running derivatives with the defaulted party. Just having a positive market value on a few derivatives does not necessarily result in a counterparty credit loss. Secondly, the immediate close-out of the open derivatives of the defaulted counterparty greatly facilitates unwinding its portfolio. A disadvantage of close-out netting is, that it may proof difficult to find an objective market value of the derivatives that have to be closed out - especially in a stressed market environment, which is likely to be present if e.g. a large investment bank defaults. The contractual obligation to perform close-out netting is agreed upon in a master agreement, which was introduced to the derivatives market by ISDA in 1985. Currently, close out netting is in effect for almost all traded derivatives and it will implicitly always be assumed to be in effect throughout this thesis. More details on close-out netting may be found in \cite[Chapter~5]{gregory2015xva}.

The second most effective measure in mitigating counterparty credit risk is the exchange of variation margin. If the obligation to post variation margin is agreed as part of a master agreement the accrued mark-to-market of the derivative portfolio has to be collateralized by the trailing counterparty. This measure effectively resets counterparty credit risk to zero for both parties every time a variation margin payment is made or the exchanged variation margin is adjusted to the current market value of the portfolio. The exchange of variation margin was common but not a given in the inter-bank market before the financial crisis of 2008. After the crisis it has become commonplace in the interbank market and recently has even been mandated by regulators\footnote{In the European Union the exchange of Variation Margin for inter bank bilateral OTC derivatives is compulsory since September of 2016 for large banks or March of 2017 for smaller banks.}. Non-financial counterparties oftentimes do not collateralize their derivatives since they are not mandated to do so, shy away from the operational burden and have a harder time funding the significant amount of cash necessary to cover the current mark-to-market value of their entire derivatives portfolio. Collateralizing a derivatives portfolio not only significantly reduces \gls{CCR} but also significantly alters how the remaining \gls{CCR} behaves. The \gls{CCR} of a collateralized portfolio may rather be driven by the terms of the \gls{CSA}\todo{Explain what a CSA is} or residual phenomenon such as collateral spikes\todo{Quote something regarding collateral Spikes} than by the underlying instruments.

As a final measure, initial margin can be exchanged. Historically, initial margin was a collateral amount that was calculated and exchanged once at the inception of a new derivative and held until maturity - hence the name \emph{initial} margin. One common formulation used w.r.t. initial margin, which has also found its way into regulatory documents \todo{cite location in regulatory document that does this} is that initial margin is collateral, that - in contrast to variation margin - is not based on the \gls{MtM} of a portfolio or derivative. The idea behind initial margin is that it secures the counterparties against losses that can incur between the last time variation margin has been exchanged prior to a default until the original position has been restored. This time period is referred to as the margin period of risk and this time period results as the sum of
\begin{enumerate}
    \item The contractually agreed regular frequency of variation margin exchanges
    \item The time it takes from a counterparty not complying with a margin call to ascertain that the counterparty has indeed defaulted
    \item The necessary time to reopen the defaulted derivatives with new counterparties to rehedge the bank and thereby restoring the risk profile held prior to the default of the counterparty\label{step3}
\end{enumerate}
Initial margin should cover the gap between the \gls{MtM} of the derivatives with the defaulted counterparty when variation margin was last exchanged to the price for which the derivatives are reopened in step \ref{step3} under the assumption that the market has moved unfavorably during the \gls{MPoR}. In the context of CCR, \emph{unfavorable} means that the banks position would have increased in value throughout the margin period of risk and therefore the market price for which the bank repurchases the derivatives is higher than the value at which variation margin was exchanged last.

A more comprehensive introduction to counterparty credit risk and its reduction through netting and margining may be found in chapters four through six of \cite{gregory2015xva}.

\subsection{Market structure and associated margining approaches}

The derivative market is divided into exchange traded derivatives, cleared OTC derivatives and uncleared bilateral derivatives. Uncleared bilateral derivatives can either be uncollateralized, collateralized with \gls{VM} or collateralized with \gls{IM} and \gls{VM}. Below, these five counterparty relations are briefly introduced. They are ordered w.r.t. their associated counterparty credit risk.

\begin{description}
    \item[Uncollateralized bilateral derivatives\label{itm:Uncollateralized bilateral derivatives}] Derivatives are arranged between two counterparties without involvement of a third party. No collateral is exchanged at any point, only the contractual cashflows of the derivatives are exchanged. The \gls{CCR} is very high as the entire \gls{MtM} of the portfolio is at risk. Since no margin is posted at all, the \gls{MPoR} is the maturity of the traded derivatives and can therefore easily eclipse multiple years. \gls{IM} posted, \gls{IM} received and \gls{VM} are zero at all times.
    \item[Bilateral derivatives collateralized with VM\label{itm:Bilateral derivatives collateralized with VM}] The \gls{CCR} is still significant. When the counterparty defaults the bank can suffer unmitigated losses for a couple of days until it can rebuild its position. The \gls{MtM} of the portfolio is collateralized with \gls{VM}. \gls{VM} exchange is subject to contract parameters such as the threshold, minimum transfer amount or the exchange frequency. Values of these parameters impact how well \gls{CCR} is mitigated.
    \item[Bilateral derivatives collateralized with \gls{VM} and \gls{IM}] Counterparty credit risk is low. Only in edge cases is it possible that the counterparties credit losses surpass the available collateral. An \gls{MPoR} of at least 10 days is mandated by the regulator. The IM is calculated with an internal or standardized bilateral initial margin model. Posted and received IM are recalculated daily. \gls{VM} exchange obeys the same mechanics as for \ref{itm:Bilateral derivatives collateralized with VM}.
    \item[Cleared OTC derivatives\label{itm:Cleared OTC derivatives}] Derivatives are initially arranged bilaterally between two counterparties and then cleared by a CCP. The CCP takes over positions in case of a default of either party mitigating any \gls{CCR} in the traditional sense. The bank has no direct counterparty credit risk. It may however suffer losses to its clearing fund contribution if another Member of the CCP defaults. The \gls{MPoR} is mandated by the regulator to be five days. The initial margin that is posted by the bank to the \gls{CCP} is calculated by the \gls{CCP} with his proprietary internal initial margin model. In line with the assumption that the CCP can not default, the CCP does not post IM to its clearing members. Therefore, the IM received from the perspective of the bank is always zero. The daily \gls{PnL} of the portfolio is exchanged as VM between the CCP and the Bank. 
    \item[Exchange traded derivatives\label{itm:Exchange traded derivatives}] Banks enter positions in exchange traded derivatives listed by a CCP. Positions are matched by the CCP and the counterparties of a transaction remain anonymous to each other. Associated \gls{CCR} and margining is largely the same as for \ref{itm:Cleared OTC derivatives} but the \gls{MPoR} is generally below five days since it is assumed that positions in exchange traded derivatives can be closed faster than in cleared \gls{OTC} derivatives. The used internal initial margin model may differ e.g. since the regulator requires coverage of a 99\% quantile instead of the 99.5\% mandated for cleared \gls{OTC} derivatives.
\end{description}

According to \cite[Figure~3.2]{gregory2015xva} based on notional 9\% of derivatives are exchange traded, 55\% are cleared OTC derivatives and 36\% are uncleared OTC derivatives. It has to be noted that these figures are from 2014 and it can be assumed, that the fraction of cleared OTC derivatives has increased since then at the expense of the fraction of uncleared OTC derivatives. The reason for this is, that regulators have incentivized or even mandated the clearing of simpler OTC derivatives over the course of the last years. In connection with this development the large CCPs such as Eurex or the LCH have extended the product range for which they offer OTC clearing in recent years.

% Exchange traded or listed derivatives are usually futures or options on a limited scope of underlyings which can be directly traded at a clearing house at a bid and ask quoted by the clearing house. As seen in figure \ref{Breakdown by underlying of exchange trades derivatives} usual underlyings are single name equities, equity indices, commodities or baskets of liquid government bonds. 
% \begin{figure}
%     \label{Breakdown by underlying of exchange trades derivatives}
%     \centering
%     \def\svgwidth{\columnwidth}
%     \input{Graphics/ListedVolumesEurex_201812_extractListedOverview.pdf_tex}
%     \caption[Breakdown by underlying of exchange trades derivatives]{Breakdown by underlying of listed contracts derivatives at Eurex in December 2018. Interest rate derivatives are almost exclusively futures and options on baskets of German government bonds. Graph has been extracted from \cite{EurexDec2018}.}
% \end{figure}
% The clearing house acts as a mediator to match market participants which are willing to buy and sell and market participants will not be informed with whom they have been matched. All market participants exchange variation margin with the CCP and have to post a unilateral initial margin to the CCP. This initial margin is calculated by the CCP and is usually calculated using a portfolio based historical simulation VaR model. If a market participants defaults to CCP inherits its positions and needs to auction them to other market participants. The purpose of the initial margin is to cover mark-to-market losses which the portfolio suffers until it can be fully auctioned or otherwise closed by the CCP.

% Cleared OTC derivatives are initially bilaterally agreed upon by two market participants. If the parametrization of the derivative is sufficiently common CCPs may offer to clear it and the market participants may choose or even be forced by regulation to clear it using a qualifying CCP. OTC clearing is especially common for simple interest rate derivatives such as interest rate swaps, forward rate agreements or cross currency swaps and for some credit derivatives. Similarly to exchange traded derivatives the CCP will charge both participants an initial margin and will also regularly determine the derivatives present value and execute the exchange of variation margin between the two market participants. If one of the market participants defaults, the CCP inherits the positions of the defaulted party and can again use the initial margin posted by the defaulted party to cover mark-to-market losses until it has auctioned or closed the inherited positions \cite{EurexDMP}.

% Assuming that the CCP cannot default itself, cleared OTC derivatives and listed derivatives bear no counterparty credit risk for the market participants.

% As of December 2018 common derivatives that are still uncleared are structured equity derivatives, swaptions and many other interest rate derivatives with optionality, derivatives on securitizations, swaps with unusual cashflow structures or in uncommon currencies and most FX derivatives \cite{BISOTCStats}.



% What are uncleared OTC derivatives


\end{document}