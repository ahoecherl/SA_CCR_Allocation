\documentclass[../Thesis_AHoecherl.tex]{subfiles}

\begin{document}
    \chapter{Detailed documentation of results}\label{Detailed documentation of results}
    The purpose of this section is to document how the results presented primarily in section \ref{Results} were computed.
    For the purpose of this thesis a library has been implemented in Python and Java that can be found at \cite{Hoecherl2020}.
    This implemented library has a very similar architecture to the one presented in \ref{sec:Architectural blueprint for SA-CCR allocation}.
    Using this library, results for this thesis were computed within Jupyter Notebooks. These Jupyter notebooks are presented in this section. 
    Some, more technical notebook cells are not displayed for brevity.  


    \section{Detailed documentation of results presented in section \ref{Allocation of initial margin}}

        

    
    \hypertarget{isda-simm-homogeneity-for-portfolio-of-a-single-trade}{%
\subsection{ISDA SIMM homogeneity for portfolio of a single
trade}\label{isda-simm-homogeneity-for-portfolio-of-a-single-trade}}

    We want to showcase that the concentration risk addOn breaks homogeneity
of the ISDA SIMM risk measure. The concentration threshold for USD
interest rate risk is 230Mn USD per Basis Point change. Considering that
IRS trades roughly have a delta of one against the interest rate this
means that a trade with a notional of
\(\frac{230\text{Mn}}{0.0001} = 2300\text{Bn}\) and a maturity of 1 year
would incur a risk above the threshold. If the maturity increases to 10
years a notional of roughly 230Bn should be enough to exceed the
concentration threshold.

    \begin{tcolorbox}[breakable, size=fbox, boxrule=1pt, pad at break*=1mm,colback=cellbackground, colframe=cellborder]
\prompt{In}{incolor}{In}{\boxspacing}
\begin{Verbatim}[commandchars=\\\{\}]
\PY{n}{notional} \PY{o}{=} \PY{l+m+mi}{200000000000}

\PY{n}{irs} \PY{o}{=} \PY{n}{IRS}\PY{p}{(}\PY{n}{notional} \PY{o}{=} \PY{n}{notional}\PY{p}{,}
          \PY{n}{timeToSwapStart}\PY{o}{=}\PY{n}{ql}\PY{o}{.}\PY{n}{Period}\PY{p}{(}\PY{l+m+mi}{2}\PY{p}{,} \PY{n}{ql}\PY{o}{.}\PY{n}{Days}\PY{p}{)}\PY{p}{,}
          \PY{n}{timeToSwapEnd}\PY{o}{=}\PY{n}{ql}\PY{o}{.}\PY{n}{Period}\PY{p}{(}\PY{l+m+mi}{10}\PY{p}{,} \PY{n}{ql}\PY{o}{.}\PY{n}{Years}\PY{p}{)}\PY{p}{,}
          \PY{n}{swapDirection}\PY{o}{=}\PY{n}{SwapDirection}\PY{o}{.}\PY{n}{PAYER}\PY{p}{,}
          \PY{n}{index} \PY{o}{=} \PY{n}{InterestRateIndex}\PY{o}{.}\PY{n}{USDLIBOR3M}\PY{p}{)}

\PY{n}{simm\PYZus{}sensis} \PY{o}{=} \PY{n}{irs}\PY{o}{.}\PY{n}{get\PYZus{}simm\PYZus{}sensis}\PY{p}{(}\PY{p}{)}
\PY{n}{ir\PYZus{}delta} \PY{o}{=} \PY{n+nb}{sum}\PY{p}{(}\PY{p}{[}\PY{n+nb}{float}\PY{p}{(}\PY{n}{entry}\PY{p}{[}\PY{l+s+s1}{\PYZsq{}}\PY{l+s+s1}{amountUSD}\PY{l+s+s1}{\PYZsq{}}\PY{p}{]}\PY{p}{)} \PY{k}{for} \PY{n}{entry} \PY{o+ow}{in} \PY{n}{simm\PYZus{}sensis} \PY{k}{if} \PY{n}{entry}\PY{p}{[}\PY{l+s+s1}{\PYZsq{}}\PY{l+s+s1}{riskType}\PY{l+s+s1}{\PYZsq{}}\PY{p}{]} \PY{o}{==} \PY{l+s+s1}{\PYZsq{}}\PY{l+s+s1}{Risk\PYZus{}IRCurve}\PY{l+s+s1}{\PYZsq{}}\PY{p}{]}\PY{p}{)}
\PY{n}{ir\PYZus{}delta}
\end{Verbatim}
\end{tcolorbox}

            \begin{tcolorbox}[breakable, size=fbox, boxrule=.5pt, pad at break*=1mm, opacityfill=0]
\prompt{Out}{outcolor}{Out}{\boxspacing}
\begin{Verbatim}[commandchars=\\\{\}]
177293785.56303406
\end{Verbatim}
\end{tcolorbox}
        
    The trade has an aggregated delta sensitivity against a 1BP move of the
USD interest rate of 177,293,786 USD. With the available market data the
delta of the IRS appears to be slightly lower than one.

    
    We create a collateral agreement with associated ISDA SIMM and SA CCR
model and load the irs in the portfolio. The collateral agreement uses
ISDA SIMM for IM calculation.

    \begin{tcolorbox}[breakable, size=fbox, boxrule=1pt, pad at break*=1mm,colback=cellbackground, colframe=cellborder]
\prompt{In}{incolor}{In}{\boxspacing}
\begin{Verbatim}[commandchars=\\\{\}]
\PY{n}{ca} \PY{o}{=} \PY{n}{CollateralAgreement}\PY{p}{(}\PY{n}{initialMargining}\PY{o}{=}\PY{n}{InitialMargining}\PY{o}{.}\PY{n}{SIMM}\PY{p}{,}
                         \PY{n}{margin\PYZus{}currency}\PY{o}{=}\PY{n}{Currency}\PY{o}{.}\PY{n}{USD}\PY{p}{)}
\PY{n}{ca}\PY{o}{.}\PY{n}{link\PYZus{}sa\PYZus{}ccr\PYZus{}instance}\PY{p}{(}\PY{n}{SA\PYZus{}CCR}\PY{p}{(}\PY{n}{ca}\PY{p}{)}\PY{p}{)}
\PY{n}{ca}\PY{o}{.}\PY{n}{add\PYZus{}trades}\PY{p}{(}\PY{n}{irs}\PY{p}{)}
\end{Verbatim}
\end{tcolorbox}

    We now want to investigate if the ISDA SIMM exhibits homogeneity for
this single trade portfolio. To do so we test if

\begin{align*}
f\left(\alpha \mathbf{x}\right) = \alpha^k f\left(\mathbf{x}\right) \\
        \text{for } \alpha >0
\end{align*}

holds. We test for \(0<\alpha\leq 2\) with an increment size of 0.01.

    \begin{tcolorbox}[breakable, size=fbox, boxrule=1pt, pad at break*=1mm,colback=cellbackground, colframe=cellborder]
\prompt{In}{incolor}{In}{\boxspacing}
\begin{Verbatim}[commandchars=\\\{\}]
\PY{n}{bumps} \PY{o}{=} \PY{n}{arange}\PY{p}{(}\PY{l+m+mi}{0}\PY{p}{,}\PY{l+m+mf}{2.01}\PY{p}{,}\PY{l+m+mf}{0.01}\PY{p}{)}
\PY{n}{resultDataframe} \PY{o}{=} \PY{n}{pd}\PY{o}{.}\PY{n}{DataFrame}\PY{p}{(}\PY{n}{columns} \PY{o}{=} \PY{p}{[}\PY{l+s+s1}{\PYZsq{}}\PY{l+s+s1}{USD}\PY{l+s+s1}{\PYZsq{}}\PY{p}{,}\PY{l+s+s1}{\PYZsq{}}\PY{l+s+s1}{k}\PY{l+s+s1}{\PYZsq{}}\PY{p}{,}\PY{l+s+s1}{\PYZsq{}}\PY{l+s+s1}{Legend}\PY{l+s+s1}{\PYZsq{}}\PY{p}{]}\PY{p}{)}
\end{Verbatim}
\end{tcolorbox}

    We create a utility function that supports IM, VM and Collateral
although we are just exploring IM right now.

    \begin{tcolorbox}[breakable, size=fbox, boxrule=1pt, pad at break*=1mm,colback=cellbackground, colframe=cellborder]
\prompt{In}{incolor}{In}{\boxspacing}
\begin{Verbatim}[commandchars=\\\{\}]
\PY{k}{def} \PY{n+nf}{bump\PYZus{}and\PYZus{}get\PYZus{}results}\PY{p}{(}\PY{n}{bumpsize}\PY{p}{,} \PY{n}{trade}\PY{p}{,} \PY{n}{collateralagreement}\PY{p}{)}\PY{p}{:}
    \PY{n}{record} \PY{o}{=} \PY{p}{\PYZob{}}\PY{p}{\PYZcb{}}
    \PY{n}{record}\PY{p}{[}\PY{l+s+s1}{\PYZsq{}}\PY{l+s+s1}{Bumpsize}\PY{l+s+s1}{\PYZsq{}}\PY{p}{]}\PY{o}{=}\PY{n}{bumpsize}
    \PY{n}{bumped\PYZus{}copy} \PY{o}{=} \PY{n}{trade}\PY{o}{.}\PY{n}{get\PYZus{}bumped\PYZus{}copy}\PY{p}{(}\PY{n}{rel\PYZus{}bump\PYZus{}size}\PY{o}{=}\PY{n}{bumpsize}\PY{o}{\PYZhy{}}\PY{l+m+mi}{1}\PY{p}{)}
    \PY{n}{collateralagreement}\PY{o}{.}\PY{n}{remove\PYZus{}all\PYZus{}trades}\PY{p}{(}\PY{p}{)}
    \PY{n}{collateralagreement}\PY{o}{.}\PY{n}{add\PYZus{}trades}\PY{p}{(}\PY{n}{bumped\PYZus{}copy}\PY{p}{)}
    \PY{n}{record}\PY{p}{[}\PY{l+s+s1}{\PYZsq{}}\PY{l+s+s1}{IM}\PY{l+s+s1}{\PYZsq{}}\PY{p}{]} \PY{o}{=} \PY{n}{collateralagreement}\PY{o}{.}\PY{n}{get\PYZus{}im\PYZus{}model}\PY{p}{(}\PY{p}{)}\PY{o}{.}\PY{n}{get\PYZus{}risk\PYZus{}measure}\PY{p}{(}\PY{p}{)}
    \PY{n}{record}\PY{p}{[}\PY{l+s+s1}{\PYZsq{}}\PY{l+s+s1}{VM}\PY{l+s+s1}{\PYZsq{}}\PY{p}{]} \PY{o}{=} \PY{n}{collateralagreement}\PY{o}{.}\PY{n}{get\PYZus{}vm\PYZus{}model}\PY{p}{(}\PY{p}{)}\PY{o}{.}\PY{n}{get\PYZus{}risk\PYZus{}measure}\PY{p}{(}\PY{p}{)}
    \PY{n}{record}\PY{p}{[}\PY{l+s+s1}{\PYZsq{}}\PY{l+s+s1}{Collateral}\PY{l+s+s1}{\PYZsq{}}\PY{p}{]} \PY{o}{=} \PY{n}{collateralagreement}\PY{o}{.}\PY{n}{get\PYZus{}C}\PY{p}{(}\PY{p}{)}
    \PY{n}{collateralagreement}\PY{o}{.}\PY{n}{remove\PYZus{}all\PYZus{}trades}\PY{p}{(}\PY{p}{)}
    \PY{n}{collateralagreement}\PY{o}{.}\PY{n}{add\PYZus{}trades}\PY{p}{(}\PY{n}{trade}\PY{p}{)}
    \PY{k}{return} \PY{n}{record}
\end{Verbatim}
\end{tcolorbox}

    \begin{tcolorbox}[breakable, size=fbox, boxrule=1pt, pad at break*=1mm,colback=cellbackground, colframe=cellborder]
\prompt{In}{incolor}{In}{\boxspacing}
\begin{Verbatim}[commandchars=\\\{\}]
\PY{k}{for} \PY{n}{bump} \PY{o+ow}{in} \PY{n}{bumps}\PY{p}{:}
    \PY{n}{result} \PY{o}{=} \PY{n}{bump\PYZus{}and\PYZus{}get\PYZus{}results}\PY{p}{(}\PY{n}{bump}\PY{p}{,} \PY{n}{irs}\PY{p}{,} \PY{n}{ca}\PY{p}{)}
    \PY{n}{im\PYZus{}record} \PY{o}{=} \PY{p}{\PYZob{}}\PY{l+s+s1}{\PYZsq{}}\PY{l+s+s1}{X}\PY{l+s+s1}{\PYZsq{}}\PY{p}{:} \PY{n}{result}\PY{p}{[}\PY{l+s+s1}{\PYZsq{}}\PY{l+s+s1}{Bumpsize}\PY{l+s+s1}{\PYZsq{}}\PY{p}{]}\PY{p}{,}
                 \PY{l+s+s1}{\PYZsq{}}\PY{l+s+s1}{Y}\PY{l+s+s1}{\PYZsq{}}\PY{p}{:} \PY{n}{result}\PY{p}{[}\PY{l+s+s1}{\PYZsq{}}\PY{l+s+s1}{IM}\PY{l+s+s1}{\PYZsq{}}\PY{p}{]}\PY{p}{,}
                 \PY{l+s+s1}{\PYZsq{}}\PY{l+s+s1}{Legend}\PY{l+s+s1}{\PYZsq{}}\PY{p}{:} \PY{l+s+s1}{\PYZsq{}}\PY{l+s+s1}{IM}\PY{l+s+s1}{\PYZsq{}}\PY{p}{\PYZcb{}}
    \PY{n}{resultDataframe} \PY{o}{=} \PY{n}{resultDataframe}\PY{o}{.}\PY{n}{append}\PY{p}{(}\PY{n}{im\PYZus{}record}\PY{p}{,} \PY{n}{ignore\PYZus{}index}\PY{o}{=}\PY{k+kc}{True}\PY{p}{)}
\end{Verbatim}
\end{tcolorbox}

    \begin{tcolorbox}[breakable, size=fbox, boxrule=1pt, pad at break*=1mm,colback=cellbackground, colframe=cellborder]
\prompt{In}{incolor}{In}{\boxspacing}
\begin{Verbatim}[commandchars=\\\{\}]
\PY{n}{fig} \PY{o}{=} \PY{n}{px}\PY{o}{.}\PY{n}{line}\PY{p}{(}\PY{n}{resultDataframe}\PY{p}{[}\PY{n}{resultDataframe}\PY{o}{.}\PY{n}{Legend} \PY{o}{==} \PY{l+s+s1}{\PYZsq{}}\PY{l+s+s1}{IM}\PY{l+s+s1}{\PYZsq{}}\PY{p}{]}\PY{p}{,} \PY{n}{x}\PY{o}{=}\PY{l+s+s1}{\PYZsq{}}\PY{l+s+s1}{X}\PY{l+s+s1}{\PYZsq{}}\PY{p}{,} \PY{n}{y}\PY{o}{=}\PY{l+s+s1}{\PYZsq{}}\PY{l+s+s1}{Y}\PY{l+s+s1}{\PYZsq{}}\PY{p}{,} \PY{n}{color}\PY{o}{=}\PY{l+s+s1}{\PYZsq{}}\PY{l+s+s1}{Legend}\PY{l+s+s1}{\PYZsq{}}\PY{p}{,} \PY{n}{line\PYZus{}dash}\PY{o}{=}\PY{l+s+s1}{\PYZsq{}}\PY{l+s+s1}{Legend}\PY{l+s+s1}{\PYZsq{}}\PY{p}{)}
\PY{n}{fig}\PY{o}{.}\PY{n}{update\PYZus{}layout}\PY{p}{(}\PY{n}{xaxis\PYZus{}title} \PY{o}{=} \PY{l+s+s1}{\PYZsq{}}\PY{l+s+s1}{a}\PY{l+s+s1}{\PYZsq{}}\PY{p}{,} \PY{n}{yaxis\PYZus{}title} \PY{o}{=} \PY{l+s+s1}{\PYZsq{}}\PY{l+s+s1}{USD}\PY{l+s+s1}{\PYZsq{}}\PY{p}{)}
\PY{n}{img\PYZus{}bytes} \PY{o}{=} \PY{n}{fig}\PY{o}{.}\PY{n}{to\PYZus{}image}\PY{p}{(}\PY{n+nb}{format}\PY{o}{=}\PY{l+s+s1}{\PYZsq{}}\PY{l+s+s1}{jpeg}\PY{l+s+s1}{\PYZsq{}}\PY{p}{)}
\PY{n}{Image}\PY{p}{(}\PY{n}{img\PYZus{}bytes}\PY{p}{)}
\end{Verbatim}
\end{tcolorbox}
 
            
            \begin{tcolorbox}[breakable, size=fbox, boxrule=.5pt, pad at break*=1mm, opacityfill=0]
\prompt{Out}{outcolor}{Out}{\boxspacing}    
    \begin{center}
    \adjustimage{max size={0.9\linewidth}{0.9\paperheight}}{JupyterNotebooksFull/output_11_0.jpg}
    \end{center}
    { \hspace*{\fill} \\}
\end{tcolorbox}
    

    As can be seen, ISDA SIMM does not exhibit homogeneity. Further
discussion of this in section \ref{Allocation of initial margin}.




    \documentclass[11pt]{article}

    \usepackage[breakable]{tcolorbox}
    \usepackage{parskip} % Stop auto-indenting (to mimic markdown behaviour)
    
    \usepackage{iftex}
    \ifPDFTeX
    	\usepackage[T1]{fontenc}
    	\usepackage{mathpazo}
    \else
    	\usepackage{fontspec}
    \fi

    % Basic figure setup, for now with no caption control since it's done
    % automatically by Pandoc (which extracts ![](path) syntax from Markdown).
    \usepackage{graphicx}
    % Maintain compatibility with old templates. Remove in nbconvert 6.0
    \let\Oldincludegraphics\includegraphics
    % Ensure that by default, figures have no caption (until we provide a
    % proper Figure object with a Caption API and a way to capture that
    % in the conversion process - todo).
    \usepackage{caption}
    \DeclareCaptionFormat{nocaption}{}
    \captionsetup{format=nocaption,aboveskip=0pt,belowskip=0pt}

    \usepackage[Export]{adjustbox} % Used to constrain images to a maximum size
    \adjustboxset{max size={0.9\linewidth}{0.9\paperheight}}
    \usepackage{float}
    \floatplacement{figure}{H} % forces figures to be placed at the correct location
    \usepackage{xcolor} % Allow colors to be defined
    \usepackage{enumerate} % Needed for markdown enumerations to work
    \usepackage{geometry} % Used to adjust the document margins
    \usepackage{amsmath} % Equations
    \usepackage{amssymb} % Equations
    \usepackage{textcomp} % defines textquotesingle
    % Hack from http://tex.stackexchange.com/a/47451/13684:
    \AtBeginDocument{%
        \def\PYZsq{\textquotesingle}% Upright quotes in Pygmentized code
    }
    \usepackage{upquote} % Upright quotes for verbatim code
    \usepackage{eurosym} % defines \euro
    \usepackage[mathletters]{ucs} % Extended unicode (utf-8) support
    \usepackage{fancyvrb} % verbatim replacement that allows latex
    \usepackage{grffile} % extends the file name processing of package graphics 
                         % to support a larger range
    \makeatletter % fix for grffile with XeLaTeX
    \def\Gread@@xetex#1{%
      \IfFileExists{"\Gin@base".bb}%
      {\Gread@eps{\Gin@base.bb}}%
      {\Gread@@xetex@aux#1}%
    }
    \makeatother

    % The hyperref package gives us a pdf with properly built
    % internal navigation ('pdf bookmarks' for the table of contents,
    % internal cross-reference links, web links for URLs, etc.)
    \usepackage{hyperref}
    % The default LaTeX title has an obnoxious amount of whitespace. By default,
    % titling removes some of it. It also provides customization options.
    \usepackage{titling}
    \usepackage{longtable} % longtable support required by pandoc >1.10
    \usepackage{booktabs}  % table support for pandoc > 1.12.2
    \usepackage[inline]{enumitem} % IRkernel/repr support (it uses the enumerate* environment)
    \usepackage[normalem]{ulem} % ulem is needed to support strikethroughs (\sout)
                                % normalem makes italics be italics, not underlines
    \usepackage{mathrsfs}
    

    
    % Colors for the hyperref package
    \definecolor{urlcolor}{rgb}{0,.145,.698}
    \definecolor{linkcolor}{rgb}{.71,0.21,0.01}
    \definecolor{citecolor}{rgb}{.12,.54,.11}

    % ANSI colors
    \definecolor{ansi-black}{HTML}{3E424D}
    \definecolor{ansi-black-intense}{HTML}{282C36}
    \definecolor{ansi-red}{HTML}{E75C58}
    \definecolor{ansi-red-intense}{HTML}{B22B31}
    \definecolor{ansi-green}{HTML}{00A250}
    \definecolor{ansi-green-intense}{HTML}{007427}
    \definecolor{ansi-yellow}{HTML}{DDB62B}
    \definecolor{ansi-yellow-intense}{HTML}{B27D12}
    \definecolor{ansi-blue}{HTML}{208FFB}
    \definecolor{ansi-blue-intense}{HTML}{0065CA}
    \definecolor{ansi-magenta}{HTML}{D160C4}
    \definecolor{ansi-magenta-intense}{HTML}{A03196}
    \definecolor{ansi-cyan}{HTML}{60C6C8}
    \definecolor{ansi-cyan-intense}{HTML}{258F8F}
    \definecolor{ansi-white}{HTML}{C5C1B4}
    \definecolor{ansi-white-intense}{HTML}{A1A6B2}
    \definecolor{ansi-default-inverse-fg}{HTML}{FFFFFF}
    \definecolor{ansi-default-inverse-bg}{HTML}{000000}

    % commands and environments needed by pandoc snippets
    % extracted from the output of `pandoc -s`
    \providecommand{\tightlist}{%
      \setlength{\itemsep}{0pt}\setlength{\parskip}{0pt}}
    \DefineVerbatimEnvironment{Highlighting}{Verbatim}{commandchars=\\\{\}}
    % Add ',fontsize=\small' for more characters per line
    \newenvironment{Shaded}{}{}
    \newcommand{\KeywordTok}[1]{\textcolor[rgb]{0.00,0.44,0.13}{\textbf{{#1}}}}
    \newcommand{\DataTypeTok}[1]{\textcolor[rgb]{0.56,0.13,0.00}{{#1}}}
    \newcommand{\DecValTok}[1]{\textcolor[rgb]{0.25,0.63,0.44}{{#1}}}
    \newcommand{\BaseNTok}[1]{\textcolor[rgb]{0.25,0.63,0.44}{{#1}}}
    \newcommand{\FloatTok}[1]{\textcolor[rgb]{0.25,0.63,0.44}{{#1}}}
    \newcommand{\CharTok}[1]{\textcolor[rgb]{0.25,0.44,0.63}{{#1}}}
    \newcommand{\StringTok}[1]{\textcolor[rgb]{0.25,0.44,0.63}{{#1}}}
    \newcommand{\CommentTok}[1]{\textcolor[rgb]{0.38,0.63,0.69}{\textit{{#1}}}}
    \newcommand{\OtherTok}[1]{\textcolor[rgb]{0.00,0.44,0.13}{{#1}}}
    \newcommand{\AlertTok}[1]{\textcolor[rgb]{1.00,0.00,0.00}{\textbf{{#1}}}}
    \newcommand{\FunctionTok}[1]{\textcolor[rgb]{0.02,0.16,0.49}{{#1}}}
    \newcommand{\RegionMarkerTok}[1]{{#1}}
    \newcommand{\ErrorTok}[1]{\textcolor[rgb]{1.00,0.00,0.00}{\textbf{{#1}}}}
    \newcommand{\NormalTok}[1]{{#1}}
    
    % Additional commands for more recent versions of Pandoc
    \newcommand{\ConstantTok}[1]{\textcolor[rgb]{0.53,0.00,0.00}{{#1}}}
    \newcommand{\SpecialCharTok}[1]{\textcolor[rgb]{0.25,0.44,0.63}{{#1}}}
    \newcommand{\VerbatimStringTok}[1]{\textcolor[rgb]{0.25,0.44,0.63}{{#1}}}
    \newcommand{\SpecialStringTok}[1]{\textcolor[rgb]{0.73,0.40,0.53}{{#1}}}
    \newcommand{\ImportTok}[1]{{#1}}
    \newcommand{\DocumentationTok}[1]{\textcolor[rgb]{0.73,0.13,0.13}{\textit{{#1}}}}
    \newcommand{\AnnotationTok}[1]{\textcolor[rgb]{0.38,0.63,0.69}{\textbf{\textit{{#1}}}}}
    \newcommand{\CommentVarTok}[1]{\textcolor[rgb]{0.38,0.63,0.69}{\textbf{\textit{{#1}}}}}
    \newcommand{\VariableTok}[1]{\textcolor[rgb]{0.10,0.09,0.49}{{#1}}}
    \newcommand{\ControlFlowTok}[1]{\textcolor[rgb]{0.00,0.44,0.13}{\textbf{{#1}}}}
    \newcommand{\OperatorTok}[1]{\textcolor[rgb]{0.40,0.40,0.40}{{#1}}}
    \newcommand{\BuiltInTok}[1]{{#1}}
    \newcommand{\ExtensionTok}[1]{{#1}}
    \newcommand{\PreprocessorTok}[1]{\textcolor[rgb]{0.74,0.48,0.00}{{#1}}}
    \newcommand{\AttributeTok}[1]{\textcolor[rgb]{0.49,0.56,0.16}{{#1}}}
    \newcommand{\InformationTok}[1]{\textcolor[rgb]{0.38,0.63,0.69}{\textbf{\textit{{#1}}}}}
    \newcommand{\WarningTok}[1]{\textcolor[rgb]{0.38,0.63,0.69}{\textbf{\textit{{#1}}}}}
    
    
    % Define a nice break command that doesn't care if a line doesn't already
    % exist.
    \def\br{\hspace*{\fill} \\* }
    % Math Jax compatibility definitions
    \def\gt{>}
    \def\lt{<}
    \let\Oldtex\TeX
    \let\Oldlatex\LaTeX
    \renewcommand{\TeX}{\textrm{\Oldtex}}
    \renewcommand{\LaTeX}{\textrm{\Oldlatex}}
    % Document parameters
    % Document title
    \title{SA-CCR and ISDA SIMM under perfect hedge}
    
    
    
    
    
% Pygments definitions
\makeatletter
\def\PY@reset{\let\PY@it=\relax \let\PY@bf=\relax%
    \let\PY@ul=\relax \let\PY@tc=\relax%
    \let\PY@bc=\relax \let\PY@ff=\relax}
\def\PY@tok#1{\csname PY@tok@#1\endcsname}
\def\PY@toks#1+{\ifx\relax#1\empty\else%
    \PY@tok{#1}\expandafter\PY@toks\fi}
\def\PY@do#1{\PY@bc{\PY@tc{\PY@ul{%
    \PY@it{\PY@bf{\PY@ff{#1}}}}}}}
\def\PY#1#2{\PY@reset\PY@toks#1+\relax+\PY@do{#2}}

\expandafter\def\csname PY@tok@w\endcsname{\def\PY@tc##1{\textcolor[rgb]{0.73,0.73,0.73}{##1}}}
\expandafter\def\csname PY@tok@c\endcsname{\let\PY@it=\textit\def\PY@tc##1{\textcolor[rgb]{0.25,0.50,0.50}{##1}}}
\expandafter\def\csname PY@tok@cp\endcsname{\def\PY@tc##1{\textcolor[rgb]{0.74,0.48,0.00}{##1}}}
\expandafter\def\csname PY@tok@k\endcsname{\let\PY@bf=\textbf\def\PY@tc##1{\textcolor[rgb]{0.00,0.50,0.00}{##1}}}
\expandafter\def\csname PY@tok@kp\endcsname{\def\PY@tc##1{\textcolor[rgb]{0.00,0.50,0.00}{##1}}}
\expandafter\def\csname PY@tok@kt\endcsname{\def\PY@tc##1{\textcolor[rgb]{0.69,0.00,0.25}{##1}}}
\expandafter\def\csname PY@tok@o\endcsname{\def\PY@tc##1{\textcolor[rgb]{0.40,0.40,0.40}{##1}}}
\expandafter\def\csname PY@tok@ow\endcsname{\let\PY@bf=\textbf\def\PY@tc##1{\textcolor[rgb]{0.67,0.13,1.00}{##1}}}
\expandafter\def\csname PY@tok@nb\endcsname{\def\PY@tc##1{\textcolor[rgb]{0.00,0.50,0.00}{##1}}}
\expandafter\def\csname PY@tok@nf\endcsname{\def\PY@tc##1{\textcolor[rgb]{0.00,0.00,1.00}{##1}}}
\expandafter\def\csname PY@tok@nc\endcsname{\let\PY@bf=\textbf\def\PY@tc##1{\textcolor[rgb]{0.00,0.00,1.00}{##1}}}
\expandafter\def\csname PY@tok@nn\endcsname{\let\PY@bf=\textbf\def\PY@tc##1{\textcolor[rgb]{0.00,0.00,1.00}{##1}}}
\expandafter\def\csname PY@tok@ne\endcsname{\let\PY@bf=\textbf\def\PY@tc##1{\textcolor[rgb]{0.82,0.25,0.23}{##1}}}
\expandafter\def\csname PY@tok@nv\endcsname{\def\PY@tc##1{\textcolor[rgb]{0.10,0.09,0.49}{##1}}}
\expandafter\def\csname PY@tok@no\endcsname{\def\PY@tc##1{\textcolor[rgb]{0.53,0.00,0.00}{##1}}}
\expandafter\def\csname PY@tok@nl\endcsname{\def\PY@tc##1{\textcolor[rgb]{0.63,0.63,0.00}{##1}}}
\expandafter\def\csname PY@tok@ni\endcsname{\let\PY@bf=\textbf\def\PY@tc##1{\textcolor[rgb]{0.60,0.60,0.60}{##1}}}
\expandafter\def\csname PY@tok@na\endcsname{\def\PY@tc##1{\textcolor[rgb]{0.49,0.56,0.16}{##1}}}
\expandafter\def\csname PY@tok@nt\endcsname{\let\PY@bf=\textbf\def\PY@tc##1{\textcolor[rgb]{0.00,0.50,0.00}{##1}}}
\expandafter\def\csname PY@tok@nd\endcsname{\def\PY@tc##1{\textcolor[rgb]{0.67,0.13,1.00}{##1}}}
\expandafter\def\csname PY@tok@s\endcsname{\def\PY@tc##1{\textcolor[rgb]{0.73,0.13,0.13}{##1}}}
\expandafter\def\csname PY@tok@sd\endcsname{\let\PY@it=\textit\def\PY@tc##1{\textcolor[rgb]{0.73,0.13,0.13}{##1}}}
\expandafter\def\csname PY@tok@si\endcsname{\let\PY@bf=\textbf\def\PY@tc##1{\textcolor[rgb]{0.73,0.40,0.53}{##1}}}
\expandafter\def\csname PY@tok@se\endcsname{\let\PY@bf=\textbf\def\PY@tc##1{\textcolor[rgb]{0.73,0.40,0.13}{##1}}}
\expandafter\def\csname PY@tok@sr\endcsname{\def\PY@tc##1{\textcolor[rgb]{0.73,0.40,0.53}{##1}}}
\expandafter\def\csname PY@tok@ss\endcsname{\def\PY@tc##1{\textcolor[rgb]{0.10,0.09,0.49}{##1}}}
\expandafter\def\csname PY@tok@sx\endcsname{\def\PY@tc##1{\textcolor[rgb]{0.00,0.50,0.00}{##1}}}
\expandafter\def\csname PY@tok@m\endcsname{\def\PY@tc##1{\textcolor[rgb]{0.40,0.40,0.40}{##1}}}
\expandafter\def\csname PY@tok@gh\endcsname{\let\PY@bf=\textbf\def\PY@tc##1{\textcolor[rgb]{0.00,0.00,0.50}{##1}}}
\expandafter\def\csname PY@tok@gu\endcsname{\let\PY@bf=\textbf\def\PY@tc##1{\textcolor[rgb]{0.50,0.00,0.50}{##1}}}
\expandafter\def\csname PY@tok@gd\endcsname{\def\PY@tc##1{\textcolor[rgb]{0.63,0.00,0.00}{##1}}}
\expandafter\def\csname PY@tok@gi\endcsname{\def\PY@tc##1{\textcolor[rgb]{0.00,0.63,0.00}{##1}}}
\expandafter\def\csname PY@tok@gr\endcsname{\def\PY@tc##1{\textcolor[rgb]{1.00,0.00,0.00}{##1}}}
\expandafter\def\csname PY@tok@ge\endcsname{\let\PY@it=\textit}
\expandafter\def\csname PY@tok@gs\endcsname{\let\PY@bf=\textbf}
\expandafter\def\csname PY@tok@gp\endcsname{\let\PY@bf=\textbf\def\PY@tc##1{\textcolor[rgb]{0.00,0.00,0.50}{##1}}}
\expandafter\def\csname PY@tok@go\endcsname{\def\PY@tc##1{\textcolor[rgb]{0.53,0.53,0.53}{##1}}}
\expandafter\def\csname PY@tok@gt\endcsname{\def\PY@tc##1{\textcolor[rgb]{0.00,0.27,0.87}{##1}}}
\expandafter\def\csname PY@tok@err\endcsname{\def\PY@bc##1{\setlength{\fboxsep}{0pt}\fcolorbox[rgb]{1.00,0.00,0.00}{1,1,1}{\strut ##1}}}
\expandafter\def\csname PY@tok@kc\endcsname{\let\PY@bf=\textbf\def\PY@tc##1{\textcolor[rgb]{0.00,0.50,0.00}{##1}}}
\expandafter\def\csname PY@tok@kd\endcsname{\let\PY@bf=\textbf\def\PY@tc##1{\textcolor[rgb]{0.00,0.50,0.00}{##1}}}
\expandafter\def\csname PY@tok@kn\endcsname{\let\PY@bf=\textbf\def\PY@tc##1{\textcolor[rgb]{0.00,0.50,0.00}{##1}}}
\expandafter\def\csname PY@tok@kr\endcsname{\let\PY@bf=\textbf\def\PY@tc##1{\textcolor[rgb]{0.00,0.50,0.00}{##1}}}
\expandafter\def\csname PY@tok@bp\endcsname{\def\PY@tc##1{\textcolor[rgb]{0.00,0.50,0.00}{##1}}}
\expandafter\def\csname PY@tok@fm\endcsname{\def\PY@tc##1{\textcolor[rgb]{0.00,0.00,1.00}{##1}}}
\expandafter\def\csname PY@tok@vc\endcsname{\def\PY@tc##1{\textcolor[rgb]{0.10,0.09,0.49}{##1}}}
\expandafter\def\csname PY@tok@vg\endcsname{\def\PY@tc##1{\textcolor[rgb]{0.10,0.09,0.49}{##1}}}
\expandafter\def\csname PY@tok@vi\endcsname{\def\PY@tc##1{\textcolor[rgb]{0.10,0.09,0.49}{##1}}}
\expandafter\def\csname PY@tok@vm\endcsname{\def\PY@tc##1{\textcolor[rgb]{0.10,0.09,0.49}{##1}}}
\expandafter\def\csname PY@tok@sa\endcsname{\def\PY@tc##1{\textcolor[rgb]{0.73,0.13,0.13}{##1}}}
\expandafter\def\csname PY@tok@sb\endcsname{\def\PY@tc##1{\textcolor[rgb]{0.73,0.13,0.13}{##1}}}
\expandafter\def\csname PY@tok@sc\endcsname{\def\PY@tc##1{\textcolor[rgb]{0.73,0.13,0.13}{##1}}}
\expandafter\def\csname PY@tok@dl\endcsname{\def\PY@tc##1{\textcolor[rgb]{0.73,0.13,0.13}{##1}}}
\expandafter\def\csname PY@tok@s2\endcsname{\def\PY@tc##1{\textcolor[rgb]{0.73,0.13,0.13}{##1}}}
\expandafter\def\csname PY@tok@sh\endcsname{\def\PY@tc##1{\textcolor[rgb]{0.73,0.13,0.13}{##1}}}
\expandafter\def\csname PY@tok@s1\endcsname{\def\PY@tc##1{\textcolor[rgb]{0.73,0.13,0.13}{##1}}}
\expandafter\def\csname PY@tok@mb\endcsname{\def\PY@tc##1{\textcolor[rgb]{0.40,0.40,0.40}{##1}}}
\expandafter\def\csname PY@tok@mf\endcsname{\def\PY@tc##1{\textcolor[rgb]{0.40,0.40,0.40}{##1}}}
\expandafter\def\csname PY@tok@mh\endcsname{\def\PY@tc##1{\textcolor[rgb]{0.40,0.40,0.40}{##1}}}
\expandafter\def\csname PY@tok@mi\endcsname{\def\PY@tc##1{\textcolor[rgb]{0.40,0.40,0.40}{##1}}}
\expandafter\def\csname PY@tok@il\endcsname{\def\PY@tc##1{\textcolor[rgb]{0.40,0.40,0.40}{##1}}}
\expandafter\def\csname PY@tok@mo\endcsname{\def\PY@tc##1{\textcolor[rgb]{0.40,0.40,0.40}{##1}}}
\expandafter\def\csname PY@tok@ch\endcsname{\let\PY@it=\textit\def\PY@tc##1{\textcolor[rgb]{0.25,0.50,0.50}{##1}}}
\expandafter\def\csname PY@tok@cm\endcsname{\let\PY@it=\textit\def\PY@tc##1{\textcolor[rgb]{0.25,0.50,0.50}{##1}}}
\expandafter\def\csname PY@tok@cpf\endcsname{\let\PY@it=\textit\def\PY@tc##1{\textcolor[rgb]{0.25,0.50,0.50}{##1}}}
\expandafter\def\csname PY@tok@c1\endcsname{\let\PY@it=\textit\def\PY@tc##1{\textcolor[rgb]{0.25,0.50,0.50}{##1}}}
\expandafter\def\csname PY@tok@cs\endcsname{\let\PY@it=\textit\def\PY@tc##1{\textcolor[rgb]{0.25,0.50,0.50}{##1}}}

\def\PYZbs{\char`\\}
\def\PYZus{\char`\_}
\def\PYZob{\char`\{}
\def\PYZcb{\char`\}}
\def\PYZca{\char`\^}
\def\PYZam{\char`\&}
\def\PYZlt{\char`\<}
\def\PYZgt{\char`\>}
\def\PYZsh{\char`\#}
\def\PYZpc{\char`\%}
\def\PYZdl{\char`\$}
\def\PYZhy{\char`\-}
\def\PYZsq{\char`\'}
\def\PYZdq{\char`\"}
\def\PYZti{\char`\~}
% for compatibility with earlier versions
\def\PYZat{@}
\def\PYZlb{[}
\def\PYZrb{]}
\makeatother


    % For linebreaks inside Verbatim environment from package fancyvrb. 
    \makeatletter
        \newbox\Wrappedcontinuationbox 
        \newbox\Wrappedvisiblespacebox 
        \newcommand*\Wrappedvisiblespace {\textcolor{red}{\textvisiblespace}} 
        \newcommand*\Wrappedcontinuationsymbol {\textcolor{red}{\llap{\tiny$\m@th\hookrightarrow$}}} 
        \newcommand*\Wrappedcontinuationindent {3ex } 
        \newcommand*\Wrappedafterbreak {\kern\Wrappedcontinuationindent\copy\Wrappedcontinuationbox} 
        % Take advantage of the already applied Pygments mark-up to insert 
        % potential linebreaks for TeX processing. 
        %        {, <, #, %, $, ' and ": go to next line. 
        %        _, }, ^, &, >, - and ~: stay at end of broken line. 
        % Use of \textquotesingle for straight quote. 
        \newcommand*\Wrappedbreaksatspecials {% 
            \def\PYGZus{\discretionary{\char`\_}{\Wrappedafterbreak}{\char`\_}}% 
            \def\PYGZob{\discretionary{}{\Wrappedafterbreak\char`\{}{\char`\{}}% 
            \def\PYGZcb{\discretionary{\char`\}}{\Wrappedafterbreak}{\char`\}}}% 
            \def\PYGZca{\discretionary{\char`\^}{\Wrappedafterbreak}{\char`\^}}% 
            \def\PYGZam{\discretionary{\char`\&}{\Wrappedafterbreak}{\char`\&}}% 
            \def\PYGZlt{\discretionary{}{\Wrappedafterbreak\char`\<}{\char`\<}}% 
            \def\PYGZgt{\discretionary{\char`\>}{\Wrappedafterbreak}{\char`\>}}% 
            \def\PYGZsh{\discretionary{}{\Wrappedafterbreak\char`\#}{\char`\#}}% 
            \def\PYGZpc{\discretionary{}{\Wrappedafterbreak\char`\%}{\char`\%}}% 
            \def\PYGZdl{\discretionary{}{\Wrappedafterbreak\char`\$}{\char`\$}}% 
            \def\PYGZhy{\discretionary{\char`\-}{\Wrappedafterbreak}{\char`\-}}% 
            \def\PYGZsq{\discretionary{}{\Wrappedafterbreak\textquotesingle}{\textquotesingle}}% 
            \def\PYGZdq{\discretionary{}{\Wrappedafterbreak\char`\"}{\char`\"}}% 
            \def\PYGZti{\discretionary{\char`\~}{\Wrappedafterbreak}{\char`\~}}% 
        } 
        % Some characters . , ; ? ! / are not pygmentized. 
        % This macro makes them "active" and they will insert potential linebreaks 
        \newcommand*\Wrappedbreaksatpunct {% 
            \lccode`\~`\.\lowercase{\def~}{\discretionary{\hbox{\char`\.}}{\Wrappedafterbreak}{\hbox{\char`\.}}}% 
            \lccode`\~`\,\lowercase{\def~}{\discretionary{\hbox{\char`\,}}{\Wrappedafterbreak}{\hbox{\char`\,}}}% 
            \lccode`\~`\;\lowercase{\def~}{\discretionary{\hbox{\char`\;}}{\Wrappedafterbreak}{\hbox{\char`\;}}}% 
            \lccode`\~`\:\lowercase{\def~}{\discretionary{\hbox{\char`\:}}{\Wrappedafterbreak}{\hbox{\char`\:}}}% 
            \lccode`\~`\?\lowercase{\def~}{\discretionary{\hbox{\char`\?}}{\Wrappedafterbreak}{\hbox{\char`\?}}}% 
            \lccode`\~`\!\lowercase{\def~}{\discretionary{\hbox{\char`\!}}{\Wrappedafterbreak}{\hbox{\char`\!}}}% 
            \lccode`\~`\/\lowercase{\def~}{\discretionary{\hbox{\char`\/}}{\Wrappedafterbreak}{\hbox{\char`\/}}}% 
            \catcode`\.\active
            \catcode`\,\active 
            \catcode`\;\active
            \catcode`\:\active
            \catcode`\?\active
            \catcode`\!\active
            \catcode`\/\active 
            \lccode`\~`\~ 	
        }
    \makeatother

    \let\OriginalVerbatim=\Verbatim
    \makeatletter
    \renewcommand{\Verbatim}[1][1]{%
        %\parskip\z@skip
        \sbox\Wrappedcontinuationbox {\Wrappedcontinuationsymbol}%
        \sbox\Wrappedvisiblespacebox {\FV@SetupFont\Wrappedvisiblespace}%
        \def\FancyVerbFormatLine ##1{\hsize\linewidth
            \vtop{\raggedright\hyphenpenalty\z@\exhyphenpenalty\z@
                \doublehyphendemerits\z@\finalhyphendemerits\z@
                \strut ##1\strut}%
        }%
        % If the linebreak is at a space, the latter will be displayed as visible
        % space at end of first line, and a continuation symbol starts next line.
        % Stretch/shrink are however usually zero for typewriter font.
        \def\FV@Space {%
            \nobreak\hskip\z@ plus\fontdimen3\font minus\fontdimen4\font
            \discretionary{\copy\Wrappedvisiblespacebox}{\Wrappedafterbreak}
            {\kern\fontdimen2\font}%
        }%
        
        % Allow breaks at special characters using \PYG... macros.
        \Wrappedbreaksatspecials
        % Breaks at punctuation characters . , ; ? ! and / need catcode=\active 	
        \OriginalVerbatim[#1,codes*=\Wrappedbreaksatpunct]%
    }
    \makeatother

    % Exact colors from NB
    \definecolor{incolor}{HTML}{303F9F}
    \definecolor{outcolor}{HTML}{D84315}
    \definecolor{cellborder}{HTML}{CFCFCF}
    \definecolor{cellbackground}{HTML}{F7F7F7}
    
    % prompt
    \makeatletter
    \newcommand{\boxspacing}{\kern\kvtcb@left@rule\kern\kvtcb@boxsep}
    \makeatother
    \newcommand{\prompt}[4]{
        \ttfamily\llap{{\color{#2}[#3]:\hspace{3pt}#4}}\vspace{-\baselineskip}
    }
    

    
    % Prevent overflowing lines due to hard-to-break entities
    \sloppy 
    % Setup hyperref package
    \hypersetup{
      breaklinks=true,  % so long urls are correctly broken across lines
      colorlinks=true,
      urlcolor=urlcolor,
      linkcolor=linkcolor,
      citecolor=citecolor,
      }
    % Slightly bigger margins than the latex defaults
    
    \geometry{verbose,tmargin=1in,bmargin=1in,lmargin=1in,rmargin=1in}
    
    

\begin{document}
    
    \maketitle
    
    

    
    \hypertarget{sa-ccr-and-isda-simm-euler-allocation-under-a-perfect-hedge}{%
\subsection{SA-CCR and ISDA SIMM Euler allocation under a perfect
hedge}\label{sa-ccr-and-isda-simm-euler-allocation-under-a-perfect-hedge}}

The goal is to create a portfolio with a perfect hedge and see if and
under which circumstances EAD allocation is still possible. We load two
perfectly offsetting IRS (one payer, one receiver). To avoid the
unrealistic case of a zero IM and EAD portfolio we add an unrelated
equity option into the portfolio.

    \begin{tcolorbox}[breakable, size=fbox, boxrule=1pt, pad at break*=1mm,colback=cellbackground, colframe=cellborder]
\prompt{In}{incolor}{2}{\boxspacing}
\begin{Verbatim}[commandchars=\\\{\}]
\PY{n}{IRS\PYZus{}pay} \PY{o}{=} \PY{n}{IRS}\PY{p}{(}\PY{n}{notional}\PY{o}{=}\PY{l+m+mi}{100000000}\PY{p}{,}
              \PY{n}{timeToSwapStart}\PY{o}{=}\PY{n}{ql}\PY{o}{.}\PY{n}{Period}\PY{p}{(}\PY{l+m+mi}{2}\PY{p}{,} \PY{n}{ql}\PY{o}{.}\PY{n}{Days}\PY{p}{)}\PY{p}{,}
              \PY{n}{timeToSwapEnd}\PY{o}{=}\PY{n}{ql}\PY{o}{.}\PY{n}{Period}\PY{p}{(}\PY{l+m+mi}{10}\PY{p}{,} \PY{n}{ql}\PY{o}{.}\PY{n}{Years}\PY{p}{)}\PY{p}{,}
              \PY{n}{swapDirection}\PY{o}{=}\PY{n}{SwapDirection}\PY{o}{.}\PY{n}{PAYER}\PY{p}{,}
              \PY{n}{index}\PY{o}{=}\PY{n}{InterestRateIndex}\PY{o}{.}\PY{n}{USDLIBOR3M}\PY{p}{)}

\PY{n}{IRS\PYZus{}rec} \PY{o}{=} \PY{n}{IRS}\PY{p}{(}\PY{n}{notional}\PY{o}{=}\PY{l+m+mi}{100000000}\PY{p}{,}
              \PY{n}{timeToSwapStart}\PY{o}{=}\PY{n}{ql}\PY{o}{.}\PY{n}{Period}\PY{p}{(}\PY{l+m+mi}{2}\PY{p}{,} \PY{n}{ql}\PY{o}{.}\PY{n}{Days}\PY{p}{)}\PY{p}{,}
              \PY{n}{timeToSwapEnd}\PY{o}{=}\PY{n}{ql}\PY{o}{.}\PY{n}{Period}\PY{p}{(}\PY{l+m+mi}{10}\PY{p}{,} \PY{n}{ql}\PY{o}{.}\PY{n}{Years}\PY{p}{)}\PY{p}{,}
              \PY{n}{swapDirection}\PY{o}{=}\PY{n}{SwapDirection}\PY{o}{.}\PY{n}{RECEIVER}\PY{p}{,}
              \PY{n}{index}\PY{o}{=}\PY{n}{InterestRateIndex}\PY{o}{.}\PY{n}{USDLIBOR3M}\PY{p}{)}

\PY{n}{eqOpt} \PY{o}{=} \PY{n}{EquityOption}\PY{p}{(}\PY{n}{notional} \PY{o}{=} \PY{l+m+mi}{1000000}\PY{p}{,}
                     \PY{n}{underlying}\PY{o}{=}\PY{n}{Stock}\PY{o}{.}\PY{n}{ADS}\PY{p}{,}
                     \PY{n}{tradeType} \PY{o}{=} \PY{n}{TradeType}\PY{o}{.}\PY{n}{CALL}\PY{p}{,}
                     \PY{n}{tradeDirection} \PY{o}{=} \PY{n}{TradeDirection}\PY{o}{.}\PY{n}{LONG}\PY{p}{,}
                     \PY{n}{maturity} \PY{o}{=} \PY{n}{ql}\PY{o}{.}\PY{n}{Period}\PY{p}{(}\PY{l+m+mi}{1}\PY{p}{,} \PY{n}{ql}\PY{o}{.}\PY{n}{Years}\PY{p}{)}\PY{p}{)}

\PY{n}{ca} \PY{o}{=} \PY{n}{CollateralAgreement}\PY{p}{(}\PY{p}{)}
\PY{n}{ca}\PY{o}{.}\PY{n}{link\PYZus{}sa\PYZus{}ccr\PYZus{}instance}\PY{p}{(}\PY{n}{SA\PYZus{}CCR}\PY{p}{(}\PY{n}{ca}\PY{p}{)}\PY{p}{)}
\end{Verbatim}
\end{tcolorbox}

    \begin{tcolorbox}[breakable, size=fbox, boxrule=1pt, pad at break*=1mm,colback=cellbackground, colframe=cellborder]
\prompt{In}{incolor}{3}{\boxspacing}
\begin{Verbatim}[commandchars=\\\{\}]
\PY{n}{ca}\PY{o}{.}\PY{n}{add\PYZus{}trades}\PY{p}{(}\PY{p}{[}\PY{n}{IRS\PYZus{}pay}\PY{p}{,} \PY{n}{IRS\PYZus{}rec}\PY{p}{,} \PY{n}{eqOpt}\PY{p}{]}\PY{p}{)}
\PY{n+nb}{print}\PY{p}{(}\PY{n}{ca}\PY{o}{.}\PY{n}{get\PYZus{}sa\PYZus{}ccr\PYZus{}model}\PY{p}{(}\PY{p}{)}\PY{o}{.}\PY{n}{get\PYZus{}risk\PYZus{}measure}\PY{p}{(}\PY{p}{)}\PY{p}{)}
\PY{n+nb}{print}\PY{p}{(}\PY{n}{ca}\PY{o}{.}\PY{n}{get\PYZus{}im\PYZus{}model}\PY{p}{(}\PY{p}{)}\PY{o}{.}\PY{n}{get\PYZus{}risk\PYZus{}measure}\PY{p}{(}\PY{p}{)}\PY{p}{)}
\end{Verbatim}
\end{tcolorbox}

    \begin{Verbatim}[commandchars=\\\{\}]
1315355.6739375147
6952717.387110085
    \end{Verbatim}

    We now create an Euler allocator which can be used to perform a
numerical Euler allocation of the ISDA-SIMM IM or the SA-CCR EAD risk
measure. The allocator can be set to use forward, backward or central
differentiation. We will see that the differentiation approach makes a
big difference for this perfectly hedged portfolio.

    \begin{tcolorbox}[breakable, size=fbox, boxrule=1pt, pad at break*=1mm,colback=cellbackground, colframe=cellborder]
\prompt{In}{incolor}{4}{\boxspacing}
\begin{Verbatim}[commandchars=\\\{\}]
\PY{n}{eulerAllocator} \PY{o}{=} \PY{n}{EulerAllocator}\PY{p}{(}\PY{n}{ca}\PY{p}{)}
\PY{n}{im\PYZus{}alloc\PYZus{}forward} \PY{o}{=} \PY{n}{eulerAllocator}\PY{o}{.}\PY{n}{allocate\PYZus{}im}\PY{p}{(}\PY{p}{)}
\PY{n}{saccr\PYZus{}alloc\PYZus{}forward} \PY{o}{=} \PY{n}{eulerAllocator}\PY{o}{.}\PY{n}{allocate\PYZus{}ead}\PY{p}{(}\PY{p}{)}

\PY{n}{eulerAllocator}\PY{o}{.}\PY{n}{fdApproach2}\PY{o}{=}\PY{n}{FdApproach2}\PY{o}{.}\PY{n}{Central}
\PY{n}{im\PYZus{}alloc\PYZus{}central} \PY{o}{=} \PY{n}{eulerAllocator}\PY{o}{.}\PY{n}{allocate\PYZus{}im}\PY{p}{(}\PY{p}{)}
\PY{n}{saccr\PYZus{}alloc\PYZus{}central} \PY{o}{=} \PY{n}{eulerAllocator}\PY{o}{.}\PY{n}{allocate\PYZus{}ead}\PY{p}{(}\PY{p}{)}

\PY{n}{eulerAllocator}\PY{o}{.}\PY{n}{fdApproach2}\PY{o}{=}\PY{n}{FdApproach2}\PY{o}{.}\PY{n}{Backward}
\PY{n}{im\PYZus{}alloc\PYZus{}backward} \PY{o}{=} \PY{n}{eulerAllocator}\PY{o}{.}\PY{n}{allocate\PYZus{}im}\PY{p}{(}\PY{p}{)}
\PY{n}{saccr\PYZus{}alloc\PYZus{}backward} \PY{o}{=} \PY{n}{eulerAllocator}\PY{o}{.}\PY{n}{allocate\PYZus{}ead}\PY{p}{(}\PY{p}{)}
\end{Verbatim}
\end{tcolorbox}

    Below the resulting allocation for the IM is displayed. The allocation
only exhibits nativ additivity when using the central difference
approach since then the allocated values sum up to the IM value of
6952717.39 USD.

    
    \begin{tcolorbox}[breakable, size=fbox, boxrule=1pt, pad at break*=1mm,colback=cellbackground, colframe=cellborder]
\prompt{In}{incolor}{7}{\boxspacing}
\begin{Verbatim}[commandchars=\\\{\}]
\PY{n}{display\PYZus{}table}\PY{p}{(}\PY{n}{im\PYZus{}alloc\PYZus{}forward}\PY{p}{,} \PY{n}{im\PYZus{}alloc\PYZus{}central}\PY{p}{,} \PY{n}{im\PYZus{}alloc\PYZus{}backward}\PY{p}{)}
\end{Verbatim}
\end{tcolorbox}

            \begin{tcolorbox}[breakable, size=fbox, boxrule=.5pt, pad at break*=1mm, opacityfill=0]
\prompt{Out}{outcolor}{7}{\boxspacing}
\begin{Verbatim}[commandchars=\\\{\}]
                       Backward       Central       Forward
IRS\_Long          -4.518969e+06 -2.421718e-01  4.518969e+06
IRS\_Short         -4.519001e+06 -6.323463e+01  4.518937e+06
EquityOption\_Long  6.952717e+06  6.952717e+06  6.952717e+06
Sum               -2.085253e+06  6.952654e+06  1.599062e+07
\end{Verbatim}
\end{tcolorbox}
        
    Below the resulting allocation for the EAD is displayed. The allocation
only exhibits nativ additivity when using the central difference
approach since then the allocated values sum up to the IM value of
1315355.67 USD.

    
    \begin{tcolorbox}[breakable, size=fbox, boxrule=1pt, pad at break*=1mm,colback=cellbackground, colframe=cellborder]
\prompt{In}{incolor}{9}{\boxspacing}
\begin{Verbatim}[commandchars=\\\{\}]
\PY{n}{display\PYZus{}table}\PY{p}{(}\PY{n}{saccr\PYZus{}alloc\PYZus{}forward}\PY{p}{,} \PY{n}{saccr\PYZus{}alloc\PYZus{}central}\PY{p}{,} \PY{n}{saccr\PYZus{}alloc\PYZus{}backward}\PY{p}{)}
\end{Verbatim}
\end{tcolorbox}

            \begin{tcolorbox}[breakable, size=fbox, boxrule=.5pt, pad at break*=1mm, opacityfill=0]
\prompt{Out}{outcolor}{9}{\boxspacing}
\begin{Verbatim}[commandchars=\\\{\}]
                       Backward       Central       Forward
IRS\_Long          -1.884133e+05  4.845671e-02  1.884134e+05
IRS\_Short         -1.884070e+05  1.265202e+01  1.884196e+05
EquityOption\_Long  1.315356e+06  1.315356e+06  1.315356e+06
Sum                9.385354e+05  1.315368e+06  1.692189e+06
\end{Verbatim}
\end{tcolorbox}
        
    The reason for the Euler allocation not working is that the SA-CCR is
not differentiable in case of a perfect hedge. This can be shown by
plotting the function SA-CCR w.r.t. the position size in the three
trades.

    \begin{tcolorbox}[breakable, size=fbox, boxrule=1pt, pad at break*=1mm,colback=cellbackground, colframe=cellborder]
\prompt{In}{incolor}{11}{\boxspacing}
\begin{Verbatim}[commandchars=\\\{\}]
\PY{n}{bumps} \PY{o}{=} \PY{n}{arange}\PY{p}{(}\PY{o}{\PYZhy{}}\PY{l+m+mf}{0.05}\PY{p}{,} \PY{l+m+mf}{0.06}\PY{p}{,} \PY{l+m+mf}{0.01}\PY{p}{)}
\end{Verbatim}
\end{tcolorbox}

    \begin{tcolorbox}[breakable, size=fbox, boxrule=1pt, pad at break*=1mm,colback=cellbackground, colframe=cellborder]
\prompt{In}{incolor}{12}{\boxspacing}
\begin{Verbatim}[commandchars=\\\{\}]
\PY{k}{def} \PY{n+nf}{bump\PYZus{}one\PYZus{}trade\PYZus{}and\PYZus{}return\PYZus{}diff}\PY{p}{(}\PY{n}{bump}\PY{p}{,} \PY{n}{trade}\PY{p}{:} \PY{n}{Trade}\PY{p}{,} \PY{n}{ca}\PY{p}{:} \PY{n}{CollateralAgreement}\PY{p}{,} \PY{n}{method}\PY{p}{)}\PY{p}{:}
    \PY{n}{base} \PY{o}{=} \PY{n}{method}\PY{p}{(}\PY{p}{)}
    \PY{n}{ca}\PY{o}{.}\PY{n}{remove\PYZus{}trades}\PY{p}{(}\PY{n}{trade}\PY{p}{)}
    \PY{n}{bumped\PYZus{}trade} \PY{o}{=} \PY{n}{trade}\PY{o}{.}\PY{n}{get\PYZus{}bumped\PYZus{}copy}\PY{p}{(}\PY{n}{rel\PYZus{}bump\PYZus{}size}\PY{o}{=}\PY{n}{bump}\PY{p}{)}
    \PY{n}{ca}\PY{o}{.}\PY{n}{add\PYZus{}trades}\PY{p}{(}\PY{n}{bumped\PYZus{}trade}\PY{p}{)}
    \PY{n}{result} \PY{o}{=} \PY{n}{method}\PY{p}{(}\PY{p}{)}
    \PY{n}{ca}\PY{o}{.}\PY{n}{remove\PYZus{}trades}\PY{p}{(}\PY{n}{bumped\PYZus{}trade}\PY{p}{)}
    \PY{n}{ca}\PY{o}{.}\PY{n}{add\PYZus{}trades}\PY{p}{(}\PY{n}{trade}\PY{p}{)}
    \PY{k}{return} \PY{n}{result}\PY{o}{\PYZhy{}}\PY{n}{base}
\end{Verbatim}
\end{tcolorbox}

    \begin{tcolorbox}[breakable, size=fbox, boxrule=1pt, pad at break*=1mm,colback=cellbackground, colframe=cellborder]
\prompt{In}{incolor}{13}{\boxspacing}
\begin{Verbatim}[commandchars=\\\{\}]
\PY{n}{result\PYZus{}df} \PY{o}{=} \PY{n}{pd}\PY{o}{.}\PY{n}{DataFrame}\PY{p}{(}\PY{p}{)}

\PY{k}{for} \PY{n}{t} \PY{o+ow}{in} \PY{n}{ca}\PY{o}{.}\PY{n}{trades}\PY{p}{:}
    \PY{k}{for} \PY{n}{bump} \PY{o+ow}{in} \PY{n}{bumps}\PY{p}{:}
        \PY{n}{record} \PY{o}{=} \PYZbs{}
            \PY{p}{\PYZob{}}\PY{l+s+s1}{\PYZsq{}}\PY{l+s+s1}{Relative change in position size}\PY{l+s+s1}{\PYZsq{}}\PY{p}{:} \PY{n}{bump}\PY{p}{,}
             \PY{l+s+s1}{\PYZsq{}}\PY{l+s+s1}{Change in EAD}\PY{l+s+s1}{\PYZsq{}}\PY{p}{:}
                 \PY{n}{bump\PYZus{}one\PYZus{}trade\PYZus{}and\PYZus{}return\PYZus{}diff}\PY{p}{(}
                     \PY{n}{bump}\PY{p}{,} \PY{n}{t}\PY{p}{,} \PY{n}{ca}\PY{p}{,}
                     \PY{n}{ca}\PY{o}{.}\PY{n}{get\PYZus{}sa\PYZus{}ccr\PYZus{}model}\PY{p}{(}\PY{p}{)}\PY{o}{.}\PY{n}{get\PYZus{}risk\PYZus{}measure}\PY{p}{)}\PY{p}{,}
             \PY{l+s+s1}{\PYZsq{}}\PY{l+s+s1}{Legend}\PY{l+s+s1}{\PYZsq{}}\PY{p}{:}
                 \PY{n}{ast}\PY{o}{.}\PY{n}{literal\PYZus{}eval}\PY{p}{(}\PY{n+nb}{str}\PY{p}{(}\PY{n}{t}\PY{p}{)}\PY{p}{)}\PY{p}{[}\PY{l+s+s1}{\PYZsq{}}\PY{l+s+s1}{Instrument}\PY{l+s+s1}{\PYZsq{}}\PY{p}{]} \PY{o}{+} \PY{l+s+s1}{\PYZsq{}}\PY{l+s+s1}{\PYZus{}}\PY{l+s+s1}{\PYZsq{}}
                 \PY{o}{+} \PY{n}{ast}\PY{o}{.}\PY{n}{literal\PYZus{}eval}\PY{p}{(}\PY{n+nb}{str}\PY{p}{(}\PY{n}{t}\PY{p}{)}\PY{p}{)}\PY{p}{[}\PY{l+s+s1}{\PYZsq{}}\PY{l+s+s1}{TradeDirection}\PY{l+s+s1}{\PYZsq{}}\PY{p}{]}
             \PY{p}{\PYZcb{}}

        \PY{n}{result\PYZus{}df} \PY{o}{=} \PY{n}{result\PYZus{}df}\PY{o}{.}\PY{n}{append}\PY{p}{(}\PY{n}{record}\PY{p}{,} \PY{n}{ignore\PYZus{}index}\PY{o}{=}\PY{k+kc}{True}\PY{p}{)}
\end{Verbatim}
\end{tcolorbox}

    Displaying \texttt{result\_df} yields:
 
            
\prompt{Out}{outcolor}{14}{}
    
    \begin{center}
    \adjustimage{max size={0.9\linewidth}{0.9\paperheight}}{output_14_0.jpg}
    \end{center}
    { \hspace*{\fill} \\}
    

    Do the same for the IM:

    \begin{tcolorbox}[breakable, size=fbox, boxrule=1pt, pad at break*=1mm,colback=cellbackground, colframe=cellborder]
\prompt{In}{incolor}{15}{\boxspacing}
\begin{Verbatim}[commandchars=\\\{\}]
\PY{n}{result\PYZus{}df} \PY{o}{=} \PY{n}{pd}\PY{o}{.}\PY{n}{DataFrame}\PY{p}{(}\PY{p}{)}

\PY{k}{for} \PY{n}{t} \PY{o+ow}{in} \PY{n}{ca}\PY{o}{.}\PY{n}{trades}\PY{p}{:}
    \PY{k}{for} \PY{n}{bump} \PY{o+ow}{in} \PY{n}{bumps}\PY{p}{:}
        \PY{n}{record} \PY{o}{=} \PYZbs{}
            \PY{p}{\PYZob{}}\PY{l+s+s1}{\PYZsq{}}\PY{l+s+s1}{Relative change in position size}\PY{l+s+s1}{\PYZsq{}}\PY{p}{:} \PY{n}{bump}\PY{p}{,}
             \PY{l+s+s1}{\PYZsq{}}\PY{l+s+s1}{Change in calculated IM}\PY{l+s+s1}{\PYZsq{}}\PY{p}{:} \PY{n}{bump\PYZus{}one\PYZus{}trade\PYZus{}and\PYZus{}return\PYZus{}diff}\PY{p}{(}
                 \PY{n}{bump}\PY{p}{,} \PY{n}{t}\PY{p}{,} \PY{n}{ca}\PY{p}{,}
                 \PY{n}{ca}\PY{o}{.}\PY{n}{get\PYZus{}im\PYZus{}model}\PY{p}{(}\PY{p}{)}\PY{o}{.}\PY{n}{get\PYZus{}risk\PYZus{}measure}\PY{p}{)}\PY{p}{,}
             \PY{l+s+s1}{\PYZsq{}}\PY{l+s+s1}{Legend}\PY{l+s+s1}{\PYZsq{}}\PY{p}{:}
                 \PY{n}{ast}\PY{o}{.}\PY{n}{literal\PYZus{}eval}\PY{p}{(}\PY{n+nb}{str}\PY{p}{(}\PY{n}{t}\PY{p}{)}\PY{p}{)}\PY{p}{[}\PY{l+s+s1}{\PYZsq{}}\PY{l+s+s1}{Instrument}\PY{l+s+s1}{\PYZsq{}}\PY{p}{]} \PY{o}{+} \PY{l+s+s1}{\PYZsq{}}\PY{l+s+s1}{\PYZus{}}\PY{l+s+s1}{\PYZsq{}}
                 \PY{o}{+} \PY{n}{ast}\PY{o}{.}\PY{n}{literal\PYZus{}eval}\PY{p}{(}\PY{n+nb}{str}\PY{p}{(}\PY{n}{t}\PY{p}{)}\PY{p}{)}\PY{p}{[}\PY{l+s+s1}{\PYZsq{}}\PY{l+s+s1}{TradeDirection}\PY{l+s+s1}{\PYZsq{}}\PY{p}{]}
             \PY{p}{\PYZcb{}}

        \PY{n}{result\PYZus{}df} \PY{o}{=} \PY{n}{result\PYZus{}df}\PY{o}{.}\PY{n}{append}\PY{p}{(}\PY{n}{record}\PY{p}{,} \PY{n}{ignore\PYZus{}index}\PY{o}{=}\PY{k+kc}{True}\PY{p}{)}
\end{Verbatim}
\end{tcolorbox}

    Displaying \texttt{result\_df} yields:
 
            
\prompt{Out}{outcolor}{16}{}
    
    \begin{center}
    \adjustimage{max size={0.9\linewidth}{0.9\paperheight}}{output_18_0.jpg}
    \end{center}
    { \hspace*{\fill} \\}
    

    The same phenomenon does not appear for if a hedge is not perfect
i.e.~if the hedge size can be increased to further diminish the risk
metric.

    \begin{tcolorbox}[breakable, size=fbox, boxrule=1pt, pad at break*=1mm,colback=cellbackground, colframe=cellborder]
\prompt{In}{incolor}{17}{\boxspacing}
\begin{Verbatim}[commandchars=\\\{\}]
\PY{n}{IRS\PYZus{}pay} \PY{o}{=} \PY{n}{IRS}\PY{p}{(}\PY{n}{notional}\PY{o}{=}\PY{l+m+mi}{200000000}\PY{p}{,}
              \PY{n}{timeToSwapStart}\PY{o}{=}\PY{n}{ql}\PY{o}{.}\PY{n}{Period}\PY{p}{(}\PY{l+m+mi}{2}\PY{p}{,} \PY{n}{ql}\PY{o}{.}\PY{n}{Days}\PY{p}{)}\PY{p}{,}
              \PY{n}{timeToSwapEnd}\PY{o}{=}\PY{n}{ql}\PY{o}{.}\PY{n}{Period}\PY{p}{(}\PY{l+m+mi}{10}\PY{p}{,} \PY{n}{ql}\PY{o}{.}\PY{n}{Years}\PY{p}{)}\PY{p}{,}
              \PY{n}{swapDirection}\PY{o}{=}\PY{n}{SwapDirection}\PY{o}{.}\PY{n}{PAYER}\PY{p}{,}
              \PY{n}{index}\PY{o}{=}\PY{n}{InterestRateIndex}\PY{o}{.}\PY{n}{USDLIBOR3M}\PY{p}{)}

\PY{n}{IRS\PYZus{}rec} \PY{o}{=} \PY{n}{IRS}\PY{p}{(}\PY{n}{notional}\PY{o}{=}\PY{l+m+mi}{100000000}\PY{p}{,}
              \PY{n}{timeToSwapStart}\PY{o}{=}\PY{n}{ql}\PY{o}{.}\PY{n}{Period}\PY{p}{(}\PY{l+m+mi}{2}\PY{p}{,} \PY{n}{ql}\PY{o}{.}\PY{n}{Days}\PY{p}{)}\PY{p}{,}
              \PY{n}{timeToSwapEnd}\PY{o}{=}\PY{n}{ql}\PY{o}{.}\PY{n}{Period}\PY{p}{(}\PY{l+m+mi}{10}\PY{p}{,} \PY{n}{ql}\PY{o}{.}\PY{n}{Years}\PY{p}{)}\PY{p}{,}
              \PY{n}{swapDirection}\PY{o}{=}\PY{n}{SwapDirection}\PY{o}{.}\PY{n}{RECEIVER}\PY{p}{,}
              \PY{n}{index}\PY{o}{=}\PY{n}{InterestRateIndex}\PY{o}{.}\PY{n}{USDLIBOR3M}\PY{p}{)}

\PY{n}{eqOpt} \PY{o}{=} \PY{n}{EquityOption}\PY{p}{(}\PY{n}{notional} \PY{o}{=} \PY{l+m+mi}{1000000}\PY{p}{)}
\end{Verbatim}
\end{tcolorbox}

    The IM for this portfolio is 11471795.91 USD and the EAD is 1538770.53
USD.

    
    For the IM the allocation yields:

            \begin{tcolorbox}[breakable, size=fbox, boxrule=.5pt, pad at break*=1mm, opacityfill=0]
\prompt{Out}{outcolor}{20}{\boxspacing}
\begin{Verbatim}[commandchars=\\\{\}]
                       Backward       Central       Forward
IRS\_Short         -4.519079e+06 -4.519078e+06 -4.519079e+06
IRS\_Long           9.038157e+06  9.038157e+06  9.038157e+06
EquityOption\_Long  6.952717e+06  6.952717e+06  6.952717e+06
Sum                1.147180e+07  1.147180e+07  1.147180e+07
\end{Verbatim}
\end{tcolorbox}
        
    and for the EAD the different differentiation approaches yield:

            \begin{tcolorbox}[breakable, size=fbox, boxrule=.5pt, pad at break*=1mm, opacityfill=0]
\prompt{Out}{outcolor}{21}{\boxspacing}
\begin{Verbatim}[commandchars=\\\{\}]
                       Backward       Central       Forward
IRS\_Short         -2.466093e+05 -2.466078e+05 -2.466063e+05
IRS\_Long           4.932094e+05  4.932156e+05  4.932217e+05
EquityOption\_Long  1.292161e+06  1.292163e+06  1.292164e+06
Sum                1.538761e+06  1.538771e+06  1.538780e+06
\end{Verbatim}
\end{tcolorbox}
        

    % Add a bibliography block to the postdoc
    
    
    
\end{document}

    
        

    
    \hypertarget{homogeneity-of-c-for-a-single-trade-portfolio}{%
\section{Homogeneity of C for a single trade
portfolio}\label{homogeneity-of-c-for-a-single-trade-portfolio}}

    We want to investigate under which circumstances the \(C\) representing
the received collateral in the calculation of SA-CCR exhibits
homogeneity. For this we consider a portfolio consisting of a single IRS
as follows

    \begin{tcolorbox}[breakable, size=fbox, boxrule=1pt, pad at break*=1mm,colback=cellbackground, colframe=cellborder]
\prompt{In}{incolor}{In}{\boxspacing}
\begin{Verbatim}[commandchars=\\\{\}]
\PY{n}{notional} \PY{o}{=} \PY{l+m+mi}{200000000000}
\PY{n}{irs} \PY{o}{=} \PY{n}{IRS}\PY{p}{(}\PY{n}{notional} \PY{o}{=} \PY{n}{notional}\PY{p}{,}
          \PY{n}{timeToSwapStart}\PY{o}{=}\PY{n}{ql}\PY{o}{.}\PY{n}{Period}\PY{p}{(}\PY{l+m+mi}{2}\PY{p}{,} \PY{n}{ql}\PY{o}{.}\PY{n}{Days}\PY{p}{)}\PY{p}{,}
          \PY{n}{timeToSwapEnd}\PY{o}{=}\PY{n}{ql}\PY{o}{.}\PY{n}{Period}\PY{p}{(}\PY{l+m+mi}{10}\PY{p}{,} \PY{n}{ql}\PY{o}{.}\PY{n}{Years}\PY{p}{)}\PY{p}{,}
          \PY{n}{swapDirection}\PY{o}{=}\PY{n}{SwapDirection}\PY{o}{.}\PY{n}{PAYER}\PY{p}{,}
          \PY{n}{index} \PY{o}{=} \PY{n}{InterestRateIndex}\PY{o}{.}\PY{n}{USDLIBOR3M}\PY{p}{)}
\end{Verbatim}
\end{tcolorbox}

    Since we have not explicitly set a fixed rate the IRS is struck at par
and should have a present value of close to 0.

    \begin{tcolorbox}[breakable, size=fbox, boxrule=1pt, pad at break*=1mm,colback=cellbackground, colframe=cellborder]
\prompt{In}{incolor}{In}{\boxspacing}
\begin{Verbatim}[commandchars=\\\{\}]
\PY{n+nb}{print}\PY{p}{(}\PY{l+s+s1}{\PYZsq{}}\PY{l+s+s1}{Fixed rate: }\PY{l+s+si}{\PYZpc{}f}\PY{l+s+s1}{\PYZsq{}} \PY{o}{\PYZpc{}}\PY{k}{irs}.get\PYZus{}fixed\PYZus{}rate())
\PY{n+nb}{print}\PY{p}{(}\PY{l+s+s1}{\PYZsq{}}\PY{l+s+s1}{Present value: }\PY{l+s+si}{\PYZpc{}.2f}\PY{l+s+s1}{\PYZsq{}} \PY{o}{\PYZpc{}}\PY{k}{irs}.get\PYZus{}price())
\end{Verbatim}
\end{tcolorbox}

    \begin{Verbatim}[commandchars=\\\{\}]
Fixed rate: 0.024093
Present value: 0.00
    \end{Verbatim}

    We also set up a fixed payer IRS that is in the money as it has a fixed
rate of 2\%.

    \begin{tcolorbox}[breakable, size=fbox, boxrule=1pt, pad at break*=1mm,colback=cellbackground, colframe=cellborder]
\prompt{In}{incolor}{In}{\boxspacing}
\begin{Verbatim}[commandchars=\\\{\}]
\PY{n}{irs\PYZus{}itm} \PY{o}{=} \PY{n}{IRS}\PY{p}{(}\PY{n}{notional} \PY{o}{=} \PY{n}{notional}\PY{p}{,}
          \PY{n}{timeToSwapStart}\PY{o}{=}\PY{n}{ql}\PY{o}{.}\PY{n}{Period}\PY{p}{(}\PY{l+m+mi}{2}\PY{p}{,} \PY{n}{ql}\PY{o}{.}\PY{n}{Days}\PY{p}{)}\PY{p}{,}
          \PY{n}{timeToSwapEnd}\PY{o}{=}\PY{n}{ql}\PY{o}{.}\PY{n}{Period}\PY{p}{(}\PY{l+m+mi}{10}\PY{p}{,} \PY{n}{ql}\PY{o}{.}\PY{n}{Years}\PY{p}{)}\PY{p}{,}
          \PY{n}{swapDirection}\PY{o}{=}\PY{n}{SwapDirection}\PY{o}{.}\PY{n}{PAYER}\PY{p}{,}
          \PY{n}{index} \PY{o}{=} \PY{n}{InterestRateIndex}\PY{o}{.}\PY{n}{USDLIBOR3M}\PY{p}{,}
          \PY{n}{fixed\PYZus{}rate}\PY{o}{=}\PY{l+m+mf}{0.02}\PY{p}{)}

\PY{n+nb}{print}\PY{p}{(}\PY{l+s+s1}{\PYZsq{}}\PY{l+s+s1}{Fixed rate: }\PY{l+s+si}{\PYZpc{}f}\PY{l+s+s1}{\PYZsq{}} \PY{o}{\PYZpc{}}\PY{k}{irs\PYZus{}itm}.get\PYZus{}fixed\PYZus{}rate())
\PY{n+nb}{print}\PY{p}{(}\PY{l+s+s1}{\PYZsq{}}\PY{l+s+s1}{Present value: }\PY{l+s+si}{\PYZpc{}.2f}\PY{l+s+s1}{\PYZsq{}} \PY{o}{\PYZpc{}}\PY{k}{irs\PYZus{}itm}.get\PYZus{}price())
\end{Verbatim}
\end{tcolorbox}

    \begin{Verbatim}[commandchars=\\\{\}]
Fixed rate: 0.020000
Present value: 7258788031.38
    \end{Verbatim}

    Additionally, we set up a collateral agreement exchanging IM in
accordance with ISDA-SIMM with a minimum transfer amount of 1Bn and a
threshold of 2Bn. For technical reasons we need to first create the
collateral agreement and afterwards link it to an instance of the SA-CCR
model. We put each of the two trades created above in a separate
portfolio and collateral agreement.

    \begin{tcolorbox}[breakable, size=fbox, boxrule=1pt, pad at break*=1mm,colback=cellbackground, colframe=cellborder]
\prompt{In}{incolor}{In}{\boxspacing}
\begin{Verbatim}[commandchars=\\\{\}]
\PY{n}{ca} \PY{o}{=} \PY{n}{CollateralAgreement}\PY{p}{(}\PY{n}{mta} \PY{o}{=} \PY{l+m+mi}{1000000000}\PY{p}{,}
                         \PY{n}{threshold}\PY{o}{=} \PY{l+m+mi}{2000000000}\PY{p}{,}
                         \PY{n}{initialMargining}\PY{o}{=}\PY{n}{InitialMargining}\PY{o}{.}\PY{n}{SIMM}\PY{p}{,}
                         \PY{n}{margin\PYZus{}currency}\PY{o}{=}\PY{n}{Currency}\PY{o}{.}\PY{n}{USD}\PY{p}{)}
\PY{n}{ca}\PY{o}{.}\PY{n}{link\PYZus{}sa\PYZus{}ccr\PYZus{}instance}\PY{p}{(}\PY{n}{SA\PYZus{}CCR}\PY{p}{(}\PY{n}{ca}\PY{p}{)}\PY{p}{)}
\end{Verbatim}
\end{tcolorbox}

    Again we explore homogeneity at this example by exploring whether

\begin{align*}
f\left(\alpha \mathbf{x}\right) = \alpha^k f\left(\mathbf{x}\right) \\
        \text{for } \alpha >0
\end{align*}

holds based on our initial portfolio for a range \(0<\alpha\leq2\)

    Next, we calculate IM, VM, and C for the two IRS. C is once calculated
with consideration of MTA and threshold and once without.

To use the MTA we need to set a current margin amount. We will set this
as the currently calculated \(C\). With the MTA in place C will
afterwards only be updated if the sum of VM and IM differ from the
current margin amount by more than the MTA.

    \begin{tcolorbox}[breakable, size=fbox, boxrule=1pt, pad at break*=1mm,colback=cellbackground, colframe=cellborder]
\prompt{In}{incolor}{In}{\boxspacing}
\begin{Verbatim}[commandchars=\\\{\}]
\PY{n}{bumps} \PY{o}{=} \PY{n}{arange}\PY{p}{(}\PY{l+m+mi}{0}\PY{p}{,}\PY{l+m+mf}{2.01}\PY{p}{,}\PY{l+m+mf}{0.01}\PY{p}{)}
\PY{n}{resultDataframe} \PY{o}{=} \PY{n}{pd}\PY{o}{.}\PY{n}{DataFrame}\PY{p}{(}\PY{n}{columns} \PY{o}{=} \PY{p}{[}\PY{l+s+s1}{\PYZsq{}}\PY{l+s+s1}{X}\PY{l+s+s1}{\PYZsq{}}\PY{p}{,}\PY{l+s+s1}{\PYZsq{}}\PY{l+s+s1}{Y}\PY{l+s+s1}{\PYZsq{}}\PY{p}{,}\PY{l+s+s1}{\PYZsq{}}\PY{l+s+s1}{Legend}\PY{l+s+s1}{\PYZsq{}}\PY{p}{]}\PY{p}{)}
\end{Verbatim}
\end{tcolorbox}

    \begin{tcolorbox}[breakable, size=fbox, boxrule=1pt, pad at break*=1mm,colback=cellbackground, colframe=cellborder]
\prompt{In}{incolor}{In}{\boxspacing}
\begin{Verbatim}[commandchars=\\\{\}]
\PY{c+c1}{\PYZsh{} At the money IRS, with threshold and mta}
\PY{n}{ca}\PY{o}{.}\PY{n}{add\PYZus{}trades}\PY{p}{(}\PY{n}{irs}\PY{p}{)}
\PY{n}{ca}\PY{o}{.}\PY{n}{set\PYZus{}start\PYZus{}collateral\PYZus{}amount}\PY{p}{(}\PY{n}{ca}\PY{o}{.}\PY{n}{get\PYZus{}C}\PY{p}{(}\PY{p}{)}\PY{p}{)}
\PY{k}{for} \PY{n}{bump} \PY{o+ow}{in} \PY{n}{bumps}\PY{p}{:}
    \PY{n}{result} \PY{o}{=} \PY{n}{bump\PYZus{}and\PYZus{}get\PYZus{}results}\PY{p}{(}\PY{n}{bump}\PY{p}{,} \PY{n}{irs}\PY{p}{,} \PY{n}{ca}\PY{p}{)}
    \PY{n}{result\PYZus{}to\PYZus{}record}\PY{p}{(}\PY{l+s+s1}{\PYZsq{}}\PY{l+s+s1}{ATM VM}\PY{l+s+s1}{\PYZsq{}}\PY{p}{,} \PY{l+s+s1}{\PYZsq{}}\PY{l+s+s1}{VM}\PY{l+s+s1}{\PYZsq{}}\PY{p}{,} \PY{n}{result}\PY{p}{)}
    \PY{n}{result\PYZus{}to\PYZus{}record}\PY{p}{(}\PY{l+s+s1}{\PYZsq{}}\PY{l+s+s1}{ATM IM}\PY{l+s+s1}{\PYZsq{}}\PY{p}{,} \PY{l+s+s1}{\PYZsq{}}\PY{l+s+s1}{IM}\PY{l+s+s1}{\PYZsq{}}\PY{p}{,} \PY{n}{result}\PY{p}{)}
    \PY{n}{result\PYZus{}to\PYZus{}record}\PY{p}{(}\PY{l+s+s1}{\PYZsq{}}\PY{l+s+s1}{ATM: C with TH and MTA}\PY{l+s+s1}{\PYZsq{}}\PY{p}{,} \PY{l+s+s1}{\PYZsq{}}\PY{l+s+s1}{Collateral}\PY{l+s+s1}{\PYZsq{}}\PY{p}{,} \PY{n}{result}\PY{p}{)}

\PY{c+c1}{\PYZsh{} In the money IRS, with threshold and mta}
\PY{n}{ca}\PY{o}{.}\PY{n}{remove\PYZus{}all\PYZus{}trades}\PY{p}{(}\PY{p}{)}
\PY{n}{ca}\PY{o}{.}\PY{n}{add\PYZus{}trades}\PY{p}{(}\PY{n}{irs\PYZus{}itm}\PY{p}{)}
\PY{n}{ca}\PY{o}{.}\PY{n}{set\PYZus{}start\PYZus{}collateral\PYZus{}amount}\PY{p}{(}\PY{n}{ca}\PY{o}{.}\PY{n}{get\PYZus{}C}\PY{p}{(}\PY{p}{)}\PY{p}{)}
\PY{k}{for} \PY{n}{bump} \PY{o+ow}{in} \PY{n}{bumps}\PY{p}{:}
    \PY{n}{result} \PY{o}{=} \PY{n}{bump\PYZus{}and\PYZus{}get\PYZus{}results}\PY{p}{(}\PY{n}{bump}\PY{p}{,} \PY{n}{irs\PYZus{}itm}\PY{p}{,} \PY{n}{ca}\PY{p}{)}
    \PY{n}{result\PYZus{}to\PYZus{}record}\PY{p}{(}\PY{l+s+s1}{\PYZsq{}}\PY{l+s+s1}{ITM VM}\PY{l+s+s1}{\PYZsq{}}\PY{p}{,} \PY{l+s+s1}{\PYZsq{}}\PY{l+s+s1}{VM}\PY{l+s+s1}{\PYZsq{}}\PY{p}{,} \PY{n}{result}\PY{p}{)}
    \PY{n}{result\PYZus{}to\PYZus{}record}\PY{p}{(}\PY{l+s+s1}{\PYZsq{}}\PY{l+s+s1}{ITM IM}\PY{l+s+s1}{\PYZsq{}}\PY{p}{,} \PY{l+s+s1}{\PYZsq{}}\PY{l+s+s1}{IM}\PY{l+s+s1}{\PYZsq{}}\PY{p}{,} \PY{n}{result}\PY{p}{)}
    \PY{n}{result\PYZus{}to\PYZus{}record}\PY{p}{(}\PY{l+s+s1}{\PYZsq{}}\PY{l+s+s1}{ITM: C with TH and MTA}\PY{l+s+s1}{\PYZsq{}}\PY{p}{,} \PY{l+s+s1}{\PYZsq{}}\PY{l+s+s1}{Collateral}\PY{l+s+s1}{\PYZsq{}}\PY{p}{,} \PY{n}{result}\PY{p}{)}

\PY{c+c1}{\PYZsh{} In the money IRS without threshold or mta}
\PY{n}{ca}\PY{o}{.}\PY{n}{threshold} \PY{o}{=} \PY{l+m+mi}{0}
\PY{n}{ca}\PY{o}{.}\PY{n}{mta} \PY{o}{=} \PY{l+m+mi}{0}

\PY{k}{for} \PY{n}{bump} \PY{o+ow}{in} \PY{n}{bumps}\PY{p}{:}
    \PY{n}{result} \PY{o}{=} \PY{n}{bump\PYZus{}and\PYZus{}get\PYZus{}results}\PY{p}{(}\PY{n}{bump}\PY{p}{,} \PY{n}{irs\PYZus{}itm}\PY{p}{,} \PY{n}{ca}\PY{p}{)}
    \PY{n}{result\PYZus{}to\PYZus{}record}\PY{p}{(}\PY{l+s+s1}{\PYZsq{}}\PY{l+s+s1}{ITM: C no TH or MTA}\PY{l+s+s1}{\PYZsq{}}\PY{p}{,} \PY{l+s+s1}{\PYZsq{}}\PY{l+s+s1}{Collateral}\PY{l+s+s1}{\PYZsq{}}\PY{p}{,} \PY{n}{result}\PY{p}{)}

\PY{c+c1}{\PYZsh{} At the money IRS without threshold or mta}
\PY{n}{ca}\PY{o}{.}\PY{n}{remove\PYZus{}all\PYZus{}trades}\PY{p}{(}\PY{p}{)}
\PY{n}{ca}\PY{o}{.}\PY{n}{add\PYZus{}trades}\PY{p}{(}\PY{n}{irs}\PY{p}{)}
\PY{n}{ca}\PY{o}{.}\PY{n}{set\PYZus{}start\PYZus{}collateral\PYZus{}amount}\PY{p}{(}\PY{n}{ca}\PY{o}{.}\PY{n}{get\PYZus{}C}\PY{p}{(}\PY{p}{)}\PY{p}{)}
\PY{k}{for} \PY{n}{bump} \PY{o+ow}{in} \PY{n}{bumps}\PY{p}{:}
    \PY{n}{result} \PY{o}{=} \PY{n}{bump\PYZus{}and\PYZus{}get\PYZus{}results}\PY{p}{(}\PY{n}{bump}\PY{p}{,} \PY{n}{irs}\PY{p}{,} \PY{n}{ca}\PY{p}{)}
    \PY{n}{result\PYZus{}to\PYZus{}record}\PY{p}{(}\PY{l+s+s1}{\PYZsq{}}\PY{l+s+s1}{ATM: C no TH or MTA}\PY{l+s+s1}{\PYZsq{}}\PY{p}{,} \PY{l+s+s1}{\PYZsq{}}\PY{l+s+s1}{Collateral}\PY{l+s+s1}{\PYZsq{}}\PY{p}{,} \PY{n}{result}\PY{p}{)}
\end{Verbatim}
\end{tcolorbox}

    Displaying the result for the at the money IRS it can be seen that the
VM is flat at zero since the IRS is at par and therefore has a PV of 0.
The IM shows the behavior described in
\ref{isda-simm-homogeneity-for-portfolio-of-a-single-trade}. C with
threshold and minimum transfer amount is not a homogeneous at all, while
C without threshold and MTA is partially. This is further discussed in
section \todo{include reference}
 
            
            \begin{tcolorbox}[breakable, size=fbox, boxrule=.5pt, pad at break*=1mm, opacityfill=0]
\prompt{Out}{outcolor}{Out}{\boxspacing}    
    \begin{center}
    \adjustimage{max size={0.9\linewidth}{0.9\paperheight}}{JupyterNotebooksFull/C_homogeneity_for_a_single_trade/output_14_0.jpg}
    \end{center}
    { \hspace*{\fill} \\}
\end{tcolorbox}
    

    Results are the same for the in the money IRS with the exception, that
the VM is not 0 but VM is still a homogeneous function.
 
            
            \begin{tcolorbox}[breakable, size=fbox, boxrule=.5pt, pad at break*=1mm, opacityfill=0]
\prompt{Out}{outcolor}{Out}{\boxspacing}    
    \begin{center}
    \adjustimage{max size={0.9\linewidth}{0.9\paperheight}}{JupyterNotebooksFull/C_homogeneity_for_a_single_trade/output_16_0.jpg}
    \end{center}
    { \hspace*{\fill} \\}
\end{tcolorbox}
    




    \documentclass[11pt]{article}

    \usepackage[breakable]{tcolorbox}
    \usepackage{parskip} % Stop auto-indenting (to mimic markdown behaviour)
    
    \usepackage{iftex}
    \ifPDFTeX
    	\usepackage[T1]{fontenc}
    	\usepackage{mathpazo}
    \else
    	\usepackage{fontspec}
    \fi

    % Basic figure setup, for now with no caption control since it's done
    % automatically by Pandoc (which extracts ![](path) syntax from Markdown).
    \usepackage{graphicx}
    % Maintain compatibility with old templates. Remove in nbconvert 6.0
    \let\Oldincludegraphics\includegraphics
    % Ensure that by default, figures have no caption (until we provide a
    % proper Figure object with a Caption API and a way to capture that
    % in the conversion process - todo).
    \usepackage{caption}
    \DeclareCaptionFormat{nocaption}{}
    \captionsetup{format=nocaption,aboveskip=0pt,belowskip=0pt}

    \usepackage[Export]{adjustbox} % Used to constrain images to a maximum size
    \adjustboxset{max size={0.9\linewidth}{0.9\paperheight}}
    \usepackage{float}
    \floatplacement{figure}{H} % forces figures to be placed at the correct location
    \usepackage{xcolor} % Allow colors to be defined
    \usepackage{enumerate} % Needed for markdown enumerations to work
    \usepackage{geometry} % Used to adjust the document margins
    \usepackage{amsmath} % Equations
    \usepackage{amssymb} % Equations
    \usepackage{textcomp} % defines textquotesingle
    % Hack from http://tex.stackexchange.com/a/47451/13684:
    \AtBeginDocument{%
        \def\PYZsq{\textquotesingle}% Upright quotes in Pygmentized code
    }
    \usepackage{upquote} % Upright quotes for verbatim code
    \usepackage{eurosym} % defines \euro
    \usepackage[mathletters]{ucs} % Extended unicode (utf-8) support
    \usepackage{fancyvrb} % verbatim replacement that allows latex
    \usepackage{grffile} % extends the file name processing of package graphics 
                         % to support a larger range
    \makeatletter % fix for grffile with XeLaTeX
    \def\Gread@@xetex#1{%
      \IfFileExists{"\Gin@base".bb}%
      {\Gread@eps{\Gin@base.bb}}%
      {\Gread@@xetex@aux#1}%
    }
    \makeatother

    % The hyperref package gives us a pdf with properly built
    % internal navigation ('pdf bookmarks' for the table of contents,
    % internal cross-reference links, web links for URLs, etc.)
    \usepackage{hyperref}
    % The default LaTeX title has an obnoxious amount of whitespace. By default,
    % titling removes some of it. It also provides customization options.
    \usepackage{titling}
    \usepackage{longtable} % longtable support required by pandoc >1.10
    \usepackage{booktabs}  % table support for pandoc > 1.12.2
    \usepackage[inline]{enumitem} % IRkernel/repr support (it uses the enumerate* environment)
    \usepackage[normalem]{ulem} % ulem is needed to support strikethroughs (\sout)
                                % normalem makes italics be italics, not underlines
    \usepackage{mathrsfs}
    

    
    % Colors for the hyperref package
    \definecolor{urlcolor}{rgb}{0,.145,.698}
    \definecolor{linkcolor}{rgb}{.71,0.21,0.01}
    \definecolor{citecolor}{rgb}{.12,.54,.11}

    % ANSI colors
    \definecolor{ansi-black}{HTML}{3E424D}
    \definecolor{ansi-black-intense}{HTML}{282C36}
    \definecolor{ansi-red}{HTML}{E75C58}
    \definecolor{ansi-red-intense}{HTML}{B22B31}
    \definecolor{ansi-green}{HTML}{00A250}
    \definecolor{ansi-green-intense}{HTML}{007427}
    \definecolor{ansi-yellow}{HTML}{DDB62B}
    \definecolor{ansi-yellow-intense}{HTML}{B27D12}
    \definecolor{ansi-blue}{HTML}{208FFB}
    \definecolor{ansi-blue-intense}{HTML}{0065CA}
    \definecolor{ansi-magenta}{HTML}{D160C4}
    \definecolor{ansi-magenta-intense}{HTML}{A03196}
    \definecolor{ansi-cyan}{HTML}{60C6C8}
    \definecolor{ansi-cyan-intense}{HTML}{258F8F}
    \definecolor{ansi-white}{HTML}{C5C1B4}
    \definecolor{ansi-white-intense}{HTML}{A1A6B2}
    \definecolor{ansi-default-inverse-fg}{HTML}{FFFFFF}
    \definecolor{ansi-default-inverse-bg}{HTML}{000000}

    % commands and environments needed by pandoc snippets
    % extracted from the output of `pandoc -s`
    \providecommand{\tightlist}{%
      \setlength{\itemsep}{0pt}\setlength{\parskip}{0pt}}
    \DefineVerbatimEnvironment{Highlighting}{Verbatim}{commandchars=\\\{\}}
    % Add ',fontsize=\small' for more characters per line
    \newenvironment{Shaded}{}{}
    \newcommand{\KeywordTok}[1]{\textcolor[rgb]{0.00,0.44,0.13}{\textbf{{#1}}}}
    \newcommand{\DataTypeTok}[1]{\textcolor[rgb]{0.56,0.13,0.00}{{#1}}}
    \newcommand{\DecValTok}[1]{\textcolor[rgb]{0.25,0.63,0.44}{{#1}}}
    \newcommand{\BaseNTok}[1]{\textcolor[rgb]{0.25,0.63,0.44}{{#1}}}
    \newcommand{\FloatTok}[1]{\textcolor[rgb]{0.25,0.63,0.44}{{#1}}}
    \newcommand{\CharTok}[1]{\textcolor[rgb]{0.25,0.44,0.63}{{#1}}}
    \newcommand{\StringTok}[1]{\textcolor[rgb]{0.25,0.44,0.63}{{#1}}}
    \newcommand{\CommentTok}[1]{\textcolor[rgb]{0.38,0.63,0.69}{\textit{{#1}}}}
    \newcommand{\OtherTok}[1]{\textcolor[rgb]{0.00,0.44,0.13}{{#1}}}
    \newcommand{\AlertTok}[1]{\textcolor[rgb]{1.00,0.00,0.00}{\textbf{{#1}}}}
    \newcommand{\FunctionTok}[1]{\textcolor[rgb]{0.02,0.16,0.49}{{#1}}}
    \newcommand{\RegionMarkerTok}[1]{{#1}}
    \newcommand{\ErrorTok}[1]{\textcolor[rgb]{1.00,0.00,0.00}{\textbf{{#1}}}}
    \newcommand{\NormalTok}[1]{{#1}}
    
    % Additional commands for more recent versions of Pandoc
    \newcommand{\ConstantTok}[1]{\textcolor[rgb]{0.53,0.00,0.00}{{#1}}}
    \newcommand{\SpecialCharTok}[1]{\textcolor[rgb]{0.25,0.44,0.63}{{#1}}}
    \newcommand{\VerbatimStringTok}[1]{\textcolor[rgb]{0.25,0.44,0.63}{{#1}}}
    \newcommand{\SpecialStringTok}[1]{\textcolor[rgb]{0.73,0.40,0.53}{{#1}}}
    \newcommand{\ImportTok}[1]{{#1}}
    \newcommand{\DocumentationTok}[1]{\textcolor[rgb]{0.73,0.13,0.13}{\textit{{#1}}}}
    \newcommand{\AnnotationTok}[1]{\textcolor[rgb]{0.38,0.63,0.69}{\textbf{\textit{{#1}}}}}
    \newcommand{\CommentVarTok}[1]{\textcolor[rgb]{0.38,0.63,0.69}{\textbf{\textit{{#1}}}}}
    \newcommand{\VariableTok}[1]{\textcolor[rgb]{0.10,0.09,0.49}{{#1}}}
    \newcommand{\ControlFlowTok}[1]{\textcolor[rgb]{0.00,0.44,0.13}{\textbf{{#1}}}}
    \newcommand{\OperatorTok}[1]{\textcolor[rgb]{0.40,0.40,0.40}{{#1}}}
    \newcommand{\BuiltInTok}[1]{{#1}}
    \newcommand{\ExtensionTok}[1]{{#1}}
    \newcommand{\PreprocessorTok}[1]{\textcolor[rgb]{0.74,0.48,0.00}{{#1}}}
    \newcommand{\AttributeTok}[1]{\textcolor[rgb]{0.49,0.56,0.16}{{#1}}}
    \newcommand{\InformationTok}[1]{\textcolor[rgb]{0.38,0.63,0.69}{\textbf{\textit{{#1}}}}}
    \newcommand{\WarningTok}[1]{\textcolor[rgb]{0.38,0.63,0.69}{\textbf{\textit{{#1}}}}}
    
    
    % Define a nice break command that doesn't care if a line doesn't already
    % exist.
    \def\br{\hspace*{\fill} \\* }
    % Math Jax compatibility definitions
    \def\gt{>}
    \def\lt{<}
    \let\Oldtex\TeX
    \let\Oldlatex\LaTeX
    \renewcommand{\TeX}{\textrm{\Oldtex}}
    \renewcommand{\LaTeX}{\textrm{\Oldlatex}}
    % Document parameters
    % Document title
    \title{Exemplary SA-CCR allocation considering an initial margin threshold}
    
    
    
    
    
% Pygments definitions
\makeatletter
\def\PY@reset{\let\PY@it=\relax \let\PY@bf=\relax%
    \let\PY@ul=\relax \let\PY@tc=\relax%
    \let\PY@bc=\relax \let\PY@ff=\relax}
\def\PY@tok#1{\csname PY@tok@#1\endcsname}
\def\PY@toks#1+{\ifx\relax#1\empty\else%
    \PY@tok{#1}\expandafter\PY@toks\fi}
\def\PY@do#1{\PY@bc{\PY@tc{\PY@ul{%
    \PY@it{\PY@bf{\PY@ff{#1}}}}}}}
\def\PY#1#2{\PY@reset\PY@toks#1+\relax+\PY@do{#2}}

\expandafter\def\csname PY@tok@w\endcsname{\def\PY@tc##1{\textcolor[rgb]{0.73,0.73,0.73}{##1}}}
\expandafter\def\csname PY@tok@c\endcsname{\let\PY@it=\textit\def\PY@tc##1{\textcolor[rgb]{0.25,0.50,0.50}{##1}}}
\expandafter\def\csname PY@tok@cp\endcsname{\def\PY@tc##1{\textcolor[rgb]{0.74,0.48,0.00}{##1}}}
\expandafter\def\csname PY@tok@k\endcsname{\let\PY@bf=\textbf\def\PY@tc##1{\textcolor[rgb]{0.00,0.50,0.00}{##1}}}
\expandafter\def\csname PY@tok@kp\endcsname{\def\PY@tc##1{\textcolor[rgb]{0.00,0.50,0.00}{##1}}}
\expandafter\def\csname PY@tok@kt\endcsname{\def\PY@tc##1{\textcolor[rgb]{0.69,0.00,0.25}{##1}}}
\expandafter\def\csname PY@tok@o\endcsname{\def\PY@tc##1{\textcolor[rgb]{0.40,0.40,0.40}{##1}}}
\expandafter\def\csname PY@tok@ow\endcsname{\let\PY@bf=\textbf\def\PY@tc##1{\textcolor[rgb]{0.67,0.13,1.00}{##1}}}
\expandafter\def\csname PY@tok@nb\endcsname{\def\PY@tc##1{\textcolor[rgb]{0.00,0.50,0.00}{##1}}}
\expandafter\def\csname PY@tok@nf\endcsname{\def\PY@tc##1{\textcolor[rgb]{0.00,0.00,1.00}{##1}}}
\expandafter\def\csname PY@tok@nc\endcsname{\let\PY@bf=\textbf\def\PY@tc##1{\textcolor[rgb]{0.00,0.00,1.00}{##1}}}
\expandafter\def\csname PY@tok@nn\endcsname{\let\PY@bf=\textbf\def\PY@tc##1{\textcolor[rgb]{0.00,0.00,1.00}{##1}}}
\expandafter\def\csname PY@tok@ne\endcsname{\let\PY@bf=\textbf\def\PY@tc##1{\textcolor[rgb]{0.82,0.25,0.23}{##1}}}
\expandafter\def\csname PY@tok@nv\endcsname{\def\PY@tc##1{\textcolor[rgb]{0.10,0.09,0.49}{##1}}}
\expandafter\def\csname PY@tok@no\endcsname{\def\PY@tc##1{\textcolor[rgb]{0.53,0.00,0.00}{##1}}}
\expandafter\def\csname PY@tok@nl\endcsname{\def\PY@tc##1{\textcolor[rgb]{0.63,0.63,0.00}{##1}}}
\expandafter\def\csname PY@tok@ni\endcsname{\let\PY@bf=\textbf\def\PY@tc##1{\textcolor[rgb]{0.60,0.60,0.60}{##1}}}
\expandafter\def\csname PY@tok@na\endcsname{\def\PY@tc##1{\textcolor[rgb]{0.49,0.56,0.16}{##1}}}
\expandafter\def\csname PY@tok@nt\endcsname{\let\PY@bf=\textbf\def\PY@tc##1{\textcolor[rgb]{0.00,0.50,0.00}{##1}}}
\expandafter\def\csname PY@tok@nd\endcsname{\def\PY@tc##1{\textcolor[rgb]{0.67,0.13,1.00}{##1}}}
\expandafter\def\csname PY@tok@s\endcsname{\def\PY@tc##1{\textcolor[rgb]{0.73,0.13,0.13}{##1}}}
\expandafter\def\csname PY@tok@sd\endcsname{\let\PY@it=\textit\def\PY@tc##1{\textcolor[rgb]{0.73,0.13,0.13}{##1}}}
\expandafter\def\csname PY@tok@si\endcsname{\let\PY@bf=\textbf\def\PY@tc##1{\textcolor[rgb]{0.73,0.40,0.53}{##1}}}
\expandafter\def\csname PY@tok@se\endcsname{\let\PY@bf=\textbf\def\PY@tc##1{\textcolor[rgb]{0.73,0.40,0.13}{##1}}}
\expandafter\def\csname PY@tok@sr\endcsname{\def\PY@tc##1{\textcolor[rgb]{0.73,0.40,0.53}{##1}}}
\expandafter\def\csname PY@tok@ss\endcsname{\def\PY@tc##1{\textcolor[rgb]{0.10,0.09,0.49}{##1}}}
\expandafter\def\csname PY@tok@sx\endcsname{\def\PY@tc##1{\textcolor[rgb]{0.00,0.50,0.00}{##1}}}
\expandafter\def\csname PY@tok@m\endcsname{\def\PY@tc##1{\textcolor[rgb]{0.40,0.40,0.40}{##1}}}
\expandafter\def\csname PY@tok@gh\endcsname{\let\PY@bf=\textbf\def\PY@tc##1{\textcolor[rgb]{0.00,0.00,0.50}{##1}}}
\expandafter\def\csname PY@tok@gu\endcsname{\let\PY@bf=\textbf\def\PY@tc##1{\textcolor[rgb]{0.50,0.00,0.50}{##1}}}
\expandafter\def\csname PY@tok@gd\endcsname{\def\PY@tc##1{\textcolor[rgb]{0.63,0.00,0.00}{##1}}}
\expandafter\def\csname PY@tok@gi\endcsname{\def\PY@tc##1{\textcolor[rgb]{0.00,0.63,0.00}{##1}}}
\expandafter\def\csname PY@tok@gr\endcsname{\def\PY@tc##1{\textcolor[rgb]{1.00,0.00,0.00}{##1}}}
\expandafter\def\csname PY@tok@ge\endcsname{\let\PY@it=\textit}
\expandafter\def\csname PY@tok@gs\endcsname{\let\PY@bf=\textbf}
\expandafter\def\csname PY@tok@gp\endcsname{\let\PY@bf=\textbf\def\PY@tc##1{\textcolor[rgb]{0.00,0.00,0.50}{##1}}}
\expandafter\def\csname PY@tok@go\endcsname{\def\PY@tc##1{\textcolor[rgb]{0.53,0.53,0.53}{##1}}}
\expandafter\def\csname PY@tok@gt\endcsname{\def\PY@tc##1{\textcolor[rgb]{0.00,0.27,0.87}{##1}}}
\expandafter\def\csname PY@tok@err\endcsname{\def\PY@bc##1{\setlength{\fboxsep}{0pt}\fcolorbox[rgb]{1.00,0.00,0.00}{1,1,1}{\strut ##1}}}
\expandafter\def\csname PY@tok@kc\endcsname{\let\PY@bf=\textbf\def\PY@tc##1{\textcolor[rgb]{0.00,0.50,0.00}{##1}}}
\expandafter\def\csname PY@tok@kd\endcsname{\let\PY@bf=\textbf\def\PY@tc##1{\textcolor[rgb]{0.00,0.50,0.00}{##1}}}
\expandafter\def\csname PY@tok@kn\endcsname{\let\PY@bf=\textbf\def\PY@tc##1{\textcolor[rgb]{0.00,0.50,0.00}{##1}}}
\expandafter\def\csname PY@tok@kr\endcsname{\let\PY@bf=\textbf\def\PY@tc##1{\textcolor[rgb]{0.00,0.50,0.00}{##1}}}
\expandafter\def\csname PY@tok@bp\endcsname{\def\PY@tc##1{\textcolor[rgb]{0.00,0.50,0.00}{##1}}}
\expandafter\def\csname PY@tok@fm\endcsname{\def\PY@tc##1{\textcolor[rgb]{0.00,0.00,1.00}{##1}}}
\expandafter\def\csname PY@tok@vc\endcsname{\def\PY@tc##1{\textcolor[rgb]{0.10,0.09,0.49}{##1}}}
\expandafter\def\csname PY@tok@vg\endcsname{\def\PY@tc##1{\textcolor[rgb]{0.10,0.09,0.49}{##1}}}
\expandafter\def\csname PY@tok@vi\endcsname{\def\PY@tc##1{\textcolor[rgb]{0.10,0.09,0.49}{##1}}}
\expandafter\def\csname PY@tok@vm\endcsname{\def\PY@tc##1{\textcolor[rgb]{0.10,0.09,0.49}{##1}}}
\expandafter\def\csname PY@tok@sa\endcsname{\def\PY@tc##1{\textcolor[rgb]{0.73,0.13,0.13}{##1}}}
\expandafter\def\csname PY@tok@sb\endcsname{\def\PY@tc##1{\textcolor[rgb]{0.73,0.13,0.13}{##1}}}
\expandafter\def\csname PY@tok@sc\endcsname{\def\PY@tc##1{\textcolor[rgb]{0.73,0.13,0.13}{##1}}}
\expandafter\def\csname PY@tok@dl\endcsname{\def\PY@tc##1{\textcolor[rgb]{0.73,0.13,0.13}{##1}}}
\expandafter\def\csname PY@tok@s2\endcsname{\def\PY@tc##1{\textcolor[rgb]{0.73,0.13,0.13}{##1}}}
\expandafter\def\csname PY@tok@sh\endcsname{\def\PY@tc##1{\textcolor[rgb]{0.73,0.13,0.13}{##1}}}
\expandafter\def\csname PY@tok@s1\endcsname{\def\PY@tc##1{\textcolor[rgb]{0.73,0.13,0.13}{##1}}}
\expandafter\def\csname PY@tok@mb\endcsname{\def\PY@tc##1{\textcolor[rgb]{0.40,0.40,0.40}{##1}}}
\expandafter\def\csname PY@tok@mf\endcsname{\def\PY@tc##1{\textcolor[rgb]{0.40,0.40,0.40}{##1}}}
\expandafter\def\csname PY@tok@mh\endcsname{\def\PY@tc##1{\textcolor[rgb]{0.40,0.40,0.40}{##1}}}
\expandafter\def\csname PY@tok@mi\endcsname{\def\PY@tc##1{\textcolor[rgb]{0.40,0.40,0.40}{##1}}}
\expandafter\def\csname PY@tok@il\endcsname{\def\PY@tc##1{\textcolor[rgb]{0.40,0.40,0.40}{##1}}}
\expandafter\def\csname PY@tok@mo\endcsname{\def\PY@tc##1{\textcolor[rgb]{0.40,0.40,0.40}{##1}}}
\expandafter\def\csname PY@tok@ch\endcsname{\let\PY@it=\textit\def\PY@tc##1{\textcolor[rgb]{0.25,0.50,0.50}{##1}}}
\expandafter\def\csname PY@tok@cm\endcsname{\let\PY@it=\textit\def\PY@tc##1{\textcolor[rgb]{0.25,0.50,0.50}{##1}}}
\expandafter\def\csname PY@tok@cpf\endcsname{\let\PY@it=\textit\def\PY@tc##1{\textcolor[rgb]{0.25,0.50,0.50}{##1}}}
\expandafter\def\csname PY@tok@c1\endcsname{\let\PY@it=\textit\def\PY@tc##1{\textcolor[rgb]{0.25,0.50,0.50}{##1}}}
\expandafter\def\csname PY@tok@cs\endcsname{\let\PY@it=\textit\def\PY@tc##1{\textcolor[rgb]{0.25,0.50,0.50}{##1}}}

\def\PYZbs{\char`\\}
\def\PYZus{\char`\_}
\def\PYZob{\char`\{}
\def\PYZcb{\char`\}}
\def\PYZca{\char`\^}
\def\PYZam{\char`\&}
\def\PYZlt{\char`\<}
\def\PYZgt{\char`\>}
\def\PYZsh{\char`\#}
\def\PYZpc{\char`\%}
\def\PYZdl{\char`\$}
\def\PYZhy{\char`\-}
\def\PYZsq{\char`\'}
\def\PYZdq{\char`\"}
\def\PYZti{\char`\~}
% for compatibility with earlier versions
\def\PYZat{@}
\def\PYZlb{[}
\def\PYZrb{]}
\makeatother


    % For linebreaks inside Verbatim environment from package fancyvrb. 
    \makeatletter
        \newbox\Wrappedcontinuationbox 
        \newbox\Wrappedvisiblespacebox 
        \newcommand*\Wrappedvisiblespace {\textcolor{red}{\textvisiblespace}} 
        \newcommand*\Wrappedcontinuationsymbol {\textcolor{red}{\llap{\tiny$\m@th\hookrightarrow$}}} 
        \newcommand*\Wrappedcontinuationindent {3ex } 
        \newcommand*\Wrappedafterbreak {\kern\Wrappedcontinuationindent\copy\Wrappedcontinuationbox} 
        % Take advantage of the already applied Pygments mark-up to insert 
        % potential linebreaks for TeX processing. 
        %        {, <, #, %, $, ' and ": go to next line. 
        %        _, }, ^, &, >, - and ~: stay at end of broken line. 
        % Use of \textquotesingle for straight quote. 
        \newcommand*\Wrappedbreaksatspecials {% 
            \def\PYGZus{\discretionary{\char`\_}{\Wrappedafterbreak}{\char`\_}}% 
            \def\PYGZob{\discretionary{}{\Wrappedafterbreak\char`\{}{\char`\{}}% 
            \def\PYGZcb{\discretionary{\char`\}}{\Wrappedafterbreak}{\char`\}}}% 
            \def\PYGZca{\discretionary{\char`\^}{\Wrappedafterbreak}{\char`\^}}% 
            \def\PYGZam{\discretionary{\char`\&}{\Wrappedafterbreak}{\char`\&}}% 
            \def\PYGZlt{\discretionary{}{\Wrappedafterbreak\char`\<}{\char`\<}}% 
            \def\PYGZgt{\discretionary{\char`\>}{\Wrappedafterbreak}{\char`\>}}% 
            \def\PYGZsh{\discretionary{}{\Wrappedafterbreak\char`\#}{\char`\#}}% 
            \def\PYGZpc{\discretionary{}{\Wrappedafterbreak\char`\%}{\char`\%}}% 
            \def\PYGZdl{\discretionary{}{\Wrappedafterbreak\char`\$}{\char`\$}}% 
            \def\PYGZhy{\discretionary{\char`\-}{\Wrappedafterbreak}{\char`\-}}% 
            \def\PYGZsq{\discretionary{}{\Wrappedafterbreak\textquotesingle}{\textquotesingle}}% 
            \def\PYGZdq{\discretionary{}{\Wrappedafterbreak\char`\"}{\char`\"}}% 
            \def\PYGZti{\discretionary{\char`\~}{\Wrappedafterbreak}{\char`\~}}% 
        } 
        % Some characters . , ; ? ! / are not pygmentized. 
        % This macro makes them "active" and they will insert potential linebreaks 
        \newcommand*\Wrappedbreaksatpunct {% 
            \lccode`\~`\.\lowercase{\def~}{\discretionary{\hbox{\char`\.}}{\Wrappedafterbreak}{\hbox{\char`\.}}}% 
            \lccode`\~`\,\lowercase{\def~}{\discretionary{\hbox{\char`\,}}{\Wrappedafterbreak}{\hbox{\char`\,}}}% 
            \lccode`\~`\;\lowercase{\def~}{\discretionary{\hbox{\char`\;}}{\Wrappedafterbreak}{\hbox{\char`\;}}}% 
            \lccode`\~`\:\lowercase{\def~}{\discretionary{\hbox{\char`\:}}{\Wrappedafterbreak}{\hbox{\char`\:}}}% 
            \lccode`\~`\?\lowercase{\def~}{\discretionary{\hbox{\char`\?}}{\Wrappedafterbreak}{\hbox{\char`\?}}}% 
            \lccode`\~`\!\lowercase{\def~}{\discretionary{\hbox{\char`\!}}{\Wrappedafterbreak}{\hbox{\char`\!}}}% 
            \lccode`\~`\/\lowercase{\def~}{\discretionary{\hbox{\char`\/}}{\Wrappedafterbreak}{\hbox{\char`\/}}}% 
            \catcode`\.\active
            \catcode`\,\active 
            \catcode`\;\active
            \catcode`\:\active
            \catcode`\?\active
            \catcode`\!\active
            \catcode`\/\active 
            \lccode`\~`\~ 	
        }
    \makeatother

    \let\OriginalVerbatim=\Verbatim
    \makeatletter
    \renewcommand{\Verbatim}[1][1]{%
        %\parskip\z@skip
        \sbox\Wrappedcontinuationbox {\Wrappedcontinuationsymbol}%
        \sbox\Wrappedvisiblespacebox {\FV@SetupFont\Wrappedvisiblespace}%
        \def\FancyVerbFormatLine ##1{\hsize\linewidth
            \vtop{\raggedright\hyphenpenalty\z@\exhyphenpenalty\z@
                \doublehyphendemerits\z@\finalhyphendemerits\z@
                \strut ##1\strut}%
        }%
        % If the linebreak is at a space, the latter will be displayed as visible
        % space at end of first line, and a continuation symbol starts next line.
        % Stretch/shrink are however usually zero for typewriter font.
        \def\FV@Space {%
            \nobreak\hskip\z@ plus\fontdimen3\font minus\fontdimen4\font
            \discretionary{\copy\Wrappedvisiblespacebox}{\Wrappedafterbreak}
            {\kern\fontdimen2\font}%
        }%
        
        % Allow breaks at special characters using \PYG... macros.
        \Wrappedbreaksatspecials
        % Breaks at punctuation characters . , ; ? ! and / need catcode=\active 	
        \OriginalVerbatim[#1,codes*=\Wrappedbreaksatpunct]%
    }
    \makeatother

    % Exact colors from NB
    \definecolor{incolor}{HTML}{303F9F}
    \definecolor{outcolor}{HTML}{D84315}
    \definecolor{cellborder}{HTML}{CFCFCF}
    \definecolor{cellbackground}{HTML}{F7F7F7}
    
    % prompt
    \makeatletter
    \newcommand{\boxspacing}{\kern\kvtcb@left@rule\kern\kvtcb@boxsep}
    \makeatother
    \newcommand{\prompt}[4]{
        \ttfamily\llap{{\color{#2}[#3]:\hspace{3pt}#4}}\vspace{-\baselineskip}
    }
    

    
    % Prevent overflowing lines due to hard-to-break entities
    \sloppy 
    % Setup hyperref package
    \hypersetup{
      breaklinks=true,  % so long urls are correctly broken across lines
      colorlinks=true,
      urlcolor=urlcolor,
      linkcolor=linkcolor,
      citecolor=citecolor,
      }
    % Slightly bigger margins than the latex defaults
    
    \geometry{verbose,tmargin=1in,bmargin=1in,lmargin=1in,rmargin=1in}
    
    

\begin{document}
    
    \maketitle
    
    

    
    \hypertarget{exemplary-sa-ccr-allocation-under-consideration-of-an-initial-margin-threshold}{%
\section{Exemplary SA-CCR allocation under consideration of an initial
margin
threshold}\label{exemplary-sa-ccr-allocation-under-consideration-of-an-initial-margin-threshold}}

Our goal is to perform an Euler allocation for the minimal example of a
one trade portfolio. We use the same 200Bn IRS as in previous examples.

    \begin{tcolorbox}[breakable, size=fbox, boxrule=1pt, pad at break*=1mm,colback=cellbackground, colframe=cellborder]
\prompt{In}{incolor}{2}{\boxspacing}
\begin{Verbatim}[commandchars=\\\{\}]
\PY{n}{irs} \PY{o}{=} \PY{n}{IRS}\PY{p}{(}\PY{n}{notional}\PY{o}{=}\PY{l+m+mi}{200000000000}\PY{p}{,}
          \PY{n}{index}\PY{o}{=}\PY{n}{InterestRateIndex}\PY{o}{.}\PY{n}{USDLIBOR3M}\PY{p}{,}
          \PY{n}{timeToSwapStart}\PY{o}{=}\PY{n}{ql}\PY{o}{.}\PY{n}{Period}\PY{p}{(}\PY{l+m+mi}{2}\PY{p}{,} \PY{n}{ql}\PY{o}{.}\PY{n}{Days}\PY{p}{)}\PY{p}{,}
          \PY{n}{timeToSwapEnd}\PY{o}{=}\PY{n}{ql}\PY{o}{.}\PY{n}{Period}\PY{p}{(}\PY{l+m+mi}{10}\PY{p}{,} \PY{n}{ql}\PY{o}{.}\PY{n}{Years}\PY{p}{)}\PY{p}{)}
\end{Verbatim}
\end{tcolorbox}

    \begin{tcolorbox}[breakable, size=fbox, boxrule=1pt, pad at break*=1mm,colback=cellbackground, colframe=cellborder]
\prompt{In}{incolor}{3}{\boxspacing}
\begin{Verbatim}[commandchars=\\\{\}]
\PY{n}{ca} \PY{o}{=} \PY{n}{CollateralAgreement}\PY{p}{(}\PY{n}{threshold}\PY{o}{=}\PY{l+m+mi}{2000000000}\PY{p}{,}
                         \PY{n}{mta}\PY{o}{=}\PY{l+m+mi}{0}\PY{p}{)}
\PY{n}{ca}\PY{o}{.}\PY{n}{link\PYZus{}sa\PYZus{}ccr\PYZus{}instance}\PY{p}{(}\PY{n}{SA\PYZus{}CCR}\PY{p}{(}\PY{n}{ca}\PY{p}{)}\PY{p}{)}
\PY{n}{ca}\PY{o}{.}\PY{n}{add\PYZus{}trades}\PY{p}{(}\PY{n}{irs}\PY{p}{)}
\end{Verbatim}
\end{tcolorbox}

    The inclusion of the threshold raises the ead since it lowers the
available overcollataralization.

    \begin{tcolorbox}[breakable, size=fbox, boxrule=1pt, pad at break*=1mm,colback=cellbackground, colframe=cellborder]
\prompt{In}{incolor}{4}{\boxspacing}
\begin{Verbatim}[commandchars=\\\{\}]
\PY{n}{ead\PYZus{}with\PYZus{}threshold} \PY{o}{=} \PY{n}{ca}\PY{o}{.}\PY{n}{get\PYZus{}sa\PYZus{}ccr\PYZus{}model}\PY{p}{(}\PY{p}{)}\PY{o}{.}\PY{n}{get\PYZus{}risk\PYZus{}measure}\PY{p}{(}\PY{p}{)}
\PY{n+nb}{print}\PY{p}{(}\PY{n}{ead\PYZus{}with\PYZus{}threshold}\PY{p}{)}
\PY{n}{ca}\PY{o}{.}\PY{n}{threshold} \PY{o}{=} \PY{l+m+mi}{0}
\PY{n}{ead\PYZus{}no\PYZus{}threshold} \PY{o}{=} \PY{n}{ca}\PY{o}{.}\PY{n}{get\PYZus{}sa\PYZus{}ccr\PYZus{}model}\PY{p}{(}\PY{p}{)}\PY{o}{.}\PY{n}{get\PYZus{}risk\PYZus{}measure}\PY{p}{(}\PY{p}{)}
\PY{n+nb}{print}\PY{p}{(}\PY{n}{ead\PYZus{}no\PYZus{}threshold}\PY{p}{)}
\PY{n}{ca}\PY{o}{.}\PY{n}{threshold} \PY{o}{=} \PY{l+m+mi}{2000000000}
\end{Verbatim}
\end{tcolorbox}

    \begin{Verbatim}[commandchars=\\\{\}]

        ---------------------------------------------------------------------------

        ConnectionRefusedError                    Traceback (most recent call last)

        \textasciitilde{}\textbackslash{}.conda\textbackslash{}envs\textbackslash{}SACCRconda\textbackslash{}lib\textbackslash{}site-packages\textbackslash{}urllib3\textbackslash{}connection.py in \_new\_conn(self)
        155         try:
    --> 156             conn = connection.create\_connection(
        157                 (self.\_dns\_host, self.port), self.timeout, **extra\_kw
    

        \textasciitilde{}\textbackslash{}.conda\textbackslash{}envs\textbackslash{}SACCRconda\textbackslash{}lib\textbackslash{}site-packages\textbackslash{}urllib3\textbackslash{}util\textbackslash{}connection.py in create\_connection(address, timeout, source\_address, socket\_options)
         83     if err is not None:
    ---> 84         raise err
         85 
    

        \textasciitilde{}\textbackslash{}.conda\textbackslash{}envs\textbackslash{}SACCRconda\textbackslash{}lib\textbackslash{}site-packages\textbackslash{}urllib3\textbackslash{}util\textbackslash{}connection.py in create\_connection(address, timeout, source\_address, socket\_options)
         73                 sock.bind(source\_address)
    ---> 74             sock.connect(sa)
         75             return sock
    

        ConnectionRefusedError: [WinError 10061] No connection could be made because the target machine actively refused it

        
    During handling of the above exception, another exception occurred:
    

        NewConnectionError                        Traceback (most recent call last)

        \textasciitilde{}\textbackslash{}.conda\textbackslash{}envs\textbackslash{}SACCRconda\textbackslash{}lib\textbackslash{}site-packages\textbackslash{}urllib3\textbackslash{}connectionpool.py in urlopen(self, method, url, body, headers, retries, redirect, assert\_same\_host, timeout, pool\_timeout, release\_conn, chunked, body\_pos, **response\_kw)
        664             \# Make the request on the httplib connection object.
    --> 665             httplib\_response = self.\_make\_request(
        666                 conn,
    

        \textasciitilde{}\textbackslash{}.conda\textbackslash{}envs\textbackslash{}SACCRconda\textbackslash{}lib\textbackslash{}site-packages\textbackslash{}urllib3\textbackslash{}connectionpool.py in \_make\_request(self, conn, method, url, timeout, chunked, **httplib\_request\_kw)
        386         else:
    --> 387             conn.request(method, url, **httplib\_request\_kw)
        388 
    

        \textasciitilde{}\textbackslash{}.conda\textbackslash{}envs\textbackslash{}SACCRconda\textbackslash{}lib\textbackslash{}http\textbackslash{}client.py in request(self, method, url, body, headers, encode\_chunked)
       1229         """Send a complete request to the server."""
    -> 1230         self.\_send\_request(method, url, body, headers, encode\_chunked)
       1231 
    

        \textasciitilde{}\textbackslash{}.conda\textbackslash{}envs\textbackslash{}SACCRconda\textbackslash{}lib\textbackslash{}http\textbackslash{}client.py in \_send\_request(self, method, url, body, headers, encode\_chunked)
       1275             body = \_encode(body, 'body')
    -> 1276         self.endheaders(body, encode\_chunked=encode\_chunked)
       1277 
    

        \textasciitilde{}\textbackslash{}.conda\textbackslash{}envs\textbackslash{}SACCRconda\textbackslash{}lib\textbackslash{}http\textbackslash{}client.py in endheaders(self, message\_body, encode\_chunked)
       1224             raise CannotSendHeader()
    -> 1225         self.\_send\_output(message\_body, encode\_chunked=encode\_chunked)
       1226 
    

        \textasciitilde{}\textbackslash{}.conda\textbackslash{}envs\textbackslash{}SACCRconda\textbackslash{}lib\textbackslash{}http\textbackslash{}client.py in \_send\_output(self, message\_body, encode\_chunked)
       1003         del self.\_buffer[:]
    -> 1004         self.send(msg)
       1005 
    

        \textasciitilde{}\textbackslash{}.conda\textbackslash{}envs\textbackslash{}SACCRconda\textbackslash{}lib\textbackslash{}http\textbackslash{}client.py in send(self, data)
        943             if self.auto\_open:
    --> 944                 self.connect()
        945             else:
    

        \textasciitilde{}\textbackslash{}.conda\textbackslash{}envs\textbackslash{}SACCRconda\textbackslash{}lib\textbackslash{}site-packages\textbackslash{}urllib3\textbackslash{}connection.py in connect(self)
        183     def connect(self):
    --> 184         conn = self.\_new\_conn()
        185         self.\_prepare\_conn(conn)
    

        \textasciitilde{}\textbackslash{}.conda\textbackslash{}envs\textbackslash{}SACCRconda\textbackslash{}lib\textbackslash{}site-packages\textbackslash{}urllib3\textbackslash{}connection.py in \_new\_conn(self)
        167         except SocketError as e:
    --> 168             raise NewConnectionError(
        169                 self, "Failed to establish a new connection: \%s" \% e
    

        NewConnectionError: <urllib3.connection.HTTPConnection object at 0x0000023761F45760>: Failed to establish a new connection: [WinError 10061] No connection could be made because the target machine actively refused it

        
    During handling of the above exception, another exception occurred:
    

        MaxRetryError                             Traceback (most recent call last)

        \textasciitilde{}\textbackslash{}.conda\textbackslash{}envs\textbackslash{}SACCRconda\textbackslash{}lib\textbackslash{}site-packages\textbackslash{}requests\textbackslash{}adapters.py in send(self, request, stream, timeout, verify, cert, proxies)
        438             if not chunked:
    --> 439                 resp = conn.urlopen(
        440                     method=request.method,
    

        \textasciitilde{}\textbackslash{}.conda\textbackslash{}envs\textbackslash{}SACCRconda\textbackslash{}lib\textbackslash{}site-packages\textbackslash{}urllib3\textbackslash{}connectionpool.py in urlopen(self, method, url, body, headers, retries, redirect, assert\_same\_host, timeout, pool\_timeout, release\_conn, chunked, body\_pos, **response\_kw)
        718 
    --> 719             retries = retries.increment(
        720                 method, url, error=e, \_pool=self, \_stacktrace=sys.exc\_info()[2]
    

        \textasciitilde{}\textbackslash{}.conda\textbackslash{}envs\textbackslash{}SACCRconda\textbackslash{}lib\textbackslash{}site-packages\textbackslash{}urllib3\textbackslash{}util\textbackslash{}retry.py in increment(self, method, url, response, error, \_pool, \_stacktrace)
        435         if new\_retry.is\_exhausted():
    --> 436             raise MaxRetryError(\_pool, url, error or ResponseError(cause))
        437 
    

        MaxRetryError: HTTPConnectionPool(host='localhost', port=8080): Max retries exceeded with url: /crifs (Caused by NewConnectionError('<urllib3.connection.HTTPConnection object at 0x0000023761F45760>: Failed to establish a new connection: [WinError 10061] No connection could be made because the target machine actively refused it'))

        
    During handling of the above exception, another exception occurred:
    

        ConnectionError                           Traceback (most recent call last)

        <ipython-input-4-61de6b79334f> in <module>
    ----> 1 ead\_with\_threshold = ca.get\_sa\_ccr\_model().get\_risk\_measure()
          2 print(ead\_with\_threshold)
          3 ca.threshold = 0
          4 ead\_no\_threshold = ca.get\_sa\_ccr\_model().get\_risk\_measure()
          5 print(ead\_no\_threshold)
    

        C:\textbackslash{}Oxford\textbackslash{}Master\_Thesis\textbackslash{}Allocation\_Thesis\textbackslash{}SA\_CCR\_Allocation\textbackslash{}src\textbackslash{}sa\_ccr\textbackslash{}sa\_ccr.py in get\_risk\_measure(self)
        306 
        307     def get\_risk\_measure(self):
    --> 308         return self.get\_ead()
        309 
        310     def get\_margining\_factor(self, trade: Trade):
    

        C:\textbackslash{}Oxford\textbackslash{}Master\_Thesis\textbackslash{}Allocation\_Thesis\textbackslash{}SA\_CCR\_Allocation\textbackslash{}src\textbackslash{}sa\_ccr\textbackslash{}sa\_ccr.py in get\_ead(self)
         25     def get\_ead(self):
         26 
    ---> 27         pfe = SA\_CCR.calculate\_pfe(self.trades, self.collateralAgreement)['PFE']
         28         rc = SA\_CCR.calculate\_rc(self.trades, self.collateralAgreement)
         29         ead = SA\_CCR.calculate\_sa\_ccr\_ead(rc, pfe)
    

        C:\textbackslash{}Oxford\textbackslash{}Master\_Thesis\textbackslash{}Allocation\_Thesis\textbackslash{}SA\_CCR\_Allocation\textbackslash{}src\textbackslash{}sa\_ccr\textbackslash{}sa\_ccr.py in calculate\_pfe(trades, ca)
        278         V = ca.get\_V()
        279 
    --> 280         C = ca.get\_C()
        281         aggregate\_addOn = addOns.sum()
        282         multiplier\_var = SA\_CCR.multiplier(V, C, aggregate\_addOn)
    

        C:\textbackslash{}Oxford\textbackslash{}Master\_Thesis\textbackslash{}Allocation\_Thesis\textbackslash{}SA\_CCR\_Allocation\textbackslash{}src\textbackslash{}collateralAgreement\textbackslash{}collateralAgreement.py in get\_C(self)
        120 
        121     def get\_C(self):
    --> 122         calculated\_collateral = self.vm\_model.get\_vm() + self.get\_nica()
        123         if abs(self.start\_collateral\_amount - calculated\_collateral)<self.mta:
        124             return self.start\_collateral\_amount
    

        C:\textbackslash{}Oxford\textbackslash{}Master\_Thesis\textbackslash{}Allocation\_Thesis\textbackslash{}SA\_CCR\_Allocation\textbackslash{}src\textbackslash{}collateralAgreement\textbackslash{}collateralAgreement.py in get\_nica(self)
        134 
        135     def get\_nica(self):
    --> 136         return max(0, self.im\_model.get\_im\_receive()-self.threshold) - self.unsegregated\_overcollateraliziation\_posted + self.segregated\_overcollateralization\_received + self.unsegregated\_overcollateralization\_received
        137 
        138     @property
    

        C:\textbackslash{}Oxford\textbackslash{}Master\_Thesis\textbackslash{}Allocation\_Thesis\textbackslash{}SA\_CCR\_Allocation\textbackslash{}src\textbackslash{}margining\textbackslash{}simm.py in get\_im\_receive(self)
         30 
         31         crif = self.\_\_createCrif\_\_()
    ---> 32         self.upload\_id = postCrif(crif)
         33         self.im\_receive = calculateCrif(self.upload\_id, imRole=im\_role, calculationCurrency=self.calculationCurrency,
         34                                         resultCurrency=self.resultCurrency)
    

        C:\textbackslash{}Oxford\textbackslash{}Master\_Thesis\textbackslash{}Allocation\_Thesis\textbackslash{}SA\_CCR\_Allocation\textbackslash{}src\textbackslash{}margining\textbackslash{}simmAPI.py in postCrif(crif)
         17 def postCrif(crif: List[Dict]) -> int:
         18     loc = 'crifs'
    ---> 19     r = requests.post(home + loc, json=crif)
         20     \_\_checkError\_\_(r)
         21     return int(r.content)
    

        \textasciitilde{}\textbackslash{}.conda\textbackslash{}envs\textbackslash{}SACCRconda\textbackslash{}lib\textbackslash{}site-packages\textbackslash{}requests\textbackslash{}api.py in post(url, data, json, **kwargs)
        117     """
        118 
    --> 119     return request('post', url, data=data, json=json, **kwargs)
        120 
        121 
    

        \textasciitilde{}\textbackslash{}.conda\textbackslash{}envs\textbackslash{}SACCRconda\textbackslash{}lib\textbackslash{}site-packages\textbackslash{}requests\textbackslash{}api.py in request(method, url, **kwargs)
         59     \# cases, and look like a memory leak in others.
         60     with sessions.Session() as session:
    ---> 61         return session.request(method=method, url=url, **kwargs)
         62 
         63 
    

        \textasciitilde{}\textbackslash{}.conda\textbackslash{}envs\textbackslash{}SACCRconda\textbackslash{}lib\textbackslash{}site-packages\textbackslash{}requests\textbackslash{}sessions.py in request(self, method, url, params, data, headers, cookies, files, auth, timeout, allow\_redirects, proxies, hooks, stream, verify, cert, json)
        528         \}
        529         send\_kwargs.update(settings)
    --> 530         resp = self.send(prep, **send\_kwargs)
        531 
        532         return resp
    

        \textasciitilde{}\textbackslash{}.conda\textbackslash{}envs\textbackslash{}SACCRconda\textbackslash{}lib\textbackslash{}site-packages\textbackslash{}requests\textbackslash{}sessions.py in send(self, request, **kwargs)
        641 
        642         \# Send the request
    --> 643         r = adapter.send(request, **kwargs)
        644 
        645         \# Total elapsed time of the request (approximately)
    

        \textasciitilde{}\textbackslash{}.conda\textbackslash{}envs\textbackslash{}SACCRconda\textbackslash{}lib\textbackslash{}site-packages\textbackslash{}requests\textbackslash{}adapters.py in send(self, request, stream, timeout, verify, cert, proxies)
        514                 raise SSLError(e, request=request)
        515 
    --> 516             raise ConnectionError(e, request=request)
        517 
        518         except ClosedPoolError as e:
    

        ConnectionError: HTTPConnectionPool(host='localhost', port=8080): Max retries exceeded with url: /crifs (Caused by NewConnectionError('<urllib3.connection.HTTPConnection object at 0x0000023761F45760>: Failed to establish a new connection: [WinError 10061] No connection could be made because the target machine actively refused it'))

    \end{Verbatim}

    When trying to allocate with threshold we realize that the allocation is
not working due to the missing homogeneity of C.

    \begin{tcolorbox}[breakable, size=fbox, boxrule=1pt, pad at break*=1mm,colback=cellbackground, colframe=cellborder]
\prompt{In}{incolor}{ }{\boxspacing}
\begin{Verbatim}[commandchars=\\\{\}]
\PY{n}{eulerAllocator} \PY{o}{=} \PY{n}{EulerAllocator}\PY{p}{(}\PY{n}{ca}\PY{p}{)}
\PY{n}{allocated\PYZus{}value} \PY{o}{=} \PY{n}{eulerAllocator}\PY{o}{.}\PY{n}{allocate\PYZus{}ead}\PY{p}{(}\PY{p}{)}\PY{p}{[}\PY{n}{irs}\PY{p}{]}
\PY{n+nb}{print}\PY{p}{(}\PY{n}{allocated\PYZus{}value}\PY{p}{)}
\end{Verbatim}
\end{tcolorbox}

    If we instead allocate without threshold, the allocation works.

    \begin{tcolorbox}[breakable, size=fbox, boxrule=1pt, pad at break*=1mm,colback=cellbackground, colframe=cellborder]
\prompt{In}{incolor}{ }{\boxspacing}
\begin{Verbatim}[commandchars=\\\{\}]
\PY{n}{ca}\PY{o}{.}\PY{n}{threshold} \PY{o}{=} \PY{l+m+mi}{0}
\PY{n}{allocated\PYZus{}value}\PY{o}{=}\PY{n}{eulerAllocator}\PY{o}{.}\PY{n}{allocate\PYZus{}ead}\PY{p}{(}\PY{p}{)}\PY{p}{[}\PY{n}{irs}\PY{p}{]}
\PY{n+nb}{print}\PY{p}{(}\PY{n}{allocated\PYZus{}value}\PY{p}{)}
\end{Verbatim}
\end{tcolorbox}

    A reasonable approach to allocate an SA-CCR EAD under consideration of a
threshold could be to allocate without threshold and then scale
accordingly:

\begin{align}
\label{eq:C threshold scaling}
    X_{t\text{, TH}} = X_{t\text{, no TH}} \frac{EAD_{\text{TH}}}{EAD_{\text{no TH}}}
\end{align}

    \begin{tcolorbox}[breakable, size=fbox, boxrule=1pt, pad at break*=1mm,colback=cellbackground, colframe=cellborder]
\prompt{In}{incolor}{ }{\boxspacing}
\begin{Verbatim}[commandchars=\\\{\}]
\PY{n+nb}{print}\PY{p}{(}\PY{n}{allocated\PYZus{}value} \PY{o}{*} \PY{p}{(}\PY{n}{ead\PYZus{}with\PYZus{}threshold}\PY{o}{/}\PY{n}{ead\PYZus{}no\PYZus{}threshold}\PY{p}{)}\PY{p}{)}
\end{Verbatim}
\end{tcolorbox}

    However, this approach does coincide with a loss of precision. If the
\(C_{calc}\) is below the threshold then \(C\) is 0 and exhibits
homogeneity, even when taking the threshold into account. We can
construct an example that shows, that the approach in
\ref{eq:C threshold scaling} does not yield the correct allocation which
can be calculated when taking the threshold into account since the IM is
below the IM threshold.

We set up an IRS and an Equity option that have a similar EAD on their
own. Here we calculate with a threshold of 50Mn which is a common value
as it is the maximum amount permitted by the regulator.
\todo{Cite source for this}

    \begin{tcolorbox}[breakable, size=fbox, boxrule=1pt, pad at break*=1mm,colback=cellbackground, colframe=cellborder]
\prompt{In}{incolor}{ }{\boxspacing}
\begin{Verbatim}[commandchars=\\\{\}]
\PY{n}{irs} \PY{o}{=} \PY{n}{IRS}\PY{p}{(}\PY{n}{notional}\PY{o}{=}\PY{l+m+mi}{100000000}\PY{p}{,}
          \PY{n}{index} \PY{o}{=} \PY{n}{InterestRateIndex}\PY{o}{.}\PY{n}{USDLIBOR3M}\PY{p}{,}
          \PY{n}{timeToSwapStart}\PY{o}{=}\PY{n}{ql}\PY{o}{.}\PY{n}{Period}\PY{p}{(}\PY{l+m+mi}{2}\PY{p}{,} \PY{n}{ql}\PY{o}{.}\PY{n}{Days}\PY{p}{)}\PY{p}{,}
          \PY{n}{timeToSwapEnd}\PY{o}{=}\PY{n}{ql}\PY{o}{.}\PY{n}{Period}\PY{p}{(}\PY{l+m+mi}{10}\PY{p}{,} \PY{n}{ql}\PY{o}{.}\PY{n}{Years}\PY{p}{)}\PY{p}{)}

\PY{n}{ca} \PY{o}{=} \PY{n}{CollateralAgreement}\PY{p}{(}\PY{n}{threshold}\PY{o}{=}\PY{l+m+mi}{50000000}\PY{p}{)}
\PY{n}{ca}\PY{o}{.}\PY{n}{link\PYZus{}sa\PYZus{}ccr\PYZus{}instance}\PY{p}{(}\PY{n}{SA\PYZus{}CCR}\PY{p}{(}\PY{n}{ca}\PY{p}{)}\PY{p}{)}
\PY{n}{ca}\PY{o}{.}\PY{n}{add\PYZus{}trades}\PY{p}{(}\PY{n}{irs}\PY{p}{)}
\end{Verbatim}
\end{tcolorbox}

    \begin{tcolorbox}[breakable, size=fbox, boxrule=1pt, pad at break*=1mm,colback=cellbackground, colframe=cellborder]
\prompt{In}{incolor}{ }{\boxspacing}
\begin{Verbatim}[commandchars=\\\{\}]
\PY{n+nb}{print}\PY{p}{(}\PY{n}{ca}\PY{o}{.}\PY{n}{get\PYZus{}im\PYZus{}model}\PY{p}{(}\PY{p}{)}\PY{o}{.}\PY{n}{get\PYZus{}risk\PYZus{}measure}\PY{p}{(}\PY{p}{)}\PY{p}{)}
\PY{n+nb}{print}\PY{p}{(}\PY{n}{ca}\PY{o}{.}\PY{n}{get\PYZus{}sa\PYZus{}ccr\PYZus{}model}\PY{p}{(}\PY{p}{)}\PY{o}{.}\PY{n}{get\PYZus{}risk\PYZus{}measure}\PY{p}{(}\PY{p}{)}\PY{p}{)}
\end{Verbatim}
\end{tcolorbox}

    \begin{tcolorbox}[breakable, size=fbox, boxrule=1pt, pad at break*=1mm,colback=cellbackground, colframe=cellborder]
\prompt{In}{incolor}{ }{\boxspacing}
\begin{Verbatim}[commandchars=\\\{\}]
\PY{n}{eqOpt} \PY{o}{=} \PY{n}{EquityOption}\PY{p}{(}\PY{n}{notional} \PY{o}{=} \PY{l+m+mi}{403106}\PY{p}{,}
                     \PY{n}{maturity} \PY{o}{=} \PY{n}{ql}\PY{o}{.}\PY{n}{Period}\PY{p}{(}\PY{l+m+mi}{1}\PY{p}{,} \PY{n}{ql}\PY{o}{.}\PY{n}{Years}\PY{p}{)}\PY{p}{,}
                     \PY{n}{underlying} \PY{o}{=} \PY{n}{Stock}\PY{o}{.}\PY{n}{ADS}\PY{p}{,}
                     \PY{n}{tradeType} \PY{o}{=} \PY{n}{TradeType}\PY{o}{.}\PY{n}{CALL}\PY{p}{,}
                     \PY{n}{tradeDirection} \PY{o}{=} \PY{n}{TradeDirection}\PY{o}{.}\PY{n}{LONG}\PY{p}{)}
\end{Verbatim}
\end{tcolorbox}

    \begin{tcolorbox}[breakable, size=fbox, boxrule=1pt, pad at break*=1mm,colback=cellbackground, colframe=cellborder]
\prompt{In}{incolor}{ }{\boxspacing}
\begin{Verbatim}[commandchars=\\\{\}]
\PY{n}{eqOpt} \PY{o}{=} \PY{n}{EquityOption}\PY{p}{(}\PY{n}{notional} \PY{o}{=} \PY{n}{eqOptNot}\PY{o}{.}\PY{n}{x}\PY{p}{[}\PY{l+m+mi}{0}\PY{p}{]}\PY{p}{)}
\PY{n}{ca2} \PY{o}{=} \PY{n}{CollateralAgreement}\PY{p}{(}\PY{n}{threshold} \PY{o}{=} \PY{l+m+mi}{50000000}\PY{p}{)}
\PY{n}{ca2}\PY{o}{.}\PY{n}{link\PYZus{}sa\PYZus{}ccr\PYZus{}instance}\PY{p}{(}\PY{n}{SA\PYZus{}CCR}\PY{p}{(}\PY{n}{ca2}\PY{p}{)}\PY{p}{)}
\PY{n}{ca2}\PY{o}{.}\PY{n}{add\PYZus{}trades}\PY{p}{(}\PY{n}{eqOpt}\PY{p}{)}
\end{Verbatim}
\end{tcolorbox}

    The initial margin of this single trade is also far below the 50Mn
threshold but differs significantly from the IM of the IRS.

    \begin{tcolorbox}[breakable, size=fbox, boxrule=1pt, pad at break*=1mm,colback=cellbackground, colframe=cellborder]
\prompt{In}{incolor}{ }{\boxspacing}
\begin{Verbatim}[commandchars=\\\{\}]
\PY{n+nb}{print}\PY{p}{(}\PY{l+s+s1}{\PYZsq{}}\PY{l+s+s1}{EAD: }\PY{l+s+s1}{\PYZsq{}} \PY{o}{+} \PY{n+nb}{str}\PY{p}{(}\PY{n}{ca2}\PY{o}{.}\PY{n}{get\PYZus{}sa\PYZus{}ccr\PYZus{}model}\PY{p}{(}\PY{p}{)}\PY{o}{.}\PY{n}{get\PYZus{}risk\PYZus{}measure}\PY{p}{(}\PY{p}{)}\PY{p}{)}\PY{p}{)}
\PY{n+nb}{print}\PY{p}{(}\PY{l+s+s1}{\PYZsq{}}\PY{l+s+s1}{IM:  }\PY{l+s+s1}{\PYZsq{}} \PY{o}{+} \PY{n+nb}{str}\PY{p}{(}\PY{n}{ca2}\PY{o}{.}\PY{n}{get\PYZus{}im\PYZus{}model}\PY{p}{(}\PY{p}{)}\PY{o}{.}\PY{n}{get\PYZus{}risk\PYZus{}measure}\PY{p}{(}\PY{p}{)}\PY{p}{)}\PY{p}{)}
\end{Verbatim}
\end{tcolorbox}

    When putting both trades in a common portfolio we observe, that the EAD
and the IM of this joint portfolio is the sum of the two separate
portfolios. This is not surprising since both, the SA CCR and ISDA SIMM
model do not recognize any hedge effect between different asset classes.

    \begin{tcolorbox}[breakable, size=fbox, boxrule=1pt, pad at break*=1mm,colback=cellbackground, colframe=cellborder]
\prompt{In}{incolor}{ }{\boxspacing}
\begin{Verbatim}[commandchars=\\\{\}]
\PY{n}{ca3} \PY{o}{=} \PY{n}{CollateralAgreement}\PY{p}{(}\PY{n}{threshold}\PY{o}{=}\PY{l+m+mi}{50000000}\PY{p}{)}
\PY{n}{ca3}\PY{o}{.}\PY{n}{link\PYZus{}sa\PYZus{}ccr\PYZus{}instance}\PY{p}{(}\PY{n}{SA\PYZus{}CCR}\PY{p}{(}\PY{n}{ca3}\PY{p}{)}\PY{p}{)}
\PY{n}{ca3}\PY{o}{.}\PY{n}{add\PYZus{}trades}\PY{p}{(}\PY{p}{[}\PY{n}{irs}\PY{p}{,} \PY{n}{eqOpt}\PY{p}{]}\PY{p}{)}
\end{Verbatim}
\end{tcolorbox}

    \begin{tcolorbox}[breakable, size=fbox, boxrule=1pt, pad at break*=1mm,colback=cellbackground, colframe=cellborder]
\prompt{In}{incolor}{ }{\boxspacing}
\begin{Verbatim}[commandchars=\\\{\}]
\PY{n}{eulerAllocator} \PY{o}{=} \PY{n}{EulerAllocator}\PY{p}{(}\PY{n}{ca3}\PY{p}{)}
\PY{n}{ca3}\PY{o}{.}\PY{n}{threshold} \PY{o}{=} \PY{l+m+mi}{50000000}
\PY{n}{allocation} \PY{o}{=} \PY{n}{eulerAllocator}\PY{o}{.}\PY{n}{allocate\PYZus{}ead}\PY{p}{(}\PY{p}{)}
\end{Verbatim}
\end{tcolorbox}

    Due to the high threshold, no IM is exchanged. Only VM is exchanged
which is not overcollateralization and therefore only reduces the \(RC\)
in formula \todo{reference RC+PFE formula} to 0 but does not impact the
\(PFE\).

    The 50/50 allocation is certainly the correct result since there are no
hedge effects between the two trades and they both have the same stand
alone EAD. However, when applying formula \ref{eq:C threshold scaling}
we yield a different result.

    \begin{tcolorbox}[breakable, size=fbox, boxrule=1pt, pad at break*=1mm,colback=cellbackground, colframe=cellborder]
\prompt{In}{incolor}{ }{\boxspacing}
\begin{Verbatim}[commandchars=\\\{\}]
\PY{n}{ead\PYZus{}with\PYZus{}threshold} \PY{o}{=} \PY{n}{ca3}\PY{o}{.}\PY{n}{get\PYZus{}sa\PYZus{}ccr\PYZus{}model}\PY{p}{(}\PY{p}{)}\PY{o}{.}\PY{n}{get\PYZus{}risk\PYZus{}measure}\PY{p}{(}\PY{p}{)}
\PY{n}{ca3}\PY{o}{.}\PY{n}{threshold}\PY{o}{=}\PY{l+m+mi}{0}
\PY{n}{ead\PYZus{}no\PYZus{}threshold} \PY{o}{=} \PY{n}{ca3}\PY{o}{.}\PY{n}{get\PYZus{}sa\PYZus{}ccr\PYZus{}model}\PY{p}{(}\PY{p}{)}\PY{o}{.}\PY{n}{get\PYZus{}risk\PYZus{}measure}\PY{p}{(}\PY{p}{)}
\PY{n}{allocation} \PY{o}{=} \PY{n}{eulerAllocator}\PY{o}{.}\PY{n}{allocate\PYZus{}ead}\PY{p}{(}\PY{p}{)}
\PY{n}{factor} \PY{o}{=} \PY{n}{ead\PYZus{}with\PYZus{}threshold}\PY{o}{/}\PY{n}{ead\PYZus{}no\PYZus{}threshold}
\PY{n+nb}{print}\PY{p}{(}\PY{n}{allocation}\PY{p}{[}\PY{n}{irs}\PY{p}{]}\PY{o}{*}\PY{n}{factor}\PY{p}{)}
\PY{n}{ca3}\PY{o}{.}\PY{n}{threshold} \PY{o}{=} \PY{l+m+mi}{50000000}
\end{Verbatim}
\end{tcolorbox}

    Therefore, the approximation of \ref{eq:C threshold scaling} should only
be used if the \(IM_{calc} > TH_{IM}\).

When we exceed the threshold we can see that the Euler allocation with
threshold returns the same results as the Euler allocation without
thresholds. For this we decrease the IM threshold from 10Mn to 1Mn.

    \begin{tcolorbox}[breakable, size=fbox, boxrule=1pt, pad at break*=1mm,colback=cellbackground, colframe=cellborder]
\prompt{In}{incolor}{ }{\boxspacing}
\begin{Verbatim}[commandchars=\\\{\}]
\PY{n}{ca3}\PY{o}{.}\PY{n}{threshold} \PY{o}{=} \PY{l+m+mi}{50000000}
\PY{n}{allocation\PYZus{}10Mn\PYZus{}threshold} \PY{o}{=} \PY{n}{eulerAllocator}\PY{o}{.}\PY{n}{allocate\PYZus{}ead}\PY{p}{(}\PY{p}{)}
\PY{n+nb}{print}\PY{p}{(}\PY{n}{ca3}\PY{o}{.}\PY{n}{get\PYZus{}sa\PYZus{}ccr\PYZus{}model}\PY{p}{(}\PY{p}{)}\PY{o}{.}\PY{n}{get\PYZus{}ead}\PY{p}{(}\PY{p}{)} \PY{o}{\PYZhy{}} \PY{n+nb}{sum}\PY{p}{(}\PY{n}{allocation\PYZus{}10Mn\PYZus{}threshold}\PY{o}{.}\PY{n}{values}\PY{p}{(}\PY{p}{)}\PY{p}{)}\PY{p}{)}


\PY{n}{ca3}\PY{o}{.}\PY{n}{threshold} \PY{o}{=} \PY{l+m+mi}{1000000}
\PY{n}{allocation\PYZus{}1Mn\PYZus{}threshold} \PY{o}{=} \PY{n}{eulerAllocator}\PY{o}{.}\PY{n}{allocate\PYZus{}ead}\PY{p}{(}\PY{p}{)}
\PY{n+nb}{print}\PY{p}{(}\PY{n}{ca3}\PY{o}{.}\PY{n}{get\PYZus{}sa\PYZus{}ccr\PYZus{}model}\PY{p}{(}\PY{p}{)}\PY{o}{.}\PY{n}{get\PYZus{}ead}\PY{p}{(}\PY{p}{)} \PY{o}{\PYZhy{}} \PY{n+nb}{sum}\PY{p}{(}\PY{n}{allocation\PYZus{}1Mn\PYZus{}threshold}\PY{o}{.}\PY{n}{values}\PY{p}{(}\PY{p}{)}\PY{p}{)}\PY{p}{)}

\PY{n}{ca3}\PY{o}{.}\PY{n}{threshold} \PY{o}{=} \PY{l+m+mi}{0}
\PY{n}{allocation\PYZus{}no\PYZus{}threshold} \PY{o}{=} \PY{n}{eulerAllocator}\PY{o}{.}\PY{n}{allocate\PYZus{}ead}\PY{p}{(}\PY{p}{)}
\PY{n+nb}{print}\PY{p}{(}\PY{n}{ca3}\PY{o}{.}\PY{n}{get\PYZus{}sa\PYZus{}ccr\PYZus{}model}\PY{p}{(}\PY{p}{)}\PY{o}{.}\PY{n}{get\PYZus{}ead}\PY{p}{(}\PY{p}{)} \PY{o}{\PYZhy{}} \PY{n+nb}{sum}\PY{p}{(}\PY{n}{allocation\PYZus{}no\PYZus{}threshold}\PY{o}{.}\PY{n}{values}\PY{p}{(}\PY{p}{)}\PY{p}{)}\PY{p}{)}
\end{Verbatim}
\end{tcolorbox}


    % Add a bibliography block to the postdoc
    
    
    
\end{document}


        

    
    \hypertarget{exemplary-sa-ccr-allocation-under-consideration-of-a-minimum-transfer-amount.}{%
\section{Exemplary SA-CCR allocation under consideration of a minimum
transfer
amount.}\label{exemplary-sa-ccr-allocation-under-consideration-of-a-minimum-transfer-amount.}}

    The result of Appendix
\ref{homogeneity-of-c-for-a-single-trade-portfolio} shows, that
inclusion of the MTA results in a local plateau of \(C\).

In this section we want to investigate if inclusion of the MTA breaks
homogeneity of the SA-CCR EAD function and what can be done to mitigate
this effect.

We initialize a 200Bn IRS, a collateral agreement with a 0 threshold and
a 1Bn MTA.

    \begin{tcolorbox}[breakable, size=fbox, boxrule=1pt, pad at break*=1mm,colback=cellbackground, colframe=cellborder]
\prompt{In}{incolor}{In}{\boxspacing}
\begin{Verbatim}[commandchars=\\\{\}]
\PY{n}{irs} \PY{o}{=} \PY{n}{IRS}\PY{p}{(}\PY{n}{notional} \PY{o}{=} \PY{l+m+mi}{200000000000}\PY{p}{,}
          \PY{n}{timeToSwapStart}\PY{o}{=}\PY{n}{ql}\PY{o}{.}\PY{n}{Period}\PY{p}{(}\PY{l+m+mi}{2}\PY{p}{,} \PY{n}{ql}\PY{o}{.}\PY{n}{Days}\PY{p}{)}\PY{p}{,}
          \PY{n}{timeToSwapEnd}\PY{o}{=}\PY{n}{ql}\PY{o}{.}\PY{n}{Period}\PY{p}{(}\PY{l+m+mi}{10}\PY{p}{,} \PY{n}{ql}\PY{o}{.}\PY{n}{Years}\PY{p}{)}\PY{p}{,}
          \PY{n}{index} \PY{o}{=} \PY{n}{InterestRateIndex}\PY{o}{.}\PY{n}{USDLIBOR3M}\PY{p}{)}
\PY{n}{ca} \PY{o}{=} \PY{n}{CollateralAgreement}\PY{p}{(}\PY{n}{threshold}\PY{o}{=}\PY{l+m+mi}{0}\PY{p}{,}
                         \PY{n}{mta}\PY{o}{=}\PY{l+m+mi}{1000000000}\PY{p}{)}
\PY{n}{ca}\PY{o}{.}\PY{n}{link\PYZus{}sa\PYZus{}ccr\PYZus{}instance}\PY{p}{(}\PY{n}{SA\PYZus{}CCR}\PY{p}{(}\PY{n}{ca}\PY{p}{)}\PY{p}{)}
\PY{n}{ca}\PY{o}{.}\PY{n}{add\PYZus{}trades}\PY{p}{(}\PY{n}{irs}\PY{p}{)}

\PY{n}{ca}\PY{o}{.}\PY{n}{set\PYZus{}start\PYZus{}collateral\PYZus{}amount}\PY{p}{(}\PY{n}{ca}\PY{o}{.}\PY{n}{get\PYZus{}C}\PY{p}{(}\PY{p}{)}\PY{p}{)}
\end{Verbatim}
\end{tcolorbox}

    The starting collateral \(C_{t-1}\) is set to 9038157077 USD which is
also the calculated IM since the VM of this par IRS is 0.

    
    The EAD is:

    \begin{tcolorbox}[breakable, size=fbox, boxrule=1pt, pad at break*=1mm,colback=cellbackground, colframe=cellborder]
\prompt{In}{incolor}{In}{\boxspacing}
\begin{Verbatim}[commandchars=\\\{\}]
\PY{n}{original\PYZus{}ead} \PY{o}{=} \PY{n}{ca}\PY{o}{.}\PY{n}{get\PYZus{}sa\PYZus{}ccr\PYZus{}model}\PY{p}{(}\PY{p}{)}\PY{o}{.}\PY{n}{get\PYZus{}risk\PYZus{}measure}\PY{p}{(}\PY{p}{)}
\PY{n}{original\PYZus{}ead}
\end{Verbatim}
\end{tcolorbox}

            \begin{tcolorbox}[breakable, size=fbox, boxrule=.5pt, pad at break*=1mm, opacityfill=0]
\prompt{Out}{outcolor}{Out}{\boxspacing}
\begin{Verbatim}[commandchars=\\\{\}]
582881953.4866074
\end{Verbatim}
\end{tcolorbox}
        
    When bumping the notional of the irs by 0.01\%, we can see that the
collateral of the portfolio does not change due to the MTA.

    \begin{tcolorbox}[breakable, size=fbox, boxrule=1pt, pad at break*=1mm,colback=cellbackground, colframe=cellborder]
\prompt{In}{incolor}{In}{\boxspacing}
\begin{Verbatim}[commandchars=\\\{\}]
\PY{n}{ca}\PY{o}{.}\PY{n}{remove\PYZus{}all\PYZus{}trades}\PY{p}{(}\PY{p}{)}
\PY{n}{ca}\PY{o}{.}\PY{n}{add\PYZus{}trades}\PY{p}{(}\PY{n}{irs}\PY{o}{.}\PY{n}{get\PYZus{}bumped\PYZus{}copy}\PY{p}{(}\PY{n}{rel\PYZus{}bump\PYZus{}size}\PY{o}{=}\PY{l+m+mf}{0.0001}\PY{p}{)}\PY{p}{)}
\PY{n}{ead\PYZus{}bumped\PYZus{}mta} \PY{o}{=} \PY{n}{ca}\PY{o}{.}\PY{n}{get\PYZus{}sa\PYZus{}ccr\PYZus{}model}\PY{p}{(}\PY{p}{)}\PY{o}{.}\PY{n}{get\PYZus{}risk\PYZus{}measure}\PY{p}{(}\PY{p}{)}
\PY{n+nb}{print}\PY{p}{(}\PY{l+s+s1}{\PYZsq{}}\PY{l+s+s1}{EAD:  }\PY{l+s+si}{\PYZpc{}d}\PY{l+s+s1}{\PYZsq{}} \PY{o}{\PYZpc{}}\PY{k}{ead\PYZus{}bumped\PYZus{}mta})
\PY{n+nb}{print}\PY{p}{(}\PY{l+s+s1}{\PYZsq{}}\PY{l+s+s1}{C:   }\PY{l+s+si}{\PYZpc{}d}\PY{l+s+s1}{\PYZsq{}} \PY{o}{\PYZpc{}}\PY{k}{ca}.get\PYZus{}C())
\end{Verbatim}
\end{tcolorbox}

    \begin{Verbatim}[commandchars=\\\{\}]
EAD:  583024482
C:   9038157077
    \end{Verbatim}

    \begin{tcolorbox}[breakable, size=fbox, boxrule=1pt, pad at break*=1mm,colback=cellbackground, colframe=cellborder]
\prompt{In}{incolor}{In}{\boxspacing}
\begin{Verbatim}[commandchars=\\\{\}]
\PY{n}{ead\PYZus{}bumped\PYZus{}mta} \PY{o}{=} \PY{n}{ca}\PY{o}{.}\PY{n}{get\PYZus{}sa\PYZus{}ccr\PYZus{}model}\PY{p}{(}\PY{p}{)}\PY{o}{.}\PY{n}{get\PYZus{}risk\PYZus{}measure}\PY{p}{(}\PY{p}{)}
\end{Verbatim}
\end{tcolorbox}

    When temporarily disabling the MTA the resulting EAD and C differ.

    \begin{tcolorbox}[breakable, size=fbox, boxrule=1pt, pad at break*=1mm,colback=cellbackground, colframe=cellborder]
\prompt{In}{incolor}{In}{\boxspacing}
\begin{Verbatim}[commandchars=\\\{\}]
\PY{n}{ca}\PY{o}{.}\PY{n}{mta} \PY{o}{=} \PY{l+m+mi}{0}
\PY{n}{ead\PYZus{}bumped\PYZus{}no\PYZus{}mta} \PY{o}{=} \PY{n}{ca}\PY{o}{.}\PY{n}{get\PYZus{}sa\PYZus{}ccr\PYZus{}model}\PY{p}{(}\PY{p}{)}\PY{o}{.}\PY{n}{get\PYZus{}risk\PYZus{}measure}\PY{p}{(}\PY{p}{)}
\PY{n+nb}{print}\PY{p}{(}\PY{l+s+s1}{\PYZsq{}}\PY{l+s+s1}{EAD:  }\PY{l+s+si}{\PYZpc{}d}\PY{l+s+s1}{\PYZsq{}} \PY{o}{\PYZpc{}}\PY{k}{ead\PYZus{}bumped\PYZus{}no\PYZus{}mta})
\PY{n+nb}{print}\PY{p}{(}\PY{l+s+s1}{\PYZsq{}}\PY{l+s+s1}{C:   }\PY{l+s+si}{\PYZpc{}d}\PY{l+s+s1}{\PYZsq{}} \PY{o}{\PYZpc{}}\PY{k}{ca}.get\PYZus{}C())
\end{Verbatim}
\end{tcolorbox}

    \begin{Verbatim}[commandchars=\\\{\}]
EAD:  582940242
C:   9039060887
    \end{Verbatim}

    Calculating the forward difference with and without consideration of the
MTA yields

    \begin{tcolorbox}[breakable, size=fbox, boxrule=1pt, pad at break*=1mm,colback=cellbackground, colframe=cellborder]
\prompt{In}{incolor}{In}{\boxspacing}
\begin{Verbatim}[commandchars=\\\{\}]
\PY{n+nb}{print}\PY{p}{(}\PY{l+s+s1}{\PYZsq{}}\PY{l+s+s1}{With MTA:    }\PY{l+s+si}{\PYZpc{}d}\PY{l+s+s1}{\PYZsq{}} \PY{o}{\PYZpc{}}\PY{p}{(}\PY{p}{(}\PY{n}{ead\PYZus{}bumped\PYZus{}mta}\PY{o}{\PYZhy{}}\PY{n}{original\PYZus{}ead}\PY{p}{)}\PY{o}{/}\PY{l+m+mf}{0.0001}\PY{p}{)}\PY{p}{)}
\PY{n+nb}{print}\PY{p}{(}\PY{l+s+s1}{\PYZsq{}}\PY{l+s+s1}{Without MTA:  }\PY{l+s+si}{\PYZpc{}d}\PY{l+s+s1}{\PYZsq{}} \PY{o}{\PYZpc{}}\PY{p}{(}\PY{p}{(}\PY{n}{ead\PYZus{}bumped\PYZus{}no\PYZus{}mta}\PY{o}{\PYZhy{}}\PY{n}{original\PYZus{}ead}\PY{p}{)}\PY{o}{/}\PY{l+m+mf}{0.0001}\PY{p}{)}\PY{p}{)}
\end{Verbatim}
\end{tcolorbox}

    \begin{Verbatim}[commandchars=\\\{\}]
With MTA:    1425289375
Without MTA:  582887602
    \end{Verbatim}

    As can be seen, only when the allocation is performed without
recognition of the MTA the allocated amount corresponds to the actual
EAD of this single trade portfolio.

    \hypertarget{impact-of-the-minimum-transfer-amount-on-rc}{%
\subsection{Impact of the minimum transfer amount on
RC}\label{impact-of-the-minimum-transfer-amount-on-rc}}

the MTA also impacts RC as displayed in table
\ref{tab:Margin in SA-CCR}. Since IM reduces the RC the most relevant
case is when the calculated IM is below the threshold.

In an example we try to allocate the EAD of a portfolio consisting of a
single 100Mn IRS. The associated collateral agreement has a threshold of
50Mn and a minimum transfer amount of 2Mn.

    \begin{tcolorbox}[breakable, size=fbox, boxrule=1pt, pad at break*=1mm,colback=cellbackground, colframe=cellborder]
\prompt{In}{incolor}{In}{\boxspacing}
\begin{Verbatim}[commandchars=\\\{\}]
\PY{n}{irs} \PY{o}{=} \PY{n}{IRS}\PY{p}{(}\PY{n}{notional}\PY{o}{=}\PY{l+m+mi}{100000000}\PY{p}{,}
          \PY{n}{timeToSwapStart}\PY{o}{=}\PY{n}{ql}\PY{o}{.}\PY{n}{Period}\PY{p}{(}\PY{l+m+mi}{2}\PY{p}{,} \PY{n}{ql}\PY{o}{.}\PY{n}{Days}\PY{p}{)}\PY{p}{,}
          \PY{n}{timeToSwapEnd}\PY{o}{=}\PY{n}{ql}\PY{o}{.}\PY{n}{Period}\PY{p}{(}\PY{l+m+mi}{10}\PY{p}{,} \PY{n}{ql}\PY{o}{.}\PY{n}{Years}\PY{p}{)}\PY{p}{,}
          \PY{n}{index}\PY{o}{=}\PY{n}{InterestRateIndex}\PY{o}{.}\PY{n}{USDLIBOR3M}\PY{p}{)}

\PY{n}{ca} \PY{o}{=} \PY{n}{CollateralAgreement}\PY{p}{(}\PY{n}{threshold}\PY{o}{=}\PY{l+m+mi}{50000000}\PY{p}{,}
                         \PY{n}{mta} \PY{o}{=} \PY{l+m+mi}{2000000}\PY{p}{)}
\PY{n}{ca}\PY{o}{.}\PY{n}{link\PYZus{}sa\PYZus{}ccr\PYZus{}instance}\PY{p}{(}\PY{n}{SA\PYZus{}CCR}\PY{p}{(}\PY{n}{ca}\PY{p}{)}\PY{p}{)}
\PY{n}{ca}\PY{o}{.}\PY{n}{add\PYZus{}trades}\PY{p}{(}\PY{n}{irs}\PY{p}{)}

\PY{n+nb}{print}\PY{p}{(}\PY{l+s+s1}{\PYZsq{}}\PY{l+s+s1}{RC:      }\PY{l+s+si}{\PYZpc{}d}\PY{l+s+s1}{ USD}\PY{l+s+s1}{\PYZsq{}} \PY{o}{\PYZpc{}}\PY{k}{ca}.get\PYZus{}sa\PYZus{}ccr\PYZus{}model().get\PYZus{}rc())
\PY{n+nb}{print}\PY{p}{(}\PY{l+s+s1}{\PYZsq{}}\PY{l+s+s1}{RC*1.4:  }\PY{l+s+si}{\PYZpc{}d}\PY{l+s+s1}{ USD}\PY{l+s+s1}{\PYZsq{}} \PY{o}{\PYZpc{}}\PY{p}{(}\PY{n}{ca}\PY{o}{.}\PY{n}{get\PYZus{}sa\PYZus{}ccr\PYZus{}model}\PY{p}{(}\PY{p}{)}\PY{o}{.}\PY{n}{get\PYZus{}rc}\PY{p}{(}\PY{p}{)}\PY{o}{*}\PY{l+m+mf}{1.4}\PY{p}{)}\PY{p}{)}
\PY{n+nb}{print}\PY{p}{(}\PY{l+s+s1}{\PYZsq{}}\PY{l+s+s1}{PFE:     }\PY{l+s+si}{\PYZpc{}d}\PY{l+s+s1}{ USD}\PY{l+s+s1}{\PYZsq{}} \PY{o}{\PYZpc{}}\PY{p}{(}\PY{n}{ca}\PY{o}{.}\PY{n}{get\PYZus{}sa\PYZus{}ccr\PYZus{}model}\PY{p}{(}\PY{p}{)}\PY{o}{.}\PY{n}{get\PYZus{}pfe}\PY{p}{(}\PY{p}{)}\PY{p}{)}\PY{p}{)}
\PY{n+nb}{print}\PY{p}{(}\PY{l+s+s1}{\PYZsq{}}\PY{l+s+s1}{PFE*1.4: }\PY{l+s+si}{\PYZpc{}d}\PY{l+s+s1}{ USD}\PY{l+s+s1}{\PYZsq{}} \PY{o}{\PYZpc{}}\PY{p}{(}\PY{n}{ca}\PY{o}{.}\PY{n}{get\PYZus{}sa\PYZus{}ccr\PYZus{}model}\PY{p}{(}\PY{p}{)}\PY{o}{.}\PY{n}{get\PYZus{}pfe}\PY{p}{(}\PY{p}{)}\PY{o}{*}\PY{l+m+mf}{1.4}\PY{p}{)}\PY{p}{)}
\PY{n+nb}{print}\PY{p}{(}\PY{l+s+s1}{\PYZsq{}}\PY{l+s+s1}{EAD:     }\PY{l+s+si}{\PYZpc{}d}\PY{l+s+s1}{ USD}\PY{l+s+s1}{\PYZsq{}} \PY{o}{\PYZpc{}}\PY{k}{ca}.get\PYZus{}sa\PYZus{}ccr\PYZus{}model().get\PYZus{}risk\PYZus{}measure())
\end{Verbatim}
\end{tcolorbox}

    \begin{Verbatim}[commandchars=\\\{\}]
RC:      2000000 USD
RC*1.4:  2800000 USD
PFE:     1179574 USD
PFE*1.4: 1651404 USD
EAD:     4451404 USD
    \end{Verbatim}

    The EAD is the sum of the RC and the PFE component time the \(\alpha\)
factor of 1.4. In all previous examples, the RC has always been floored
at 0 since the received IM was higher than the MTA or since the MTA was
0.

Again, Euler allocation is not possible, because the EAD is a sum of the
PFE, which is a function of the portfolio notional and the RC which, at
least locally, is a constant. Similar to the issue with threshold
described in
\ref{exemplary-sa-ccr-allocation-under-consideration-of-an-initial-margin-threshold}
one can only allocate without MTA and then allocate the remainder
\(EAD_{\text{MTA}}-EAD_{\text{no MTA}}\) according to some rule.

Below, we allocate the ead assuming a mta of 0.

    \begin{tcolorbox}[breakable, size=fbox, boxrule=1pt, pad at break*=1mm,colback=cellbackground, colframe=cellborder]
\prompt{In}{incolor}{In}{\boxspacing}
\begin{Verbatim}[commandchars=\\\{\}]
\PY{n}{ca}\PY{o}{.}\PY{n}{mta} \PY{o}{=} \PY{l+m+mi}{0}
\PY{n}{eulerAllocator} \PY{o}{=} \PY{n}{EulerAllocator}\PY{p}{(}\PY{n}{ca}\PY{p}{)}
\PY{n}{allocation} \PY{o}{=} \PY{n}{eulerAllocator}\PY{o}{.}\PY{n}{allocate\PYZus{}ead}\PY{p}{(}\PY{p}{)}
\PY{n}{ca}\PY{o}{.}\PY{n}{mta} \PY{o}{=} \PY{l+m+mi}{2000000}
\PY{n}{allocation}\PY{p}{[}\PY{n}{irs}\PY{p}{]}
\end{Verbatim}
\end{tcolorbox}

            \begin{tcolorbox}[breakable, size=fbox, boxrule=.5pt, pad at break*=1mm, opacityfill=0]
\prompt{Out}{outcolor}{Out}{\boxspacing}
\begin{Verbatim}[commandchars=\\\{\}]
1651404.724563472
\end{Verbatim}
\end{tcolorbox}
        
    As we can see the result equals \(PFE*1.4\) of the entire portfolio but
the RC has not been allocated.

The RC is also \(>0\), if \(TH_{IM} < IM_{calc} < MTA\). Based on the
available marketdata, chosen MTA and chosen threshold this is e.g.~the
case for an IRS with a notional of 1130Mn USD.

    \begin{tcolorbox}[breakable, size=fbox, boxrule=1pt, pad at break*=1mm,colback=cellbackground, colframe=cellborder]
\prompt{In}{incolor}{In}{\boxspacing}
\begin{Verbatim}[commandchars=\\\{\}]
\PY{n}{irs2} \PY{o}{=} \PY{n}{IRS}\PY{p}{(}\PY{n}{notional}\PY{o}{=}\PY{l+m+mi}{1130000000}\PY{p}{,}
          \PY{n}{timeToSwapStart}\PY{o}{=}\PY{n}{ql}\PY{o}{.}\PY{n}{Period}\PY{p}{(}\PY{l+m+mi}{2}\PY{p}{,} \PY{n}{ql}\PY{o}{.}\PY{n}{Days}\PY{p}{)}\PY{p}{,}
          \PY{n}{timeToSwapEnd}\PY{o}{=}\PY{n}{ql}\PY{o}{.}\PY{n}{Period}\PY{p}{(}\PY{l+m+mi}{10}\PY{p}{,} \PY{n}{ql}\PY{o}{.}\PY{n}{Years}\PY{p}{)}\PY{p}{,}
          \PY{n}{index}\PY{o}{=}\PY{n}{InterestRateIndex}\PY{o}{.}\PY{n}{USDLIBOR3M}\PY{p}{)}

\PY{n}{ca}\PY{o}{.}\PY{n}{remove\PYZus{}all\PYZus{}trades}\PY{p}{(}\PY{p}{)}
\PY{n}{ca}\PY{o}{.}\PY{n}{add\PYZus{}trades}\PY{p}{(}\PY{n}{irs2}\PY{p}{)}

\PY{n+nb}{print}\PY{p}{(}\PY{l+s+s1}{\PYZsq{}}\PY{l+s+s1}{RC:        }\PY{l+s+si}{\PYZpc{}d}\PY{l+s+s1}{ USD}\PY{l+s+s1}{\PYZsq{}} \PY{o}{\PYZpc{}}\PY{k}{ca}.get\PYZus{}sa\PYZus{}ccr\PYZus{}model().get\PYZus{}rc())
\PY{n+nb}{print}\PY{p}{(}\PY{l+s+s1}{\PYZsq{}}\PY{l+s+s1}{PFE:     }\PY{l+s+si}{\PYZpc{}d}\PY{l+s+s1}{ USD}\PY{l+s+s1}{\PYZsq{}} \PY{o}{\PYZpc{}}\PY{p}{(}\PY{n}{ca}\PY{o}{.}\PY{n}{get\PYZus{}sa\PYZus{}ccr\PYZus{}model}\PY{p}{(}\PY{p}{)}\PY{o}{.}\PY{n}{get\PYZus{}pfe}\PY{p}{(}\PY{p}{)}\PY{p}{)}\PY{p}{)}
\PY{n+nb}{print}\PY{p}{(}\PY{l+s+s1}{\PYZsq{}}\PY{l+s+s1}{EAD/1.4: }\PY{l+s+si}{\PYZpc{}d}\PY{l+s+s1}{ USD}\PY{l+s+s1}{\PYZsq{}} \PY{o}{\PYZpc{}}\PY{p}{(}\PY{n}{ca}\PY{o}{.}\PY{n}{get\PYZus{}sa\PYZus{}ccr\PYZus{}model}\PY{p}{(}\PY{p}{)}\PY{o}{.}\PY{n}{get\PYZus{}risk\PYZus{}measure}\PY{p}{(}\PY{p}{)}\PY{o}{/}\PY{l+m+mf}{1.4}\PY{p}{)}\PY{p}{)}
\PY{n+nb}{print}\PY{p}{(}\PY{l+s+s1}{\PYZsq{}}\PY{l+s+s1}{EAD:     }\PY{l+s+si}{\PYZpc{}d}\PY{l+s+s1}{ USD}\PY{l+s+s1}{\PYZsq{}} \PY{o}{\PYZpc{}}\PY{k}{ca}.get\PYZus{}sa\PYZus{}ccr\PYZus{}model().get\PYZus{}risk\PYZus{}measure())
\end{Verbatim}
\end{tcolorbox}

    \begin{Verbatim}[commandchars=\\\{\}]
RC:        934412 USD
PFE:     13329195 USD
EAD/1.4: 14263607 USD
EAD:     19969050 USD
    \end{Verbatim}

    These observations are discussed further in
\ref{sec:Incorporation of a minimum transfer amount}.




\end{document}