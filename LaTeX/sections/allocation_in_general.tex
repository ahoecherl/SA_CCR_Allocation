\documentclass[../Thesis_AHoecherl.tex]{subfiles}

\begin{document}


\section{Allocation of Risk Measures}\label{Allocation of Risk Measures}

With increasing sophistication of risk, own capital and margining models the need for equally sophisticated tools for attributing these measures rises as well. Allocating the variation margin or models that disregard portfolio effect entirely such as the current exposure method (CEM) to individual trades is trivial as these measures may just be calculated for an individual trade and then added up across all trades to obtain the correct aggregate value. 
For measures which take portfolio effects into account such as a \gls{VaR} model, \gls{ISDA SIMM} or \gls{SA-CCR} however, this approach is not possible. The advent of portfolio based models for internal risk measurement in the late 1990s and for regulatory risk measurement in the late 2000s sparked research into how such measures should be reallocated. Gregory \cite[Chapter~10.7]{gregory2015xva} states that three approaches are used in practice: 
\begin{enumerate}
\item Incremental allocation
\item Marginal allocation which will be called Euler allocation in this thesis  
\item Pro rata allocation
\end{enumerate}

Based on the paper of Koyluoglu and Stoker \cite{koyluoglu2002risk} the list of approaches can be complemented by:
\begin{enumerate}[resume]
    \item Discrete marginal allocation
    \item Shapley value
\end{enumerate}
 
Unfortunately, naming conventions for the different allocation approaches are not consistent between the different publications.
Therefore, a definition of the five approaches is following based on the notation used by Tasche \cite{tasche2007}. In the following we will always assume that $X_1, \dots, X_n$ are real valued random variables that are representing the profits and losses of the trades in a portfolio. $1, \dots, n$ represents the order in which the trades have been added to the portfolio. $X$ denotes the portfolio-wide \gls{PnL}, s.t.

\begin{align}
    X = \sum_{i=1}^n{X_i} \text{.}
\end{align}

$\rho(X)$ is a risk measure that is supposed to estimate the profit or loss of the portfolio at a certain quantile for a certain time period. 
Both, the \gls{ISDA SIMM} model and the \gls{SA-CCR} model are in their core such risk measures.

The allocation or contribution of trade $i$ to risk measure $\rho(X)$ is denoted as $\rho\left(X_i|X\right)$. Position sizes in the portfolio can be notated through a vector $u = (u_1,\dots, u_n)$:

\begin{align}
    X(u) = X(u_1, \dots, u_n) = \sum_{i=1}^{n}{u_iX_i}
\end{align}

Then, with $\mathbf{1}$ being a vector of ones, $\rho\left(X\left(\mathbf{1}\right)\right) = \rho\left(X\right)$.

\begin{definition}
    Assuming that $\rho\left(X\right)$ is a risk measure, the \textbf{incremental allocation} of trade $n$ can be calculated as
    \begin{gather}
        \begin{split}
            \text{with } u_{i\neq n} = 1 \text{ and }u_n = 0\\
            \rho_{inc}\left(X_n|X\right) = \rho\left(X\right) - \rho\left(X\left(u\right)\right)    
        \end{split}
    \end{gather}
    The incremental allocation can only be calculated for trade $n$.
\end{definition}

\begin{definition}
    Assuming that $\rho\left(X\right)$ is a risk measure that is homogeneous of degree one and continuously differentiable, the \textbf{Euler allocation} of an arbitrary trade $i$ can be calculated assume
    \begin{gather}
        \rho_{Euler}\left(X_i|X\right) = \frac{d\rho}{dh}\left(X+hX_i\right)|_{h\rightarrow0} = \frac{\partial \rho(X(\mathbf{1}))}{\partial u_i}
    \end{gather}
\end{definition}

\begin{definition}
    \label{def:pro rata allocation}
    Assuming that $\rho\left(X\right)$ is a risk measure, the \textbf{pro rata allocation} of an arbitrary trade $i$ can be calculated as
    \begin{gather}
        \text{with } u_{i} = 1 \text{ and } u_{\neq i} = 0\\
        \rho_{ProRata}\left(X_i|X\right) = \frac{\rho(X(u))}{\rho(X)}
    \end{gather}
\end{definition}

\begin{definition}
    Assuming that $\rho(X)$ is a risk measure, the \textbf{discrete marginal allocation} of an arbitrary trade $i$ can be calculated as
    \begin{gather}
        \begin{split}
            \text{with } u_{i} = 0 \text{ and } u_{\neq i} = 1\\
            \rho_{discrete}\left(X_n|X\right) = \rho\left(X\right) - \rho\left(X\left(u\right)\right)    
        \end{split}
    \end{gather}
\end{definition}

\begin{definition}
    \todo{Enter definition of Shapley allocation}
\end{definition}

The usefulness of any of the five allocation approaches listed above is dependent on the individual application of the allocation. Criteria by which the allocation approach is judged are also highly dependent of the application. However, the two criteria
\begin{enumerate}
    \item Native additivity
    \item Risk sensitivity
\end{enumerate}
are usually regarded to be the most important one. They are for example the criteria by which Koyluoglu and Stoker \cite{koyluoglu2002risk} compare the different allocation approaches.

A allocation algorithm $alloc$ exhibits native additivity if equation \ref{eq:native additivity} holds.

\begin{align}
    \sum_{i=1}^n{\rho_{alloc}\left(X|X_i\right)} = \rho\left(X\right)
    \label{eq:native additivity}
\end{align}

Risk sensitivity means that $\rho_{alloc}\left(X|X_i\right)$ should indicate how the trade $i$ impacts the overall risk $\rho\left(X\right)$. As an example, the pro rata allocation defined in \ref{def:pro def:pro rata allocation} exhibits native additivity by design but is not risk sensitive. Even a trade that is hedging the portfolio will always have a positive allocation since $\rho\left(X\right) > 0$ for any $u$.

Depending on the application of the allocation other criteria might be important such as
\begin{itemize}
    \item Non-negativity of allocations
    \item The value allocated to a trade must not change through time
    \item The allocated value needs to be independent from the order in which trades are entered 
\end{itemize}
Generally, such auxiliary requirements raise through operational or technical limitations. Some of the allocation algorithms presented above comply with them, while others do not. 
Such requirements might be the reason that allocation algorithms that are dismissed as inappropriate in this thesis still find application in the field.

The incremental allocation excels for use at the trading desk. It is well suited as an input when making investment decisions or for calculating the remuneration of traders and trading desks after entering a new trade.
Both, Gregory \cite{gregory2015xva} and Koyluoglu and Stoker \cite{koyluoglu2002risk} state that incremental allocation is the best suited allocation for these purposes. It does, however, perform poorly for risk analysis of an existing portfolio. Incremental allocation is further investigated in section \ref{sec:Incremental alloaction}.

Euler allocation on the other hand is well suited for analysis of an existing portfolio. It is well suited to identify concentration risk within an portfolio or for portfolio optimization purposes. In the literature, Euler allocation is generally regarded as the best allocation approach for such purposes as the allocations naturally sum up to the risk measure, 


% Given a risk measure $\rho$, a portfolio $P$ and an individual trade  

% Additionally \cite{koyluoglu2002risk}
% Allocation properties
% \begin{itemize}
% \item Native additivity
% \item Risk sensitivity
% \item Independent of portfolio constellation
% \item Stable through time
% \end{itemize}

\subsection{Incremental allocation\label{sec:Incremental alloaction}}
Incremental allocation can only be applied when observing the development of a portfolio through time. Given a pre-existing portfolio $P$ consisting of $n$ trades $t_1$ through $t_n$ and a portfolio-based measure $M$ the incremental contribution of the first and second additional trade may be calculated as:
\begin{align*}
M_{\text{inc},t_{n+1}} & =M\left(t_1\dots t_{n+1}\right)- M\left(t_1\dots t_{n}\right) \\
M_{\text{inc},t_{n+2}} & =M\left(t_1\dots t_{n+2}\right)- M\left(t_1\dots t_{n+1}\right)
%IncM_{t_{n+1}} =M\left(t_1\dots t_{n+1}\right) - M\left(t_1\dots t_{n+1}\right) \\
%IncM_{t_{n+2}} =M\left(t_1\dots \t_{n+2}\right) - M\left(t_1\dots t_{n+1}\right)
\end{align*}
It can be easily seen that this approach yields a natively additive allocation since it forms a telescoping sum\footnote{For brevity in Notation let $M(t_i)$ be equivalent to $M(t_1\dots t_i)$ 
} :
\begin{align*}
M_{\text{inc},t_1}&=M(t_1) \\
M_{\text{inc},t_i}&= M(t_i)-M(t_{i-1}) \\
M_{\text{inc},t_n}&= M(t_n) - M(t_{n-1})\\
\sum_{i=1}^{n}{M_{\text{inc},i}} &= M(t_1)-M(t_1)+\dots+M(t_{n-1})-M(t_{n-1})+M(t_n) = M(t_n)
\end{align*}
The incremental allocation can be calculated as or before a new trade is added to the portfolio. It is a risk sensitive value when it is calculated as it accurately reflect how the additional trade changes the risk measure. 
If the trade is mitigating risk at the time of its inception according to $M$ its incremental allocation $M_{inc}$ is negative. If it increases the risk its $M_{inc}$ is positive. However, $M_{inc}$ does not adapt over time and is likely to loose its accurate risk depiction as additional trades are added to the portfolio. 
As a portfolio develops it may well be possible, that a trade for which a negative $M_{inc}$ was calculated at its inception may loose its risk mitigation. 
Due to this property $M_{inc}$ of a given trade should ideally only be used at or before trade inception. 
One such use case is the PnL calculation of a new trade to determine the performance of the trading desk or trader which initiated the trade. 
Another would be to use it prior to an investment decision \cite{tibiletti2001incremental}. 
It can however not be used to analyze an existing portfolio to e.g. identify trades which drive risk or determine how increases or decreases in a given position would impact the portfolio measure. 
It also cant be calculated deterministically a posteriori for a portfolio without knowing its composition through time.

% In the past, some academic work focused on approximating the incremental VaR as it requires a recalculation of the VaR for the entire portfolio. 
% An overview of these works and their potential pitfalls may be found in \cite{tibiletti2001incremental}. Since this work will rather focus on marginal than incremental allocation further details may be found in the referred paper. 
% In the empirical analysis the incremental allocation will be calculated exactly.

\subsection{Euler allocation}

Discrete marginal contribution vs continuous marginal contribution. The latter is Euler allocation. Widely used in banking for market risk refer to Jorion

Closed form formulas for contribution derived for standard deviation based models, VaR models, conditional VaR models.

Could show this for a single variance covariance submatrix of ISDA SIMM?!

\subsection{Shapley allocation}

\todo{Inclusion of this section optional}

\end{document}