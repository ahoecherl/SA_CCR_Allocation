\documentclass[../Thesis_AHoecherl.tex]{subfiles}

\begin{document}
    \chapter{Discussion of results}\label{sec:Discussion of results}

    In most cases an Euler allocation of the \gls{SA-CCR} \gls{EAD} model is possible and yields risk sensitive results that can support risk analysis or portfolio optimization tasks.

    The \gls{SA-CCR} \gls{EAD} model is a rather unconventional risk measure as it embeds another risk measure, namely the initial margin model whose result mitigates risk of the \gls{EAD} model.
    This risk mitigation is a major part of the model and must not be dismissed when an allocation of the risk measure is required. For the exemplary equity portfolio in \ref{sec:Exemplary allocation of SA-CCR for a small portoflio of equity options} we can see that inclusion of risk mitigation through \gls{IM} reduces the calculated \gls{EAD} by a factor of ten.

    The exemplary portfolios also showed that a risk sensitive allocation can differ tremendously between the different margining approaches. 
    
    When allocating the \gls{SA-CCR} \gls{EAD} without variation margining the risk characteristics of the trades can be overshadowed by its present value. This means that a significant fraction of the \gls{EAD} may stem from its $RC$ which unlike the PFE does not take future hedge characteristics of the portfolio into account but is rather based on the current \gls{PV} of the trades in the portfolio.

    When allocating the \gls{SA-CCR} under consideration of \gls{VM} but not \gls{IM}, the $RC$ should, assuming a \gls{MTA} and \gls{VM} threshold of zero, be zero and the portfolio \gls{EAD} consists solely of the $PFE$ component. No overcollateralization is in place and therefore the risk sensitive allocation is entirely based on the PFE mechanics of the \gls{SA-CCR} model.

    Finally, with initial margining in play, the allocation depends a lot on the interplay between the \gls{SA-CCR} PFE risk measure and the risk measure used to calculate the initial margin. In this thesis, we performed exemplary allocations assuming the \gls{ISDA SIMM} model as the initial margin model. 
    Depending on the portfolio all constellations between the two risk measures could be observed. Examples have been found in which a trade locally reduces the \gls{SA-CCR} risk measure without overcollateralization while simultaneously increasing the received initial margin. These two effects amplify each other leading to a very high, but explicable negative amount of the \gls{EAD} being allocated to the trade. The opposite phenomenon of a trade increasing the \gls{SA-CCR} risk measure without overcollateralization while simultaneously decreasing the received initial margin could also be observed and justified.

    These results indicate that risk sensitive allocation of \gls{SA-CCR} is only possible when fully recognizing margining during the allocation. Simpler workarounds such as e.g. allocating \gls{SA-CCR} without margining and then scaling the results can yield deceptive results that the user might assume to be risk sensitive although they may be very far from capturing the actual portfolio mechanics.
    
    The interaction of the two risk measures as part of \gls{SA-CCR} \gls{EAD} recognizing initial margin also increases the value that a risk sensitive allocation approach such as the Euler allocation may add to the risk analytics function of a bank.
    The exemplary portfolios analyzed in section \ref{sec:Exemplary Euler allocation of SA-CCR under consideration of margining} were quite simple but their Euler allocation still unveiled some surprising portfolio mechanics especially for \gls{SA-CCR} recognizing IM.
    Once unveiled by the allocation the underlying mechanics could be understood upon further analysis. This demonstrates that a risk sensitive allocation is a powerful tool for \gls{SA-CCR} risk \gls{EAD} analysis and portfolio optimization purposes.

    Although not explicitly shown, a numerical allocation of \gls{SA-CCR} under consideration of initial margin is also possible if the received \gls{IM} is calculated under the standard approach discussed in section \ref{The standard approach} instead of the internal \gls{ISDA SIMM} model. 

    \section{Evaluation of the numerical allocation approach\label{Evaluation of the numerical allocation approach}}

    To calculate the Euler allocation a numerical, finite difference approach was chosen in this thesis. Alternatively, one could have tried to establish an analytical solution similar to the approach that Schulze has chosen in \cite{schulze2018capital} for allocation of the FRTB standard approach.
    The \gls{ISDA SIMM} model's structure is very related to the FRTB standard approach but it lends itself less to calculating analytical derivatives as its convoluted formulation e.g. nests more Min and Max expressions that need to be handled on a case by case basis when trying to establish partial derivatives analytically.
    When nesting the \gls{ISDA SIMM} risk measure within the \gls{SA-CCR} model this complexity increases further as the \gls{SA-CCR} model too e.g. has maximum conditions in its calculation of the $RC$ component.
    Considering this complexity, a manual analytic calculation of the partial derivatives appears to be prohibitively complex, error prone and difficult to implement and maintain in practice.
    A more practical approach to establish an analytic solution would probably be an adjoint algorithmic differentiation, which was outside the scope of this thesis but is briefly broached in section \ref{sec:Outlook}.

    Overall, the numerical, finite difference approach for calculation of both the \gls{ISDA SIMM} and \gls{SA-CCR} allocation worked well. Without extensive analysis of the aggregation formulas the user can always check whether the risk measure exhibits the prerequisite homogeneity and differentiability by just checking if equation \ref{eq:Euler theorem of positive homogeneous functions} holds after performing the allocation. 
    Within an appropriate architecture it can be very simple and require very little additional code to implement classes to perform a numerical Euler allocation.
    
    Despite the increased computational cost, a central difference approach for numerical finite difference calculation is generally advisable.
    The increased accuracy of the central difference approach makes it easier to evaluate whether equation \ref{eq:Euler theorem of positive homogeneous functions} holds and the allocated results are trustworthy or whether the theorem of homogeneous functions is violated inferring, that the prerequisites for Euler allocation are violated which should generally trigger an investigation into the processed portfolio.
    Additionally, the case of a perfectly hedged portfolio that is investigated in \ref{sec:Allocation of hedged portfolios} showcases a scenario that could have sufficient practical relevance in which the central difference approach unlike a forward difference, backward difference or even an analytic partial derivative yields a result that does not break the native additivity of the Euler allocation.

    The numerical approach also lends itself to direct allocation on subportfolios as illustrated in section \ref{sec:Exemplary allocation of SA-CCR on subportfolios}.
    If subportfolio results rather then trade results are required, simultaneously bumping multiple trades may significantly improve runtimes.

    \section{Limitations of an Euler allocation of SA-CCR\label{Limitations of an Euler allocation of SA-CCR}}

    In practice, the incorporation of parameters of the collateral agreement such as thresholds or the minimum transfer amount is a challenge.

    The issues caused by the minimal transfer amount are relatively minor.
    If the received initial margin is high enough to cap the $RC$ at zero, the \gls{MTA} can be ignored for the allocation calculation and still yield risk sensitive results that exhibit native additivity. If the initial margin is too low or when the \gls{MTA} is taken into account for margin recalculations on the other hand, the \gls{SA-CCR} risk measure does not exhibit homogeneity anymore and an Euler allocation is not possible.
    
    The observations made for the \gls{MTA} do also indicate that treating margin as an immutable portfolio-independent external constant will render an Euler allocation impossible as the \gls{SA-CCR} risk measure does not exhibit homogeneity under this assumption.

    Euler allocation under the presence of a variation margin threshold is not a problem.
    A variation margin threshold increases the \gls{EAD} of a portfolio by a flat amount under all circumstances and this amount is not really attributable to the trades.

    Out of the collateral agreements parameters the initial margin threshold is the hardest to handle. As long as the threshold is not exceeded, i.e. no initial margin is exchanged the allocation works fine but when the threshold is exceeded and is subtracted from the calculated initial margin, the \gls{SA-CCR} model does not exhibit homogeneity anymore.
    It is possible to identify relative bounds how much should allocated to each trade within these bounds \emph{some} metric can be use to find an allocation that adds up to the portfolio risk measure but an undisputable risk sensitive way of doing so could not be identified in this thesis.
    In practice, allocating the impact on \gls{EAD} of the \gls{IM} threshold could turn out to be the biggest issue when performing and Euler allocation of \gls{SA-CCR} EAD.

    Finally, whether the inability to perform an Euler allocation when an \gls{ISDA SIMM} threshold is exceeded as described in section \ref{sec:Allocation when an ISDA-SIMM liquidity threshold is exceeded} prohibits a bank from using Euler allocation entirely or is rather just a theoretical observation depends on the structure and size of the portfolios that should be allocated.
    While the USD interest rate concentration threshold chosen as an example in section\ref{sec:Allocation when an ISDA-SIMM liquidity threshold is exceeded} is extremely high due to the high liquidity, other threshold might be easier to exceed. If a threshold is exceeded the calculated allocation will not fulfill equation \ref{eq:positive homogeneous function} anymore and it should therefore be relatively easy for a risk department to identify such a case. However, there is no simple workaround to adjust for such a case and Euler allocation appears to be generally not possible anymore once the portfolio exceeds any \gls{ISDA SIMM} thresholds.

\end{document}