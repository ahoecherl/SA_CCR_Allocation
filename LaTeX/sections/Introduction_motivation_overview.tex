\documentclass[../Thesis_AHoecherl.tex]{subfiles}

\begin{document}
    \section{Introduction and motivation}

    With increasing sophistication of risk models, own capital models and margining models the need for equally sophisticated tools for allocating these measures rises as well. For any risk metric that considers portfolio effects calculating the contribution to the risk measure of individual trades is a challenge. As part of the Basel 3 reform, \todo{Add appropriate citation to Basel 3 or CRD 2} regulators have updated the standardized models for market risk own capital requirement and credit risk own capital requirements. The new standardized model for market risk own capital requirements is the FRTB-SA \todo{Citation to FRTB-SA needed} and the new standardized model for credit risk own capital requirements is the SA-CCR model \todo{Enter Citation of SA-CCR}. Both of these models are portfolio based risk models. Gregory \citep[Chapter~10.7]{gregory2015xva} states that three allocation approaches are used in practice:

    \begin{itemize}
        \item Incremental allocation for pre trade risk checks and for front office incentivization  
        \item Marginal allocation for risk analytics of existing portfolios 
        \item Pro rata allocation if trade contributions must not be negative, if risk sensitivity is not required or if the allocated risk measure does not take portfolio effects into account
    \end{itemize}

    These allocation approaches and their advantages and disadvantages are analyzed in further detail in section \ref{Allocation of Risk Measures}. Calculation of pro rata and incremental allocation is fairly straightforward and can generally be performed under any circumstances. Marginal allocation on the other hand is more challenging. This thesis will also include results for incremental and pro rata allocations for the sake of completeness but the main focus will be the analysis of marginal allocation.

    Schulze \citep{schulze2018capital} has analytically calculated the marginal allocation for the FRTB-SA. However, an approach to marginal allocation of SA-CCR has not been published yet. This thesis intends to close this gap by showcasing a numerical marginal allocation approach for SA-CCR. A particular challenge in the allocation of SA-CCR is its consideration of margining. This makes the allocation of SA-CCR dependent on margin models which can themselves be significantly more complex than the SA-CCR model itself.

    The goal of this thesis is to find an approach for allocating SA-CCR while considering margining. Particular focus is put on the allocation of SA-CCR under consideration of variation margin an an internal initial margin model. 
    
    \subsection{Literature review}
    \subsection{Overview}

    Throughout this document a small but diverse selection of derivatives is used for exemplary calculations and to explore edge cases of the different models. Section \ref{Instruments, pricing and market data} does briefly introduce these instruments and the financial models and market data that is used to price them.
    
    Since margining is an integral part of the SA-CCR model, section \ref{Margining} will afterwards introduce different margin types and typical counterparty relations and margin models observed in the market. Section \ref{Margining} will also establish why out of the different margin models, the ISDA-SIMM model needs to be investigated the closest for the purpose of this thesis.

    Section \ref{ISDA SIMM} introduces the ISDA SIMM which is the most commonly used model for initial margin calculation of uncleared derivatives. As the ISDA SIMM is based on first order sensitivities, section \ref{ISDA SIMM} also describes how to calculate \gls{ISDA SIMM} compliant sensitivities for the financial instruments introduced in section \ref{Instruments, pricing and market data}.

    After the different margin components have been introduced, section \ref{SA CCR} presents the SA CCR model that is used for the calculation of \gls{EAD} of derivatives.



\end{document}