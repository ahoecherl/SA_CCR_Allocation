\documentclass[../Thesis_AHoecherl.tex]{subfiles}

\begin{document}
    \chapter{Introduction}
    \section{Motivation}
    With increasing sophistication of risk models, own capital models and margining models, the need for equally sophisticated tools for allocating these measures rises as well. For any risk metric that considers portfolio effects calculating the contribution to the risk measure of individual trades is a challenge. As part of the Basel 3 reform, regulators have updated the standardized models for market risk own capital requirement and credit risk own capital requirements. The new standardized model for market risk own capital requirements is the \gls{FRTB-SA} and the new standardized model for credit risk own capital requirements is the \gls{SA-CCR} model. Both of these models are portfolio based risk models. Gregory \cite[Chapter~10.7]{gregory2015xva} states that three allocation approaches are used in practice:

    \begin{itemize}
        \item Incremental allocation for pre trade risk checks and for front office incentivization  
        \item Marginal allocation for risk analytics of existing portfolios 
        \item Pro rata allocation if trade contributions must not be negative, if risk sensitivity is not required or if the allocated risk measure does not take portfolio effects into account
    \end{itemize}

    These allocation approaches and their advantages and disadvantages are analyzed in further detail in section \ref{sec:Allocation of Risk Measures}. Calculation of pro rata and incremental allocation is fairly straightforward and can generally be performed under any circumstances. Marginal allocation on the other hand is more challenging. This thesis will investigate if and how a marginal allocation of the SA-CCR model is possible.

    Schulze \cite{schulze2018capital} has analytically calculated the marginal allocation for the FRTB-SA. However, an approach to marginal allocation of SA-CCR has not been published yet. This thesis intends to close this gap by showcasing a numerical marginal allocation approach for SA-CCR. A particular challenge in the allocation of SA-CCR is its consideration of margining. This makes the allocation of SA-CCR dependent on margin models which can themselves be significantly more complex than the SA-CCR model itself.

    The goal of this thesis is to find an approach for allocating SA-CCR while considering margining. Particular focus is put on the allocation of SA-CCR under consideration of variation margin an an internal initial margin model.

    Unlike the analytical approach of Schulze \cite{schulze2018capital} this thesis will instead apply a numerical approach to calculating a marginal allocation.
    
    % \section{Literature review}

    % % asdf

    \section{Structure of the thesis}

    The rest of the thesis is segmented in four chapters. Chapter \ref{sec:Applied models and methods} introduced applied methods and models.
    This includes used financial instruments, risk models, an introduction of the concept of margining and approaches to risk measure allocation.
    Next, chapter \ref{sec:Results} proposes a numerical approach to Euler allocation of the \gls{SA-CCR} model and highlights some of the strengths and weaknesses of this approach by means of small exemplary portfolios.
    These results are discussed in chapter \ref{sec:Discussion of results} where the results of chapter \ref{sec:Results} are summarized and the theoretical and practical usefulness of the numerical allocation of SA-CCR is assessed.
    Finally, chapter \ref{sec:Outlook and Conclusion} provides an outlook to adjacent topics and concludes the thesis.

    Due to the scope of chapter \ref{sec:Applied models and methods} and \ref{sec:Results} their structure is presented in additional detail below.

    \subsection{Structure of chapter \ref{sec:Applied models and methods}}

    Throughout this document a small but diverse selection of derivatives is used for exemplary calculations and to explore edge cases of the different models. Section \ref{Instruments, pricing and market data} does briefly introduce these instruments and the financial models and market data that is used to price them.
    
    Since margining is an integral part of the \gls{SA-CCR} model, section \ref{Margining} will afterwards introduce different margin types and typical counterparty relations and margin models observed in the market. Section \ref{Margining} will also establish why out of the different margin models, the \gls{ISDA SIMM} model is investigated the closest for the purpose of this thesis.

    Section \ref{Bilateral initial margin} introduces the ISDA SIMM which is the most commonly used model for initial margin calculation of uncleared derivatives. As the ISDA SIMM is based on first order sensitivities, section \ref{Bilateral initial margin} also describes how to calculate \gls{ISDA SIMM} compliant sensitivities for the financial instruments introduced in section \ref{Instruments, pricing and market data}.

    After the different margin components have been introduced, section \ref{SA-CCR} presents the SA-CCR model that is used for the calculation of the \gls{EAD} of derivatives. Special emphasis will be put on the inclusion of margining.

    Section \ref{sec:Allocation of Risk Measures} presents literature results regarding the allocation of risk measures. Additionally, the theoretical foundation for the Euler allocation is laid out as the investigation if and when an Euler allocation is possible is the main subject for the analysis section of the thesis.

    At this point all relevant concepts, models and financial instruments have been introduced to perform the analysis that is then presented in the results chapter of the thesis. 
    
    \subsection{Structure of chapter \ref{sec:Results}}
    
    The main goal of the thesis is to establish a numerical allocation approach for SA-CCR.

    An approach to numerical allocation of SA-CCR is introduced with the aid of two exemplary portfolios in chapters \ref{sec:Exemplary allocation of SA-CCR for a small portoflio of equity options} and \ref{sec:Exemplary allocation of SA-CCR for a small portfolio of interst rate derivatives}. 
    In these chapters both, ISDA-SIMM and SA-CCR are allocated and the portfolios are used to showcase analytical insights that an Euler allocation may provide. Finally, \ref{sec:Exemplary allocation of SA-CCR on subportfolios} showcases the value of an Euler allocation not only for an analysis on trade level but also on subportfolios.

    In contrast to the encouraging results of section \ref{sec:Exemplary Euler allocation of SA-CCR under consideration of margining}, section \ref{sec:Consideration of edge cases} analyzes when and why Euler allocation of SA-CCR is not possible in certain edge cases.
    A property of the ISDA-SIMM model that may prevent risk-sensitive allocation is discussed in section \ref{sec:Allocation when an ISDA-SIMM liquidity threshold is exceeded}. 
    On the other hand, contractual agreements governing the exchange of margin may interfere with the allocability of the SA-CCR model which is discussed in section \ref{sec:Incorporation of a minimum transfer amount and threshold}.
    Finally, section \ref{sec:Allocation of hedged portfolios} discusses a possible, specific hedge constellation that infringes a necessary precondition for an Euler allocation.


    % As the received initial margin is a subcomponent of SA-CCR it might be necessary to also allocate initial margin. Due to the complexity of initial margin models allocation of this subcomponent is investigated separately. Section \ref{Allocation of initial margin} establishes numerical approaches for marginal and incremental allocation of initial margin figures.\todo{Rephrase after Result restructuring}

    % Finally, section \ref{Allocation of SA-CCR} uses the results of all previous sections as the basis to analyze the marginal and incremental allocation of SA-CCR. The prerequisites for marginal allocation are higher than those for incremental allocation. Therefore, the section investigates in detail under which circumstances marginal allocation of SA-CCR is possible and examines portfolios that represent edge cases.\todo{Rephrase after Result restructuring}

\end{document}