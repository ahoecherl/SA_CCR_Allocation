\documentclass[../Thesis_AHoecherl.tex]{subfiles}

\begin{document}
    \section{ISDA SIMM}\label{ISDA SIMM}

    \subsection{Reducing counterparty credit risk with initial margin}

    Counterparty credit risk refers to the risk of a default of the counterparty of a derivative. Derivatives are contracts between two institutions which give raise to future cash flows dependent of the performance of its underlying. These future cash flows may be at risk if the counterparty defaults during the lifetime of the derivative. 
    
    Over the past decades several measures were established in the derivatives markets to mitigate counterparty credit risk.
    
    The most impactful measure is close-out netting. Close-out netting is a contractual agreement of two counterparties how their bilateral derivative contracts have to be settled if one of the counterparties defaults. With close-out netting, in case one of the two counterparties defaults all derivatives which are yet to mature are immediately closed out based on the current market value. The market values of the individual derivatives are summed up and the netted amount needs to be paid by whichever party is trailing across the portfolio. In case of a default this close-out netting procedure takes priority over all other claims of creditors against the defaulted counterparty. Close-out netting has two major advantages. First, the non-defaulting counterparty only suffers a counterparty credit loss, if it is ahead across the entire portfolio of currently running derivatives with the defaulted party. Just having a positive market value on a few derivatives does not necessarily result in a counterparty credit loss. Secondly, the immediate close-out of the open derivatives of the defaulted counterparty greatly facilitates unwinding its portfolio. A disadvantage of close-out netting is, that it may proof difficult to find an objective market value of the derivatives that have to be closed out - especially in a stressed market environment, which is likely to be present if e.g. a large investment bank defaults. The contractual obligation to perform close-out netting is agreed upon in a master agreement, which was introduced to the derivatives market by ISDA in 1985. More details on close-out netting may be found in \cite[Chapter~5]{gregory2015xva}.
    
    The second most effective measure in mitigating counterparty credit risk is posting variation margin. If the obligation to post variation margin is agreed as part of a master agreement the accrued mark-to-market of the derivative portfolio has to be collateralized by the trailing counterparty. This measure effectively resets counterparty credit risk to zero for both parties every time a variation margin payment is made or the exchanged variation margin is adjusted to the current market value of the portfolio. The exchange of variation margin was common but not a given in the inter-bank market before the financial crisis of 2008. After the crisis it has become commonplace in the interbank market and recently has even been mandated by regulators\footnote{In the European Union the exchange of Initial Margin for inter bank bilateral OTC derivatives is compulsory since September of 2016 for large banks or March of 2017 for smaller banks.}. Non-financial counterparties oftentimes do not collateralize their derivatives since they are not mandated to do so, shy away from the operational burden and have a harder time funding the significant amount of cash necessary to cover the current mark-to-market value of their entire derivatives portfolio. Collateralizing a derivatives portfolio not only significantly reduces CCR but also significantly alters how the remaining CCR behaves. The CCR of a collateralized portfolio may rather be driven by the terms of the CSA\footnote{Explain what a CSA is} or residual phenomenon such as collateral spikes\footnote{Quote something regarding collateral Spikes} than by the underlying instruments.
    
    
    Close-out Netting
    Variation margin
    Initial margin
    
    A more comprehensive introduction to counterparty credit risk and its reduction through netting and margining may be found in chapters four through six of \cite{gregory2015xva}.
    \subsubsection{Market structure and regulatory background}
    The derivative market is divided into exchange traded derivatives, cleared OTC derivatives and uncleared OTC derivatives. According to \cite[Figure~3.2]{gregory2015xva} based on notional 9\% of derivatives are exchange traded, 55\% are cleared OTC derivatives and 36\% are uncleared OTC derivatives. It has to be noted that these figures are from 2014 and it can be assumed, that the fraction of cleared OTC derivatives has increased since then at the expense of the fraction of uncleared OTC derivatives. The reason for this is, that regulators have incentivized or even mandated the clearing of simpler OTC derivatives over the course of the last years. In connection with this development the large CCPs such as Eurex or the LCH have extended the product range for which they offer OTC clearing in recent years.
    
    Exchange traded or listed derivatives are usually futures or options on a limited scope of underlyings which can be directly traded at a clearing house at a bid and ask quoted by the clearing house. As seen in figure XX usual underlyings are single name equities, equity indices, commodities or baskets of liquid government bonds. 
    \begin{figure}
        \centering
        \def\svgwidth{\columnwidth}
        \input{Graphics/ListedVolumesEurex_201812_extractListedOverview.pdf_tex}
        \caption{Breakdown by underlying of listed contracts derivatives at Eurex in December 2018. Interest rate derivatives are almost exclusively futures and options on baskets of German government bonds. Graph has been extracted from \cite{EurexDec2018}.}
    \end{figure}
    The clearing house acts as a mediator to match market participants which are willing to buy and sell and market participants will not be informed with whom they have been matched. All market participants exchange variation margin with the CCP and have to post a unilateral initial margin to the CCP. This initial margin is calculated by the CCP and is usually calculated using a portfolio based historical simulation VaR model. If a market participants defaults to CCP inherits its positions and needs to auction them to other market participants. The purpose of the initial margin is to cover mark-to-market losses which the portfolio suffers until it can be fully auctioned or otherwise closed by the CCP.
    
    Cleared OTC derivatives are initially bilaterally agreed upon by two market participants. If the parametrization of the derivative is sufficiently common CCPs may offer to clear it and the market participants may choose or even be forced by regulation to clear it using a qualifying CCP. OTC clearing is especially common for simple interest rate derivatives such as interest rate swaps, forward rate agreements or cross currency swaps and for some credit derivatives. Similarly to exchange traded derivatives the CCP will charge both participants an initial margin and will also regularly determine the derivatives present value and execute the exchange of variation margin between the two market participants. If one of the market participants defaults, the CCP inherits the positions of the defaulted party and can again use the initial margin posted by the defaulted party to cover mark-to-market losses until it has auctioned or closed the inherited positions \cite{EurexDMP}.
    
    Assuming that the CCP cannot default itself, cleared OTC derivatives and listed derivatives bear no counterparty credit risk for the market participants.
    
    As of December 2018 common derivatives that are still uncleared are structured equity derivatives, swaptions and many other interest rate derivatives with optionality, derivatives on securitizations, swaps with unusual cashflow structures or in uncommon currencies and most FX derivatives \cite{BISOTCStats}.
    
    
    
    What are uncleared OTC derivatives
    
    
    \subsection{Regulatory requirements for an internal initial margin model}
    After the 2008 financial crisis the G20 agreed to reduce systematic, counterparty and operational risk and as a result of this commitment has been put into effect by regulators worldwide. In Europe the European Market Infrastructure Regulation (EMIR) came into force in August 2012 and focused on promoting or mandating central clearing as the primary measure to reduce counterparty risk.

    
    \subsection{The ISDA-SIMM model}
    In December of 2013 the International Swap and Derivatives Association (ISDA) published a motivation and basic methodological outline of a common internal initial margin model called Standard Initial Margin Model (SIMM\texttrademark)\cite{ISDADec2013}. The goal of the model is to meet the model requirements to an internal model of all regulators, while being among others easy to replicate, quick to calculate and relatively cheap to operate, implement and validate.

    SIMM is a Delta-Gamma VaR model using Delta and Vega sensitivities calculated by the banks themselves and risk weights and correlations provided and recalibrated annually by ISDA. ISDA provides member with a methodological paper \cite{SIMM} and a paper describing the input format of sensitivities \cite{RiskDataStandard}.

    \subsection{Calculation of ISDA SIMM compliant sensitivities}\label{Calculation of ISDA SIMM compliant sensitivities}
\end{document}