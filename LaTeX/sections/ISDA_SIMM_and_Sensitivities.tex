\documentclass[../Thesis_AHoecherl.tex]{subfiles}

\begin{document}
    \section{Bilateral initial margin}\label{Bilateral initial margin}
    
    After the 2008 financial crisis the G20 agreed to reduce systematic, counterparty and operational risk and as a result of this commitment has been put into effect by regulators worldwide. In Europe, the European Market Infrastructure Regulation (EMIR) came into force in August 2012 and focused on promoting or mandating central clearing as the primary measure to reduce counterparty risk.

    \subsection{The standardized approach}\label{The standard approach}

    The standardised approach to calculate bilateral initial margin has been proposed by the bank for international settlement in \cite[Requirement 3.5 and 3.6]{BCBS_MarginRequirements} and has been implemented in European law in \cite{OTC_Margin_EU_Regulation}. It is a schedule based approach that calculates an \gls{IM} contribution on a trade by trade basis multiplying the trades notional with a regulatory factor based on the asset class and term to maturity of the trade.
    The resulting sum may be reduced by up to 60\% through the so called net-gross ratio, if the portfolio has a negative present value from the perspective of the calculating bank.
    For a detailed specification of the aggregation the reader may refer to \cite{BCBS_MarginRequirements} and \cite{OTC_Margin_EU_Regulation}. The implementation of this approach is addressed in \ref{sec:ISDA SIMM implementation}.
    
    \subsection{The ISDA SIMM model}\label{sec:The ISDA-SIMM model}
    In December of 2013 the International Swap and Derivatives Association (ISDA) published a motivation and basic methodological outline of a common internal initial margin model called Standard Initial Margin Model (SIMM\texttrademark)\cite{ISDADec2013}. The goal of the model is to meet the model requirements to an internal model of all regulators, while being, among others, easy to replicate, quick to calculate and relatively cheap to operate, implement and validate.

    SIMM is a Delta-Gamma \gls{VaR} model using Delta and Vega sensitivities calculated by the banks themselves and risk weights and correlations provided and recalibrated annually by \gls{ISDA}. \gls{ISDA} provides member with a methodological paper \cite{SIMM} and a paper describing the input format of sensitivities \cite{RiskDataStandard}.
    Additionally, the authors of \gls{ISDA SIMM} have provided a technical paper \cite{SIMM_technical_paper} that lays out the mathematical foundation of the model. 
    The core idea of the model is to multiply sensitivities with risk weights and aggregate them with nested variance-covariance computations. For a detailed introduction to the structure, requirements against input data and underlying mathematical concepts of \gls{ISDA SIMM} the reader may please refer to sources \cite{SIMM}, \cite{RiskDataStandard} and \cite{SIMM_technical_paper}.

    \subsubsection{Implementation}\label{sec:ISDA SIMM implementation}
    As already pointed out, \gls{ISDA SIMM} is standardized despite being an internal model. Therefore, all market participants using an internal model for bilateral initial margin calculation are forced to calculate \gls{ISDA SIMM} compliant sensitivities, convert them into the \gls{CRIF} format and to aggregate them to an initial margin figure using the \gls{ISDA SIMM} aggregation.
    The process to create \gls{ISDA SIMM} compliant sensitivities is individual to each bank. Many vendor solutions for trading and risk have incorporated the creation of \gls{ISDA SIMM} compliant sensitivities and a \gls{CRIF} into their products but the most suitable way to produce a \gls{CRIF} still needs to be established on a bank to bank basis.
    
    Aggregation on the other hand is absolutely standardized. It uses a single file, the \gls{CRIF}, as input, does not need any auxiliary market data and returns a single value, the IM. 
    Considering this, Acadiasoft\footnote{\url{https://acadiasoft.com/}} decided to provide an open source implementation of the \gls{ISDA SIMM} aggregation \cite{simm-lib}. Acadiasoft is an \gls{ISDA}-affiliated company who also offers a dispute resolution platform for bilateral initial margin that has become the market standard.
    For the analysis shown in chapter \ref{sec:Results}, this open source library was used for aggregation of \gls{ISDA SIMM}. 
    Therefore, only the \gls{ISDA SIMM} compliant sensitivities needed to be calculated and parsed into a \gls{CRIF} entry minimizing potential sources of error and necessary testing effort. The open source library by Acadiasoft also offers functionality to calculate bilateral initial margin according to the standard approach presented in section \ref{The standard approach} using an extended \gls{CRIF} standard.


\end{document}