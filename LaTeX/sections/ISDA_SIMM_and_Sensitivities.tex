\documentclass[../Thesis_AHoecherl.tex]{subfiles}

\begin{document}
    \section{ISDA SIMM}\label{ISDA SIMM}
    
    \subsection{Regulatory requirements for an internal initial margin model}
    After the 2008 financial crisis the G20 agreed to reduce systematic, counterparty and operational risk and as a result of this commitment has been put into effect by regulators worldwide. In Europe the European Market Infrastructure Regulation (EMIR) came into force in August 2012 and focused on promoting or mandating central clearing as the primary measure to reduce counterparty risk.

    
    \subsection{The ISDA-SIMM model}
    In December of 2013 the International Swap and Derivatives Association (ISDA) published a motivation and basic methodological outline of a common internal initial margin model called Standard Initial Margin Model (SIMM\texttrademark)\cite{ISDADec2013}. The goal of the model is to meet the model requirements to an internal model of all regulators, while being among others easy to replicate, quick to calculate and relatively cheap to operate, implement and validate.

    SIMM is a Delta-Gamma VaR model using Delta and Vega sensitivities calculated by the banks themselves and risk weights and correlations provided and recalibrated annually by ISDA. ISDA provides member with a methodological paper \cite{SIMM} and a paper describing the input format of sensitivities \cite{RiskDataStandard}.

    \subsection{Calculation of ISDA SIMM compliant sensitivities}\label{Calculation of ISDA SIMM compliant sensitivities}
\end{document}