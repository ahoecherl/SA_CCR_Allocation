\documentclass[../Thesis_AHoecherl.tex]{subfiles}

\begin{document}
    \section{Allocation of initial margin}\label{Allocation of initial margin}
    The goal of this section is to investigate if a numerical Euler allocation of \gls{ISDA SIMM} is possible.
    
    As pointed out in section \ref{sec:Euler allocation} a risk measure needs to exhibit positive homogeneity of degree 1 to be able to perform an Euler allocation.
    In a first step we can investigate by calculating ISDA SIMM for a single trade whether ISDA SIMM does exhibit positive homogeneity for a minimal example.

    For this we set up an USD Libor IRS with ten years time to maturity a notional of 200 billion USD. This is our initial portfolio $\mathbf{u}$. ISDA SIMM would fulfill the required positive homogeneity condition if $a \rho(\mathbf{u}) = \rho(a \mathbf(u)$ for $a>0$.

    Even a trivial example with just one trade is sufficient to show that Euler allocation does not work in the inhomogeneous part of the ISDA SIMM equation.
    For this, we compare two sample portfolios one consisting of one USD IRS with 200 bn notional and one consisting of one USD IRS with 400 bn notional.
    Critically, the second portfolio is penalized by the model since its USD IRS risk is too large while the first one does not. We calculate the Euler calculation with a forward finite difference approach as displayed in equation \ref{eq:forward difference}. However, since the notionals of the trades are so large it appears to be more sensible to calculate the Euler allocation based on the notional as shown in equation \ref{eq:notional based euler allocation} instead of normalizing the current position size as $\mathbf{u} = 1$ as in equation \ref{eq:relative bump euler allocation}.
    Assuming that we calculate the finite difference with an $\epsilon$ of 1 USD and we notate 

    \section{Allocation of SA-CCR}\label{Allocation of SA-CCR}
    \subsection{Allocation without margining}
    \subsection{Allocation under VM collateralization}
    \subsection{Allocation under VM and IM collateralization}
\end{document}