\documentclass[../Thesis_AHoecherl.tex]{subfiles}

\begin{document}

    In this chapter the results produced for the thesis are presented. We will first show with small sample portfolio that numerical Euler allocation of SA-CCR is possible in section \ref{sec:General approach to numerical Euler allocation of SA-CCR}. On these exemplary portfolios we will also point out a couple of observations highlighting how an Euler allocation can offer great insight due to its risk sensitive nature.

    Afterwards, section \ref{sec:Consideration of edge cases} lists scenarios, under which Euler allocation prerequisites are violated and suggests approaches on how to mitigate these issues or proposes a workaround.

    \section{General approach to numerical Euler allocation of SA-CCR\label{sec:General approach to numerical Euler allocation of SA-CCR}}

    In this section we assume, that the minimum transfer amount, variation margin threshold and initial margin threshold as defined in \todo{Enter reference} are all 0. This means, that the margin calculated by the used variation margin and initial margin model is entirely incorporated in the SA-CCR model for EAD calculation.

    The reason for this is, that assumptions other than 0 for the thresholds and MTA generally violate the homogeneity prerequisite for Euler allocation. In practice, this is a strong and somewhat unrealistic assumption as for example an initial margin threshold of 50Mn is usual for bilateral portfolios as it is the highest amount allowed by the regulator \todo{Reference to where this is stated}. Due to the high practical relevance the impact of thresholds and MTA on Euler allocation is analyzed in detail in section \ref{sec:Consideration of edge cases}.   

    \subsection{Exemplary allocation of SA-CCR for a small portfolio of equity options}

    \todo{Ensure that there are no issues with the currencies. EAD should actually be EUR and needs to be converted to USD}

    In this section we analyze an Euler allocation of a small portfolio of equity options. The detailed computation steps are demonstrated in appendix \ref{sa-ccr-euler-allocation-of-exemplary-equity-portfolio}. 
    First, we consider a portfolio consisting of two million call options and three million put option on Adidas. All options are struck at the current stock price and long. Obviously, the two positions are in a hedge relation and being at the money long options both positions do have a significant, positive present value.
    
    With ISDA-SIMM and SA-CCR portfolio risk measures introduced in \todo{add reference} and considering different margining approaches we can calculate portfolio risk measure values displayed the portfolio risk measure column of table \ref{tab:2TradeEquityResults}.
    % Table generated by Excel2LaTeX from sheet 'Sheet1'
    \begin{table}[htbp]
        \centering
        \begin{tabular}{l||r|r|r}
                & 2Mn ADS Call & 3Mn ADS Put & Portfolio Risk Measure\\
                \toprule
        SIMM  & -33.75\% & 133.75\% & 14,231,564 EUR \\
        No margin & 99.21\% & 0.79\% & 37,643,536 EUR \\
        VM only & 232.47\% & -132.47\% & 3,519,458 EUR \\
        VM+IM & 622.10\% & -522.10\% & 345,874 USD \\
        \end{tabular}%
        \caption{}
        \label{tab:2TradeEquityResults}%
    \end{table}%
    It can be seen that incorporation of the variation margin significantly drops the portfolio EAD. The main reason for this is that the RC in formula \todo{ref needed} drops from 18.5Mn which is equal to the portfolio PV to 0 when VM is incorporated.
    The additional overcollateralization of 14.2Mn USD through IM is then the reason for the EAD to again drops by 90\% to 346k USD when IM is incorporated in addition to VM.

    We can now calculate the Euler allocation by applying the forward difference formula \ref{eq:forward difference} with a bump size of $\epsilon = 0.0001$ as this is the first time this is explicitly done we are writing this down step by step for EAD under consideration of VM and IM.

    \begin{enumerate}
        \item We increase the position in Adidas call options by $\epsilon$ adding 200 call options to the portfolio.
        \item We recalculate the EAD of the updated portfolio as $\rho_{\text{Inc Call}}$ and remove the 200 Mn call options again.
        \item We repeat this with a position increase of 100 Adidas put options yielding result $\rho_{\text{Inc Put}}$.
        \item We now use the \emph{bumped} risk measure results and the original $\rho$ of 345,874 USD as displayed in table \ref{tab:2TradeEquityResults} to yield the allocation.
        \item The allocated value for the call position is $\frac{\rho{\text{Inc Call}-\rho}}{0.0001} = 2.152 Mn \text{USD}$.
        \item The allocated value for the put position is $\frac{\rho{\text{Inc Put}-\rho}}{0.0001} = -1.806 Mn \text{USD}$.
        \item This yield the relative allocated values of 622\% and -522\% displayed in table \ref{tab:2TradeEquityResults}.
    \end{enumerate}

    Comparing the allocations of the different risk measures in table \ref{tab:2TradeEquityResults} with each other uncovers a couple of interesting observations. 
    
    First, under consideration of no margin the contribution of the put position is close to zero. The decrease of risk potential future exposure (PFE), i.e. a lower risk of the portfolio is counteracted by an increase of current exposure through the positive market value of the put position.

    Secondly, for the ISDA-SIMM risk measure the call position is considered as a hedge position while for the VM only EAD model the put position is considered a hedge. A marginal increase in the Call position decreases the charged IM while it increases the calculated EAD under consideration of VM only. These two effects reinforce each other when allocating the EAD under consideration of VM and IM. Here, a marginal increase in the call position simultaneously increases the portfolio risk under the SA-CCR EAD risk measure while also resulting in a decrease in received IM and therefore overcollateralization which further raises the calculated EAD. On the other hand, a marginal increase in the put position results decreases the EAD risk measure while simultaneously increasing overcollateralization. This is the reason for the stark increase of the relative allocated risk from the VM only case to the VM+IM case.

    Finally, it is worth discussing why the ISDA-SIMM risk measure and the SA-CCR risk measure evaluate the portfolio so differently with one considering the put a hedge position while the other one considers the call as the hedge position.
    Due to the large differences between the two models and the dependency of the ISDA-SIMM model on market data is is difficult to pinpoint a single driving factor for this phenomenon.
    However, the different holding periods of ten days for the ISDA-SIMM model and one year for the SA-CCR model appears to be a likely candidate. Indeed, if we reduce the maturity of the options from one year to ten days and thereby effectively reduce the holding period of the SA-CCR model to ten days we can see from the results in table \ref{tab:2TradeEquity10dayEAD} that the SA-CCR model then considers the smaller call position to be the hedge trade.

    % Table generated by Excel2LaTeX from sheet 'Sheet1'
    \begin{table}[htbp]
        \centering
        \begin{tabular}{l||r|r|r}
                & 2Mn ADS Call 10D &3Mn ADS Put 10D & Portfolio Risk Measure \\
                \toprule
        VM only & -358.06\% & 458.06\% & 1,701,707 EUR \\
        \end{tabular}%
        \caption{}
        \label{tab:2TradeEquity10dayEAD}%
    \end{table}%

    A reliable test, whether the Euler allocation was successful and therefore if the allocated function exhibits homogeneity is to calculate if equation \todo{Reference formula of natural additivity under homogeneity} holds. For the allocation of EAD under VM and IM we yield a residual of 1068 EUR or about 0.3\% of the portfolio risk measure. This deviation appears to be within the expected order of magnitude of a forward difference approximation of the two derivatives. 
    The approximated derivate for the call position is $6.221 * 345,874 = 2.151Mn$ while the approximated derivative of the put position is $-1.806Mn$. In absolute terms this results in a sum of about 4Mn. Considering the $\epsilon = 0.0001$ this indicates, that the sum of the error of the two derivatives should be in the order of $\mathcal{O}\left(4Mn/0.0001 = 400\right)$ which is in line with the observed deviation.

    Theoretically, application of a central difference approach should bring the order of magnitude of the error down to $\mathcal{O}\left(4Mn/0.0001^2 = 0.04\right)$.
     This behavior can also be observed. 
    If the partial derivatives are instead calculated as $(\rho_{Incr}-\rho_{Decr})/0.0001$ the sum of the allocations deviates from the portfolio risk measures by only 0.01 EUR. 
    This indicates that for proper native additivity of the Euler allocation of EAD, the computationally more expensive central difference approach should be used.
    An even stronger case for application of a central difference approach will be made in section \ref{sec:Allocation of hedged portfolios}.

    \subsection{Exemplary allocation of SA-CCR for a small portfolio of interest rate derivatives\label{sec:Exemplary allocation of SA-CCR for a small portfolio of interest rate derivatives}}

    In this section we investigate a small equity portfolio. This section will further highlight how the interaction of the two EAD and initial margin risk measures can yield surprising results further highlighting the need for a risk sensitive allocation methodology for analysis purposes. The detailed computation steps are demonstrated in appendix \ref{sa-ccr-euler-allocation-of-exemplary-rates-portfolio}.

    Initially, we consider a 1Bn USD Receiver IRS and a 180Mn EUR Payer IRS and create three portfolios. The first only contains the USD IRS, the second only contains the EUR IRS and the third contains both trades.
    When calculating the ISDA-SIMM initial margin, the EAD under consideration of VM and the EAD under consideration of VM and IM we yield the results displayed in table \ref{tab:2TradeRatesResults}.

    \begin{table}[htbp]
        \centering
        \begin{tabular}{l||r|r|r}
                & EAD VM only &ISDA-SIMM & EAD VM + IM \\
                \toprule
        EUR Swap & 1,957,315 USD & 6,079,460 USD & 286,420 USD \\
        USD Swap & 10,873,970 USD & 28,762,683 USD & 2,014,873 USD \\
        Both Swaps & 12,831,284 USD & 28,059,093 USD & 3,074,959 USD \\
        \end{tabular}%
        \caption{}
        \label{tab:2TradeRatesResults}%
    \end{table}%

    Interest rate risks across different currencies are handled differently between the SA-CCR and the ISDA SIMM model.
    In the SA-CCR model each interest rate currency forms a separate so called \emph{hedging set}. SA-CCR does not allow for any hedge effects across the borders of a hedging set.
    We can observe this in the EAD VM only column of table \ref{tab:2TradeRatesResults}, since the EAD of the portfolio containing both trades is simply the sum of the two standalone portfolios.
    The ISDA SIMM risk measure on the other hand does allow hedge effects across currencies within the interest rate asset class. When aggregating sensitivities across multiple currency buckets, ISDA SIMM assumes a correlation of 22\% \cite[Section D.2]{SIMM}.
    This does show in the ISDA SIMM column of table \ref{tab:2TradeRatesResults} with the ISDA SIMM charged for for the portfolio containing both trades is smaller than the sum of the two standalone portfolios.
    
    This difference in models leads to the phenomenon that can be observed in the EAD under VM and IM column. The calculated EAD for the portfolio of both IRS is significantly higher than the sum of the EAD of the two standalone portfolios.
    We have found a counterexample showing that SA-CCR under consideration of VM and IM is not a sub-additive risk measure.
    Subadditivity is one of the properties of a coherent risk measure \todo{Reference to coherent risk measure} and counterexamples showing that a risk measure does not exhibit subadditivity can for example constructed for all VaR-based risk measures \todo{Reference to this}. However, for EAD under IM and VM it appears to be especially simple to construct a counterexample.

    The SA-CCR model considers the portfolio of both trades to be just as risky as the two trades independently. However, the available overcollateralization of the portfolio with both trades is relatively lower than the overcollateralization of the two standalone portfolios since the ISDA-SIMM model does recognize hedge effect trades. This constellation leads to the observed effect that the EAD of the joint portfolio is higher than the sum of the standalone EAD of the trades.

    When performing an Euler allocation of the different risk measures of the portfolio containing both IRS we yield the allocation as depicted in table \ref{tab:2TradeRatesAllocation}.

    \begin{table}[htbp]
        \centering
        \begin{tabular}{l||r|r|r}
                & \makecell{Allocated \\ EAD VM only} & \makecell{Allocated \\ ISDA-SIMM} & \makecell{Allocated \\ EAD VM + IM} \\
                \toprule
        180Mn EUR Swap & 15.25\% & -0.19\% & 34.95\% \\
        1000Mn USD Swap & 84.75\% & 100.19\% & 65.05\% \\
        \end{tabular}%
        \caption{}
        \label{tab:2TradeRatesAllocation}%
    \end{table}%
    
    The results are related to those observed for the standalone portfolio. As $15.25\% * 12.83Mn = 1.96Mn$ the Euler allocation results exactly in the standalone results when allocating the EAD under consideration of VM only. 
    For the allocation of ISDA SIMM on the other hand the EUR trade is considered a hedge trade and almost none of the risk measure is allocated to it. 
    The fact that the EUR trade reduces overcollateralization then also leads to the overproportionate fraction of the risk measure that is allocated to it under consideration of VM and IM.
    
    Considering the larger USD swap as the baseline trade the EUR swap contribution to the portfolio EAD under IM and VM is overproportionate since it is considered to be a risk mitigating trade by the ISDA SIMM model while the SA-CCR model considers it to increase the risk. The opposite phenomenon can be observed when we add a USD receiver swaption as a third trade to the portfolio.

    For this we add a one year swaption on a five year 500Mn USD receiver swap to the portfolio. The risk measures of the resulting portfolio are allocated and the result displayed in table \ref{tab:3TradeRatesAllocation}.
    \begin{table}[htbp]
        \centering
        \begin{tabular}{l||r|r|r}
                & \makecell{Allocated \\ EAD VM only} &\makecell{Allocated \\ ISDA-SIMM} &\makecell{Allocated \\ EAD VM + IM} \\
                \toprule
        180Mn EUR Swap & 17.70\% & 0.50\% & 43.69\% \\
        1000Mn USD Swap & 98.33\% & 80.07\% & 125.93\% \\
        500Mn USD Swaption & -16.04\% & 19.44 \% & -69.62\\
        \end{tabular}%
        \caption{}
        \label{tab:3TradeRatesAllocation}%
    \end{table}%
    As can be seen, the swaption is considered to be marginally risk decreasing by the EAD risk measure with only VM while it is considered to be a risk increasing trade by the ISDA SIMM risk measure. Consequently, the swaption reduces the EAD under consideration of VM and IM by an overproportionate amount as it decreases risk while increasing overcollateralization which is indicated by the large negative Euler allocation.

    In line with the results of the Euler allocation we can see in table \ref{tab:3TradeRatesResults} that inclusion of the swaption increases portfolio ISDA-SIMM by 24\%, while decreasing portfolio EAD under VM by -14\% and portfolio EAD und VM and IM by -48\%. This is exactly the opposite effect as observed for the EUR swap beforehand.

    \begin{table}[htbp]
        \centering
        \begin{tabular}{l||r|r|r}
                & EAD VM only &ISDA-SIMM & EAD VM + IM \\
                \toprule
        Portfolio & 11,058,114 USD & 34,796,088 USD & 1,586,748 USD \\
        \end{tabular}%
        \caption{}
        \label{tab:3TradeRatesResults}%
    \end{table}%

    For the swaption it is more difficult to pinpoint where the difference between the two models that results in the very different risk assessment of the swaption in relation to the rest of the portfolio is originating.
    Sensitivities of the swaps and swaption are calculated very differently between the two models with the ISDA SIMM model being based on current market data whilst the SA-CCR makes much more simplifying assumptions.

    \subsection{Exemplary allocation of SA-CCR on subportfolios}

    One of the advantages of the Euler allocation is, that once allocated, values on trade level can be aggregated to also produce risk sensitive results for subportfolios. It is also possible to directly allocate risk metrics on subportfolios if no further granularity is required and thereby saving computation time. The detailed computation steps for the results of this section are demonstrated in appendix \ref{sa-ccr-euler-allocation-of-an-exemplary-multi-asset-class-portfolio}.

    For this we put the five exemplary derivatives of the two previous sections in a joint portfolio, i.e. the EUR IRS, the USD IRS, the USD swaption and the position in Adidas put and call options.
    The resulting portfolio measure risk measures are displayed in \ref{tab:multiAssetResult} with the portfolio results of the standalone portfolios from the previous section displayed as reference.

    \begin{table}[htbp]
        \centering
        \begin{tabular}{l||r|r|r}
                & EAD VM only &ISDA SIMM & EAD VM + IM \\
                \toprule
        Equity Portfolio & 3,519,458 USD & 14,231,564 USD & 345,874 USD \\
        Rates Portfolio & 11,058,114 USD & 34,796,088 USD & 1,586,748 USD \\
        \midrule
        Equity \& Rates Portfolio & 14,577,571 USD & 49,027,652  USD & 1,890,742 USD \\
        \end{tabular}%
        \caption{}
        \label{tab:multiAssetResult}%
    \end{table}%

    As we can see, in for EAD under VM and ISDA SIMM the multi asset class portfolio is simply the sum of the risk measures of the portfolios containing only trades of the individual asset class.
    The reason for this is, that for both models, SA-CCR and ISDA SIMM no hedge effects are recognized between different asset classes.
    Nevertheless, for EAD under VM and IM we can observe in table \ref{tab:multiAssetResult} that the EAD of the multi asset class portfolio is slightly lower than the sum of equity and rates portfolio. 
    The reason for this is that initial margin pledged for interest rates trades may also be used to mitigate losses caused by equity trades and vice versa. This slightly increases the overcollateralization of the joint portfolio leading to a lower EAD und VM and IM. 

    Figure \ref{fig:C for ITM IRS} also showcases the trivial result that the Variation Margin is a homogeneous function - the value of the trade scales linearly with the notional.
    
    Critically, this result can also be transferred to any SA-CCR allocation approach that would treat C as an externally given constant value as locally, treating C as a constant value is the same as consideration of a \gls{MTA}. 
    In both cases the slope of C is zero.  

    \section{Consideration of edge cases\label{sec:Consideration of edge cases}}

    While numerical Euler allocation of SA-CCR and ISDA SIMM is generally possible as shown in section \ref{sec:General approach to numerical Euler allocation of SA-CCR} a couple of cases can be identified, in which Euler allocation fails since its prerequisites of homogeneity and differentiability are violated.
    Both, ISDA SIMM and SA-CCR are complex, convoluted formulas, make it different to make general observations on differentiability and homogeneity.
    In fact, we will see that depending on the portfolio and parametrization of the collateral agreement, both ISDA SIMM and SA-CCR are not homogeneous risk measures, and that they are both occasionally not partially differentiable w.r.t. position size.
    
    In this section, all identified cases under which prerequisites for Euler allocation are violated are presented and for some a workaround is presented to allow nevertheless for risk sensitive and naturally additive allocation.
    
    \subsection{Allocation when an ISDA-SIMM liquidity threshold is exceeded\label{sec:Allocation when an ISDA-SIMM liquidity threshold is exceeded}}
    
    As pointed out in section \ref{sec:Euler allocation} a risk measure needs to exhibit positive homogeneity of degree 1 to be able to perform an Euler allocation.
    This precondition is violated if a liquidity threshold of the ISDA SIMM model is exceeded.

    We can show this by exploring whether ISDA SIMM does exhibit positive homogeneity for a minimal example.
    
    For this we set up an USD Libor IRS with ten years time to maturity and a notional of 200 billion USD. This is our initial portfolio $\mathbf{u}$. ISDA SIMM would fulfill the required positive homogeneity condition if $a \rho(\mathbf{u}) = \rho(a \mathbf{u}$ for $a>0$. In figure \ref{fig:homogeneity of ISDA SIMM} $\rho(a\mathbf{u})$ is plotted for $0<a\leq 2$ in blue. 
    \begin{figure}
        \centering
        \includegraphics{Graphics/ISDA_SIMM_homogenity.pdf}
        \caption{}
        \label{fig:homogeneity of ISDA SIMM}
    \end{figure}
    The function exhibits homogeneity for $0<a<1.4$ \todo{find the exact point where homogeneity breaks} but not for higher $a$. 
    The reason for this is, that at this point the concentration risk charge of ISDA SIMM does kick in.
    The concentration risk for interest rate risks for our minimal example is defined as \cite[Article 7.b]{SIMM}
    \begin{align*}
        CR = \max\left(1,\left(\frac{\lvert\sum{s}\rvert}{T}\right)^{1/2}\right)
    \end{align*}
    with $s$ being the sensitivities against USD interest rate risk and T being 230Mn USD as specified in \cite[Article 74]{SIMM}. Due to subsequent variance-covariance aggregation the concentration risk impacts the calculated IM as
    \begin{align*}
        IM_{\text{with conc. risk}} = CR^2 \cdot IM_{\text{without conc. risk}}
    \end{align*}
    This causes the change in slope and implied loss of homogeneity visible in figure \ref{fig:homogeneity of ISDA SIMM}. If the portfolio would consist of a more diverse set of risk factors than the minimal example displayed in figure \ref{fig:homogeneity of ISDA SIMM} the associated concentration risk would kick in at different levels of $a$.
    The slope of the function would increase with each additional concentration risk not being floored at one any more. 
    
    It is important to note that as soon as the sensitivity against a single risk factor in the portfolio is above the concentration threshold the ISDA SIMM risk measure does not exhibit homogeneity anymore.
    
    Even a trivial example with just one trade is sufficient to show that Euler allocation does not work in the inhomogeneous part of the ISDA SIMM equation.
    For this, we compare two sample portfolios one consisting of one USD IRS with 200 bn notional and one consisting of one USD IRS with 400 bn notional.
    Critically, the second portfolio is penalized by the model since its USD IRS risk is too large. We calculate the Euler calculation with a forward finite difference approach as displayed in equation \ref{eq:forward difference}.
    
    Assuming that we calculate the finite difference with an $\epsilon = 0.0001$ this means that we calculate the ISDA SIMM of an IRS with 200Bn notional ($SIMM_{200Bn}$) and the ISDA SIMM of an equivalent IRS with 200.02 Bn notional ($SIMM_{200.02Bn}$) and this yields an Euler allocation to this trade as
    \begin{align*}
        \frac{SIMM_{200.02Bn} - SIMM_{200Bn}}{0.0001} = 9.04Bn
    \end{align*}
    We can see in figure \ref{fig:homogeneity of ISDA SIMM} that this value is both, the slope and the IM value at $a = 1$. The portfolios IM was correctly fully allocated to the single trade of which it consists. 
    
    However, performing the same calculation for an equivalent IRS with 400Bn notional yields
    \begin{align*}
        \frac{SIMM_{400.04Bn} - SIMM_{400Bn}}{0.0001} = 33.67Bn
    \end{align*}
    again, we can refer to figure \ref{fig:homogeneity of ISDA SIMM} to check if this is a reasonable allocation result. As $a=1$ represents the IM charge for investing 200Bn of notional in the IRS, $a=2$ represents an investment of 400Bn notional. The associated IM is just 22.44Bn - allocating 33.67Bn of the risk measure to the only trade in the portfolio is therefore clearly wrong. The Euler allocation of 33.67Bn can also be read off figure \ref{fig:homogeneity of ISDA SIMM} - it is the slope at $a=2$ times two.

    Euler allocation of SA-CCR also does not work if the calculated C does not exhibit homogeneity, i.e. if a concentration risk threshold of the ISDA-SIMM model is exceeded. Again using the 400Bn IRS from \ref{Allocation of initial margin} we calculate an EAD\textsubscript{400Bn IRS} of 843.5Mn but, despite not applying an MTA, Euler allocation yields a vastly different amount of $ \frac{\text{EAD\textsubscript{200.02Bn IRS}}-\text{EAD\textsubscript{200Bn IRS}}}{0.0001} = 201.0Mn $. If for a given portfolio C does not exhibit homogeneity, neither does SA-CCR and therefore Euler allocation of SA-CCR is not possible for such portfolios. \todo{Paragraph needs to be tidied up as it has been relocated}
    
    As can already seen by the notional of the exemplary trade, the liquidity thresholds imposed by ISDA-SIMM are relatively high and will not be exceeded by the majority of bilateral portfolios.
    
    \todo{Insert explanation that this can't really be fixed.}

    \subsection{Incorporation of a minimum transfer amount and threshold}

    To allocate \gls{SA-CCR} under consideration of margining, the available collateral $C$ is of special interest. As pointed out in table \ref{tab:Margin in SA-CCR} depending on the margining approach, C can be calculated as $C = \text{VM}$ or $C = \text{VM} + \text{IM}_{\text{received}}$.
    
    The actually exchanged collateral, however has to be calculated under consideration of the threshold and minimum transfer amount as displayed in equation \ref{eq:C} \todo{Fix reference}. With this consideration of threshold and minimum transfer amount $C$ is not a homogeneous function. 

    \begin{figure}
        \centering
        \includegraphics{Graphics/C_and_its_components_for_ATM_IRS.pdf}
        \caption{VM, IM, C and C under consideration of TH and MTA for a portfolio consisting of a single at the money interest rate swap. Values are calculated based on different notionals invested in the IRS with $a=1$ referring to a notional of 200Bn USD. More details on creation can be found in Appendix \ref{homogeneity-of-c-for-a-single-trade-portfolio}.}
        \label{fig:C for ATM IRS}
    \end{figure}

    \begin{figure}
        \centering
        \includegraphics{Graphics/C_and_its_components_for_ITM_IRS.pdf}
        \caption{Same as figure \ref{fig:C for ATM IRS} but for an in the money trade with an payer interest rate at 2\% which is below par.}
        \label{fig:C for ITM IRS}
    \end{figure}
    
    This can exemplary seen in figures \ref{fig:C for ATM IRS} and \ref{fig:C for ITM IRS}. These figures display $C$ for an at the money 10Y USD interest swap and the same swap with a lower fixed rate making it an in the money swap. Again, a very high notional of 200 Bn is chosen to showcase the concentration risk charge of ISDA SIMM and the threshold and minimum transfer amount have also been chosen to be very high at 2Bn and 1Bn USD.
    
    \subsubsection{Incorporation of a minimum transfer amount}

    \todo{This paragraph probably needs to be removed} To be able to calculate an Euler allocation of SA-CCR one has to calculate C for use in \ref{eq:multiplier} without recognition of the minimum transfer amount as the $C_{calc}$ as defined in equation \ref{eq:C_calc}.
    \begin{align}
        \label{eq:C_calc}
        C_{calc} &= VM + IM_{rec}
    \end{align}
    
    If only a MTA but no threshold is in place, no further adjustment is necessary.
    This can exemplary been seen in the next example.
    Again, similar to the exemplary calculation in section \ref{sec:Allocation when an ISDA-SIMM liquidity threshold is exceeded} it can be shown with a trivial example, that Euler allocation of SA-CCR is not possible under recognition of a \gls{MTA}. 
    For this we consider the same 200Bn \gls{IRS} as in section \ref{sec:Allocation when an ISDA-SIMM liquidity threshold is exceeded}. We assume that the currently posted margin is 9.04Bn which is the calculated ISDA-SIMM margin. 
    Setting $C$ at 9.04Bn in \ref{eq:multiplier} and then calculating the SA-CCR EAD for this single IRS \todo{as specified in ...} yields an regulatory EAD of 582.8Mn USD. Any natively additive allocation should allocate this full amount to the IRS.
    The \gls{VM} is zero as the IRS is struck at par.
    \begin{table}[htbp]
        \label{tab:Allocate SA-CCR with MTA calculation}
        \centering
        \begin{tabular}{c|c|c}
            & $\text{SA-CCR}_{\text{MTA}}$ & $\text{SA-CCR}_{\text{No \gls{MTA}}}$ \\
            \toprule
            Initial C & 9038.2Mn & 9038.2Mn \\
            \midrule
            EAD\textsubscript{200Bn IRS} & 582.8Mn & 582.8Mn \\
            \midrule
            C\textsubscript{200Bn IRS} & 9038.2Mn & 9038.2Mn \\
            \midrule
            C\textsubscript{200.02Bn IRS} & 9038.2Mn & 9039.1Mn \\
            \midrule
            EAD\textsubscript{200.02Bn IRS} & 583.0Mn & 582.9Mn \\
            \midrule
            $\frac{\text{EAD\textsubscript{200.02Bn IRS}}-\text{EAD\textsubscript{200Bn IRS}}}{0.0001}$ & 1425.3Mn & 582.8Mn  \\
        \end{tabular}%
        \caption{Numerical Euler allocation of SA-CCR with and without consideration of a minimum transfer amount for an example of a portfolio with a single 200Bn notional IRS. Euler allocation is only successful if the \gls{MTA} is not considered for the recalculation of the received margin C. A threshold of 0 is assumed.}
    \end{table}
    In table \ref{tab:Allocate SA-CCR with MTA calculation} we assume that the initially received collateral is the currently calculated collateral. 
    When calculating the Euler allocation with a forward difference in line with \ref{eq:forward difference} the received collateral when rising the notional to 200.02Bn increases when no MTA is assumed while it remains unchanged with consideration of an MTA. 
    Ultimately, this difference leads to a correct allocation of the entire EAD to the single IRS with the \emph{No MTA} approach while the \emph{MTA} approach obviously yields a wrong result by allocating 244\% of the portfolios EAD to its only trade.
    
    \subsubsection{Incorporation of a VM threshold\label{sec:Incorporation of a VM threshold}}
    
    \subsubsection{Incorporation of an IM threshold\label{sec:Incorporation of an IM threshold}}
    
    Consideration of the threshold is challenging. When calculating $C$ with a threshold, $C$ does not exhibit homogeneity and an allocation will fail. On the other hand, when calculating $C$ without threshold the considered margin is to high and the sum of allocated EAD will be too low.
    One solution to this is to first allocate the SA-CCR assuming the threshold is 0 and then scaling up the allocation by $EAD_{TH = 0} \ EAD$ where $EAD_{TH = 0}$ is the calculated SA-CCR EAD assuming a zero threshold and $EAD$ is the actual SA-CCR EAD.
    
    
    \subsection{Allocation of hedged portfolios\label{sec:Allocation of hedged portfolios}}
    
    As pointed out in \ref{Allocation of Risk Measures} Euler allocation is a risk sensitive allocation and as such does generally attribute negative contributions to trades that are decreasing the risk of the portfolio. If we consider a portfolio of a 200Mn payer IRS, an equivalent 100Mn receiver IRS and 1Mn long stock call options we yield the result depicted in \ref{tab:hedge trade sample results} when calculating the Euler allocation numerically with a forward difference.
    \begin{table}[htbp]
        \label{tab:hedge trade sample results}
        \centering
        \begin{tabular}{c|c|c}
            & SA-CCR & ISDA SIMM \\
            \toprule
            IRS\textsubscript{100Mn Rec} & -246k & -4.52Mn \\
            \midrule
            IRS\textsubscript{200Mn Pay} & 493k & 9.04Mn \\
            \midrule
            Equity Option & 1.30Mn & 6.95Mn \\
            \bottomrule
            Sum of allocations & 1.54Mn & 11.47Mn \\
            \midrule
            Portfolio risk measure & 1.54Mn & 11.47Mn \\
        \end{tabular}%
        \caption{}
    \end{table}
    The 100Mn receiver IRS partially hedges the risk induced by the 200Mn payer IRS. 
    Both, the ISDA SIMM and the SA-CCR model do not recognize any hedge effects across asset classes and therefore the risk associated with the equity option is completely independent from the two IRS trades. 
    Appropriately, a negative IM and EAD is allocated to the smaller IRS trade and the allocation exhibits native additivity as the sum of the allocation of the three trades coincides with the risk measures of the portfolio.
    
    However, an allocation 
    
    \begin{table}[htbp]
        \label{tab:EAD perfect hedge}
        \centering
        \begin{tabular}{c|c|c|c}
            & Forward & Central & Backward \\
            \toprule
            IRS\textsubscript{200Mn Rec} & 188k & 0k & -188k \\
            \midrule
                IRS\textsubscript{200Mn Pay} & 188k & 0k & -188k\\
                \midrule
                Equity Option & 1.32Mn & 1.32Mn & 1.32Mn\\
                \bottomrule
                Sum of allocations & 1.69Mn & 1.32Mn & 944k \\
                \midrule
                Portfolio risk measure & \multicolumn{3}{c}{1.32Mn} \\
            \end{tabular}%
            \caption{}
    \end{table}
    
    \begin{table}[htbp]
        \label{tab:IM perfect hedge}
        \centering
        \begin{tabular}{c|c|c|c}
            & Forward & Central & Backward \\
            \toprule
            IRS\textsubscript{200Mn Rec} & 4.52Mn & 0.00Mn & -4.52Mn \\
            \midrule
            IRS\textsubscript{200Mn Pay} & 4.52Mn & 0.00Mn & -4.52Mn \\
            \midrule
            Equity Option & 6.95Mn & 6.95Mn & 6.95Mn \\
            \bottomrule
            Sum of allocations & 15.99Mn & 6.95Mn & -2.09Mn \\
            \midrule
            Portfolio risk measure & \multicolumn{3}{c}{6.95Mn}  \\
        \end{tabular}%
        \caption{}
    \end{table}
    
    \end{document}
    de. 